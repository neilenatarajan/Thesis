%%%%%%%%%%%%%%%%%%%%%%%%%%%%%%%%%%%%%%%%%%%%%%%%%%%%%%%%%%%%%%%
%% OXFORD THESIS TEMPLATE

% Use this template to produce a standard thesis that meets the Oxford University requirements for DPhil submission
%
% Originally by Keith A. Gillow (gillow@maths.ox.ac.uk), 1997
% Modified by Sam Evans (sam@samuelevansresearch.org), 2007
% Modified by John McManigle (john@oxfordechoes.com), 2015
%
% This version Copyright (c) 2015-2023 John McManigle
%
% Broad permissions are granted to use, modify, and distribute this software
% as specified in the MIT License included in this distribution's LICENSE file.
%

% I've (John) tried to comment on this file extensively, so read through it to see how to use the various options.  Remember
% that in LaTeX, any line starting with a % is NOT executed.  In several places below, you have a choice of which line to use
% out of multiple options (eg draft vs final, PDF vs binding, etc.)  When you pick one, add a % to the beginning of
% the lines you don't want.


%%%%% CHOOSE PAGE LAYOUT
% The most common choices should be below.  You can also do other things, like replacing "a4paper" with "letterpaper", etc.

% This one will format for two-sided binding (ie left and right pages have mirror margins; blank pages inserted where needed):
% \documentclass[a4paper,twoside]{ociamthesis}
% This one will format for one-sided binding (ie left margin > right margin; no extra blank pages):
%\documentclass[a4paper]{ociamthesis}
% This one will format for PDF output (ie equal margins, no extra blank pages):
\documentclass[a4paper,nobind]{ociamthesis} 

%%%%% SELECT YOUR DRAFT OPTIONS
% Three options going on here; use in any combination.  But remember to turn the first two off before
% generating a PDF to send to the printer!

% This adds a "DRAFT" footer to every normal page.  (The first page of each chapter is not a "normal" page.)
% \fancyfoot[C]{\emph{DRAFT Printed on \today}}  

% This highlights (in blue) corrections marked with (for words) \mccorrect{blah} or (for whole
% paragraphs) \begin{mccorrection} . . . \end{mccorrection}.  This can be useful for sending a PDF of
% of your corrected thesis to your examiners for review.  Turn it off, and the blue disappears.
% \correctionstrue


%%%%% BIBLIOGRAPHY SETUP
% Note that your bibliography will require some tweaking depending on your department, preferred format, etc.
% The options included below are basic "sciencey" and "humanities" options to get started.
% If you've not used LaTeX before, I recommend reading about biblatex/biber and getting started with it.
% If you're already a LaTeX pro and are used to natbib or something, modify as necessary.
% Either way, you'll have to choose and configure an appropriate bibliography format...

% The science-type option: numerical in-text citation with references in order of appearance.
\usepackage[style=numeric-comp, sorting=nyt, backend=biber, maxcitenames=2, doi=false, isbn=false]{biblatex}


% \usepackage[style=authoryear, sorting=nyt, backend=biber, maxcitenames=2, useprefix, doi=false, isbn=false]{biblatex}
\newcommand*{\bibtitle}{References}

% This makes the bibliography left-aligned (not 'justified') and slightly smaller font.
\renewcommand*{\bibfont}{\raggedright\small}

% Change this to the name of your .bib file (usually exported from a citation manager like Zotero or EndNote).
\addbibresource{references.bib}


% Uncomment this if you want equation numbers per section (2.3.12), instead of per chapter (2.18):
%\numberwithin{equation}{subsection}


%%%%% INCLUDING PACKAGES

% Useful for including algorithms and pseudocide
\usepackage{algorithm}
\usepackage{algpseudocode}

% Required for inserting images
\usepackage{tabularx} 

% For subfigures; subfigure and subfigure are both superseded by subscription
\usepackage{subcaption}
% \usepackage{subfig}
% \usepackage{subfigure}

\usepackage{amsmath}

% For graphics (and sets path to the figure path)
\usepackage{graphicx}
\graphicspath{{./figures/}}


\usepackage{array}
\usepackage{multirow}

\usepackage{wrapfig}
\usepackage[utf8]{inputenc} % allow utf-8 input
\usepackage[T1]{fontenc}    % use 8-bit T1 fonts
\usepackage{hyperref}       % hyperlinks
\usepackage{url}            % simple URL typesetting
\usepackage{booktabs}       % professional-quality tables
\usepackage{amsfonts}       % blackboard math symbols
\usepackage{nicefrac}       % compact symbols for 1/2, etc
\usepackage{microtype}      % microtypography
\usepackage{xcolor}         % colors

% For wider tables
\usepackage{adjustbox}

% For the cancelto command
\usepackage[makeroom]{cancel}

% For correct real number formatting
\usepackage{amssymb}
\usepackage{amsthm}

% For theorems in the body
\newtheorem{theorem}{Theorem}
\newtheorem{corollary}[theorem]{Corollary}
\newtheorem{proposition}{Proposition}
% \newenvironment{proof}[1][Proof]

% \usepackage{dcolumn} % Booktabs column spacing
% \usepackage{etoolbox}
% \usepackage{tikz}
% \usepackage[makeroom]{cancel}
% \usepackage{adjustbox}
% \usepackage{comment}
% \usepackage{longtable}
% \usepackage{geometry}
% \geometry{verbose,tmargin=1in,bmargin=1in,lmargin=1in,rmargin=1in}
% \usepackage{setspace}
% \usepackage{makecell}
% \usepackage{soulpos} % for the command \hl
% \usepackage{ulem}
% \usepackage{eurosym}
% \usepackage{color}
% \usepackage{sectsty}
% \usepackage{palatino}
% \usepackage{caption}
% \usepackage{pdflscape}
% \usepackage{csquotes}
% \usepackage{changepage}
% \usepackage{alphalph}
% \renewcommand*{\thesubfigure}{%
% \alphalph{\value{subfigure}}%
% }%

%%%%% THESIS / TITLE PAGE INFORMATION
% Everybody needs to complete the following:
\title{Selection-Oriented AI: The Role of HCI in Supporting Solutions to Explainability, Plagiarism, and Diversity in Global Scholarship Selection}
\author{Neil Natarajan}
\college{New College}

% Master's candidates who require the alternate title page (with candidate number and word count)
% must also un-comment and complete the following three lines:
%\masterssubmissiontrue
%\candidateno{933516}
%\wordcount{28,815}

% Uncomment the following line if your degree also includes exams (eg most masters):
%\renewcommand{\submittedtext}{Submitted in partial completion of the}
% Your full degree name.  (But remember that DPhils aren't "in" anything.  They're just DPhils.)
\degree{Doctor of Philosophy}
% Term and year of submission, or date if your board requires (eg most masters)
\degreedate{Trinity 2024}


%%%%% YOUR MACROS
% This is a good place to dump your LaTeX macros as they come up.

% To make text superscripts shortcuts
	\renewcommand{\th}{\textsuperscript{th}} % ex: I won 4\th place
	\newcommand{\nd}{\textsuperscript{nd}}
	\renewcommand{\st}{\textsuperscript{st}}
	\newcommand{\rd}{\textsuperscript{rd}}

\DeclareLanguageMapping{latin}{english}

\counterwithout{footnote}{chapter}

%%%%% THE ACTUAL DOCUMENT STARTS HERE
\begin{document}

%%%%% CHOOSE YOUR LINE SPACING HERE
% This is the official option.  Use it for your submission copy and library copy:
\setlength{\textbaselineskip}{22pt plus2pt}
% This is closer spacing (about 1.5-spaced) that you might prefer for your copies:
%\setlength{\textbaselineskip}{18pt plus2pt minus1pt}

% You can set the spacing here for the Roman-numbered pages (acknowledgements, table of contents, etc.)
\setlength{\frontmatterbaselineskip}{17pt plus1pt minus1pt}

% Leave this line alone; it gets things started for the real document.
\setlength{\baselineskip}{\textbaselineskip}


%%%%% CHOOSE YOUR SECTION NUMBERING DEPTH HERE
% You have two choices.  First, how far down are sections numbered?  (Below that, they're named but
% don't get numbers.)  Second, what level of section appears in the table of contents?  These don't have
% to match: you can have numbered sections that don't show up in the ToC, or unnumbered sections that
% do.  Throughout, 0 = chapter; 1 = section; 2 = subsection; 3 = subsubsection, 4 = paragraph...

% The level that gets a number:
\setcounter{secnumdepth}{3}
% The level that shows up in the ToC:
\setcounter{tocdepth}{1}


%%%%% ABSTRACT SEPARATE
% This is used to create the separate, one-page abstract that you are required to hand to the Exam
% Schools.  You can comment it out to generate a PDF for printing or whatnot.
% \begin{abstractseparate}
% 	Selecting people for opportunities like jobs, universities, loans, or scholarships pervades and shapes society. And while processes exist for people to make these decisions at scale, these processes are unequipped to handle the elevated demands of modernity. The work in this thesis explores the use of data-driven Decision Support Tools (DSTs) to improve selection processes, focusing on two global scholarship programmes.

We frame our investigation in terms of the \emph{Decision Matrix} framework, categorising decisions by stage (in-process or ex-post) and stakes (high or low). We then explore using existing AI tools as DSTs, focusing on post-hoc explainable AI and generative AI detectors. We find them ineffective for in-process decisions but useful ex-post. We engage in participatory design to create six design prototypes to assist with in-process decision-making, with a focus on diversity. Participants demonstrated enthusiasm for using these tools across the Decision Matrix. To validate this enthusiasm, we implemented one design as a technology probe and evaluated its impact. The selected cohort's diversity and performance improved, demonstrating the tool's ability to support high-stakes in-process decisions. 

Our findings highlight the need for data-driven and AI-based DSTs across the Decision Matrix. We propose \emph{Selection-Oriented AI}, a design paradigm focused on the social goals of selection, and provide design recommendations. We conclude with a call for AI-driven DSTs that balance practitioners' needs while optimising selection outcomes for social benefit. % Create an abstract.tex file in the 'text' folder for your abstract.
% \end{abstractseparate}


% JEM: Pages are Roman numbered from here, though page numbers are invisible until ToC.  This is in
% keeping with most typesetting conventions.
\begin{romanpages}

% JEM: By default, this template uses the traditional Oxford "Belt Crest". Un-comment the following
% line to use the newer, "Blue Square" logo:
% \renewcommand{\crest}{{\includegraphics[width=4.2cm, height=4.2cm]{figures/newlogo.pdf}}}

% Title page is created here
\maketitle

%%%%% DEDICATION -- If you'd like one, un-comment the following.
%\begin{dedication}
%This thesis is dedicated to\\
%someone\\
%for some special reason\\
%\end{dedication}

%%%%% ACKNOWLEDGEMENTS -- Nothing to do here except comment out if you don't want it.
\begin{acknowledgements}
 	First and foremost, I would like to thank my supervisor, Reuben Binns, for his dedication, patience, and wisdom. I would also like to thank my secondary supervisor, Nigel Shadbolt, for his efficient wisdom.

I would like to thank my coauthors and co-conspirators: Sruthi Viswanathan, for teaching me the basics of HCI and for keeping me sane; Thomas Serban Von Davier, for his infinite willingness to read each others' work; Ulrik Lyngs, for teaching me statistics; Elías Hanno, for making much of my action research possible; Kadeem Noray, for an astonishingly deep knowledge of the economics of talent; Logan Gittelson, for writing code that I couldn't have; and Elijah Mayfield, for teaching me how to write a thesis.

I would like to thank my rotating cast of assessors: Max Van Kleek, Jun Zhao, Marina Jirotka, and Tim Miller.

I would like to thank my research community, for fostering an environment of collaboration and support. Thank you Jake Stein, Tyler Reinmund, Lize Alberts, Laura Csuka, Jumana Baghabrah, Sarah Aldaweesh, Sarah Alromaih, Tala Ross, and Helen Gee.

I would like to thank my fiancée for her copyediting prowess; my parents, for their eternal belief in me; and my sister, for teaching me that we go far together.
\end{acknowledgements}

%%%%% ABSTRACT -- Nothing to do here except comment out if you don't want it.
\begin{abstract}
	Selecting people for opportunities like jobs, universities, loans, or scholarships pervades and shapes society. And while processes exist for people to make these decisions at scale, these processes are unequipped to handle the elevated demands of modernity. The work in this thesis explores the use of data-driven Decision Support Tools (DSTs) to improve selection processes, focusing on two global scholarship programmes.

We frame our investigation in terms of the \emph{Decision Matrix} framework, categorising decisions by stage (in-process or ex-post) and stakes (high or low). We then explore using existing AI tools as DSTs, focusing on post-hoc explainable AI and generative AI detectors. We find them ineffective for in-process decisions but useful ex-post. We engage in participatory design to create six design prototypes to assist with in-process decision-making, with a focus on diversity. Participants demonstrated enthusiasm for using these tools across the Decision Matrix. To validate this enthusiasm, we implemented one design as a technology probe and evaluated its impact. The selected cohort's diversity and performance improved, demonstrating the tool's ability to support high-stakes in-process decisions. 

Our findings highlight the need for data-driven and AI-based DSTs across the Decision Matrix. We propose \emph{Selection-Oriented AI}, a design paradigm focused on the social goals of selection, and provide design recommendations. We conclude with a call for AI-driven DSTs that balance practitioners' needs while optimising selection outcomes for social benefit.
\end{abstract}

%%%%% MINI TABLES
% This lays the groundwork for per-chapter, mini tables of contents.  Comment on the following line
% (and remove \minitoc from the chapter files) if you don't want this.  Un-comment either of the
% next two lines if you want a per-chapter list of figures or tables.
\dominitoc % include a mini table of contents
%\dominilof  % include a mini list of figures
%\dominilot  % include a mini list of tables

% This aligns the bottom of the text of each page.  It generally makes things look better.
\flushbottom

% This is where the whole-document ToC appears:
\tableofcontents

% Uncomment to generate a list of figures:
% \listoffigures
% 	\mtcaddchapter

% \mtcaddchapter is needed when adding a non-chapter (but chapter-like) entity to avoid confusing minitoc

% Uncomment to generate a list of tables:
%\listoftables
%	\mtcaddchapter

%%%%% LIST OF ABBREVIATIONS
% This example includes a list of abbreviations.  Look at text/abbreviations.tex to see how that file is
% formatted.  The template can handle any kind of list though, so this might be a good place for a
% glossary, etc.
% First parameter can be changed eg to "Glossary" or something.
% Second parameter is the max length of bold terms.

% [FIXED] Selection – May be useful to scope that this thesis explicitly deals with positive discrimination in talent selection (in the cohort or individual contexts as discussed above) as opposed to other aspects of selection.

% [FIXED] Section 1.2: I don't entirely agree that HCAI would centre the task around Selectors. It would centre around anyone involved in the process: selectors, their team/bosses, the applicants, etc…. if done properly! Perhaps it is correct to say that many people would choose to centre it around selection?

\begin{mclistof}{Glossary}{3.2cm}
    \item[Human-Computer Interaction (HCI)] A subfield of Computer Science that deals primarily with how people interact with computers and to what extent computers are or are not developed for successful interaction with human beings. This thesis is a work of HCI.

    \item[Participatory Design (PD)] A paradigm within HCI that engages participants as co-designers in an iterative design process, recognising the user as ideally positioned to understand user needs and preferences. Research outputs are usually designs and design recommendations driven by careful analysis of user feedback. Much of the work in this thesis is inspired by the PD paradigm.

    \item[Action Research (AR)] A family of methods within HCI that engages a group of practitioners as co-researchers and co-participants in the research process; in this case, preparation is only one part of the research process, while action and reflection are equally valuable. Research outputs are ordinarily learnings that arise from the action. Much of the work in this thesis is inspired by the AR paradigm.

    \item[Value-Sensitive Design (VSD)] A family of methods within HCI engaging participants, where particular values of participants are elicited and used as a guide for design. Research outputs are usually designs and design recommendations driven by careful analysis of user values. This thesis engages with VSD in supporting diversity.

    \item[Human-Centred Computing (HCC)] A subfield of Computer Science that designs and develops computer systems around the needs and desires of a group of humans, thus `centring' that group of humans. This thesis's central contribution (Selection-Oriented AI) is offered in contrast to HCC.

    \item[Artificial Intelligence (AI)] Various definitions exist, including the study of intelligent behaviour in computers \cite{wang2008you}, computational requirements for tasks like perception or reasoning \cite{Leake2001ArtiicialI}, or large models such as ChatGPT or DALL-E \cite{du2020ai}. AI is often construed as definitionally aspirational, i.e., it is taken as a given that current computer systems are not AI \cite{wang2008you}. In this thesis, any computer system that can be said to exhibit behaviour similar to human intelligence is included, and all work herein seeks to build or evaluate AI tools.

    \item[Explainable Artificial Intelligence (XAI)] A subfield of AI that develops and assesses explanations that make AI systems more legible to a group of humans. Chapter \ref{ch:xai}, in particular, engages in a debate over the usefulness of xAI.

    \item[Generative Artificial Intelligence (GenAI)] A subfield of AI that develops and assesses AI systems, usually large machine learning models, that generate new data, such as text, images, or audio. Chapter \ref{ch:genai} concerns itself with genAI and the detection of genAI.

    \item[Diversity] In its broadest sense, refers to variety, difference, or heterogeneity within a given collection of entities. The seminal definition by Page describes it as: ``The heterogeneity of elements in a set about a class that takes different values, such as species in an eco-environment, or ethnicity in a population'' \cite{page_diversity_2010}. While this definition is broad enough for contexts such as ecology, a more nuanced understanding is required in the context of applicant selection (see Chapter \ref{sec:context_diversity}).

    \item[Human-Centred Artificial Intelligence (HCAI)] A subfield of HCC that concerns itself with AI systems, rather than all computer systems. This thesis's central contribution (Selection-Oriented AI) is offered in contrast to HCAI.

    \item[Selection] Occurs in a variety of forms throughout society, from recruitment to matchmaking. In this thesis, selection refers exclusively to the processes of scholarships and other academic or talent investment opportunities, with a primary interest in selection processes for social benefit rather than organisational benefit.

    \item[Selector] Practitioners responsible for making selection decisions, both direct and supporting, within selection teams and organisations. These practitioners are referred to as selectors, and tools are built to support their decision-making processes.

    \item[Selection-Oriented AI (SOAI)] A family of methodologies designed to achieve the social values of properly selecting scholars. In contrast to HCAI, which would centre the point of view of various stakeholders, SOAI orients itself around the social benefits of selection, deviating from the point of view of the selectors, applicants, and other stakeholders when their values differ.
\end{mclistof}

% The Roman pages, like the Roman Empire, must come to its inevitable close.
\end{romanpages}

%%%%% CHAPTERS
% Add or remove any chapters you'd like here, by file name (excluding '.tex'):
\flushbottom
\chapter{\label{ch:intro}Introduction} 

% As noted below, your introduction does a nice job in setting the scene for the overall thesis. However, it could do better with properly clarifying what the research questions of the thesis are, and how all the chapters included fit into the overall narrative. 

\minitoc

\section{Motivation}

% [FIXED] Cohort/Team Selection vs Aspirational Quota Selection – The football example used to open the introduction is a fun example, but it would seem to differ significantly from the selection scenarios you focus on in the thesis. How does choosing candidates for scholarship applications, hiring, or university admissions differ from team/cohort selection, as in the football case?  For instance, this often requires assembling groups with specific, predefined compositions. In contrast, the other scenarios generally aim to meet aspirational diversity goals across the entire intake, with metrics analysed post-hoc rather than determining each group's precise makeup. This raises an important question: how do the methods presented in your thesis apply to strict compositional requirements compared to broader, trend-focused selection goals?  (You may, in fact, want to discuss this in 1.2. Scope and Terms as well–as all selection problems aren't equal, it might be useful to introduce fundamental terms relating to different kinds of candidate selection to better define the scope. )

Consider a prestigious global scholarship program receiving thousands of applications from aspiring scholars worldwide. The selection committee faces a complex decision: from this diverse pool of talented individuals, they must choose a cohort that will best advance the program's mission to develop future leaders who will tackle global challenges. Unlike a simple ranking of academic merit, this selection process involves balancing multiple, sometimes competing values. Should they prioritise applicants with the highest test scores, those who demonstrate the greatest potential for social impact, candidates from underrepresented regions, or individuals who have overcome significant adversity? The committee's approach will depend on how they interpret the program's goals and what they believe constitutes the ``best'' cohort, and this decision carries profound implications for both the selected scholars and the communities they will eventually serve.

This ``selection problem'', choosing whom to include, echoes throughout society. Employers select whom to hire. Creditors select whom they lend to. Universities select whom to admit. Scholarship programmes, like our hypothetical one, select whom to award. An organisation's goals and values determine its selections. An employer may seek the best candidate for a specific task, while a creditor seeks debtors likely to repay loans. Universities and scholarship programmes, however, often differ. Their selection is not for organisational benefit but for a social one \cite{Warikoo_2019}. Thus, instead of selecting only applicants who yield the best returns to the organisation, they seek to select those who are most deserving, will learn the most, have the greatest need, whose presence will benefit others, or who will use their education to most improve society.\footnote{Hirers and lenders are often bound by law to select the most deserving, and they occasionally select those who will benefit others. However, these institutions are generally motivated to maximise profits within legal bounds \cite{schmidt1998validity}.}

Although a wealth of research explores when and how algorithms can make hiring or lending decisions (and whether they should) \cite{schmidt1998validity,schumann2017diverse,raghavan2020mitigating,horodyski_applicants_2023,Leung_Zhang_Jibuti_Zhao_Klein_Pierce_Robert_Zhu_2020}, little research explores how algorithms can support selection decisions in the scholarship context. The research that exists in the university context often likens this problem more to hiring than to scholarship \cite{schumann2017diverse,Steel_Multiple_2018,ijcai2023p819}. Furthermore, just as research on hiring and lending finds flaws in many applications of algorithms \cite{raghavan2020mitigating,horodyski_applicants_2023,Peng_Nushi_Kıcıman_Inkpen_Suri_Kamar_2019}, the scant research on human-led scholarship and university selection finds similar problems \cite{schumann2017diverse}.

In a world flattened by the global proliferation of technology \cite{Friedman_2005}, new global scholarship initiatives aim to select scholars from every part of the world. These programmes offer applicants worldwide access to educational resources that may have previously been inaccessible, but they also exacerbate the problems found in related work. If these programmes can select the ``best'' cohorts, they can deliver on their stated missions to improve the world by broadening access to elite higher education and training scholars to solve the world's most significant problems. This thesis aims to enable that mission by supporting these selection processes with algorithmic decision-support tools.

% [FIXED] Scope – Despite the title, the section doesn't establish a scope – e.g. what is in and what is out of scope of this thesis.  What sort of selection problems are within scope and what is not? More importantly, the questions you present in this section read like research questions; indeed, it is appropriate for introductions of theses to articulate a set of central research questions for the entire thesis.  However, it is unclear that these questions relate to some of the chapters; ensure they are able to admit and support the work on explanations (Ch 4), AI based plagiarism (Ch5), or even very much to Chapter 6.  Also, in RQ1 you mention applicant-based decisions and cohort-based decisions interchangeably – are they both in scope?
\section{Scope of the Thesis}

Global scholarship programs, such as the Rise and Ellison Scholars programmes, with whom we conducted this research, face daunting challenges in their mission to select and cultivate future leaders. These challenges include navigating different interpretations of the ``best'' cohort, ensuring fair comparisons of applicants from diverse global contexts, and mitigating risks such as applicants using generative AI to misrepresent their aptitude. Existing low-tech decision-making systems are often unequipped to handle these newfound complexities \cite{Latzer_Hollnbuchner_Just_Saurwein_2014}. While selection practitioners (selectors) design innovative solutions, they often lack easy access to the information needed to robustly support their decisions.

With now all selection problems being equal, to properly define this thesis's scope, we must distinguish between different types of candidate selection, and narrow our scope to the kind of problem at hand. We identify two primary categories of selection that differ fundamentally in their approach and objectives:

\paragraph{Strict Quota Selection} involves assembling groups with specific, predefined compositions to fulfil particular functional requirements. Examples include selecting a sports team where specific positions must be filled (e.g., two defenders, two midfielders, and one goalkeeper) or hiring a team of specific roles (e.g., one software engineer and one salesperson). In these contexts, selectors know in advance how many individuals with particular characteristics they need. The selection process is constrained by these structural requirements, and success requires that the final composition match the predetermined template, but is otherwise evaluated by evaluating each individual independently.

\paragraph{Aspirational Selection} aims to meet broader diversity and excellence goals across an entire intake rather than strictly constraining each individual selection decision. This approach characterises most scholarship, university admissions, and similar social programmes. Selectors work toward aspirational targets (e.g., achieving geographic diversity or representing underrepresented groups) but do not have strict quotas. Here, these compositional elements are a goal, rather than a constraint. Success is evaluated by examining whether the cohort reflects the programme's values and mission, which requires considering the group holistically.

This thesis focuses primarily on \emph{Aspirational Selection}, as global scholarship programmes like Rise and Ellison Scholars primarily use \emph{Aspirational Selection} to achieve diversity and excellence. We designed the methods and tools herein to support selection processes where values-based decision-making predominates over structural composition requirements and where the ``best'' cohort is determined by balancing multiple competing objectives rather than fulfilling predetermined quotas.\footnote{Historically, some programmes have used \emph{Strict Quota Selection} as a means of operationalising the aspirational goals of the programme in structures that allow stakeholders to evaluate the programme, but this process has fallen out of favour and is not used by either the Rise or Ellison Scholars programmes.}

% [FIXED] You discuss "competing values" abstractly. Can you give some examples of values?  Please include "values" as a key term, and examples of these.

\paragraph{Values} In scholarship selection, a programme's values are the fundamental principles that guide decisions about which applicants to select. Common values include: \emph{academic excellence} (prioritising applicants with high test scores or grades); \emph{potential for social impact} (favouring candidates with compelling visions for addressing global challenges); \emph{diversity} (seeking applicants from a variety of backgrounds); \emph{need} (supporting those with limited access to education); \emph{integrity} (valuing truth-seeking and truth-telling); or \emph{resilience} (recognising applicants who have overcome adversity). These values often conflict; an applicant with top academic credentials may not show the greatest potential for social impact, and those with the greatest need may not come from the most underrepresented regions. In practice, these conflicts lead to vastly different understandings of the ``best'' cohort \cite{zimmerman_research_2014}. Selectors must balance these competing values to select cohorts that best embody their programme's mission.

We term the primary decision to overcome this conflict \emph{Selection}: choosing the most apt cohort of applicants according to a specific organisation set of values. This central decision is supported by many subordinate decisions, such as: ``What criteria make one applicant (or cohort) more apt than another?'' and ``How can we apply these criteria to select the most apt cohort?'' Each subordinate decision is itself supported by further decisions regarding programme purpose, metrics, and their application.

The scope of this thesis is the development and evaluation of Decision Support Tools (DSTs) to enhance \emph{Selection} processes within global scholarship programmes. Specifically, it focuses on building and assessing DSTs that address three critical challenges selectors face:

\begin{enumerate}
    \item How to ensure that AI recommendations are interpretable and trustworthy (Chapter \ref{ch:xai}).
    \item How to maintain integrity when applicants may use generative AI (Chapter \ref{ch:genai}).
    \item How to understand, operationalise, and foster diversity within selected cohorts (Chapters \ref{ch:diversity} and \ref{ch:spf}).
\end{enumerate}

We conducted this research in collaboration with Rise and Ellison Scholars. While this thesis engages with the ethical dimensions of AI-supported selection, it does not aim to provide definitive solutions to all underlying societal inequities. Instead, it seeks to improve the tools and processes available to selectors. We acknowledge work outside this direct scope, such as critical theory perspectives on selection, in Chapter \ref{ch:discussion}.

\section{Research Questions}

This thesis is guided by the following central research questions:

\begin{enumerate}
    \item[\textbf{RQ1:}] How can Decision Support Tools (DSTs) be effectively designed, implemented, and evaluated to aid selectors in global talent investment programs in making values-driven selection decisions that balance multifaceted objectives like excellence, diversity, and fairness, across both aspirational quota and cohort-based selection paradigms?
    \item[\textbf{RQ2:}] In what ways do emerging AI technologies, particularly explainable AI (XAI) and generative AI (GenAI), impact the decision-making processes, integrity, and perceived legitimacy of scholarship selection, and what frameworks or interventions can mitigate potential harms while leveraging benefits for selectors?
    \item[\textbf{RQ3:}] How can complex and evolving conceptualizations of diversity be translated into practical support mechanisms within DSTs to assist scholarship programs in achieving their diversity-related goals, and what is the real-world efficacy of such mechanisms when deployed in live selection processes?
\end{enumerate}

\section{Contributions} 
The contributions of this thesis are twofold. There are meta-level conceptual distinctions introduced, and also some substantive contributions associated with body chapters.

The meta-level contributions of this thesis are:

\begin{itemize}
    \item A list of decision points facing scholarship programmes uncovered through longitudinal HCI research with Rise and Ellison Scholars.
    \item The SOAI paradigm for designing AI systems that support one of the social benefits of good selection processes.
    \item A set of design recommendations for designers seeking to apply SOAI to build a DST to support selectors.
\end{itemize}

However, this thesis is composed of several papers that make more specific, core contributions to support tools seeking to solve issues of Explainability, Plagiarism, and Diversity. These are detailed in the relevant chapters but are also described here.

\paragraph{Explainability}
\begin{itemize}
    \item Quantitative findings indicating that the problem of explanation-induced unwarranted trust extends to generic post-hoc justifications, but that such criticism only applies in process (Chapter \ref{ch:xai}).
    \item Qualitative findings that post-hoc explanations, properly presented, can make useful ex-post DSTs (Chapter \ref{ch:xai}).
\end{itemize}

\paragraph{Plagiarism}
\begin{itemize}
    \item An evaluation of GenAI detectors GPTZero and Originality.ai on Rise's 2022 and 2023 application data (Chapter \ref{ch:genai}).
    \item The Decision Matrix framework for evaluating the suitability of AI systems as support tools for differing decision points.
    \item A case study using GPTZero to support two decision points facing Rise (Chapter \ref{ch:genai}).
\end{itemize}

\paragraph{Diversity}
\begin{itemize}
    \item The Diversity Triangle, categorising diversity-related themes according to our three definitions of diversity uncovered through inductive thematic analysis (Chapter \ref{ch:diversity}).
    \item Six design prototypes developed through PD for supporting the diversity needs of a given organisation (Chapter \ref{ch:diversity}).
    \item Design recommendations grounded in PD for system implementers supporting the diversity needs of a given organisation (Chapter \ref{ch:diversity}).
    \item A field deployment of Prototype \ref{fig:diversity} to the Rise selection process selecting $500$ finalists from a pool of $2000$ demonstrating the efficacy of this prototype in practice (Chapter \ref{ch:spf}).
    \item A demonstration of a hypothetical application of Prototype \ref{fig:diversity} as an ex-post DST (Chapter \ref{ch:spf}).
\end{itemize}

\section{Thesis Structure}
Chapter \ref{ch:context} serves as an extended introduction and background chapter, including situating this thesis in related work. Following this, Chapter \ref{ch:methods} explores the paradigms that guide research design throughout this thesis, lists methods used throughout the thesis, and ties these methods to specific chapters.

Chapter \ref{ch:xai} responds to common criticisms of post-hoc XAI and explores this approach as a scholarship selection DST via PD workshops with selectors from Rise. Chapter \ref{ch:genai} engages selectors from Rise in an AR process and explores the role of generative AI in selection decisions. Chapter \ref{ch:diversity} engages selectors from both Rise and Ellison Scholars in participatory design to explore selector notions of diversity and potential ways to support these considerations; this chapter ultimately develops 6 design prototypes. Chapter \ref{ch:spf} implements one of these prototypes in a field deployment with Rise, evaluates that deployment, and explores other applications of the technology.

Chapter \ref{ch:discussion} discusses the scope of the thesis, including references to critical theory work falling outside the scope; the SOAI paradigm; design recommendations that developers can use to follow SOAI; methodological and technical limitations of the work herein; and this thesis's broader significance in a quickly changing landscape. 

\section{Papers}
\subsection{Archival and Under Review}
\begin{itemize}
    \item Kadeem Noray and Neil Natarajan. 2024. "Selecting for Diverse Talent: Theory and Evidence." Under review at Economics of Talent Meeting, Fall 2024.
    \item Neil Natarajan, Sruthi Viswanathan, Reuben Binns, Nigel Shadbolt. 2024. "'Diversity is Having the Diversity': Unpacking and Designing for Diversity in Applicant Selection." Under review at CHI 2025.
    \item Neil Natarajan, Reuben Binns, Ulrik Lyngs, Nigel Shadbolt. 2024. "XAI: Misleading In Process, but Useful Post Hoc." Under review at CHI 2025.
    \item Neil Natarajan, Elías Hanno, Logan Gittelson, Reuben Binns, Nigel Shadbolt. 2024. "What Are Generative AI Detectors Good For? Evaluating and Implementing with the Decision Matrix." Under review at CHI 2025.
\end{itemize}

\subsection{Peer Reviewed}
\begin{itemize}
    \item \fullcite{natarajan_detecting_2024}
    \item \fullcite{ijcai2023p819} 
    \item \fullcite{natarajan_trust_2023}
\end{itemize}




\chapter{\label{ch:context}Background and Context}
\minitoc

\section{The Social Value of Selection}\label{sec:social_value}
Scholarship programmes offer long-term benefits to their chosen scholars, often under the theory that providing these benefits improves not only the welfare of the scholars themselves but of society as a whole \cite{DilraboJonbekova_Ruby_2023,Dassin_Marsh_Mawer_2018}. Theories of the mechanisms of this social benefit vary. Some theories rely on the future actions of the chosen scholars. For example, \textcite{Dassin_Marsh_Mawer_2018} argue that scholars are often empowered and disposed to devote themselves to solving global problems. \textcite{Dassin_Marsh_Mawer_2018} also note that these scholars may bring additional returns to their communities, thereby improving the welfare of a broad group of people (though they also express concern over `brain drain', where these scholars do not return to their communities). In contrast, others contend that the mere provision of scholarships to the correct recipients is itself pro-social. \textcite{minkin2023diversity} note that society benefits from making space for a breadth of perspectives; providing scholarships to those who would otherwise be unable to afford higher education may create that breadth of perspectives. Some evidence suggests that this broader range of perspectives brings additional benefits in the form of increased productivity \cite{autor2008does,noray2023systemic}. Besides the gain to organisations, though, some argue that the social mobility brought about by the existence of scholarship programmes yields inherent benefits to society \cite{Dassin_Marsh_Mawer_2018}. Under any of these theories, selecting the ``best'' applicants as scholars is clearly in society's best interest. However, as we explore throughout this thesis, different theories of change yield different definitions of ``best''.

Traditional selection is a human-led process, and many suggest it should remain that way \cite{Latzer_Hollnbuchner_Just_Saurwein_2014}. However, as the number of applicants to scholarship programmes grows, the need for scalable selection processes has grown; with traditional selection processes unequipped to handle these new challenges, organisations are forced to innovate, often turning to algorithmic solutions that offer to solve these difficult problems \cite{Latzer_Hollnbuchner_Just_Saurwein_2014}. This has led to the development of Decision Support Tools (DSTs) to support selection processes. These DSTs range from simple tools like automated essay scoring to more complex tools like AI-driven selection algorithms. The use of these tools is not without controversy. Critics argue that these tools may be biased \cite{dwork_fairness_2012}, or may dehumanise the selection process \cite{binns_its_2018}. However, while certain features of these tools may succumb to some critiques, evolving discussions about fairness in selection render older, human-led selection processes equally vulnerable to critique \cite{Ahnaf2023AHPAP,pmlr-v80-kearns18a}.\footnote{More discussion on the value-laden aspects of selection can be found in Chapter \ref{ch:discussion}. In particular, we revisit fairness in Chapter \ref{ssec:fairness}; we also discuss the position of this research in structures of power in Chapter \ref{sec:reflexivity}.}

\section{Related Work}
This thesis engages primarily with literature seeking to use AI tools to support decision-making in global scholarship selection processes. Unfortunately (or perhaps fortunately), this particular niche of literature is fairly sparse. While many tools do exist, especially those seeking to automate the job of the scholarship selector entirely, we are forced to look beyond this niche for a larger body of related literature. Primarily, we do this by considering other selection contexts (e.g., universities or recruiters). Note that while this body of literature is large, and contains work ranging from automated essay scoring to intellect testing \cite{cozma_automated_2018,condon2014international}, our work is only tangentially related to these fields. Some work explores applicant perceptions of scholarship selection processes \cite{10.1145/3351095.3372867}, but this work is primarily interested in the decision subjects and does not engage with the design of selection processes.

More closely related is the body of algorithmic fairness literature engaging with recruitment, much of which seeks to ensure that AI tools do not discriminate against protected classes \cite{dwork_fairness_2012}. However, as we explore in this chapter, disanalogies between recruitment and global scholarship selection limit the applicability of this work.

Similarly, much work explores the impact of algorithms on educational outcomes. For example, \textcite{NISSENBAUM1998237} explore the risk of algorithmic involvement in education dehumanising the experience. However, though these works are conducted in the same environment, they do not touch on selection itself.

This chapter also explores a model of selection as subordinate decisions and investigations supporting the ultimate decision about which subset of applicants to select as scholars. Many of these subordinate decisions relate to their bodies of literature; these are explored in the relevant chapters.

\section{On Working with Scholarship Programmes}
Unsurprisingly, little research explores this unique and novel context. In part, this is due to the necessity of longitudinal research with or on organisations. The novelty of these programmes, combined with an understandable desire to avoid the scrutiny arising from public identification in research, has made research with or on these programmes challenging.\footnote{Critics may point to the Rhodes Scholarship, founded in 1902, as an example of a long-standing global scholarship organisation \cite{Ziegler_2008}. However, the programme's history of intentional exclusionism challenges its claim to global reach; the Rhodes Scholarship did not accept women until legally mandated in 1977 \cite{Ziegler_2008}, and continued to discriminate against Black South Africans until the legalisation of the African National Congress and the end of Apartheid in 1991 \cite{Ziegler_2008}. Furthermore, though the Rhodes Scholarship does select a range of applicants from around the world, they do so via regional committees that compare applicants from similar backgrounds, thus sidestepping the central difficulty of global selection.}

Fortunately, this thesis engages with two such organisations, Rise and Ellison Scholars. Both are already seeking to embed AI and computational DSTs into their processes (e.g., Ellison Scholars already uses an AI-based automated essay scoring system). Both programmes have asked that they not be identified in public-facing research, and thus we request that reviewers not share details on either programme and to protect the identities of the programmes where possible. For more details on the programmes, see Appendix \ref{app:programmes}.

\begin{table}[htbp]
  \centering
  \caption{This table enumerates relevant challenges facing selectors from the Rise and Ellison Scholars selection teams. Challenges are drawn from discussions with selectors, where descriptions are framed in terms of decisions these programs make.}
  \label{tab:full_decision_list}
  \adjustbox{max width=\textwidth}{
  \begin{tabular}{l r p{0.33\linewidth}p{0.33\linewidth}}
      \toprule
      Challenges & Chapter(s) & Description & Supporting Information \\
      \midrule
      \emph{Refinement} & \ref{ch:xai} and \ref{ch:spf} & A programme may refine its scoring algorithm each year to better score applicants. & Explanations of perplexing AI-generated scores; information about implications of scoring methods for cohort diversity \\ 
      \emph{Diligence} & \ref{ch:genai} & A programme may make holistic decisions about when and how to consider applicants. & Information about which essays (and which parts of essays) were written by genAI; information about whether the genAI-written passages are hallucinations. \\ 
      \emph{Partners} & \ref{ch:genai} & A programme may determine whether to continue channel partnerships, which encourage and support applicants. & Whether any channel partners' affiliated applicants use genAI disproportionately. \\
      \emph{Pipeline} & \ref{ch:genai} & A programme may decide whether to modify their application material or process. & Information about the usage of genAI throughout the application pipeline. \\
      \emph{Gameability} & \ref{ch:genai} & A programme may decide how to modify their application material or process. & Information about how AI-generated essays are scored under the current application process. \\
      \emph{Disqualification} & \ref{ch:genai} & A programme may decide to disqualify an applicant that violates their application guidelines. & Information about whether essays violate application guidelines around genAI usage. \\
      \emph{Diversity} & \ref{ch:diversity} and \ref{ch:spf} & A programme may make cohort-level decisions regarding the diversity of their cohort. & Information about the diversity of possible cohorts. \\
      \emph{Contribution} & \ref{ch:diversity} and \ref{ch:spf} & A programme may make decisions about which applicants to move forward based on their contribution to diversity. & Information about the impact of including different applicants on cohort diversity. \\
      \bottomrule
  \end{tabular}
  }
\end{table}

We engage these programmes, variously, in AR, VSD, and PD. In working with the Rise and Ellison Scholars programmes to support solutions to the central \emph{Selection} decision, the programmes expressed interest in supporting many subordinate decisions and challenges. While some of them, such as automating the essay scoring process, fall outside the scope of this thesis, we isolate three families of challenges that engage with AI or HCI literature and are thus of both programme and research interest. These families are challenges supported by explainable AI algorithms, challenges arising from applicant usage of genAI, and challenges relating to the diversity of selected cohorts. Table \ref{tab:full_decision_list} enumerates challenges of interest to us. 
\chapter{\label{ch:methods}Methodology}

\minitoc

\section{Paradigms}
\subsection{Why Paradigms?}
While much of the work in this thesis draws idiosyncratically from common methods in HCI, we are inspired by a number of ``paradigms'' that group families of methods into a more coherent whole. We discuss here three particularly influential paradigms. 

\subsection{Participatory Design}\label{ssec:participatory_design}
Participatory Design (PD) is a design paradigm in HCI emphasising active involvement of all stakeholders, particularly end-users, in the design process \cite{Hussain2014OverviewOV}. This approach recognises these users as best-positioned to speak to their subjective needs and preferences \cite{Hussain2014OverviewOV}, ensuring that designs meet user needs.

PD is a fundamentally collaborative process in which the designers and users work together to create solutions \cite{Tokranova2022ApplyingPD}. By involving users in each iteration of the development of a design, PD empowers participants to build tools that serve them \cite{Hussain2014OverviewOV}.

Historically, PD has placed special emphasis on the inclusion of a broad variety of user groups \cite{Brankaert2019IntersectionsIH}. For example, \textcite{10.1145/3544549.3573821,10.1145/3544548.3580933,Chowdhury2023ReflectionsOO} employ child-centric PD. \textcite{Brankaert2019IntersectionsIH} call for (and employ) PD aiming to serve users with dementia.

This focus on inclusion, especially a focus on the inclusion of oft-overlooked groups, is in itself a value-laden assumption of the PD process. That is, to employ PD is to assert that the voices of the selected participants are valuable, both in the design of the tool and in the broader context of research.

\subsection[Action Research]{\label{ssec:action_research}Action Research\footnote{This section reproduces text from Chapter \ref{ch:genai}.}}
Action Research (AR) is a research philosophy that emphasises ``research with, rather than on, people'' \cite{bradbury_action_2003}. Rather than one specific method, AR is best seen as a collection of related methods all embodying this ethos, usually to produce research contributions useful to the target group of people \cite{lu_organizing_2023}. Among these are semiotic inspection \cite{DeSouza_Leitão_2009,Alvarado_Waern_2018} and participatory design  (PD) \cite{braun_using_2006,Griffiths_Johnson_Hartley_2007,blythe2014research,Knapp_Zeratzky_Kowitz_2016}. AR is most often used in the context of social work, but can be applied across a variety of fields \cite{dombrowski_social_2016,lu_organizing_2023}. 

In education, AR is often used in a classroom setting \cite{Mertler_2019}. \textcite{venn-wycherley_realities_2024} argue that it is crucial in this setting to perform AR on both educators (teachers) and educatees (students), as failing to do so is liable to yield contributions useful to one group but not the other. While this holds for classroom settings, engagement across the stakeholder map is less feasible or desirable in scholarship selection. Unlike teacher and student, who share the common goal that the student learn, selector and applicant are at cross purposes: practitioners seek to choose the `best' cohort of applicants (although they often disagree on what constitutes `best'), while applicants seek to be included in the chosen cohort \cite{bergman2021seven}. Thus, when elucidating the interests and desires of one group, the other will merely act as noise. (E.g., applicants who use genAI to assist in writing their application will, of course, oppose using systems that monitor genAI usage to disqualify applicants.)

AR is comparatively new to HCI \cite{Hayes_2011,lu_organizing_2023}, but its methods and philosophies closely mirror longstanding pillars of HCI \cite{Hayes_2011}. Much like other HCI methods, AR seeks to democratise the research and design processes. However, AR extends beyond building solutions democratically, and sees learning through action as the ultimate research contribution \cite{Hayes_2011}. For example, AR sees all parties become: ``Co-investigators of, co-participants in, and co-subjects of...the project'' \cite{Hayes_2011}.  Thus, research questions are formulated by and with participants, actions and interventions are designed by and with participants, and results are found by and with participants \cite{Hayes_2011}.

\subsection{Value-Sensitive Design}\label{ssec:value_sensitive_design}
Value-Sensitive Design (VSD) is a theoretically grounded design philosophy that seeks to procedurally account for human values in design \cite{batyavalue}. It, like AR, employs a three-part structure: theoretical, empirical, and technical investigations. All investigations seek to understand and incorporate human values into the design process \cite{10.1145/242485.242493}. The theoretical investigation seeks to understand the values at play in the design space, the empirical investigation seeks to understand how these values are enacted in the world, and the technical investigation seeks to incorporate these values into the design \cite{10.1145/242485.242493}.

\subsection{The Relationship between and Significance of Paradigms}
While PD and AR share the notion that research participants should be closely involved in the research process, they differ in their ontology and the nature of their research outputs. PD is primarily concerned with the design process of products, systems, or interfaces; thus, research contributions are primarily those finalised designs \cite{zimmerman_research_2014}. AR, in contrast, is primarily concerned with the learning that occurs through action; contributions in AR are more often learnings that occur in the act of doing \cite{Hult1980TOWARDSAD}. They also differ in the kinds of research they inspire. While PD employs an iterative design process (and often involves evaluation), each stage in this iteration is a design stage. AR, in contrast, involves cycles of three distinct activities: planning (i.e., design), acting, and reflecting \cite{Hult1980TOWARDSAD}. Finally, and most importantly, they differ in the relative role of the user. In PD, the user \emph{participates} in the act of design; a distinction remains between researchers, who facilitate discussion and implement user requirements, and participants, who discuss and require \cite{Hussain2014OverviewOV}. In AR, this distinction is elided, rendering the participant a co-researcher; it is not uncommon for participants to appear as authors in AR processes \cite{Hayes_2011}. In PD, the user is involved in the design process, while in AR, the user is involved in the research process.

VSD, unlike both PD and AR, does not primarily centre a group of users. Rather, it is a principled approach seeking to orient itself around a particular set of values \cite{10.1145/242485.242493}. Unlike both AR and PD, VSD's tripartite structure only needs to involve the end user in one part, the empirical study \cite{10.1145/242485.242493}. This possibility of separation from the user allows the researcher to examine both the conceptual and the practical in a context removed from end-user desires. 

Our focus on the social aims of selection is ultimately inspired by VSD, as is our focus on diversity in Chapter \ref{ch:diversity}, but our research design draws more heavily from PD and AR. We follow a PD approach in Chapters \ref{ch:xai} and \ref{ch:diversity}, and engage in AR in Chapters \ref{ch:genai} and \ref{ch:spf}.

\section{Methods}
\subsection{Online Surveys}
The practice of running online surveys to gather quantitative data is well-established and often used both within and without HCI \cite{zhao2023fairness,pillai_adoption_2020,krishna_disagreement_2022,mai_user_nodate,bansal_does_2021,binns_its_2018,dzindolet_role_2003,papenmeier_its_2022}. Chapter \ref{ch:xai} makes use of one such survey. We use Prolific Academic to gather participants and Formr to administer our survey \cite{binns_its_2018,Arslan_formr_2019}. We follow \textcite{caldwell_power_nodate} in designing our survey based on a power analysis of the statistical tests we intend to run on the output data.

\subsection{Design Workshops}
Chapters \ref{ch:xai} and \ref{ch:diversity} both make use of group design workshops to refine and evaluate design prototypes. Both follow an experience-prototype methodology \cite{Buchenau_Suri_2000}, and incorporate a few specific methodologies.

Both chapters follow \textcite{Zimmerman_Forlizzi_2017}'s scenario speed dating approach, which sees participants rapidly applying different design prototypes to (real or hypothetical) scenarios.

\textcite{Gatian_1994} has researchers asking participants to choose a favourite among a series of options as a means of comparison, while \textcite{Griffiths_Johnson_Hartley_2007} brings this method to HCI. Chapter \ref{ch:diversity} makes use of this method.

\subsection{Individual Interviews}
Chapter \ref{ch:diversity} makes use of one-on-one interviews with participants to first elucidate participant understanding of diversity. In these interviews, we incorporate several methods.

\textcite{Knapp_Zeratzky_Kowitz_2016}'s `crazy 8s' exercise sees participants give eight feature requests in eight minutes. Ordinarily, this exercise is done with a writing surface, but we have participants do this verbally.

\textcite{blythe2014research} introduces the concept of design fiction, where participants more detail their ideal app. We adapt this to create a ``magic app'', capable of doing anything the participant desires and asking the participant to describe this app.

\subsection{Quantitative Analysis}
Chapters \ref{ch:xai}, \ref{ch:genai}, and \ref{ch:spf} rely on several standard statistical tests. Primarily, we use Student's t-test \cite{Mishra_Singh_Pandey_Mishra_Pandey_2019}, the Analysis of Variance (ANOVA) \cite{Mishra_Singh_Pandey_Mishra_Pandey_2019}, Pearson's test of correlation \cite{Schober_Boer_Schwarte_2018}, Tukey's Honestly Significant Difference test \cite{Kim_2015}, and the Receiver Operating Characteristic curve \cite{hanley1989receiver}. Additionally, we develop a permutation test in Chapter \ref{ch:spf} based on \textcite{good2013permutation}.

\subsection{Qualitative Analysis}
Chapters \ref{ch:xai} and \ref{ch:diversity} engage in inductive thematic analyses of their qualitative results. In doing so, we follow the methodology introduced by \textcite{braun_using_2006} and developed in \textcite{braun_conceptual_2022,braun_toward_2023,noauthor_thematic_nodate}.

\section{Research Design}
Chapters \ref{ch:xai}, \ref{ch:genai}, \ref{ch:diversity}, and \ref{ch:spf} all detail studies conducted according to different research paradigms and employing different methodologies. Each chapter contains a self-encapsulated section on research design. However, Table \ref{tab:method_subsections} provides a high-level overview of the methods and paradigms employed in each chapter.

\begin{table}[htbp]
    \centering
    \begin{tabular}{|l|c|c|c|c|}
    \hline
    & \textbf{Chapter \ref{ch:xai}} & \textbf{Chapter \ref{ch:genai}} & \textbf{Chapter \ref{ch:diversity}} & \textbf{Chapter \ref{ch:spf}} \\
    \hline
    \textit{Participatory Design} & Yes & & Yes & \\ 
    \textit{Action Research} & & Yes & & Yes \\ 
    \textit{Value-Sensitive Design} & & & Yes & \\ 
    \hline
    \textit{Online Surveys} & Yes & & & \\ 
    \textit{Design Workshops} & Yes & & Yes & \\ 
    \textit{Individual Interviews} & & & Yes & \\ 
    \textit{Quantitative Analysis} & Yes & Yes & & Yes \\ 
    \textit{Qualitative Analysis} & Yes & Yes & Yes & \\
    \hline
    \end{tabular}
    \caption{This table answers, for each method or paradigm and each chapter: does this methodology appear in this chapter?}
    \label{tab:method_subsections}
\end{table}

\chapter[XAI]{\label{ch:xai}XAI: Misleading In-Process, but Useful Post-Hoc\footnote{This chapter is based on a paper written in concert with Reuben Binns, Ulrik Lyngs, and Nigel Shadbolt. The paper is currently under review as: Neil Natarajan, Reuben Binns, Ulrik Lyngs, and Nigel Shadbolt. 2024. “XAI: Misleading In Process, but Useful Post Hoc.” Under review at CHI 2025.}}

\minitoc

\section{Motivation}
In exploring the array of decision support tools applicable to global scholarship selection, explainable AI (xAI) offers a natural starting point. XAI tools are often offered as a fair and responsible way to support a decision-subject's right to explanation, empowering them while improving decision-making in potentially sensitive fields \cite{Goodman_Flaxman_2017}. A wealth of research explores and evaluates xAI in these contexts \cite{molnar_interpretable_2019,barocas_hidden_2020,wachter_counterfactual_2017,Barocas_Hood_Ziewitz_2013,raghavan2020mitigating}. Despite this, much of this research cautions against blind applications of these tools to decision-making processes \cite{Lipton,miller_explainable_2023,kumar_problems_2020,Bastounis_Campodonico_vanderSchaar_Adcock_Hansen_2024}. In this chapter, we explore the potential for post-hoc notions of interpretability in supporting selection-related decision-making processes, and consider the potential for these tools to be applied in different supporting contexts.

\section{Introduction}
Despite the promise of xAI in enabling fair and explainable decision-making \cite{Goodman_Flaxman_2017}, post-hoc xAI systems are oft-maligned as poor tools for decision support. \textcite{Lipton} cautions that well-intentioned explanation design may yield ``misleading but plausible'' explanations, and \textcite{miller_explainable_2023} notes that these post-hoc xAI methods justify the underlying AI models and their outputs rather than allowing users to make their own informed decisions \cite{miller_explainable_2023}. This has, in recent years led to a shift away from post-hoc xAI entirely \cite{Lipton,miller_explainable_2023,kumar_problems_2020,Bastounis_Campodonico_vanderSchaar_Adcock_Hansen_2024}. The core of both critiques is that explanations may induce misplaced explainee trust in model outputs. Indeed, there is evidence that such xAI systems do induce trust in the underlying model \cite{lai_human_2019,jacobs_how_2021}. In response, research using xAI as a decision support tool (DST) often eschews post-hoc, model-agnostic explanations in favour of new paradigms such as \textcite{miller_explainable_2023}'s evaluative AI for decision-makers and \textcite{karimi_algorithmic_2021}'s causal models for decision subjects.

But have we been too hasty in rejecting these older methods? Underlying this shift away from xAI is the assumption that approaches will be deployed to increase trust in particular outputs in decision support contexts, even though such trust may be unwarranted when outputs are wrong. But is this always the case? Might these trust-inducing xAI be usefully deployed elsewhere, such as for post-decision evaluation of models and decision-making processes?

We utilise a decision stage distinction from Chapter \ref{ch:context} to help consider the benefits and risks of post-hoc interpretability tools as DSTs. We distinguish between: the in-process stage, where AI outputs and post-hoc explanations are used to support human decision-makers in confirming or overriding the `primary' decision the model output advises on, and the ex-post stage, where the primary decision has already been made, and the xAI is offered to inform second-order decisions about the decision-making process. (E.g., between application cycles, recruitment and selection practitioners examine their prior decision-making procedures and seek to make decisions that improve them for the next cycle \cite{li2020hiring}.) We seek to answer two research questions (RQs):

\begin{enumerate}
    \item[(RQ1)] Do post-hoc explanations, when used as in-process DSTs, induce unwarranted trust in the explainee?
    \item[(RQ2)] If post-hoc xAI methods induce unwarranted trust in-process, could they still be useful ex-post?
\end{enumerate}

At the in-process stage, we run an online study to discern whether the problem of unwarranted trust is specific to xAI. We investigate \textcite{lundberg_unified_2017,ribeiro_anchors_2018}'s popular methods, SHapley-based Additive exPlanations (SHAP) \cite{lundberg_unified_2017} and Scoped Rules (Anchor) \cite{ribeiro_anchors_2018}, respectively, to see if they induce unwarranted trust. We also investigate a `Confidence' explanation consisting of the model's confidence statistic to determine if the problem of unwarranted trust is unique to explanations per se, or applies more generally to the presentation of any information that could increase positive perceptions of the AI's performance. We ask participants to \emph{estimate a person's salary} \cite{kohavi_scaling_1996} or \emph{predict whether someone will be severely delinquent in making a credit payment} \cite{GiveMeSomeCredit} with the help of an AI output and with or without an explanation. We find that SHAP explanations do increase unwarranted trust in AI outputs, but that Confidence explanations do as well. We find no such effect for Anchor. This suggests the problem of unwarranted trust is not unique to xAI per se and is rather a symptom of generic post-hoc justifications.

Having identified a core problem with some kinds of in-process xAI, we consider whether they have any potentially redeeming features if deployed at the ex-post level through a series of participatory design workshops. We refine our attention to SHAP, as critiques by \textcite{Lipton} and \textcite{miller_explainable_2023} are most germane. We contend that, in an ex-post context, this induction of unwarranted trust is less problematic, as primary decisions have already been made and thus trust in the AI outputs is not at issue. We ask participants to \emph{refine a scholarship selection algorithm} with the help of SHAP-based explanations. Through these workshops, we find that, while SHAP explanations may induce unwarranted trust in specific model outputs, they can still be useful to drive process change in organisations.

Our primary contributions are:

\begin{enumerate}
    \item Quantitative findings indicating that the problem of explanation-induced unwarranted trust extends to generic post-hoc justifications, but that such criticism only applies in-process.
    \item Qualitative findings that post-hoc explanations, properly presented, can make useful ex-post DSTs.
\end{enumerate}

\section{Background}
\subsection{A Brief History of Explainable AI}\label{ssec:history}
\textcite{ribeiro_why_2016}'s Local Interpretable Model-agnostic Explanations (LIME), which identifies important features (creating a `feature-based' explanation) grew popular as an offering for explanations of Computer Vision (CV) models. \textcite{ribeiro_why_2016} demonstrated the explanation method on a task classifying huskies and wolves. By highlighting that the model used snow to recognise huskies, LIME would help a user spot what portions of an image were likely to be causing the model's classifications. By highlighting that the model used snow to recognise huskies, rather than e.g. their coat pattern, the user might come to understand the model's abilities and limitations and calibrate their trust accordingly.

The explanation paradigm established by LIME was shortly thereafter applied to non-CV tasks. Tasks based on tabular data, in particular, were often explained using LIME \cite{zerilli_explaining_2020}. Subsequent xAI methods have revised the feature-centric paradigm established by LIME to better fit this new data type. \textcite{lundberg_unified_2017}'s SHAP offers another way of calculating the influence of each feature wherein the sum of the influences of each feature (plus a `bias' term) equals the model's output, and has since grown ubiquitous in the xAI field \cite{weerts_human-grounded_2019}.

But while the feature-based methodology underlying LIME and SHAP is useful for identifying error in the computer vision case, this relies on the human evaluator comparing explanation outputs to their (at least partial) knowledge of the ground truth (i.e., when the LIME explanation highlights a portion of the image, the human evaluator must rely on their knowledge of the ground truth to identify whether what is highlighted is relevant). While this may make good sense in cases such as the task classifying huskies and wolves, more contentious applications involve less obvious decisions where, even with access to feature space, a human reviewer might not be able to deduce the correct ground truth classification \cite{kumar_problems_2020,markus_role_2021}.\footnote{This is just one of many problems with feature importance explanations. \textcite{miller_explanation_2017} outlines a list of desiderata that model interpretability methods should satisfy. He points to a need for contrastive, counterfactual, selective, and social explanations; the feature importance statistics yielded by LIME and SHAP are none of these.} But if these explanations are provided in cases where they do not help users identify cases where the AI output is incorrect, they can only serve to increase trust in the outputs, and cannot serve to decrease said trust where it is not warranted; explanations of AI outputs should only be reassuring if they might have not been.

`Example-based' explanations respond to many of these critiques. \textcite{wachter_counterfactual_2017}'s Counterfactual Explanations (CE) presents an example counterfactual to the input, showing the user a similar feature value with a different model output. \textcite{mothilal_explaining_2019}'s DIverse Contrastive Explanations (DICE) operate similarly, presenting multiple intentionally diverse counterfactuals to give the explainee a better sense of the many ways that the feature space could be perturbed. These counterfactuals serve to give explainees a glimpse into local model behaviour – if the model behaves strangely in any of these similar examples, it is likely untrustworthy in the original instance. However, these explanations are not without their critiques. In particular, counterfactual explanations are subject to a \emph{Rashomon effect}; i.e., for any explainable point, multiple counterfactuals may be equally valid. \textcite{miller_explanation_2017} argues that xAI methods should be selective in what they show explainees to not overwhelm them, but if there are many valid counterfactuals, we must then choose between showing an explainee all counterfactuals and violating selectivity, or selecting only a few counterfactuals and biasing the explanations.

\textcite{ribeiro_anchors_2018}'s Anchor retains the contrastive benefits of CE and DiCE without suffering the same Rashomon paradox. The algorithm yields a set of conditional statements that help situate the model's output in a region surrounding the actual point. Similarly, \textcite{ustun_actionable_2019}'s Actionable Recourse (Recourse) offers decision subjects rules guiding what they must change to receive a different determination. These explanations meet \textcite{miller_explanation_2017}'s selectivity maxim without ad-hoc simplifications.

\subsection{A Taxonomy of Explanation Algorithms}
In much of the xAI literature, descriptors such as ``post-hoc'' or ``model-specific'' are used to describe types of explainability. We provide a taxonomy of them here and in Figure \ref{fig:taxonomy}.

\begin{figure}[htbp]
    \centering
    \includegraphics[width=.9\textwidth]{xai/taxonomy.png}
    \caption{Explainable AI methods are separated into types by this taxonomy. We focus primarily on kinds of post-hoc (model-agnostic) explanations and highlight several explanations following different philosophies (SHAP and Anchor are of particular interest, as both recur in our analyses), but we also consider model confidence statistics, which follow none of these philosophies.}
    \label{fig:taxonomy}
\end{figure}

We begin by broadly categorising the space of explainable AI into two subgroups: intrinsically interpretable models and post-hoc explanations \cite{molnar_interpretable_2019}. Though intrinsically interpretable models offer solutions for interpretability and often present their own explanations, we (and the xAI critiques we engage with) focus primarily on post-hoc explanations \cite{molnar_interpretable_2019}. Crucially, while intrinsically interpretable models are inherently model-specific, post-hoc methods may reap the benefits of model agnosticism; thus, a single method may be used on any number of underlying models. \textcite{Lipton}'s argument about misleading explanations restricts itself to post-hoc models, we contend, because of this frequent ignorance about the model's internal processes.

Alternatively, explanations can be separated into global methods, which explain entire models, and local methods, which explain individual decisions \cite{molnar_interpretable_2019}. Though similar, the distinction between local and global differs from our distinction between decision stages, as our distinction does not identify the scope of the explanation itself, but rather identifies the scope of the decision. Local explanations, of course, lend themselves best to in-process decisions, but one could supply a global explanation here instead. And ex-post decisions are varyingly informed best by local or global explanations. We focus on local explanations.

Finally, explainability methods could be instead subdivided by the design philosophy of their outputs \cite{friedrich_taxonomy_2011}. Some methods are `Feature-Based' in that they offer feature-importance statistics. Other `Example-Based' explanations give examples. Yet other `Rule-Based' explanations offer rules that guide decision-making. Many explanations follow none of these paradigms. Model confidence statistics, for example, do not subscribe to any such design philosophies, but are often used as explanations in place of explainable AI systems \cite{zhang_effect_2020}. All explanations we mention, as well as where they fit on this taxonomy, are detailed in Figure \ref{fig:taxonomy}.

\subsection{Ethical Considerations of Employing xAI Systems as DSTs}
Researchers raise many concerns about using AI systems in general as DSTs. The most relevant bodies of concern for us come from the literature on scholarship selection and recruitment. One major concern is that AI algorithms amplify human biases \cite{MikePerkins_JasperRoe_2023}. Though this concern does and should limit applications of certain AI systems, it must be weighed against pre-existing human bias in pipelines. That is, a system might be less biased than what it replaces. Furthermore, a system might even be used to help holistic reviewers identify and mitigate their own biases \cite{alvero_ai_2020}.

More broadly, \textcite{alvero_ai_2020} note that, while AI researchers tend to think of fairness on a population level, recruiters (and, we contend, practitioners in related fields) think of fairness on an individual level. I.e., when making decisions based on qualitative information about a candidate in a process known as holistic review, reviewers are thinking of being fair to the candidate in front of them. But new AI tools blur the quantitative and the qualitative: what was once a simple score can now be a full explanation, and what was once a human-gradable essay could be automatically given a score. There is, we contend, space in these processes for tools that can help reviewers understand the qualitative and quantitative contexts surrounding the decisions they make, and thus make more just decisions.

\subsection{Evaluating Unwarranted Trust in a Model}
To evaluate whether an explanation induces unwarranted trust, we must evaluate both trust and warrantedness. We begin with trust. \textcite{jacovi_formalizing_2021} define trust in an AI system as characterised by two properties: ``the vulnerability of the user, and the ability to anticipate the impact of the AI model's decisions''; \textcite{vereschak_how_2021} similarly isolate three elements: ``trust is linked to a situation of vulnerability and positive expectations, and is an attitude''; \textcite{lee_trust_2004} give a similar definition of trust: ``An attitude that an agent will achieve an individual's goal in a situation characterized by uncertainty and vulnerability''. In all definitions, we see \emph{vulnerability} emerge as a key concept, and we variably also see that trust is characterised by \emph{uncertainty} and \emph{expectations}. Finally, we see that the definitions used concerning evaluations of AI systems tend to be attitudinal. However, though \textcite{vereschak_how_2021} suggest that trust is an unobservable variable, they term behaviours emergent from trust, rather than the attitude itself, `reliance'. 

The form of trust detailed by \textcite{vereschak_how_2021} is elsewhere called `attitudinal' trust \cite{crites_measuring_1994}. This form of trust is the one most often considered in human-centred AI, and much of the research discussing explainable AI focuses on this form of trust \cite{vereschak_how_2021, ford_play_2020, bansal_does_2021, yin_understanding_2019}. However, less frequently documented is a second, `behavioural' form of trust \cite{crites_measuring_1994}. When \textcite{jacovi_formalizing_2021,lee_trust_2004} argue that trust is measurable through such behaviours as reliance, we contend that they are speaking not of attitudinal trust, but rather of behavioural trust. Unlike attitudinal trust, behavioural trust does not rely on the participants' understanding of the term matching academic definitions or their estimates of their trust accurately capturing what we seek to measure \cite{jacovi_formalizing_2021}. This more closely aligns with how explanations impact the actual decisions made by humans-in-the-loop in practice.

Though attitudinal and behavioural trust are conceptually similar, research is mixed on the existence and strength of correlation between the two constructs \cite{ahmed_relationship_2009, kim_relation_2018}. As we wish to take a full account of trust in AI systems, we must consider both forms of trust.

We now turn to the warrantedness of said trust. \textcite{Vereschak_Alizadeh_Bailly_Caramiaux_2024} establish a clear distinction between trust and trustworthiness and observe this distinction in decision-makers and decision subjects alike. That is to say, all parties involved agree that it is at least possible to place unwarranted trust in an algorithm \cite{Vereschak_Alizadeh_Bailly_Caramiaux_2024}. However, when it comes to the question of whether a given model output is trustworthy, opinions are more mixed. Some argue that certain uses of AI systems are inherently untrustworthy, but for many others, the trustworthiness of these systems is merely a matter of their accuracy \cite{Rebitschek_Gigerenzer_Wagner_2021}. \textcite{Rebitschek_Gigerenzer_Wagner_2021} find that, in the aforementioned case of credit estimation, potential decision subjects tend to simply place higher accuracy requirements on their willingness to trust an AI system. \textcite{jacovi_formalizing_2021} similarly conclude that a model's trustworthiness is related to properties such as accuracy, robustness, and bias. However, unlike accuracy, a model may be trustworthy in some contexts, and not in others. I.e., a model may be sufficiently accurate to be trustworthy when used to give rough estimates, but insufficiently accurate for use in high-stakes decisions. In the case of specific instances of model outputs, though, the question of trustworthiness is more straightforward: when a system outputs a correct (or an incorrect) result, that result is trustworthy (or untrustworthy).

\subsection{A Taxonomy of Critiques}
There is a large body of studies that empirically evaluate xAI methods. In some cases, an xAI method is evaluated not with human subject experiments, but rather with analysis of mathematical properties \cite{doshi-velez_towards_2017}. These critiques range from \textcite{kumar_problems_2020}'s argument that SHAP explanations lack desired properties like contrastiveness to \textcite{lundberg_unified_2017}'s argument that LIME's mathematical properties make it unsuited to tabular data.

However, while these evaluations are instructive in that they provide interesting new perspectives on technologies, they do not help us evaluate whether utilising these explanations as DSTs is problematic in practice. To determine this, we turn to evaluations that involve humans using an AI system to perform some task \cite{ribeiro_why_2016,ribeiro_anchors_2018, rader_explanations_2018, jacobs_how_2021, bansal_does_2021}. These critiques, either explicitly or implicitly, tend to evaluate the impact of explanations on (attitudinal or behavioural) trust in AI systems. We enumerate some in Table \ref{tab:studies}.

\begin{table}
    \centering
    \caption{This table documents human-centric evaluations of the effects of different explanation types on trust and warrentedness of trust. Many studies find evidence that all Feature-Based explanations increase trust in AI outputs, while results are mixed for other explanation types, but warrantedness of trust is rarely explored.}
    \label{tab:studies}
    \begin{tabular}{p{0.15\textwidth} p{0.15\textwidth} p{0.15\textwidth} p{0.15\textwidth} p{0.15\textwidth}}
        \toprule
        Study & Domain & Explanation Philosophy & Explanation Effect on Trust & Incorrect Explanation Effect on Warrantedness \\
        \midrule
        \textcite{lai_human_2019} & Deception Detection & Feature-Based, Example-Based, Model Confidence & Conditional increase & No observed effect \\
        \textcite{binns_its_2018} & Multiple Domains & Feature-Based, Rule-Based (Recourse), Example-Based & Conditional decrease & Not investigated \\
        \textcite{ford_play_2020} & Image Recognition & Example-Based & Explanation presence or style has no main effect on end-user trust & Not investigated \\
        \textcite{jacobs_how_2021} & Clinical Treatment & Feature-Based, Others & Increase & Conditional increase. \\
        \textcite{bansal_does_2021} & Sentiment Classification & Feature-Based (LIME) & No observed effect & Not investigated \\
        \textcite{mohseni_trust_nodate} & Fake News Detection & Feature-Based and Others & Increase and decrease & Not investigated \\
        \textcite{Spitzer_Holstein_Morrison_Holstein_Satzger_Kühl} & Architecture Style Classification & Free Text (LLM) & Increase & Increase \\
        \bottomrule
    \end{tabular}
\end{table}

Despite \textcite{Lipton}'s ``misleading but plausible'' critique, the verdict of these human-centric evaluations is mixed. \textcite{lai_human_2019,jacobs_how_2021} both find that their Feature- and Example-Based explanations increase explainee trust in AI outputs, but \textcite{binns_its_2018} find that Example-Based explanations decrease explainee trust in AI outputs. The answer, we must conclude, depends on the explanation, the target group, or even the domain \cite{mohseni_trust_nodate}. 

Among those studies that do find an explanation-induced trust, few consider whether this trust is warranted. Among those that do, evidence is mixed; while some find unwarranted trust in some cases, others find such no effect \cite{lai_human_2019,jacobs_how_2021}. Thus, we must ask: are we too hasty in moving away from these explanations? 

\section[Experimental Study]{Experimental Study: In-Process Decision Support}\label{sec:online}
\subsection{Research Questions}
Our online study seeks to answer RQ1:

\begin{enumerate}
    \item[(RQ1)] Does post-hoc xAI used as an in-process DST induce unwarranted trust in the explainee?
\end{enumerate}

To do this, we compare three alternate conditions: \textcite{lundberg_unified_2017}'s SHAP explanations, \textcite{ribeiro_anchors_2018}'s Anchor-based explanation, which obeys contrastive, selective, and counterfactual paradigms for explanations, and a Confidence condition consisting of the model's intrinsic confidence measurement. We measure trust in the AI system in two ways: attitudinal trust, measured by self-report, and behavioural trust, measured by the participant's decision to follow the AI system's recommendation. We also measure the change in trust from before to after the explanation and compare this change across the three conditions. These measurements are done across two tasks. Each participant sees six cases, with the explanatory and task conditions held constant. Finally, we calculate the correlation between attitudinal and behavioural trust, and between the change in attitudinal and behavioural trust.

\subsection{Methodology}
\subsubsection{Participants}
Participants were recruited via Prolific Academic's standard sampling method restricted to the United States.\footnote{\url{www.prolific.co}} They were paid at a rate of \$15 per hour. Participants were first shown an information sheet detailing the study's methodology and what was being asked of them. They were then asked to give informed consent. After consenting to participate in the study, participants were routed to Formr, our chosen survey design and hosting platform, to complete the online study.\footnote{\url{www.formr.org}} All data collected was anonymous and was stored on secure servers. Ethics review was performed by the University of Oxford's Central University Research Ethics Committee.

\subsubsection{Tasks}
We restrict our research question to tasks familiar to lay people with a well-defined but difficult-to-ascertain ground truth. Our two tasks are chosen from a gamut of well-known algorithmic decision-making tasks as two particularly related to \emph{Selection}, articulated in Chapter \ref{ch:context} \cite{10.1111/j.1467-954X.2007.00740.x,Pasquale_2006,Latzer_Hollnbuchner_Just_Saurwein_2014}: \emph{estimating a hypothetical person's salary} based on census information of that individual, and \emph{predicting whether someone will be severely delinquent in making a credit payment}. We use two datasets: the Adult dataset collected from the 1994 US Census for the former task and the Give Me Some Credit dataset for the latter task \cite{kohavi_scaling_1996, GiveMeSomeCredit}. In both tasks, the participant aims to accurately estimate the dependent variable with the help of the AI system and one of several possible explanations of the AI system's estimate (which is possibly just the confidence rating of the model). In our analyses, we index these tasks as \emph{Salary} and \emph{Credit}. Note that each participant only receives one task to complete throughout all 6 cases. 

\begin{figure}[htbp]
    \centering
    \begin{subfigure}[b]{0.45\textwidth}
        \includegraphics[width=\textwidth]{xai/survey-shap.png}
        \caption{SHAP explanations for \emph{Salary}}
        \label{fig:shapsalary}
    \end{subfigure}
    \hfill
    \begin{subfigure}[b]{0.45\textwidth}
        \includegraphics[width=\textwidth]{xai/survey-shap-2.png}
        \caption{SHAP explanations for \emph{Credit}}
        \label{fig:shapcredit}
    \end{subfigure}
    \medskip
    \begin{subfigure}[b]{0.45\textwidth}
        \includegraphics[width=\textwidth]{xai/survey-anchor.png}
        \caption{Anchor explanations for \emph{Salary}}
        \label{fig:anchorsalary}
    \end{subfigure}
    \hfill
    \begin{subfigure}[b]{0.45\textwidth}
        \includegraphics[width=\textwidth]{xai/survey-anchor-2.png}
        \caption{Anchor explanations for $credit$}
        \label{fig:anchorcredit}
    \end{subfigure}
    \medskip
    \begin{subfigure}[b]{0.45\textwidth}
        \includegraphics[width=\textwidth]{xai/survey-confidence.png}
        \caption{Confidence explanations for \emph{Salary}}
        \label{fig:confidencesalary}
    \end{subfigure}
    \hfill
    \begin{subfigure}[b]{0.45\textwidth}
        \includegraphics[width=\textwidth]{xai/survey-confidence-2.png}
        \caption{Confidence explanations for \emph{Credit}}
        \label{fig:confidencecredit}
    \end{subfigure}
    \caption{This figure shows sample explanations for all cases. Larger images and more detailed descriptions of explanations can be seen in Appendix \ref{app:xaifigures}.}
    \label{fig:online_explanations}
\end{figure}

\subsubsection{Models}
In both tasks, we construct a predictor model using random forests and augment this predictor with three different explanatory conditions. Our random forest classifier achieves $86\%$ test accuracy on the Adult dataset and $93\%$ test accuracy on the Give Me Some Credit dataset. We use a SHAP explainer to produce one of our explanatory conditions, and an Anchor explainer to produce another; our final explanatory condition is an intrinsic explanation produced by the random forest model. Figure \ref{fig:online_explanations} shows sample explanations produced by these methods. These ultimately form three explanatory conditions: SHAP, Anchor, and Confidence. Note that each participant only receives one model of explanation throughout all 6 cases. 

\subsubsection{Design}
Both tasks rely on the same 3-between-by-2-within design using repeated measures to capture the same data before and after the presentation of each explanation. The between-subjects factor determines which model is used to generate the explanation a given participant will receive. The within-subjects factor is the repeated-measures `explanation presence' factor. This is either `before explanation' or `after explanation', indexed $before$ or $after$. A flowchart of the study design can be found in Figure \ref{fig:online_flowchart}.

\begin{figure*}[htbp]
    \centering
    \includegraphics[width=\textwidth]{xai/online_flowchart.png}
    \caption{Participants in the online study are sorted into six buckets, where each bucket is segregated by explanatory condition and task and shown a brief description of the task (i.e., each participant sees only one of the explanations in Figure \ref{fig:online_explanations}). Then, each participant is shown 6 cases. In each case, participants are shown an applicant profile and an AI output. Participants are asked to agree or disagree with the AI output. Then, participants are given explanations based on their explanatory condition scores. They are then asked again to agree or disagree with the AI output.}
    \label{fig:online_flowchart}
\end{figure*}

\subsubsection{Questions and Variables}\label{sssec:q_and_v}
Each participant was shown a brief explanation of the task in question and was then asked to complete the 6 cases, with participants given a random mix of correct and incorrect cases. In each case, participants are first shown a table identifying the subject of the case and an AI output of what determination they should make. They are then asked to estimate the dependent variable and rate both their confidence in the estimate and their trust in the AI output on sliding scales (this is discretised to 20 points).

We code the participant's estimate as a binary $y_{human}$ variable:

\begin{equation}
    y_{human} := \begin{cases}
        \text{How much money does this person make?} & (Salary) \\
        \text{Will this person experience severe credit delinquency?} & (Credit)
    \end{cases}
\end{equation}

\noindent The two sliding scale responses are coded as $selfconfidence$ and $trust_{attitudinal}$ and have values between 1 and 20. These are defined as:

\begin{equation}
    confidence := \begin{cases}
        \text{How confident are you in your estimation?} & (Salary) \\
        \text{How confident are you in your prediction?} & (Credit)
    \end{cases}
\end{equation}

\begin{equation}
    trust_{attitudinal} := \begin{cases}
        \text{How much do you trust the AI's estimation?} & (Salary) \\
        \text{How much do you trust the AI's prediction?} & (Credit)
    \end{cases}
\end{equation}

As we ask all questions in both the $before$ and $after$ conditions, we collect six responses from each participant in each case: $y_{human}^{before}$, $selfconfidence^{before}$, $trust_{attitudinal}^{before}$, $y_{human}^{after}$, $selfconfidence^{before}$, and $trust_{attitudinal}^{after}$.  We additionally have the binary variables $y_{True}$ and $y_{AI}$ that are the true value and the AI output of the dependent variable.

In addition to these, we define $agreement^{x}$ as:

\begin{equation}
    agreement^{x} := \begin{cases}
        1 & \text{if } y_{human}^{x} = y_{AI} \\
        0 & \text{otherwise}
    \end{cases}
\end{equation}

\noindent and $correct_{x}$ as:

\begin{equation}
    correct_{x} := y_{x} = y_{True}
\end{equation}

We define $trust_{behavioural}^{before}$ and $trust_{behavioural}^{after}$ to be the extent to which the participant's confidence agrees with the AI output:

\begin{equation}
    trust_{behavioural}^{x} := \begin{cases}
        confidence^{x}      & \text{if } agreement^{x} \\
        1-confidence^{x}    & \text{otherwise}
    \end{cases}
\end{equation}

Finally, to reason about the change in a variable due to the explanation, we define `$\Delta$' constructs for all variables with a $before$ and an $after$ as:

\begin{equation}
    \Delta variable := variable^{after} - variable^{before}
\end{equation}

\noindent so, e.g.:

\begin{equation}
    \Delta trust_{attitudinal} := trust_{attitudinal}^{after} - trust_{attitudinal}^{before}
\end{equation}

\subsubsection{Data Analysis}
We preregistered many of our analyses. Though we also include some post-hoc analysis below, we wish to delineate between the two types of analyses. The former are listed in full here.

We first wish to test for the presence of unwarranted trust. To do this, we measure the difference between ratings in each condition in the cases where the AI output is incorrect, i.e. where $\neg correct_{AI}$. To do this, we run three one-sided t-tests \cite{caldwell_power_nodate} to determine:

\begin{equation}
    \Delta trust \geq 0 | \neg correct_{AI}
\end{equation}

\noindent for both $\Delta trust_{behavioural}$ and $\Delta trust_{attitudinal}$. Note that this is identical to repeated-measures t-tests on the $trust_{behavioural}$ and $trust_{attitudinal}$ variables where $\neg correct_{AI}$. Following this, as this is also a between-subjects experiment, we wish to compare the varying effects of different explanation methods in this case. Thus, we run two between-subjects ANOVAs \cite{caldwell_power_nodate} on $\Delta trust_{behavioural}$ and $\Delta trust_{attitudinal}$ across the explanatory conditions again filtered on $\neg correct_{AI}$. When these ANOVAs have significant results ($p < 0.05$), we run Tukey's Honestly Significant Difference (HSD) test \cite{caldwell_power_nodate}.

Finally, in our Salary Estimation survey, we observed a strong positive correlation between our two trust variables, though we did not preregister it for that study. Indeed, correlation analysis is a well-documented method for confirming that two measurements indeed measure the same concept \cite{westen_quantifying_2003, morata-ramirez_construct_2013}. Thus, we additionally included the calculation of Pearson's correlation between $trust_{behavioural}$ and $trust_{attitudinal}$ and between $\Delta trust_{attitudinal}$ and $\Delta trust_{behavioural}$ in our preregistration for the Credit Delinquency Prediction survey. 

\paragraph{Power Analysis}
We run power analyses using \textcite{caldwell_power_nodate}'s Superpower. Specifically, we desire sufficiently powerful results in our primary analysis. We select a moderate effect size of interest (Cohen’s $f$) of $0.15$ (yielding group means $-0.15$ and $0.15$ with unit variance), and target a power of at least $0.90$. We note that using a one-way t-test at $p = 0.05$ and assuming a sample size of $200$, we get power far above $0.90$. We also test for the ANOVA assuming unit variance and group means of $-0.15$, $0.15$, and $0.15$, respectively. Under these conditions, we achieve a power of $0.90$ with $200$ samples per condition. We ask each participant a total of 6 questions, and the AI output is incorrect in slightly less than half of them. To achieve $200$ samples per condition, therefore, we aim to recruit a total of roughly $66$ participants per condition, or roughly $200$ participants in total.

\paragraph{Preregistration}
We have preregistered analyses for both of our tasks in the OSF registries \cite{natarajan_binns_2022}. 

\subsection{Results}\label{ssec:os_results}
In both tasks, though we originally set $200$ as our target participants, some participants did not complete our task following Prolific Academic's guidelines. Data from these participants were marked incomplete and removed from consideration. 

After this removal, we had a total of $192$ participants complete the Salary Estimation study. These were split randomly into our three explanatory groups. By gender, $115$ were Male, $76$ were Female, and $1$ did not provide gender information. By ethnicity, $137$ were white, $10$ did not provide ethnicity, and the remaining $45$ were split among non-white ethnicities. Our participants were an average of $36.7$ years old, with the youngest being $18$ and the oldest $74$. Each applicant completed an introductory page and six cases. The average completion time for these tasks was $7$ minutes $43$ seconds, the minimum was $2$ minutes $25$, and the maximum was $36$ minutes $46$.

We had a total of $197$ participants complete the Credit Delinquency Prediction study. These were similarly split into groups. By gender, $106$ were Male, $90$ were Female, and $1$ did not provide gender information. By ethnicity, $143$ were white, $11$ did not provide ethnicity, and the remaining $43$ were split among non-white ethnicities. Our participants were an average of $38.4$ years old, with the youngest being $20$ and the oldest $77$. Each applicant completed an introductory page and six cases. The average completion time for these tasks was $7$ minutes $53$ seconds, the minimum was $2$ minutes $17$, and the maximum was $30$ minutes $13$.

\paragraph{SHAP and Confidence Increase Unwarranted Trust}
We first run the one-sided t-tests on the two trust variables ($attitudinal$ and $behavioural$). I.e., we test:

\begin{equation}
    \Delta trust_{x} > 0 | \neg correct_{AI} \text{ for } x \in \{behavioural, attitudinal\}
\end{equation}

\noindent for both \emph{Salary} and \emph{Credit} across SHAP, Anchor, and Confidence. A positive $F$ statistic here indicates $trust^{after} > trust^{before}$ and a negative $F$ statistic indicates $trust^{after} < trust^{before}$, but, as these are one-sided tests, $p$-values will only be meaningful when $F > 0$. This test was preregistered in both of our tasks \cite{natarajan_binns_2022}. Table \ref{tab:delta-trust-t} contains the results of these analyses.  

\begin{table}[htb]
    \centering
    \caption{These one-sided t-tests test for $\Delta trust > 0 | \neg correct_{AI}$ for all explanatory conditions and both tasks. We find that SHAP and Confidence increase unwarranted trust in the AI system.}
    \label{tab:delta-trust-t}
    \begin{tabular}{l l l r r}
        \toprule
        Task & Explanation & Variable & Test Statistic & p Value \\ 
        \midrule
        \emph{Salary} & Anchor & $\Delta trust_{behavioural}$ & $0.509$ & $0.306$ \\
        & & $\Delta trust_{attitudinal}$ & $0.165$ & $0.434$ \\
        & SHAP & $\Delta trust_{behavioural}$ & $\mathbf{3.811}$ & $\mathbf{<0.001}$ \\
        & & $\Delta trust_{attitudinal}$ & $-0.886$ & $0.812$ \\
        & Confidence & $\Delta trust_{behavioural}$ & $\mathbf{2.196}$ & $\mathbf{0.015}$ \\
        & & $\Delta trust_{attitudinal}$ & $0.945$ & $0.173$ \\
        \midrule
        \emph{Credit} & Anchor & $\Delta trust_{behavioural}$ & $1.396$ & $0.082$ \\
        & & $\Delta trust_{attitudinal}$ & $-2.364$ & $0.990$ \\
        & SHAP & $\Delta trust_{behavioural}$ & $1.516$ & $0.066$ \\
        & & $\Delta trust_{attitudinal}$ & $\mathbf{2.475}$ & $\mathbf{0.007}$ \\
        & Confidence & $\Delta trust_{behavioural}$ & $\mathbf{1.835}$ & $\mathbf{0.034}$ \\
        & & $\Delta trust_{attitudinal}$ & $0.940$ & $0.174$ \\
        \bottomrule
    \end{tabular}
\end{table}

This indicates that SHAP and Confidence appear to lead users to trust the AI system more when that system is wrong. We find this result more strongly for behavioural trust than attitudinal trust in all but one test.

Notably, Anchor does not follow this pattern and instead shows no significant increase in either attitudinal or behavioural trust on these one-sided t-tests. (However, as we explore in Section \ref{sec:anchor-attitudinal}, they may show a significant decrease.)

\paragraph{Different Explanation Styles Have Different Effects on Unwarranted Trust}
We have shown already that SHAP and Confidence induce unwarranted trust relative to no explanation; we now show that there is a significant difference in the effect of some explanatory conditions relative to others. To do this, we examine the $\Delta trust_{behavioural}$ and $\Delta trust_{attitudinal}$ variables across explanatory conditions with an ANOVA test. For this test, we filter on $\neg correct_{AI}$. I.e.:

\begin{equation}
    \begin{split}
        \Delta \text{ any}(trust_{x1,x2} \neq trust_{x1,x3}) | \neg correct_{AI} & \text{ for } x1 \in \{behavioural, attitudinal\} \\
        & \text{ and } x2,x3 \in \{SHAP, Anchor, Confidence\}
    \end{split}
\end{equation}

\noindent This test was preregistered in both of our tasks \cite{natarajan_binns_2022}. Table \ref{tab:delta-trust-anova} contains the results of these analyses.

\begin{table}[htb]
    \centering
    \caption{These ANOVAs compare $\Delta trust$ between SHAP, Confidence, and Anchor to indicate where significant differences exist. We find two statistically significant differences, indicating that we should focus post-hoc analyses on these two.}
    \label{tab:delta-trust-anova}
    \begin{tabular}{lrrr}
        \toprule
        Task & Variable & Test Statistic & p Value \\
        \midrule
        \emph{Salary} & $\Delta trust_{behavioural}$ & $\mathbf{3.671}$ & $\mathbf{0.026}$ \\
        & $\Delta trust_{attitudinal}$ & $0.925$ & $0.397$ \\
        \midrule
        \emph{Credit} & $\Delta trust_{behavioural}$ & $0.066$ & $0.936$ \\
        & $\Delta trust_{attitudinal}$ & $\mathbf{6.213}$ & $\mathbf{0.002}$ \\
        \bottomrule
    \end{tabular}
\end{table}

Note from Table \ref{tab:delta-trust-anova} that in the Salary Estimation task, we find no significant results for our ANOVA $trust_{attitudinal}$, but do find significant results for $trust_{behavioural}$. However, in the Credit Delinquency Prediction task, we find significant results for our ANOVA $trust_{attitudinal}$, but none for $trust_{behavioural}$. We examine these two findings separately.

\paragraph{SHAP Increases Behavioural Trust More than Anchor in the Salary Estimation Task}
We now show that:

\begin{equation}
    \Delta trust_{behavioural,SHAP,salary} > \Delta trust_{behavioural,Anchor,salary} | \neg correct_{AI}
\end{equation}

\noindent Note that, while the result of the ANOVA test in c{tab:delta-trust-anova} supports that there are indeed statistically significant differences in the three group means of the $\Delta trust_{behavioural}$ variable, it does not specify which means are greater and which are less. For an indication of which means are greater, following our preregistered protocol for significant ANOVA results, we turn to Tukey's Honestly Significant Difference (HSD) test as a post-hoc test in Table \ref{tab:delta-trust-hsd}. As we found significant results in the ANOVA test of $\Delta trust_{behavioural}$ filtered on $\neg correct_{AI}$, we restrict our post-hoc analysis to this variable.

\begin{table}[htb]
    \centering
    \caption{Tukey's HSD test compares $\Delta trust_{behavioural,x}$ in \emph{Salary} with $\neg correct_{AI}$. We find that SHAP increases behavioural trust in incorrect AI outputs more than Anchor.}
    \label{tab:delta-trust-hsd}
    \begin{tabular}{lllrr}
        \toprule
        Explanation A & Explanation B & Variable & Test Statistic & p Value \\
        \midrule
        SHAP & Anchor & $\Delta trust_{behavioural}$ & $\mathbf{2.310}$ & $\mathbf{0.022}$ \\
        Confidence & Anchor & $\Delta trust_{behavioural}$ & $0.855$ & $0.599$ \\
        SHAP & Confidence & $\Delta trust_{behavioural}$ & $1.455$ & $0.198$ \\
        \bottomrule
    \end{tabular}
\end{table}

As can be seen in Table \ref{tab:delta-trust-hsd}, we observe a significant difference in the mean of $\Delta trust_{behavioural}$ between the SHAP and Anchor conditions with $\neg correct_{AI}$, but we do not observe a significant difference between the other conditions. This indicates that, beyond increasing behavioural trust in incorrect AI outputs, SHAP increases behavioural trust in incorrect AI outputs \emph{more} than Anchor.

\paragraph{Anchor Decreases Unwarranted Attitudinal Trust Relative to SHAP and Confidence in the Credit Delinquency Prediction Task}
Note that we found a large negative F for $\Delta trust_{attitudinal}$ in the Anchor case in the Credit Delinquency Prediction portion of Table \ref{tab:delta-trust-t} – an effect that is not significant due to the one-sidedness of our tests. However, as we found only positive F values for $\Delta trust_{attitudinal}$ in the SHAP and Confidence cases, we might expect that Anchor has a negative effect on $\Delta trust_{attitudinal}$ relative to SHAP and Confidence. Indeed, the result of the ANOVA test in Table \ref{tab:delta-trust-anova} supports that there are indeed statistically significant differences in the three group means of the $\Delta trust_{attitudinal}$ variable in this task, though it again does not specify which groups are different, or how.

For an indication of which means are greater, following our preregistered protocol for significant ANOVA results, we turn again to Tukey's HSD test as a post-hoc test in Table \ref{tab:delta-trust-hsd-2}. As we found significant results in the ANOVA test of $\Delta trust_{attitudinal}$ filtered on $\neg correct_{AI}$, we restrict our post-hoc analysis to this variable. We test:

\begin{equation}
    \begin{split}
        & \Delta trust_{behavioural,x1,credit} > \Delta trust_{behavioural,x2,credit} | \neg correct_{AI} \\
        & \qquad \text{ for } x1,x2 \in \{SHAP, Anchor, Confidence\}
    \end{split}
\end{equation}

\begin{equation}
    \begin{split}
        \Delta \text{ any}(trust_{x1,x2} \neq trust_{x1,x3}) | \neg correct_{AI} & \text{ for } x1 \in \{behavioural, attitudinal\} \\
        & \text{ and } x2,x3 \in \{SHAP, Anchor, Confidence\}
    \end{split}
\end{equation}

\begin{table}[htb]
    \centering
    \caption{Tukey's HSD test compares $\Delta trust_{attitudinal}$ across explanations in \emph{Credit} with $\neg correct_{AI}$. We find that Anchor decreases unwarranted trust relative to SHAP and Confidence.}
    \label{tab:delta-trust-hsd-2}
    \begin{tabular}{lllrr}
        \toprule
        Explanation A & Explanation B & Variable & Test Statistic & p Value \\
        \midrule
        SHAP & Anchor & $\Delta trust_{attitudinal}$ & $\mathbf{1.213}$ & $\mathbf{<0.001}$ \\
        Confidence & Anchor & $\Delta trust_{attitudinal}$ & $\mathbf{1.030}$ & $\mathbf{<0.001}$ \\
        SHAP & Confidence & $\Delta trust_{attitudinal}$ & $0.183$ & $0.708$ \\
        \bottomrule
    \end{tabular}
\end{table}

As can be seen in Table \ref{tab:delta-trust-hsd-2}, we observe a significant difference in the mean of $\Delta trust_{behavioural}$ between the Anchor condition and both other conditions, but we do not observe a significant difference between SHAP and Confidence. This indicates that, relative to both other conditions, Anchor actually \emph{reduces} attitudinal trust in the AI output.

Note that this does not prove that Anchor reduces attitudinal trust relative to no explanation. For this analysis, we will need another t-test. As we did not preregister this test, an analysis of this phenomenon is included in the exploratory proportion of our results below.

\paragraph{Behavioural and Attitudinal Trust are Highly Correlated}
It should be noted that some patterns observed for $trust_{behavioural}$ do not hold for $trust_{attitudinal}$ and vice-versa. However, while they are mathematically distinct constructs, they are both intended to measure the same underlying phenomenon. We apply Pearson's correlation analysis across all explanatory conditions in both the before- and after-cases. We also perform this analysis on $\Delta trust_{attitudinal}$ and $\Delta trust_{behavioural}$. 

For this analysis, we do not filter out positive cases. Rather, we consider all cases together. Results can be seen in Table \ref{tab:trust-correlation}.\footnote{This analysis was only partially preregistered; we did not register this analysis in the Salary Estimation task, but we did in the Credit Delinquency Prediction task \cite{natarajan_binns_2022}.}

\begin{table}[htb]
    \centering
    \caption{Pearson's test shows a high correlation between $trust_{attitudinal}$ and $trust_{behavioural}$ in both tasks. This correlation extends to the relationship between $\Delta trust_{attitudinal}$ and $\Delta trust_{behavioural}$.}
    \label{tab:trust-correlation}
    \begin{tabular}{lllrr}
        \toprule
        Task & Variable A & Variable B & Test Statistic & p Value \\
        \midrule
        \emph{Salary} & $trust_{attitudinal}$ & $trust_{behavioural}$ & $\mathbf{0.630}$ & $\mathbf{<0.001}$ \\
        & $\Delta trust_{attitudinal}$ & $\Delta trust_{behavioural}$ & $\mathbf{0.265}$ & $\mathbf{<0.001}$ \\
        \midrule
        \emph{Credit} & $trust_{attitudinal}$ & $trust_{behavioural}$ & $\mathbf{0.612}$ & $\mathbf{<0.001}$ \\
        & $\Delta trust_{attitudinal}$ & $\Delta trust_{behavioural}$ & $\mathbf{0.179}$ & $\mathbf{<0.001}$ \\
        \bottomrule
    \end{tabular}
\end{table}

Note that, though the attitudinal and behavioural trust variables display different behaviours in other analyses, Table \ref{tab:trust-correlation} indicates that they are indeed highly correlated across both of our tasks. Furthermore, though the correlation between the $\Delta trust$ is more modest, it is still statistically significant. These together leave little doubt that $trust_{attitudinal}$ and $trust_{behavioural}$ measure related, and perhaps even identical, concepts. In other words, when a participant says they trust the AI output, they generally act accordingly.

\paragraph{Anchor Decreases Attitudinal Trust in AI Outputs}\label{sec:anchor-attitudinal}
Having found no significant result indicating the presence of the hypothesised effect of Anchor explanations on unwarranted trust, we explore what effect Anchor explanations have on end users' trust in incorrect AI outputs. For this analysis, we filter on $\neg correct_{AI}$.\footnote{This analysis was not preregistered.}

We noted already that SHAP and Confidence appear to increase trust in cases where the AI output is incorrect. However, we noticed no such result for Anchor. However, we did observe an apparent negative effect on $\Delta trust_{attitudinal}$ in the t-tests for the Credit Delinquency Prediction task, though, as the tests were one-sided, we could not confirm the significance. Thus, we repeat this test as a two-sided test, as shown in Table \ref{tab:delta-trust-t-2}. We also repeat other t-tests on Anchor as two-sided, though, as they all have positive F-values, note that none can yield significant results.

\begin{table}[htb]
    \centering
    \caption{Two-sided t-tests compare $\Delta trust_{x,Anchor} \neq 0 for x \in \{attitudinal, behavioural\}$ in both tasks. We find that Anchor explanations decrease attitudinal trust in incorrect AI outputs in \emph{Credit}, but all other results are inconclusive.}
    \label{tab:delta-trust-t-2}
    \begin{tabular}{lllrr}
        \toprule
        Task & Explanation & Variable & Test Statistic & p Value \\ 
        \midrule
        \emph{Salary} & Anchor & $trust_{behavioural}$ & $0.509$ & $0.611$ \\
        & & $trust_{attitudinal}$ & $0.165$ & $0.869$ \\
        \midrule
        \emph{Credit} & Anchor & $trust_{behavioural}$ & $1.396$ & $0.164$ \\
        & & $trust_{attitudinal}$ & $\mathbf{-2.364}$ & $\mathbf{0.019}$ \\
        \bottomrule
    \end{tabular}
\end{table}

Note here that, on the two-sided t-test, we \textit{do} find that the provision of Anchor explanations decreases participant attitudinal trust in incorrect AI output, at least in the Credit Delinquency Prediction task. However, we do not see a similar effect on behavioural trust. 

\paragraph{Anchor and SHAP Increase Participant Self-Confidence in their Determinations}
Noting that behavioural trust is an index variable constructed from $selfconfidence$, we ask: does providing an Anchor explanation increase participant confidence in their own decisions when the AI output is incorrect? Similarly, we ask this question for both the SHAP and Confidence conditions, following exactly the format of the preregistered one-sided t-tests in Table \ref{tab:delta-trust-t}, but applied to the variable $\Delta selfconfidence$. Again, we filter on $\neg correct_{AI}$.:

\begin{equation}
    \Delta selfconfidence > 0 | \neg correct_{AI}
\end{equation}

Analysis can be seen in Table \ref{tab:delta-confidence-t}.\footnote{This analysis was not preregistered.} Note for clarity that $selfconfidence$ is the variable indicating participant confidence in their own decisions, and Confidence is the condition in which the explanation consists of the AI's own confidence in its suggestion.

\begin{table}[htb]
    \centering
    \caption{One-sided t-tests determine whether $\Delta selfconfidence > 0$ where $\neg correct_{AI}$ for all explanatory conditions and both tasks. We find that Anchor and SHAP increase participant self-confidence in their determinations, while Confidence yields inconclusive results.}
    \label{tab:delta-confidence-t}
    \begin{tabular}{lllrr}
        \toprule
        Task & Explanation & Variable & Test Statistic & p \\
        \midrule
        \emph{Salary} & Anchor & $selfconfidence$ & $\mathbf{2.171}$ & $\mathbf{0.016}$ \\
        & SHAP & $selfconfidence$ & $\mathbf{1.694}$ & $\mathbf{0.046}$ \\
        & Confidence & $selfconfidence$ & $1.047$ & $0.296$ \\
        \midrule
        \emph{Credit} & Anchor & $selfconfidence$ & $\mathbf{1.742}$ & $\mathbf{0.042}$ \\
        & SHAP & $selfconfidence$ & $\mathbf{3.473}$ & $\mathbf{<0.001}$ \\
        & Confidence & $selfconfidence$ & $0.752$ & $0.226$ \\
        \bottomrule
    \end{tabular}
\end{table}

We note that, while Confidence shows no significant effects on either task, participants shown an Anchor or SHAP explanation grow significantly more confident in their prediction, indicating that providing an Anchor or SHAP explanation serves to increase a participant's confidence in their own estimate.

\paragraph{Explanations Impact Trust Differently When the AI Output is Correct}
Note that ideally calibrated trust would involve both distrusting the AI output when it is wrong and trusting it when it is right. To assess the latter, we now turn to an evaluation of what happens in the cases where the AI is correct, i.e. $correct_{AI}$. Namely, we conduct two-sided t-tests on both trust variables in all three cases. Table \ref{tab:delta-trust-t-positives} contains the results of these analyses.\footnote{These analyses were not preregistered.}

\begin{table}[htb]
    \centering
    \caption{These two-sided t-tests compare $\Delta trust$ when $correct_{AI}$. We find that Confidence and SHAP increase trust in the AI system when it is correct, while Anchor decreases attitudinal trust in the AI system, but increases behavioural trust.}
    \label{tab:delta-trust-t-positives}
    \begin{tabular}{lllrr}
        \toprule
        Task & Explanation & Variable & Test Statistic & p Value \\ 
        \midrule
        \emph{Salary} & Anchor & $trust_{behavioural}$ & $0.502$ & $0.616$ \\
        & & $trust_{attitudinal}$ & $\mathbf{-2.337}$ & $\mathbf{0.020}$ \\
        & SHAP & $trust_{behavioural}$ & $0.295$ & $0.768$ \\
        & & $trust_{attitudinal}$ & $-1.385$ & $0.168$ \\
        & Confidence & $trust_{behavioural}$ & $\mathbf{2.410}$ & $\mathbf{0.017}$ \\
        & & $trust_{attitudinal}$ & $\mathbf{3.254}$ & $\mathbf{0.001}$ \\
        \midrule
        \emph{Credit} & Anchor & $trust_{behavioural}$ & $\mathbf{3.013}$ & $\mathbf{0.003}$ \\
        & & $trust_{attitudinal}$ & $\mathbf{-2.487}$ & $\mathbf{0.014}$ \\
        & SHAP & $trust_{behavioural}$ & $0.207$ & $0.836$ \\
        & & $trust_{attitudinal}$ & $\mathbf{3.538}$ & $\mathbf{0.001}$ \\
        & Confidence & $trust_{behavioural}$ & $\mathbf{2.863}$ & $\mathbf{0.005}$ \\
        & & $trust_{attitudinal}$ & $\mathbf{2.461}$ & $\mathbf{0.015}$ \\
        \bottomrule
    \end{tabular}
\end{table}

\paragraph{Confidence Explanations Increase Warranted Trust When the AI Output is Correct}
Note that, for both trust variables and both tasks, the Confidence condition boasts a significant positive $\Delta trust$. In other words, when the AI output is correct, providing confidence in its own prediction increases both behavioural and attitudinal trust towards the AI.

\paragraph{Anchor Explanations Decrease Warranted Attitudinal Trust but Increase Warranted Behavioural Trust When the AI Output is Correct}
In the Anchor case, it is clear that providing Anchor explanations yields a large decrease in $trust_{attitudinal}$. This, along with the finding that Anchor explanations decrease $trust_{attitudinal}$ when $\neg correct_{AI}$, would indicate that Anchor explanations have an overall negative impact on $trust_{attitudinal}$, regardless of case. Despite this, providing Anchor explanations yields an increase in $trust_{behavioural}$ (though this is only significant in the \emph{Credit} case). This suggests that, though participants report lower trust in AI outputs when shown an Anchor explanation, they behave as though their trust in the outputs is appropriately calibrated.\footnote{Though $trust_{attitudinal}$ and $trust_{behavioural}$ are closely correlated overall, this represents one instance in which they appear to reveal differences in participant behaviour.}

\paragraph{SHAP Explanations Increase Warranted Attitudinal Trust in the Credit Delinquency Prediction Task When the AI Output is Correct}
In the SHAP case, we find a large significant positive $F$ for $trust_{attitudinal}$ when $correct_{AI}$ in the credit delinquency prediction task. However, not only is the effect not mirrored in either test of $trust_{behavioural}$, but the same test on the salary estimation task has a negative $F$ statistic.

\subsection{Study Findings}\label{ssec:os_discussion}
Our results indicate that both SHAP and Confidence induce unwarranted trust in the explainee. I.e., on the \emph{Salary} and \emph{Credit} tasks, neither SHAP nor Confidence serves to correctly calibrate trust in AI outputs. Rather, they blindly increase trust in these outputs, encouraging users to incorrectly agree with the AI outputs. While this confirms the cautionary critique of \textcite{Lipton} as applied to SHAP in our domain, our results relating to Confidence suggest this critique is too narrow. Namely, issues of unwarranted trust do not seem confined to post-hoc xAI but are rather a function of providing post-hoc justification of the model's output. This suggests that even un-optimised notions of interpretability when provided post-hoc as justifications of model outputs, induce unwarranted trust.

Challenging this interpretation, we find no similar effect for Anchor. Instead, we find a significant decrease in participants' stated confidence in AI and a simultaneous increase in participant self-confidence in their own decisions. \textcite{miller_explanation_2017} identifies several features that social sciences would suggest make a good explanation; we conclude, following Section \ref{ssec:history}, that Anchor explanations serve to highlight strange model behaviour that correctly undermines explainee confidence in model outputs. In short, the problem is not the use of explanations as justification tools, but rather the use of ``bad'' explanations as justification tools.

This raises another question: if SHAP and Confidence are ``bad'' explanations as in-process justification tools, are they ``good'' explanations for something else?

\section[Participatory Design]{Participatory Design: Ex-Post Decision Support}\label{sec:xaicase}
\subsection{Motivation}
In Section \ref{sec:online}'s investigation of post-hoc explainable AI, we found that SHAP-based explanations can lead to unwarranted trust when used to justify decisions. It is clear, at least, that we should not use SHAP-based explanations as a DST in this context. However, we do not suggest that SHAP should be discarded entirely. In fact, though \emph{Salary} and \emph{Credit} both prove unsuitable tasks for applications of SHAP-based explanations as DSTs, we suggest that we might still make use of the explanations. In particular, in use cases where explainee trust in the underlying model is not at issue, SHAP's induction of unwarranted trust need not undermine its utility.

It is clear that when making a `primary' decision (i.e., the same decision the model output seeks to make), human reviewers working from AI outputs are preeminently concerned with whether to agree with or overrule the model's output. However, after making this decision, they are no longer necessarily concerned with whether to agree or disagree with the model's output. Consider the task of \emph{refining a scholarship selection algorithm}, the \emph{Refinement} task from Chapter \ref{ch:context}. Scholarship and talent investment programmes ordinarily select cohorts in a series of application cycles \cite{li2020hiring}. Between application cycles, they seek to examine their previous selection decisions, and possibly modify their processes to improve these decisions in the future \cite{li2020hiring}; if AI algorithms are used to support these decisions, the review will naturally include refinements to the AI algorithms. In this case, though, we are no longer concerned with whether the AI was correct; rather, we are concerned with whether the way the AI informs decision-making is conducive to ideal selection pipelines.

We conducted a human-centric study using SHAP explanations as an ex-post explanation tool to help selectors from Rise (see Appendix \ref{app:programmes}) with the \emph{Refinement} task. Through this participatory design study, we assess SHAP's usefulness based on the insights these explanations provide.

This study aims to answer RQ2:

\begin{enumerate}
    \item[(RQ2)] If post-hoc xAI methods induce unwarranted trust in-process, could they still be useful ex-post?\footnote{Recall the distinction between ex-post and post-hoc. We use the term `post-hoc' to refer to explanation algorithms that are applied after a model; ex-post, in contrast, refers to the explanation tools applied in decision support scenarios where the primary decisions have already been made.}
\end{enumerate}

\noindent In doing so, we restrict our attention specifically to SHAP, as we have already demonstrated its induction of unwarranted trust.

\subsection{Methodology}\label{ssec:cs_methods}
We test SHAP's usefulness in a human-centred context. To do so, we ran a study with scholarship and talent investment selectors (N=8) from Rise. Though Rise already used algorithmic scoring to support their decision-making, no selectors possessed experience with post-hoc xAI before the study. The study consisted of two participatory design workshops with said selectors – we term these `G1' and `G2'.\footnote{As Rise only employs a small number of easily identifiable selection selectors, to preserve the anonymity of the participants, we do not number or identify participants. Rather, we attribute quotes based on the workshop group.}

Before this study, we obtained informed consent from all participants. As we ran group workshops (and as we do not attribute comments to individual participants), participants were informed that they would not be able to recuse themselves after the study. Participants also gave consent to be recorded, and to have these recordings stored on a secure server. All recording, transcribing, and data analysis was conducted on secure servers. Ethics review was performed by the University of Oxford's Central University Research Ethics Committee.

\begin{figure}[htbp]
    \centering
    \includegraphics[width=.9\textwidth]{xai/case_flowchart.png}
    \caption{Each workshop consisted of a series of cases relating to a past application decision that was flagged by programme reviewers. In each case, participants were shown slides like in Figure \ref{fig:sample_case} and were asked to analyse the algorithm itself and whether the case warrants changes to the algorithm in future years.}
    \label{fig:case_flowchart}
\end{figure}

Both workshops followed an identical protocol. The flow of these workshops is shown in Figure \ref{fig:case_flowchart}, and more detail on the protocol followed can be found in Appendix \ref{app:xaiprotocol}.

In each workshop, participants discussed several cases, each examining a (possibly successful) applicant from a past application cycle who was flagged by programme reviewers for having perplexing algorithm scores relative to other known information. In each case, participants are asked to use visual, SHAP-based explanations to first understand why programme reviewers found these cases worth noting, then to explore why programme reviewers gave the feedback they did and what caused the algorithm's perplexing outputs, and finally to opine on whether the case suggests that changes should be made to the algorithm (or to the selection process as a whole) for future years. The cases themselves have been redacted, as they contain sensitive information about programme applicants, but a sample case can be seen in Figure \ref{fig:sample_case}.

In analysing our data, we follow \textcite{braun_using_2006}'s methodology for reflexive thematic analysis. 

\begin{figure}[htbp]
    \centering
    \includegraphics[width=.9\textwidth]{xai/sample_case.png}
    \caption{Each case explores one applicant from past years chosen by past programme reviewers after being flagged as having perplexing algorithm scores. Each case contains the applicant's profile (overall algorithm scores alongside demographic information; the profile is redacted to preserve applicant anonymity), the programme reviewers' comments, and the SHAP-based explanation (the score names are replaced with generic labels to preserve programme anonymity).}
    \label{fig:sample_case}
    
\end{figure}

\subsection{Results}\label{sec:cs_results}
Our case study yielded two key themes. Firstly, SHAP Explanations yield useful ex-post insights about feature importance. Second, even though SHAP yields useful information, the accessibility of such information depends on careful presentation. We now cover these in depth.

\paragraph{Ex-Post Insights on Important Features}
In both groups, several useful insights emerged due to the SHAP visualisations. For example, the relationships between scores and contextual factors (e.g., markers of applicants' socioeconomic status) revealed that context plays little to no role in scoring; despite this, an applicant's context has a strong impact on how selectors read scores.\footnote{Meanwhile, we find in Chapter \ref{ch:diversity} that selectors consider contextualising applications a key motivation behind considerations of diversity.} E.g., an applicant with high test scores from a poor region of Kenya is more impressive than one with high test scores from a rich part of the United Kingdom. When discussing one applicant who was selected, but had particularly low algorithmic scores, one participant said: ``This is one of the candidates that... [was from a] different country and [had] very low income'' (G1). Such cases raise difficult questions about fairness, including whether markers of disadvantage should be factored into decision-making, or if decision-makers should focus only on `task-relevant' factors \cite{dwork_fairness_2012}. If algorithms are not taking into account such information, decision-makers are all but guaranteed to find discrepancies between algorithmically-driven selections and their human-made ones.

It was also remarked upon that scores calculated based on the feedback of external experts often disagreed with scores calculated based on the feedback of programme applicants. It was discovered here that, contrary to programme expectations, the programme's expert reviews appeared less prone to biases than the peer ones. For one applicant: ``There was a question about why his peer and expert review were so different...I think confirms that it's not actually that they were seeing dramatically different things...his peers were dinging him for not seeming like he needed the award'' (G1). For another: ``I think this applicant has been significantly brought down by peer reviews; [their] scores are substantially lower than those that were, perhaps, given to [another applicant]'' (G1).

Similarly, it was observed that, unlike project reviews, group activities, and test results, the best candidates do not appear to have particularly good interview scores: ``I'm seeing also quite a few top-ranked candidates whose interview score was really low'' (G1). In some cases, this appeared to create a discrepancy between algorithmically generated overall scores and the participants' perceptions of the best candidates: ``[The applicant's] staff reviews imply that [they] should be top 30, and even if you factor in low interview scores...[they] are still pretty low on the algorithm score'' (G2).

\paragraph{Presentation is Key}
Besides insights about the selection process, the workshops yielded direct feedback on how the presentation of SHAP explanations should be improved. One major point was that contextual factors describing an applicant were missing. One participant said, of the explanation: ``It doesn't give me the context'' (G1). Another from the same group said: ``But I think without the context, it's really hard to decipher what's going on here'' (G1). This could be interpreted as participants asking for supporting information. However, when the researchers read out the information in the explanation's caption, this cleared up the participant's confusion: ``Yeah, that makes sense'' (G1). This suggests that, rather than needing more information, participants needed the information presented differently.

Several times, participants were unclear on the meaning of different aspects of explanations: ``Should I be alarmed and I see it going blue in the context of this? It's really hard for me to if you threw this at me...to compare'' (G1). Participants asked for the more complicated information as a ``Pre-reading'' (G2), and asked for simpler information, i.e., ``Maybe just colour coding things that are positive in one colour and then things that are negative in the other colours'' (G2).

One request that several participants echoed was that axes be kept constant, even between different types of scores: ``It's also different scaling. So, that massive bar...does not mean the same thing as the massive bar meant last time'' (G2). Another solution suggested to a similar problem was the provision of benchmark information: ``Like, give me the benchmark for that'' (G1).

\subsection{Participatory Design Study Findings}\label{ssec:cs_discussion}
Exposure to SHAP explanations appears to have yielded useful insights when used as part of an ex-post decision-making process. In particular, participants appeared to find SHAP explanations useful for indicating when information may have been over- or under-used in selectors' holistic review and in the algorithm itself. In this context, where decisions about individuals have already been made, and the organisation is looking into how to go about selecting its next cohort, the possibility of unwarranted trust in a particular output is not a concern. Rather, these explanations can be regarded as probes or provocations, helping decision-makers hone in on particular cases that highlight potential areas for improvement in decision-making, whether through changes in the model itself, or changes in human evaluation processes (e.g., placing greater or lesser weight on certain features). Thus, we can conclude that SHAP explanations may be useful ex-post when trust in the primary output is not at issue. Interestingly, the wolf-husky example from \textcite{ribeiro_why_2016}'s original LIME proposal could be interpreted as being used similarly; using an explanation of an existing classification to guide changes in the model in future (e.g. by adding an edge detection step before classification, to ignore snow in the background). In both cases, the explanation may reveal unwarranted reliance on (or lack of reliance on) a particular part of the feature space.

However, we also find that the SHAP-based waterfall explanations we provide, alone, lack the detail and presentation required. Practitioners desired additional context, additional modes of interaction with our explanatory materials, and points of comparison to clarify their investigation. This allays \textcite{miller_explainable_2023}'s concern that post-hoc xAI methods might discourage explainees from engaging deeply with the facts of the task. Rather, in this case, the explanations served as a platform for explainees to seek additional information to inform their decisions.

\section{Discussion}
\subsection{Implications}
By delineating DSTs by the stage of the decision they inform, we can answer the question: Should we use post-hoc xAI methods?'' separately for in-process and ex-post decisions. While we find them misleading, and thus dangerous, for in-process decisions, Section \ref{ssec:os_discussion} indicates that these misleading tendencies are not limited to post-hoc xAI, and are rather a symptom of the practice of post-hoc justification more broadly. Furthermore, Section \ref{ssec:cs_discussion} indicates that certain ex-post use cases do not necessarily require that explanations appropriately modulate trust. Thus, post-hoc xAI methods might still inform ex-post decision-making (e.g., selection process refinement, evaluations of selector bias). This yields two direct implications for the xAI field:

\begin{enumerate}
    \item While we reiterate caution around post-hoc justification of model outputs \cite{miller_explainable_2023, Lipton, bansal_does_2021, ford_play_2020, jacobs_how_2021}, we extend this caution from xAI methods to any form of post-hoc justification.
    \item We qualify this caution in its application to ex-post decision-making. We encourage a field that has, in large part, moved on from post-hoc notions of interpretability \cite{kumar_problems_2020,barocas_hidden_2020,Lipton,karimi_algorithmic_2021} to engage with and identify ex-post applications for these tools.
\end{enumerate}

\subsection{The Anchor Problem}\label{ssec:anchor_problem}
Suppose a farmer sees what they believe to be a sheep on a hill, and states ``there is a sheep on that hill''. Now, suppose this farmer sees a cleverly disguised goat, but that there is also a sheep on the hill, only invisible to the farmer. In this case, the farmer has a true belief (``there is a sheep on that hill'') and has justification for it (the goat), but the justification is unrelated to the truth of the belief. In a seminal paper on Epistemology, \textcite{Gettier_1963} discusses this class of problem (now called `Gettier Problems') and maintains that, despite the truth of the farmer's belief, that farmer does not know. In keeping with this tradition, \textcite{Cabitza_Fregosi_Campagner_Natali_2024} argue that, if an explainee is presented with a trust-inducing misleading explanation, even if that explanation induces trust in correct output, then the induced trust is misplaced.

In Section \ref{ssec:os_discussion}, we present what we believe is the most likely explanation for why Anchor explanations do not induce unwarranted trust: unlike SHAP and Confidence, these explanations might reveal concerns in the underlying model's local behaviour. However, other explanations exist. \textcite{miller_explanation_2017} describes desiderata that make explanations well-suited to most explainees: explanations should be contrastive, counterfactual, selective, and social. While Anchor explanations are not social, they are contrastive, counterfactual, and selective. However, it may be that these explanations' beneficial effects on trust stem not from an ability to reveal concerns in the underlying model, but rather from subjective desiderata. In this case, Anchor might still mislead \cite{Lipton}.

\subsection{Limitations and Future Work}
One core limitation of our work relates to the choice of tasks. While \emph{Salary}, \emph{Credit}, and \emph{Refinement} are closely related tasks, the distinction between in-process and ex-post decisions may be complicated by other distinctions between the three tasks. Future work should investigate this distinction in other contexts.

Another major limitation of our work stems from Section \ref{ssec:anchor_problem}'s Anchor problem. We recognise here that, though Sections \ref{ssec:history} and \ref{ssec:os_discussion} give a compelling theory for the surprising results surrounding Anchor explanations, we do not investigate why Anchor explanations do not induce unwarranted trust. We expect future work will examine this.

Finally, differences between the personalities of our online study and participatory design participants may limit the external validity of our results. Similarly, though we choose popular post-hoc xAI methods \cite{barocas_hidden_2020,kumar_problems_2020,weerts_human-grounded_2019,ribeiro_nothing_2016}, our choice of SHAP and Anchor limits applicability to other methods.

\subsection{Conclusion}
\textcite{miller_explainable_2023} likens explanations provided by SHAP and related explanation systems to ``Bluster'', a hypothetical person that always gives a recommendation, even when unsure, and does their best to justify this. They note that, in the context of decision support, such a person is less valuable than ``Prudence'', who asks the decision-maker's opinion first and then provides feedback, as Bluster risks discouraging explainee engagement with the decision. Here, we distinguish in-process and ex-post \emph{decision stages}. We conclude that, while \textcite{miller_explainable_2023}'s conclusion applies straightforwardly to the in-process stage, post-hoc xAI might still drive engagement and inform ex-post decisions, and urge that more be done to identify and apply post-hoc xAI where it is useful. This chapter both hints at the significance of the distinction between in-process and ex-post decisions and hints towards the question: might other DSTs be more useful in-process?

\chapter[What Are Generative AI Detectors Good For?]{\label{ch:genai}What Are Generative AI Detectors Good For? Evaluating and Implementing with the Decision Matrix\footnote{This chapter is based on a paper written in concert with Elías Hanno, Logan Gittelson, Reuben Binns, and Nigel Shadbolt. The paper is currently under review as: Neil Natarajan, Elías Hanno, Logan Gittelson, Reuben Binns, and Nigel Shadbolt. 2024. “What Are Generative AI Detectors Good For? Evaluating and Implementing with the Decision Matrix.” Under review at CHI 2025.}}

\minitoc

\section{Motivation}
Chapter \ref{ch:xai} elucidates an important distinction between in-process and ex-post decisions and finds post-hoc notions of interpretability suitable for supporting ex-post decisions, but not in-process ones. This chapter extends this work by conducting Action Research (AR) with the Rise programme to codify this stage distinction alongside another axis of stakes. We apply this distinction to the problem of generative AI (genAI) detection and again evaluate existing technology as a DST in the scholarship selection context.

\section{Introduction}\label{sec:genaiintro}
Since \textcite{ashish_vaswani_attention_2017} introduced transformer architecture, AI has made rapid progress. More recently, large language models (LLMs) like BERT and GPT3 have demonstrated the ability to generate human-like text \cite{jacob_devlin_bert_2018,brown_language_2020}. The releases of GPT3.5 and GPT4o have made these models more powerful and ubiquitous, and students are increasingly using them to write essays \cite{openai_gpt-4_2023,dehouche_plagiarism_2021}. This has raised concerns about plagiarism, and while many genAI-users are caught by detectors, these detectors often fall short of their goal \cite{liang_gpt_2023,kalpesh_krishna_paraphrasing_2023,mitchell_detectgpt_2023,tharindu_kumarage_stylometric_2023,dehouche_plagiarism_2021}. Essay-writers' ability to use genAI has changed the role of essay-based teaching and evaluation, especially in competitive contexts like scholarship selection.

Many bodies of research seek to understand this new role. Among them, theoretical research attempts to determine what uses of genAI constitute plagiarism, or should otherwise be banned, and what uses do not \cite{yu_huang_reflection_2023,MikePerkins_JasperRoe_2023}. Meanwhile, practical research builds and evaluates genAI detectors to determine whether a ban on such use is enforceable \cite{mitchell_detectgpt_2023,tharindu_kumarage_stylometric_2023,kalpesh_krishna_paraphrasing_2023}. Even work that does not overtly concern itself with plagiarism still grapples with issues of detecting and removing cohorts or passages, inadvertently viewing the problem through a plagiarism lens \cite{mitchell_detectgpt_2023,liang_gpt_2023}. This focus on plagiarism neglects other problems genAI has created for selectors.

We address this here with AR. We partner and ``research with'' \cite{bradbury_action_2003} Rise to understand the needs of their team of scholarship selectors (N=8; excludes authors) with regards to categorising and interpreting application essays and undertake this research while supporting their 2022 and 2023 application cycles. We identify ``decision points'' when the programme requires or desires to make a decision we might support with contextual information about genAI usage (see Table \ref{tab:decisions}). Through conversations with internal stakeholders \cite{Hayes_2011}, we identify two axes: stage and stakes, on which these decisions exist \cite{braun_using_2006}; we use these axes to construct the Decision Matrix in Figure \ref{fig:decision_matrix}. We then evaluate three genAI detectors on these decisions.

AR revealed several decisions that selectors desire to make surrounding genAI. Though the literature focuses on the decision of whether or not to disqualify applicants using genAI, Rise dismissed this decision as irrelevant, as their application guidelines do not forbid the usage of genAI. Instead, the programme expressed interest in decisions such as diligence, where the programme provides information about genAI usage alongside other facts about the applicant to make a holistic decision about when and how to consider an applicant. We conceptualise these decisions in a Decision Matrix. The Decision Matrix framework identifies challenges that selectors and selection teams are faced with, reframing them in terms of ``decision points''. We then categorise them on two axes: stage and stakes. The stage axis captures the important distinctions between decisions made in the process of selection (in-process) and decisions made after the primary \emph{Selection} decision of ``What cohort of people do we select?'' has been made (ex-post). The stakes axis, in contrast, captures the sensitivity (or severity) of the decisions. E.g., choosing to disqualify or select an applicant is a high-stakes decision, while choosing to assign extra staff to evaluate the truthfulness of an applicant's claims is a comparatively low-stakes decision. We then identify the properties desired from detectors so that they might support different decisions on different parts of our Matrix. In our evaluation of detectors for these decisions, we find that organisational needs are not met by current detectors, particularly concerning in-process decision-making. Our results suggest that detectors can be useful in aggregate analyses to support ex-post decisions. As a case study, we use one of our detectors, GPTZero, to support two decisions: \emph{Partners} and \emph{Pipeline} (described in Table \ref{tab:decisions}). We demonstrate two applications of detectors to support ex-post decisions. The flow of this study can be seen in Figure \ref{fig:flow}.

At a high level, this chapter has four major contributions:

\begin{enumerate}
    \item We identify decision points facing scholarship and talent investment programmes regarding genAI through AR.
    \item We introduce the Decision Matrix framework for categorising decisions and evaluating the decision support capabilities of genAI detectors.
    \item We apply this framework to two detectors, GPTZero and Originality.ai, on Rise's 2022 and 2023 application data and determine their suitability for supporting specific decisions.
    \item As a case study, We use GPTZero to support two decision points, \emph{Partners} and \emph{Pipeline}, in Rise's context.
\end{enumerate}

\section{Related Works}\label{sec:rw}
GenAI models have quickly surpassed what was previously thought possible with a natural language model \cite{brown_language_2020,chowdhery_palm_2022,openai_gpt-4_2023}. While their ability to respond with natural language is impressive, their use in academic writing raises questions; of particular note are those of fairness and integrity \cite{hu_challenges_2023}. In part due to these questions, there are many approaches to detecting AI usage, a vast majority of them commercial. Researchers have developed methods such as DetectGPT and stylometric detection, while commercial approaches include AI Writing Check, CatchGPT, Copyleaks, GPT Radar, GPTZero,  Turnitin, and Originality.ai \cite{mitchell_detectgpt_2023,kalpesh_krishna_paraphrasing_2023,tharindu_kumarage_stylometric_2023,gptzero_gptzero_2023,kirchner_new_2023}. These genAI detectors, much like genAI itself, have advanced rapidly in capability.

Many research institutions make their stances on pedagogical integrity clear. \textcite{h_holden_thorp_chatgpt_2023}, in an editorial from \emph{Science}, declares that ``the word `original' is enough to signal that text written by ChatGPT is not acceptable: It is, after all, plagiarized from ChatGPT''. The journal adopts a policy that: ``text generated by ChatGPT (or any other AI tools) cannot be used in the work, nor can figures, images, or graphics be the products of such tools'' \cite{h_holden_thorp_chatgpt_2023}. However, while research institutions face pressure to publish universal guidelines, researchers themselves are free to debate the theoretical and ethical implications of genAI usage \cite{lav_r_varshney_limits_2020,h_holden_thorp_chatgpt_2023,yu_huang_reflection_2023,weber-wulff_testing_2023,otterbacher_why_2023}. \textcite{yu_huang_reflection_2023} argues that these guidelines emerge as a result of pressure placed on organisations, and that, in the pursuit of academic integrity, these institutions hamper learning. A better system would encourage the use of these new technologies and instead make appropriate adjustments to teaching methods and examination standards 
\cite{yu_huang_reflection_2023}. 

While the complete ban of genAI systems is perhaps disagreeable and unrealistic, it is at least clear. \textcite{MikePerkins_JasperRoe_2023} point out a problematic lack of clarity in policies that allow genAI for some uses, but not others. Under many of these guidelines, detection of genAI is not sufficient to determine whether programme policy has been violated, as the nature of the usage of genAI must be compared to the programme guidelines (e.g., if a line or paragraph is rewritten by a genAI, is this a form of writing, or editing?). \textcite{MikePerkins_JasperRoe_2023} do not encourage bans of genAI usage, instead arguing the virtues of ``integrating technology, education, policy reform, and assessment restructuring''.

Despite these urgings, the field remains preoccupied with the question of detecting AI-based plagiarism. Many papers have undertaken the task of determining whether genAI detectors can help censure writers who use AI. Some studies conclude that detection algorithms can be effective across a variety of simple detection tasks \cite{dugan_raid_2024,weber-wulff_testing_2023,tharindu_kumarage_stylometric_2023,elkhatat_evaluating_2023,mitchell_detectgpt_2023}. Other evaluations find that, for more complicated tasks such as detecting paraphrased AI-generated text, these detectors perform less well \cite{kalpesh_krishna_paraphrasing_2023}. Ostensibly unrelated to plagiarism, researchers are beginning to evaluate the potential implications of using imperfect AI detectors \cite{liang_gpt_2023} or the implications of genAI for content creation \cite{kalpesh_krishna_paraphrasing_2023}. For example, comparing detectors' False Positive Rates (FPRs) for genuinely human-generated essays from a US-based essay competition ($N = 88$) against those from Chinese English-language test takers ($N = 91$), \textcite{liang_gpt_2023} conclude AI detectors are biased against non-native English writers. Even still, these works concern themselves primarily with the question of whether a single essay includes any AI-written content; in adopting this viewpoint, they ascribe value to this individual-level determination, inadvertently viewing this problem through the lens of plagiarism.

Though the problem of detecting AI-written plagiarism is interesting, it is only one of the many problems selectors evaluating essays are faced with. While theoretical work explores the implications of these other problems \cite{otterbacher_why_2023,yu_huang_reflection_2023}, practical work often overlooks selectors' real and present need for solutions.

\section{Methodology and Action Research}\label{sec:embedded}
\subsection{What is Action Research?}\label{ssec:par}
Action Research (AR) is a research philosophy that emphasises ``research with, rather than on, people'' \cite{bradbury_action_2003}. Rather than one specific method, AR is best seen as a collection of related methods all embodying this ethos, usually to produce research contributions useful to the target group of people \cite{lu_organizing_2023}. Among these are semiotic inspection \cite{DeSouza_Leitão_2009,Alvarado_Waern_2018} and participatory design  (PD) \cite{braun_using_2006,Griffiths_Johnson_Hartley_2007,blythe2014research,Knapp_Zeratzky_Kowitz_2016}. AR is most often used in the context of social work, but can be applied across a variety of fields \cite{dombrowski_social_2016,lu_organizing_2023}. 

In education, AR is often used in a classroom setting \cite{Mertler_2019}. \textcite{venn-wycherley_realities_2024} argue that it is crucial in this setting to perform AR on both educators (teachers) and educatees (students), as failing to do so is liable to yield contributions useful to one group but not the other. While this holds for classroom settings, engagement across the stakeholder map is less feasible or desirable in scholarship selection. Unlike teacher and student, who share the common goal that the student learn, selector and applicant are at cross purposes: selectors seek to choose the `best' cohort of applicants (although they often disagree on what constitutes `best'), while applicants seek to be included in the chosen cohort \cite{bergman2021seven}. Thus, when elucidating the interests and desires of one group, the other will merely act as noise. (E.g., applicants who use genAI to assist in writing their application will, of course, oppose using systems that monitor genAI usage to disqualify applicants.)

AR is comparatively new to HCI \cite{Hayes_2011,lu_organizing_2023}, but its methods and philosophies closely mirror longstanding pillars of HCI \cite{Hayes_2011}. Much like PD and other HCI methods, AR seeks to democratise the research and design processes; unlike PD, AR extends beyond building solutions democratically, and sees learning through action as the ultimate research contribution \cite{Hayes_2011}. For example, AR sees all parties become: ``Co-investigators of, co-participants in, and co-subjects of...the project'' \cite{Hayes_2011}.  Thus, research questions are formulated by and with participants, actions and interventions are designed by and with participants, and results are found by and with participants \cite{Hayes_2011}.

\subsection{Our Action Research}
In many scholarship selection contexts, organisations typically select groups of talented young people based on applications consisting variously of essays, videos, test scores, interviews, project results, group or individual activities, etc.. The organisation will begin by narrowing down the pool of candidates, often based primarily on test results, project results, or other easy-to-gather information. The organisation likely then compiles information from the application into short-form summaries of each applicant, complete with internally generated scores, and often even a recommendation. This recommendation is then often reviewed by a selection committee, who craft a cohort from the recommended applicants.

We engage Rise in AR to investigate this selection context. In engaging this organisation in AR, several of the authors also functioned in supporting roles for the organisation's 2022 and 2023 selection cycles. In addition to working alongside the selectors themselves, the authors were, at the time, a part of Rise.

As with all AR, we first worked with participants (N=8; excludes authors) to identify research questions. We did this via a mixture of synchronous and asynchronous interactions. After this, we evaluated two genAI detectors, GPTZero and Originality.ai, according to our research questions. Finally, we implemented one genAI detector as an intervention. The flow of this study can be seen in Figure \ref{fig:flow}.

\begin{figure}[htbp]
  \centering
  \includegraphics[width=\textwidth]{genai/study_flow.png}
  \caption{This figure describes the flow of our research.}
  \label{fig:flow}
\end{figure}

Before our study, we obtained consent from all 8 participants to be included. Applicant essay data was collected by Rise, who obtained consent to use these essays (anonymously) for research purposes. Participants also gave consent to be recorded, and to have these recordings stored on a secure server. All recording, transcribing, and data analysis was conducted on secure servers. Ethics review was performed by the University of Oxford's Central University Research Ethics Committee.

\subsection{Positionality}
Following \textcite{venn-wycherley_realities_2024}, we state researcher positionality here. The research team is comprised of five researchers split between the United States and the United Kingdom; all researchers are men; four out of five researchers are ethnically White, while the fifth is South Asian; three researchers are affiliated with the University of Oxford and two are affiliated with Schmidt Futures.

\begin{table}[htbp]
  \centering
  \caption{Rise discussed a desire to make several decisions concerning genAI. These decisions and the information desired to support them are detailed in Chapter \ref{ch:context}, but replicated here. Though the programme discussed disqualification, the programme has no application guidelines surrounding the use of genAI and expressed no interest in disqualification. We include it here due to its ubiquity elsewhere in the literature \cite{liang_gpt_2023,mitchell_detectgpt_2023,tharindu_kumarage_stylometric_2023,kalpesh_krishna_paraphrasing_2023}.}
  \label{tab:decisions}
  \begin{tabular}{ p{0.2\textwidth}p{0.3\textwidth}p{0.45\textwidth}}
      \toprule
      Decision Point & Decision Description & Supporting Information \\
      \midrule
      \emph{Diligence} & The programme makes holistic decisions about when and how to consider applicants. & Information about which essays (and which parts of essays) were written by genAI; information about whether the genAI-written passages are hallucinations. \\ 
      \emph{Partners} & The programme must determine whether to continue channel partnerships, which encourage and support applicants. & Whether any channel partners' affiliated applicants use genAI disproportionately. \\
      \emph{Pipeline} & The programme decides whether to modify their application material or process. & Information about the usage of genAI throughout the application pipeline. \\
      \emph{Gameability} & The programme decides how to modify their application material or process. & Information about how AI-generated essays are scored under the current application process. \\
      \midrule
      \emph{Disqualification} & A programme may decide to disqualify an applicant that violates their application guidelines. & Information about whether essays violate application guidelines around genAI usage. \\
      \bottomrule
  \end{tabular}
\end{table}


\section{Constructing the Decision Matrix}\label{ssec:poi}
We began from a starting point of ``What do we do about genAI?''. Literature about genAI elsewhere prompted early discussions to focus on: ``Can we determine whether an applicant used genAI to plagiarise?'' but Rise quickly discarded this; Rise had no application guidelines forbidding genAI usage, and hoped to accommodate ``innovative uses of this powerful technology'' in their selection process. From there, we quickly moved to ``What decisions do talent investment organisations want to make around generative AI usage?''. We then sought out other decisions the programme wished to make in response to genAI. Ultimately, we identified the decision points in Table \ref{tab:decisions}; our final research question, then, is ``Do existing generative AI detectors support these decisions?''.

Notably, the decision to disqualify applicants for using genAI (\emph{Disqualification}) appears several times in the literature surrounding potential use cases for detectors \cite{gptzero_gptzero_2023,kalpesh_krishna_paraphrasing_2023}. Thus, despite Rise's dismissal, we list it alongside the different decisions discussed in Table \ref{tab:decisions}.

\begin{figure}[htbp]
  \centering
  \includegraphics[width=0.8\textwidth]{genai/decision_matrix.png}
  \caption{This figure places the decisions from Table \ref{tab:decisions} on the Decision Matrix, with axes of stage and stakes.}
  \label{fig:decision_matrix}
\end{figure}

In conversation with Rise selectors, we isolated two `axes' on which the decision points exist: stage and stakes. Decision stages vary from entirely in-process (`primary' decisions during the selection process) to entirely ex-post (`secondary' decisions about future selection processes). Meanwhile, decision stakes vary from very low (e.g., a due-diligence decision to investigate further) to very high (e.g., a decision to disqualify an applicant). Figure \ref{fig:decision_matrix} uses these axes to categorise the decisions in Table \ref{tab:decisions}.

We also work with selectors to determine what type of properties a detection score should have to support each decision. For example, a disqualification decision may have stricter requirements than a diligence one. We lay out these desiderata on the same axes as Figure \ref{fig:decision_matrix} in Figure \ref{fig:desiderata_matrix}. 

\begin{figure}[htbp]
  \centering
  \includegraphics[width=0.8\textwidth]{genai/desiderata_matrix.png}
  \caption{This figure places uses our Decision Matrix to understand the properties of genAI detectors required (or, in the case of low-stakes decisions, desired) to support each decision.}
  \label{fig:desiderata_matrix}
\end{figure}

\section{Applying the Decision Matrix}\label{sec:data}
\subsection{Evaluating for Decisions}
We seek to evaluate genAI detection software for its suitability in supporting decisions across the Decision Matrix in Figure \ref{fig:decision_matrix}. To do this, we first evaluate detectors on the properties outlined in Figure \ref{fig:desiderata_matrix}; if a detector has the properties outlined in both low-stakes and ex-post decisions, it is likely to be useful in supporting decisions from this quadrant. However, although Rise selectors indicated that these properties are necessary, we still seek to test their sufficiency. Thus, where the desiderata are satisfied, we apply the detectors to specific decisions from Figure \ref{fig:decision_matrix}. In particular, we select one decision from each quadrant (unless all detectors are deemed to lack the properties required for decision support in that quadrant). Finally, in case multiple detectors have sufficient properties, we compare detectors directly to make a recommendation about which method selectors should use for these use cases.

\subsection{Data}
\subsubsection{Applicant Data}\label{sssec:applicant_data}
We use data from two of Rise's application cycles, Cycle 2022 and Cycle 2023.

Applications for Cycle 2022 were due in early 2022, well before ChatGPT's public release, so we assume that these submissions were written without the use of genAI \cite{openai_gpt-4_2023}. Applications for Cycle 2023 were due in early 2023, so genAI tools were widely available and AI detection tools were already emerging \cite{kirchner_new_2023,gptzero_gptzero_2023,liu_deid-gpt_2023}; we thus make no such assumption for these applications.\footnote{Several avoidance detection strategies (e.g., paraphrasers) have been proven to severely hamper state-of-the-art detection \cite{kalpesh_krishna_paraphrasing_2023}. However, as of the 2023 deadline, the efficacy of these detection avoidance strategies was not well-known. Thus, we assume these strategies were not widely employed.}

\begin{table}[htbp]
    \centering
    \caption{This table shows the total number of submitted essays evaluated across years and demographic groups.}
    \label{tab:demo_counts}
    \begin{tabular}{ l r r }
        \toprule
        Demographic Group & 2022 Essays & 2023 Essays \\
        \midrule
        Male & $6,475$  & $11,080$ \\
        Female & $8,710$  & $13,380$ \\
        Other & $176$ & $355$ \\
        \midrule
        Caribbean & $128$ & $75$ \\
        East and Southeast Asia & $1,332$ & $395$ \\
        Core Anglosphere & $522$ & $1,865$ \\
        Eastern Europe / Central Asia & $85$ & $705$ \\
        South Asia & $2,130$ & $2,560$\\
        Latin America & $709$ & $2,855$ \\
        Middle East / North Africa & $1,363$ & $4,100$ \\
        Sub-Saharan Africa & $8,972$& $8,375$\\
        Western Europe & $59$ & $560$ \\
        Pacific Islands & $5$& $0$ \\
        \midrule
        Total & $15,149$ & $24,815$ \\
        \bottomrule
    \end{tabular}
\end{table}

Each applicant submitted essays in response to five prompts. At the request of Rise, these prompts are not described in more detail here; what information the program wishes to share can be found in Appendix \ref{app:programmes}. Applicants in both cycles provided information on their background, including gender identity and countries of citizenship. 

Applicants were asked their gender identity and given options including `male', `female', `non-binary', `other', and `prefer not to say'. The programme uses all categories. However, the vast majority selected `male' or `female', so, to avoid spurious results from small samples, we group `non-binary' and `prefer not to say' under `other' for this analysis. 

Similarly, applicants reported their primary, secondary, and tertiary nationalities (where they exist). The programme uses only primary nationalities in this context; throughout the 2022 and 2023 application cycles, the program has grouped applicant nationalities in various ways. Rise uses the `semi-regional' groupings presented here to capture relevant similarities between nationalities.\footnote{E.g., while `Core Anglosphere' is not a distinct geographic region, applicants from Core Anglosphere countries tend to speak English as their first language, come from similar school systems, and share similar facets of culture.} These regional groupings are the Rise's, and at the programme's request, we do not share specific country-to-region codings.

Table \ref{tab:demo_counts} describes the analytic sample of essays by these demographic dimensions for each cycle. 

\subsubsection{Synthetic Data}\label{sssec:chatgpt}
To obtain a set of known AI-generated essays, we generate $5,002$ synthetic essays using OpenAI's ChatGPT API and Cycle 2022 prompts. These sit alongside our $15,149$ human-written, applicant-submitted Cycle 2022 essays and form our Cycle 2022 corpus. In contrast, our Cycle 2023 corpus consists entirely of applicant-submitted essays, though it is unknown how many of these are AI-generated. Our entire corpus is detailed in Table \ref{tab:cycle_counts}.

\begin{table}[htbp]
  \centering
  \caption{This table shows the total number of (submitted and generated) essays. The top row, Cycle 2022 submissions, is assumed to be human-written, as the Cycle 2022 submission deadline predates the release of ChatGPT. The second row lists essays generated by the authors in response to the Cycle 2022 prompts. The third row lists submitted essays of unknown providence from the 2023 application cycle.}
  \label{tab:cycle_counts}
  \begin{tabular}{ l r }
      \toprule
      Source  & Essays \\
      \midrule
      Cycle 2022 Submissions (Assumed Not AI) & $15,149$ \\
      ChatGPT Responses to Cycle 2022 Prompts & $5,002$ \\
      Cycle 2023 Submissions (Potentially AI) & $24,815$ \\
      \bottomrule
  \end{tabular}
\end{table}

All of our synthetic essays were generated via OpenAI's ChatGPT API using GPT-3.5 \cite{brown_language_2020}. More details on our prompts can be found in Appendix \ref{app:prompt}.

\subsubsection{Detectors}\label{sssec:detectors}
Despite the myriad of available genAI detection tools, standardised comparisons of detectors are few and far between, but the benchmarks that do exist list similar models as leaders in accuracy across standard FPR thresholds. \textcite{dugan_raid_2024} introduce RAID, a standardised genAI detection benchmark, and apply it to twelve popular models. Their results are mixed, but demonstrate a clear advantage for three detectors: the open-source Binocular model, and the commercial GPTZero and Originality.ai models \cite{dugan_raid_2024}. \textcite{verma_ghostbuster_2023} compare GPTZero, DetectGPT, two baselines, and their own model, Ghostbuster. Their results are similarly mixed but, in a variety of scenarios, GPTZero or Ghostbuster variously perform best \cite{verma_ghostbuster_2023}. Rise reached out to both GPTZero and Originlity.ai, and both offered access to their models for research purposes.

To calculate scores, we use the API under default settings for both GPTZero and Originality.ai. Note that we calculated our GPTZero-based detection scores in early 2023, while the Originality.ai scores were calculated more recently in mid-2024. Thus, a more recent version of GPTZero's model may have since been made available; our results apply only to the 2023 version we used. This yielded various statistics from each detector for each essay, but we are primarily interested in the overall likelihood statistic from each.

\begin{figure}[htb]
  \centering
  \includegraphics[width=0.4\textwidth]{genai/roc_curve.jpg}
  \caption{This Receiver Operating Characteristic (ROC) curve shows the performance of each detector on our data of known providence (Cycle 2022 submissions and ChatGPT responses to Cycle 2022 prompts). While GPTZero's ROC curve has a moderate area under the curve (AUC), Originality.ai's ROC curve has a very high AUC.}
  \label{fig:roc_auc}
\end{figure}

\subsection{Results}
\subsubsection{Both Detectors Possess the Properties Desired for Low-Stakes Decision Support}
We identified no properties required to support low-stakes decisions, but list several desiderata that we consider more a matter of utility than of necessity. In particular, detectors should have:

\begin{enumerate}
    \item Low cost
    \item High utility
\end{enumerate}

We measure cost as the price per essay evaluated at the highest tier of subscription service available from each detector in Table \ref{tab:detector_cost}. We measure utility according to ROC AUC.

Table \ref{tab:detector_cost} shows that both detectors have a low cost, with GPTZero having a slightly lower cost.\footnote{As we used both detectors for research purposes, we reached out to the companies involved and were given research access. These cost estimates are based not on our access, but on public information assuming the highest tier of subscription service available from each detector.}

Figure \ref{fig:roc_auc} demonstrates high utility from both detectors, though Originality.ai has a higher ROC AUC than GPTZero. Overall, both detectors satisfy the desiderata specific to low-stakes decisions.

\begin{table}[htbp]
  \centering
  \caption{This table displays the estimated per-essay costs of each detector; both detectors are deemed sufficiently cost-effective.}
  \label{tab:detector_cost}
  \begin{tabular}{l r}
      \toprule
      Detector & Cost per Essay \\
      \midrule
      GPTZero & $\$0.023$ \\
      Originality.ai & $\$0.06$\\
      \bottomrule\\
  \end{tabular}
\end{table}

\subsubsection{GPTZero Possesses the Properties Required for High-Stakes Decision Support}\label{sssec:highstakes}
We identified two properties required to support high-stakes decisions. Detectors should have:

\begin{enumerate}
    \item High (predictive) discrimination
    \item Low bias
\end{enumerate}

We measure discrimination according to whether the detector discriminates positive and negative cases (see Table \ref{tab:detector_cost}). We measure bias as the difference in FPR between demographic groups.

\begin{figure}[htb]
  \centering
  \includegraphics[width=0.8\textwidth]{genai/histogram.jpg}
  \caption{These histograms demonstrate the high bimodality and predictive discrimination of both detectors' outputs.}
  \label{fig:histogram}
\end{figure}

As can be seen in Figure \ref{fig:histogram}, both detectors' outputs are close to either 0 or 1, though Originality.ai has far higher discrimination than GPTZero. This is reflected in the confusion matrices in Figure \ref{fig:confusion}. As can be seen, both output statistics discriminate well between positive and negative cases. However, while GPTZero discriminates between positive and negatives, Figure \ref{fig:confusion} shows that Originality.ai has less error of both types.

\begin{figure}[htb]
  \centering
  \includegraphics[width=0.8\textwidth]{genai/confusion.jpg}
  \caption{These confusion matrices again demonstrate the high predictive discriminative capabilities of both detectors' outputs.}
  \label{fig:confusion}
\end{figure}

We see bias in both statistics in Table \ref{tab:demo_means_c2}, as the FPRs (average detection statistic on known-human essays) differ by demographic in both cases. Interestingly, though GPTZero has much higher false positive rates in all cases, these rates differ less by demographic than those of Originality.ai (this can be seen in the lower ANOVA statistic). This results in us observing Originality.ai ANOVA statistics $2.75$ and $3.40$ greater than GPTZero ones, respectively. This is, practically, the difference between minor between-group variance and measurable and harmful bias. For this reason, although both detectors have bias, we believe that GPTZero's bias does not render it unsuitable for ex-post decisions across our applicant pool, while Originality.ai's does.\footnote{Note that, in the case of Originality.ai, this appears to be driven by disparate performance across regions (and the ``Other'' gender category). Organisations with more regionally homogeneous applicants may wish to re-analyse Originality.ai's biases across their relevant demographics.} In this case, Rise saw the high risk of false positives from the GPTZero statistic as sub-optimal but deemed the bias demonstrated by Originality.ai, particularly across different regions, untenable.

\begin{table*}[htbp]
  \centering
  \caption{This table displays FPRs for both detectors split by applicant demographics and compared by ANOVA in the (human-written) Cycle 2022 submissions. Differences in FPR here can be interpreted as bias: both detectors display a bias, but Originality.ai displays a far higher bias than GPTZero.}
  \label{tab:demo_means_c2}
  \begin{tabular}{l r r}
      \toprule
      Demographic Group & GPTZero & Originality.ai \\
      \midrule
      Male                    & $0.23$ & $0.06$ \\
      Female                  & $0.23$ & $0.07$ \\
      Other                   & $0.21$ & $0.16$ \\
      ANOVA Statistic         & $\mathbf{0.11}$ & $\mathbf{2.86}$ \\
      \midrule
      Caribbean               & $0.17$ & $0.00$ \\
      East and Southeast Asia     & $0.24$ & $0.06$ \\
      Core Anglosphere               & $0.21$ & $0.17$ \\
      Eastern Europe / Central Asia    & $0.23$ & $0.17$ \\
      South Asia     & $0.22$ & $0.07$ \\
      Latin America           & $0.21$ & $0.02$ \\
      Middle East / North Africa   & $0.25$ & $0.05$ \\
      Sub-Saharan Africa      & $0.23$ & $0.06$ \\
      Western Europe            & $0.20$ & $0.00$ \\
      Pacific Islands         & $0.04$ & $0.00$ \\
      ANOVA Statistic         & $\mathbf{0.57}$ & $\mathbf{3.97}$ \\
      \midrule
      All Submissions         & $0.23$ & $0.05$ \\
      \bottomrule
  \end{tabular}
\end{table*}

\subsubsection{Both Detectors Possess the Properties Required for Ex-Post Decision Support}
We identified two properties required to support ex-post decisions. Detectors should have:

\begin{enumerate}
    \item Moderate Discrimination
    \item Calibrated Probability\footnote{Calibrated statistics are ones whose distributions are probability-like in expectation.}
\end{enumerate}

We measure discrimination as in Section \ref{sssec:highstakes}. We measure statistic calibration using calibration curves in Figure \ref{fig:calibration}.

We have already determined in Figure \ref{fig:roc_auc} that Originality.ai has particularly high predictive discrimination, while GPTZero has moderate predictive discrimination suitable for ex-post analyses.

\begin{figure}[htb]
  \centering
  \includegraphics[width=0.4\textwidth]{genai/calibration.jpg}
  \caption{These calibration curves demonstrate the bimodality of both detectors' outputs. As can be seen, both output statistics fall far below the calibration curve; these output statistics are not calibrated on our data.}
  \label{fig:calibration}
\end{figure}

In aggregate analyses, we desire to reason about the probability that a particular essay was generated by AI ($P(AI)$), or even the expected number of AI-generated essays in a given group ($E(P(AI))$). This yields convenient general properties, e.g., $E(P(AI))$ is just the mean $P(AI)$ within a given group. Figure \ref{fig:calibration} provides a calibration curve for both detectors and demonstrates a comparative lack of calibration in both cases. In both cases, output statistics fall far below the calibration curve, indicating a bias towards extrema. In effect, this entails that we should not treat these output statistics like probabilities. However, so long as we have a large provenance of labelled data (which we do in the form of $15,149$ applicant-submitted and $5,002$ ChatGPT-generated Cycle 2022 essays), we can calibrate an uncalibrated statistic to achieve a more probability-like output. In our case, we calibrate our statistics on our data by applying a monotonic transformation to ensure that, within any subset of our body of essays, the mean predicted probability aligns with the fraction of positive cases.

Thus, we conclude that both calibrated statistics possess the properties of probabilities that we would require for use in ex-post analyses. We caution other organisations against using these detectors for these purposes without first calibrating their output statistics on data of known provenance.

\subsubsection{Neither Detector Possesses the Properties Required for In-Process Decision Support}
We identified four properties required to support in-process decisions. Detectors should have:

\begin{enumerate}
    \item Minimal Bias
    \item Interpretable Outputs
    \item Acceptable Performance at Low FPR
    \item Bimodal Statistic
\end{enumerate}

We measure bias as in Section \ref{sssec:highstakes}. We reason about model interpretability. We measure acceptable performance at low FPR based on TPR rates at an FPR of $\%1$. We measure bimodality using the histogram in Figure \ref{fig:histogram} and the calibration curve in Figure \ref{fig:calibration}.

Recall that, in Table \ref{tab:demo_means_c2}, we observe bias in both outputs and conclude that, while GPTZero is sufficiently balanced across demographic groups, Originality.ai's outputs display too much regional bias for our purposes. 

We also note that GPTZero provides interpretability in the form of local-level scores that might direct a human overseer's attention to particularly problematic phrases or sentences. Originality.ai, in contrast, only provides a single overall statistic. 

Figure \ref{fig:histogram} demonstrates the bi-modality of both detectors. Though it is clear that Originality.ai's output is more bimodal, we consider both detectors sufficiently bimodal for our purposes.

\begin{table*}[htbp]
  \centering
  \caption{This table displays TPRs at $\%1$ FPR.}
  \label{tab:tprs}
  \begin{tabular}{l r r}
      \toprule
      Detector & TPR \\
      \midrule
      GPTZero & $\%36.0$ \\
      Originality.ai & $\%98.9$ \\
      \bottomrule
  \end{tabular}
\end{table*}

Finally, to analyse performance at low FPR, our organisation set the acceptable FPR at $\%1$. We see a TPR of $\%36.0$ for GPTZero and $\%98.9$ for Originality.ai in Table \ref{tab:tprs}. While GPTZero's output statistics are interpretable and display limited bias, it fails at low FPR rates.

\begin{figure}[htbp]
  \centering
  \includegraphics[width=0.8\textwidth]{genai/satisfaction_matrix.png}
  \caption{This figure demonstrates the results of our application of the Decision Matrix, marking the use cases we consider suited for each detector.}
  \label{fig:satisfaction_matrix}
\end{figure}

\section{Analysis of \emph{Pipeline} and \emph{Partners}}\label{ssec:decisions}

\subsection{Low-Stakes Ex-Post: Analysing Overall GenAI Usage in the 2023 Application Cycle (\emph{Pipeline})}
We focus our subsequent analysis primarily on the potential use of genAI by applicants in Cycle 2023. Seeking to avoid the disproportionate effects induced by GPTZero's heterogeneous biases (see Table \ref{tab:demo_means_c2}), we focus primarily on within-group changes. We note here that the mean probability of an essay being AI-generated within a corpus is exactly the expected proportion of AI-generated content within that corpus. Thus, we test for changes in mean $P(AI)$. As we have previously confirmed that GPTZero is suitable for these analyses, we use our calibrated GPTZero score going forward. Table \ref{tab:demo_means_c23} presents mean $P(AI)$ for 2023 (column 3) and 2022 (column 5), as well as test statistics of whether there is a difference in means (columns 6 and 7). 

\begin{table*}[htbp]
  \centering
  \caption{These t-tests comparing $\widehat{E(P(AI))}$ across the 2022 and 2023 application cycles for each demographic reveal several changes by demographic, but only a small overall increase in AI-generated content.}
  \label{tab:demo_means_c23}
  \begin{tabular}{p{0.2\textwidth} r r r r}
      \toprule
      & Cycle 2022 & Cycle 2023 & \multicolumn{2}{c}{Inter-Cycle $\Delta$} \\
      Demographic Group & $\widehat{E(P(AI))}$ & $\widehat{E(P(AI))}$ & Test Statistic & p Value \\
      \midrule
      Male    & $0.09$ & $0.11$ & $\mathbf{8.63}$ & $\mathbf{<0.01}$ \\
      Female  & $0.11$ & $0.11$ & $-0.40$ & $0.69$ \\
      Other   & $0.14$ & $0.12$ & $-1.10$ & $0.27$ \\
      \midrule
      Caribbean               & $0.17$ & $0.15$ & $-0.47$ & $0.64$ \\
      East and Southeast Asia     & $0.13$ & $0.13$ & $0.74$ & $0.46$ \\
      Core Anglosphere               & $0.21$ & $0.14$ & $\mathbf{-5.86}$ & $\mathbf{<0.01}$ \\
      Eastern Europe / Central Asia     & $0.13$ & $0.15$ & $0.59$ & $0.56$ \\
      South Asia     & $0.08$ & $0.09$ & $\mathbf{2.65}$ & $\mathbf{0.01}$ \\
      Latin America           & $0.13$ & $0.09$ & $-1.44$ & $0.15$ \\
      Middle East / North Africa   & $0.11$ & $0.12$ & $1.67$ & $0.09$ \\
      Sub-Saharan Africa      & $0.09$ & $0.09$ & $\mathbf{2.15}$ & $\mathbf{0.03}$ \\
      Western Europe            & $0.20$ & $0.12$ & $\mathbf{-2.75}$ & $\mathbf{0.01}$ \\
      Pacific Islands         & $0.04$ & N/A      &  &  \\
      \midrule
      All Submissions         &$0.10$ & $0.11$ &  &  \\
      \bottomrule
  \end{tabular}
\end{table*}

We find statistically significant increases in the $P(AI)$ for only two overlapping subgroups: male applicants and applicants from the Indian subcontinent. In both cases the magnitude of the increase is small, suggesting that at least in Cycle 2023, the use of genAI was limited. However, other findings preclude interpreting this change as a direct measure of increased genAI use. In two regions, the Core Anglosphere and Western Europe, we found a statistically significant decrease in the calibrated estimated probability that essays were completely AI-generated. Since we can assume very few if any applicants in Cycle 2022 had access to genAI, this cannot be interpreted as a decrease in AI use. It may be that both regions' high average scores in 2022 were a fluke of the cohort and that these regions reverted to the mean in 2023. Alternatively, it is possible that, in these regions, Cycle 2023 applicants used AI detection tools to ensure that their content would not be flagged by our detector (although this would require such detector usage to offset any actual genAI use) \cite{gptzero_gptzero_2023}. In either case, this analysis surfaces interesting discrepancies demanding further interrogation in future cycles but does not demand the programme alter its application material or process.

\begin{table}[htb]
    \centering
    \caption{This Tukey's Honestly Significant Difference (HSD) test looks for differences in the proportion of predicted AI-generated essays across different programme channel partners in the 2023 application cycle but finds no evidence of any channel partners' applications being more likely to be AI-generated than the average.}
    \label{tab:c3_partner_anova}
    \begin{tabular}{ c c c c c }
        \toprule
        \multicolumn{3}{c}{} & \multicolumn{2}{c}{HSD Results} \\
        \cmidrule(lr){4-5}
        Channel Partner & Essays & Share `AI-Generated' & Test Statistic & p Value \\
        \midrule
        Partner A & $540$ & $0.041$ & $0.064$ & $0.80$ \\
        Partner B & $390$ & $0.044$ & $0.004$ & $0.95$ \\
        Partner C & $325$ & $0.049$ & $0.320$ & $0.57$ \\
        Partner D & $315$ & $0.029$ & $1.599$ & $0.21$ \\
        Partner E & $235$ & $0.064$ & $2.526$ & $0.11$ \\
        Partner F & $205$ & $0.039$ & $0.076$ & $0.78$ \\
        Partner G & $170$ & $0.047$ & $0.071$ & $0.79$ \\
        Partner H & $150$ & $0.027$ & $0.970$ & $0.33$ \\
        Partner I & $150$ & $0.007$ & $\mathbf{4.829}$ & $\mathbf{0.03}$ \\
        Partner J & $135$ & $0.022$ & $1.415$ & $0.23$ \\% can suppress to save space
        \midrule
        All Submissions & $24,815$ & $0.040$ & \\
        \bottomrule
    \end{tabular}
\end{table}

\subsection{High-Stakes Ex-Post: Analysing GenAI Usage in the Programme's 2023 Channel (\emph{Partners})}
As a final analysis, we evaluate whether GPTZero detects any suspicious patterns in the essays associated with the programme's channel partners, who refer and support applicants to the programme. The organisation partners with `channel partner' organisations that encourage applications and these are attributed to channel partners based on the custom links that applicants use to reach the programme's website, as well as questions in the application about how applicants learned of the programme. Evidence that any of these channel partners used genAI to create large volumes of applications would warrant further investigation before continuing affected partnerships. 

For this analysis, we use a cutoff of $0.5$ on our calibrated predictor, meaning that essays flagged as `AI-Generated' are more likely than not to be so. This threshold has a TPR of $76\%$ and an FPR of $4\%$. We limited the analysis to the partners with the most essays and have hidden individual partners' identities.

As Table \ref{tab:c3_partner_anova} shows, only one of the channel partners studied was associated with essays identified as AI-generated at a rate meaningfully different from the overall pool. However, the partner in question, Partner I, was associated with fewer essays than average identified as AI-generated. This could be driven either by the chance of finding at least one statistically significant result when testing multiple hypotheses or by the partner, based in sub-Saharan Africa, working primarily with demographics identified in Cycle 2022 analyses as naturally having below-average calibrated scores. In either case, we find no evidence of widespread genAI use among the programme's channel partners, supporting the organisation's channel partnership approach and choice of partners.

\section{Discussion}\label{sec:genaidisc}
\subsection{Implications for the Programme}
The programme we work with has several use cases for genAI detection, and we have found that GPTZero is suitable for the ex-post use cases, both high- and low-stakes. However, the programme's use of genAI detection for in-process decisions is not yet feasible, as, although Originality.ai has very high accuracy, it suffers from a lack of interpretability and its FPRs exhibit large heterogeneous biases across demographics, especially regions. We recommend that the programme continue to use GPTZero for ex-post decisions, but that it seek out a new detector or design its own if it wishes to make data-driven, in-process decisions surrounding genAI usage. A summary of our recommendations can be found in Figure \ref{fig:satisfaction_matrix}.

\subsection{Implications for Other Programmes}
By identifying decision points, placing them on the Decision Matrix, aligning these desired and required properties, and then testing genAI detection tools for the relevant properties, programmes can determine the suitability of detectors in supporting and informing decision points. If a programme's decision points and desired properties align closely with Rise's, they may find that they consider the same use cases for GPTZero and Originality.ai, and have no need for our framework. But if the genAI detection landscape changes in response to new developments \cite{ashish_vaswani_attention_2017,jacob_devlin_bert_2018,openai_gpt-4_2023,liang_gpt_2023,mitchell_detectgpt_2023,liu_deid-gpt_2023,kalpesh_krishna_paraphrasing_2023}, or if programmes' priorities do not align with those of Rise, programmes should use the Decision Matrix framework to replicate the analysis done here.

\subsection{Implications for the Field}
Our results suggest that, while genAI detection tools address some issues faced by scholarship selectors, deeper involvement with these institutions is key in designing technology to meet the specific needs of these selectors. Currently, a narrow focus on academic integrity, plagiarism, and decisions to censure essay writers limits broader discussions about how genAI should be integrated into academic workflows. While selectors shift towards embracing genAI as a tool rather than a threat, the field of HCI is lagging behind. 

Our AR process raises the question: should it matter whether genAI was used at all? Rather than aiming to detect AI-generated plagiarism, selectors are perennially concerned with determining whether an applicant's submission indicates their aptitude for the programme; the need for detector-based decision support, in all cases, is to support decisions impacting that ultimate determination. GenAI tools pose problems to selectors' abilities to make that determination with contemporary essay-based assessment, but outright bans and disqualification \cite{h_holden_thorp_chatgpt_2023} unenforceable with current technology offer no solutions. A shift in assessment methodology, then, is a practical and desirable alternative.

HCI and genAI detection can enable this shift through respectful design \cite{VanKleek_Seymour_Binns_Shadbolt_2018} by working with selectors to understand their needs and support them (e.g., by improving assessment design or by building digital feedback mechanisms). Institutions, meanwhile, may wish to design essay prompts that encourage or even require applicants to use genAI in a meaningful way. In such cases, the role of detectors would shift from merely identifying AI-generated content to evaluating how well applicants have leveraged these technologies.

\subsection{Ethical Implications}
The Decision Matrix framework developed throughout this chapter reveals a variety of desiderata for the usage of genAI detectors in application essay settings. As can be seen in Figure \ref{fig:desiderata_matrix}, the decision to disqualify a candidate demands much of detectors. Indeed, even when detectors possess all of the desired properties, taking automated adverse action against applicants is ethically fraught \cite{Lashkari_Cheng_2023}. However, when opting to make no decisions about applicants using this technology, these demands fall away.

In practice, our research sits between these extremes. The process of holistic review sees organisations incorporate a mass of disparate information into an opaque decision-making process. \textcite{hirschman_dequantifying_2016} note this lack of transparency as a benefit of holistic review insofar as it shields institutions from regulators. But while this may offer some legal insulation to the holistic review process, no such moral insulation exists. In-process algorithmic decision support, even for low-stakes decisions such as \emph{Diligence}, still bears a high moral burden \cite{Lashkari_Cheng_2023}.

\section{Limitations and Future Work}
One participant highlighted: ``There are two problems here: Did this applicant use [generative AI]? And if so, is this essay based in fact?''. This points to a key limitation of our evaluation of detectors: while we could determine whether text was AI-generated, we had no basis for evaluating the truthfulness of AI-generated text. As genAI is known for its hallucination \cite{alkaissi_artificial_2023}, frequently yielding convincing falsehoods, this represents a significant limitation. While this distinction is ultimately irrelevant to those who consider a genAI usage plagiarism \cite{h_holden_thorp_chatgpt_2023}, decisions such as \emph{Diligence} would be well-informed by an understanding of how genAI was used. Future work should investigate detecting the nature of genAI usage (writing, editing, etc.) and then determining the truthfulness of genAI-written text.

As we select only two detectors, our results as applied to these detectors may not apply to others. Furthermore, as the field of genAI detection moves so rapidly, our results from Cycles 2022 and 2023 may not even apply to Cycle 2024. For this reason, we develop the Decision Matrix framework to support ongoing evaluations of detectors in response to new developments such as paraphrasing tools and hybrid human-genAI writing processes \cite{kalpesh_krishna_paraphrasing_2023}. Future work should use this framework in re-evaluating detectors in response to these new developments.

We deliberately do not engage applicants in our process. \textcite{venn-wycherley_realities_2024} argue that, when conducting human-centred research in a classroom setting, it is important to gather perspectives of both educators and students. Here, we conduct AR centred on scholarship selectors, but we omit the perspectives of their young decision subjects. Unlike in the classroom context, the evaluation context sees an adversarial relationship between the educational institution and its target population – while the scholarship seeks to select only the most well-fit candidates, each candidate seeks to be selected, and therefore to make themselves seem most fit. Thus, in making the selector ``Co-investigators of, co-participants in, and co-subjects of'' our research \cite{Hayes_2011}, we necessarily exclude the perspectives of their decision subjects. Future work should seek to engage these young decision subjects in this context and may explore concepts like the essayist's sense of authorship, the line between writing and editing, essayist thoughts on plagiarism, and applicant perceptions of programme decisions driven by AI.

\section{Conclusion}
In summary, this research examines the various decisions that may arise when scholarship selection organisations consider the problems posed by genAI in practice, emphasising the need for tools designed to support decisions besides simply disqualifying applicants. Our findings reveal that, although state-of-the-art detectors may be unsuitable as automated disqualification tools, they can be used as-is to support ``integrating technology, education, policy reform, and assessment restructuring'' \cite{MikePerkins_JasperRoe_2023} and support ex-post decisions organisations may wish to make. By engaging in action research, we catalogued real decisions scholarship selection organisations seek to make in response to the problems posed by genAI usage. We then worked with them to develop the Decision Matrix, which serves as a tool for selectors to evaluate genAI detectors on their data. As we move forward, we call for a broader view of the purpose of genAI detection, and for a restructuring of what it is to learn and assess in an era so heavily influenced by easy access to genAI tools. We also call for more research into the Decision Matrix, specifically examining the support of the ever-elusive in-process decision points; Chapter \ref{ch:diversity} seeks to support one such in-process decision point.
\chapter[``Diversity is Having the Diversity'']{\label{ch:diversity}``Diversity is Having the Diversity'': Unpacking and Designing for Diversity in Applicant Selection\footnote{This chapter is based on a paper written in concert with Sruthi Viswanathan, Reuben Binns, and Nigel Shadbolt. The paper is currently under review as: Neil Natarajan, Sruthi Viswanathan, Reuben Binns, and Nigel Shadbolt. 2024. ```Diversity is Having the Diversity': Unpacking and Designing for Diversity in Applicant Selection.'' Under review at CHI 2025.}}

\minitoc

\section{Motivation}
Chapters \ref{ch:xai} and \ref{ch:genai} both reveal flaws of existing algorithmic support tools in their applications to decision points facing scholarship selection organisations: these tools lack the properties required to support in-process decision-making. In this chapter, we select a family of in-process decision points of particular interest to the literature: ensuring diversity in selection. This chapter, thus, seeks to understand what diversity is and how to support its consideration.
  
\section{Introduction}\label{sec:divintro}
Processes for selecting people for jobs, universities, prizes, and other opportunities have often failed to reflect the diversity of their actual and potential candidate pool. Recognising this, various sectors have in recent years shifted towards recognising and promoting diversity through the establishment of a variety of related norms: DEI (Diversity, Equity, and Inclusion), EDI (Equality, Diversity and Inclusion), JEDI (Justice, Equality, Diversity, and Inclusion), DEIB (Diversity, Equity, Inclusion, and Belonging), etc. \cite{pinkett2023data,hsieh2019allocation,minkin2023diversity}. These norms are frequently operationalised through changes to application, evaluation, and decision-making procedures designed to result in greater representation of different demographic groups in final selection decisions \cite{pinkett2023data}. Such efforts are in part a response to widespread societal concerns about racial, gender, and other injustices, but have also frequently been justified in economic terms by evidence suggesting that diverse teams perform better than homogeneous ones on a variety of tasks \cite{deming2017growing,page_diversity_2017,noray2023systemic}. Meanwhile, the concept of diversity itself has become swept up in `culture war' discourse, criticised by right-wing commentators as part of a sinister `woke' agenda, and by progressives as mere window dressing that fails to meaningfully address deeper societal injustices. 

However, for practitioners involved in selection processes on the ground, the question of how to meaningfully measure and promote diversity in their decisions is a real and challenging one. The proliferation of software to help with hiring and talent selection – in the form of application management platforms, decision-support tools, and more recently, widespread use of AI – provides additional complications, as well as potential opportunities \cite{Lashkari_Cheng_2023}. For instance, AI-driven tools may discriminate against candidates who don't resemble their training data \cite{chen2018investigating,li2020hiring,lambrecht2019algorithmic}; but they could potentially assist in mitigating human biases in recruitment and selection processes, resulting in greater diversity \cite{yarger2020algorithmic,avery2024does,will2023people, suhr2021does}. This context thus challenges Human-Computer Interaction (HCI) and Human-Centred Artificial Intelligence (HCAI) research to improve diversity in selection processes without amplifying or codifying pre-existing inequities.

\begin{figure}
    \includegraphics[width=\textwidth]{diversity/diversity_is.png}
    \caption{This figure shows participant codes defining what ``diversity is''. In this chapter, we seek to answer: what do they mean and how do we design for that?}
    \label{fig:diversity_is_teaser}
\end{figure}

But to improve diversity in selection processes, we must first understand it. We undertake two studies to do just that. In Study 1, we conducted 15 one-to-one interviews with scholarship and talent investment selection practitioners (selectors). All participants were involved in scholarship selection for an international cohort of students in a global academic programme. We aimed to understand how these selectors define and operationalise diversity and how technology can aid this process. Each interview lasted 45 minutes. We analysed this data following \textcite{braun_using_2006}'s inductive thematic analysis methodology. Our analysis highlighted the ad-hoc nature of current diversity considerations and the need for more structured, data-supported approaches, and surfaced three distinct definitions of diversity: one involves placing people with `different perspectives' in the same room; another, ensuring `representativeness' of some target population; a third, `contextualising applications', e.g., with information such as the applicants' relative privilege or required level of support. We conclude that technological interventions designed to promote diversity in selection processes should first identify specific definition(s) of diversity they aim to promote; we construct the \emph{Diversity Triangle} (Figure \ref{fig:div_triangle}) as a guide.

For Study 2, we developed six design prototypes for diversity-supporting tools based on the Diversity Triangle from Study 1 \cite{Buchenau_Suri_2000}. The prototypes included tools for visualising cohort representativeness, measuring entropy (average number of in-group differences), and assessing individual applicant (dis)advantage scores. They were presented to participants via participatory design workshops \cite{Zimmerman_Forlizzi_2017}. Participants provided feedback on the utility and integration of these tools into their selection processes. The workshops revealed that participants use specific, idiosyncratic lenses to navigate their diversity considerations and that while quantitative tools are essential for making informed decisions, qualitative assessments remain crucial.

This research contributes to the understanding of how diversity can be supported through data-driven tools in selection processes. Specifically, our contributions are:

\begin{enumerate}
    \item Three definitions of diversity that impact selectors' decisions uncovered through inductive thematic analysis.
    \item The Diversity Triangle, categorising diversity-related themes according to our definitions of diversity.
    \item Design recommendations grounded in participatory design for system implementers supporting the diversity needs of a given organisation.
\end{enumerate}

\noindent More broadly, this work demonstrates that by providing structured, data-supported approaches to diversity, organisations can better navigate the complexities of DEI (EDI, JEDI, DEIB, etc.). While differing in some respects, we believe these implications will generalise from scholarship selection to various other talent identification contexts, including recruitment for jobs and admission to universities. By helping these organisations achieve their desired outcomes in selection processes, we aim to ultimately contribute to and help build a more diverse society.

% [FIXED] ⭐️ 6.3.1/2 – Race after Technology – This section could benefit from additional literatures on historical injustice carried into power structures and technological artefacts – as well as embedded in western science. See Ruha Benjamin's book "Race After Technology",  "Algorithms of Oppression" (S. Noble), and "Weapons of Math Destruction" (O'Neill) as well as  "Do Artefacts Have Politics?" (L. Winner).

\section{Background}\label{sec:back}
\subsection{Diversity as a societal and organisational value}\label{ssec:value}
Despite its global reach, contemporary discourse on diversity derives (in large part) from the political context of the United States in the latter half of the 20th century \cite{nkomo2019diversity}. Civil rights activists identified gender, race, disability, and other forms of group identity as loci of discrimination and oppression, and constructed political actions around these identities \cite{morris1984origins}. This yielded civil rights laws, including equal treatment laws to protect applicants from discrimination (e.g., in employment). \textcite{nkomo2019diversity} argue that this initial, U.S.-centric perspective on anti-discrimination in the workplace, which focused on the under-representation of women and racial minorities, has evolved into a more global concept of diversity encompassing a variety of identities. With an increase in social pressure for representation, organisations increasingly prioritise diversity in their selection procedures \cite{hsieh2019allocation,minkin2023diversity}.

However, this evolution has not fully addressed the historical injustices embedded in technological systems and power structures. As \textcite{benjamin2019race} argues, technological systems often encode and perpetuate racial hierarchies, with ostensibly neutral technologies serving to reinforce existing power structures. This is particularly evident in algorithmic systems, where \textcite{noble2018algorithms} demonstrates how search engines and other technologies can reproduce and amplify racial and gender biases. Similarly, \textcite{oneill2016weapons} shows how mathematical models and algorithms can systematically disadvantage already marginalized groups, while \textcite{winner1980artefacts} reminds us that technological artefacts themselves can embody political values and power relations. These insights suggest that achieving true diversity requires not just equal treatment under the law, but also critical examination of how technological systems and power structures perpetuate historical injustices.

However social and political pressures do not uniformly push for diversity. Critics on all sides challenge the value of diversity, both to organisations and society. \textcite{Ahmed_2012} argues that organisational prioritisation of diversity often limits their appetite to prioritise more meaningful changes. The attention paid to diversity may encourage organisations to merely document social injustice, rather than do something to change it \cite{Ahmed_2012,Rossi2020-ROSWNA-2}. Worse, \textcite{Warikoo_2019} argues that diversity among elite institutions reinforces social injustices. On the other side, critics such as \textcite{Goodhart} position diversity as opposed to the obligations of a good society, while others position this as less a rejection of `diversity' than a rejection of `bad diversity' \cite{lentin_Multiculturalism_2011}.

However, political struggles and social pressure for greater equality are not the only motivations for pursuing diversity. More recent discourse and research also emphasise the instrumental benefits that greater diversity can bring to workplaces and institutions. While it is sometimes assumed that diversity could reduce the quality of a workforce or cohort by favouring more diverse but `worse' candidates, research on selection processes suggests many organisations can both improve the diversity of their organisations and select `better' candidates (as measured by their metrics) \cite{autor2008does,noray2023systemic}. Generally, research on the performance of diverse teams yields mixed results \cite{daubner2017dovetailing,page_diversity_2017,noray2023systemic,muller_learning_2019}.

Other arguments in favour of diversity emphasise its benefits in the context of knowledge production. For instance, a long strand of work in the philosophy of science and social epistemology has made the case for diversity as key to success in epistemic communities \cite{mill1998liberty,merton1942note,wylie2006introduction}. While diversity in this context often refers to cognitive diversity rather than demographic diversity \cite{page2008difference}, the latter may often precede the former, as: ``membership in different social groups (e.g., gender or race) often comes with different task-relevant information, perspectives, or experiences'' \cite{peters2021hidden}. For instance, in biomedical science, the inclusion of scientists from more diverse backgrounds could lead to ``novel findings and treatment of diverse populations'' \cite{swartz2019science}.

More generally, ``the more robust a community's mechanisms for bringing diverse perspectives to bear on epistemic questions of public import, the more effective it is in generating and ensuring responsiveness to all the information and insights held by its members'' \cite{wylie2006introduction}. A related argument is that marginalised groups may occupy epistemically advantageous standpoints when it comes to understanding unjust power structures \cite{harding2004feminist,dror2023there,steel_multiple_2018}, as they may have more informative experiences of oppression \cite{mills2015blackness}, and incentives to learn about it \cite{jaggar1983feminist}.

\subsection{Defining and Measuring Diversity}
With the wealth of disparate motivations for diversity, it is unsurprising that its definition is often unclear. \textcite{page_diversity_2010} offers a helpful generic definition of diversity: ``The heterogeneity of elements in a set about a class that takes different values, such as species in an eco-environment, or ethnicity in a population''. While suitably broad, this definition lacks the specificity required to build supporting technologies \cite{hupont2021diverse,page_diversity_2010}. 

% [FIXED] Biology and ecology – more details here – are the challenges with biology/ecology transferrable to characterising 'diversity' of human society? I feel that this section currently does not get to the core that reducing any characteristic designed to capture the "diversity" of humans in a society is extremely difficult to do, and particularly so to quantify precisely.

The variety of approaches to measuring diversity reflects the ambiguity of its definition. For instance, the natural way to measure diversity, on \textcite{page_diversity_2010}'s definition, is to report percentages of those different elements, e.g., demographics in a population. But in scientific fields like biology and ecology, diversity is often measured with different methods and occasionally conflicting results \cite{xu2020diversity}. Some use measures from information theory (including entropy measured through the Shannon index \cite{shannon1948mathematical}); others adopt the Herfindahl-Hirschman Index \cite{rhoades1993herfindahl}, commonly used to measure economic market concentration \cite{budescu2012measure,acuna2021ai,shannon1948mathematical,rhoades1993herfindahl}.

However, measuring diversity in human societies presents unique challenges that go beyond those faced in non-human contexts. Human diversity encompasses complex, socially constructed categories that resist simple quantification. As \textcite{scheuerman2019computers} notes, attempts to reduce human diversity to measurable attributes often fail to capture the nuanced, intersectional nature of human identity. This challenge is compounded by the fact that many aspects of human diversity are not merely descriptive but carry significant social, political, and historical meaning \cite{abdu2023empirical}. For instance, while a biologist might measure species diversity by counting distinct genetic markers, attempting to measure racial diversity requires navigating contested categories that vary across cultures and historical contexts \cite{scheuerman2021auto}. This complexity makes it particularly difficult to develop precise, universally applicable metrics for human diversity, as any such attempt risks oversimplifying the rich tapestry of human experience and potentially reinforcing harmful social categorizations.

% [FIXED] Diversity vs fairness – be more explicit on the connection here; these are very distinct concepts by most definitions, and so it's important to clarify why one might think these might "collapse on each other" – perhaps it would be more constructive to discuss the relevance of achieving diversity in outcomes to the goals of work to improve fairness.

Another complication arises when we consider the relationship between diversity and similar notions from AI ethics like fairness. While these concepts are distinct in their traditional definitions, they can become intertwined in practice, particularly in the context of algorithmic decision-making. \textcite{zhao2023fairness} argue that ``fairness works can be re-interpreted through the lens of diversity, and strategies enhancing diversity have proven efficacious in improving fairness''. This connection emerges most clearly when we consider distributive justice frameworks that emphasize equality of outcome across protected characteristics. In such frameworks, achieving fairness may require ensuring diverse representation in outcomes, effectively making diversity a necessary condition for fairness.

This relationship becomes particularly relevant in algorithmic contexts. The algorithmic fairness literature often operationalizes fairness through constraints that equalize performance (i.e., positive predictions) between different demographic groups \cite{barocas2023fairness}. While these approaches were developed to address fairness concerns, they can also serve as mechanisms for promoting diversity in outcomes. This dual purpose highlights how fairness and diversity can become mutually reinforcing goals in practice, even though they remain conceptually distinct.

It's worth noting that while many of these measurements were originally defined for non-human notions of `diversity' (e.g., diversity of items in a recommender system or diversity of flora in an ecosystem), HCI and related research communities have adapted them for human contexts. These adaptations are evident in measuring the diversity of research participants, AI training data records, or author backgrounds in published papers \cite{linxen2021weird,himmelsbach2019we,zhao2024position,rojas2022dollar,acuna2021ai}. This application of diversity metrics to human contexts further demonstrates how diversity considerations can serve both fairness and representation goals simultaneously.

% [FIXED] Last paragraph – individual respect – Another danger: reducing people to finite sets of attributes – or indeed characterising a group as "X diverse" essentialises the individuals within that group to be "no more than the sum of their attributes" – check citations...

More practical criticisms of measurement and implementation of notions of diversity often point to this confusion. \textcite{steel_multiple_2018} argue that disparate definitions of diversity: ``Can generate unclarity about the meaning of diversity, lead to problematic inferences from empirical research, and obscure complex ethical-epistemic questions about how to define diversity in specific cases''. Similarly, \textcite{abdu2023empirical} caution that the specifics of measurement are, at least in the context of race: ``a value-laden process with significant social and political consequences''. And, if done poorly, this measurement process may exclude already marginalised groups (e.g., gender-fluid or mixed-race individuals) \cite{scheuerman2019computers}. While poor categorisation is harmful in any context, larger institutions must be doubly cautious, as, in using constructs to measure diversity, they may inadvertently reify them as natural rather than social \cite{scheuerman2021auto}. 

A deeper concern lies in the reduction of individuals to their impacts on diversity metrics. When we characterize a group as `X diverse' and make decisions based on that, we risk treating individuals as no more than the sum of their attributes. This reductionist approach holds twin dangers: we may not only fail to capture the complexity of human identity, but can also reify social constructs as natural rather than social \cite{steel_multiple_2018,scheuerman2021auto}. This concern is particularly acute in selection contexts, where the pressure to achieve measurable diversity outcomes can lead to what \textcite{scheuerman2019computers} terms "categorical violence" - the harm done when complex human experiences are forced into rigid classification systems. The challenge is compounded by the fact that many diversity metrics were originally developed for non-human contexts (like ecological diversity) and may not adequately capture the nuanced nature of human identity \cite{page_diversity_2010}.

As \textcite{scheuerman2019computers} notes, this reduction can lead to the flattening of complex human experiences into simplified categories. This is particularly problematic in selection contexts, where the richness of individual experience and potential may be lost in the pursuit of measurable diversity metrics. The challenge, then, is to develop approaches to diversity that respect individual complexity while still enabling meaningful measurement and action. This tension between measurement and respect for individual complexity is not unique to diversity metrics; \textcite{benjamin2019race} argues that similar issues arise in many technological systems that attempt to categorize human identity. The solution, according to \textcite{scheuerman2021auto}, may lie in developing more fluid, contextual approaches to diversity that acknowledge the socially constructed nature of identity categories while still enabling meaningful action.

\textcite{abdu2023empirical} caution that the specifics of measurement are, at least in the context of race: ``a value-laden process with significant social and political consequences''; in neglecting this, organisations may inadvertently perpetuate harmful stereotypes and reinforce existing power structures. This warning extends beyond race to other aspects of identity. \textcite{noble2018algorithms} demonstrates how seemingly neutral classification systems can encode and perpetuate existing power structures, while \textcite{oneill2016weapons} shows how mathematical models can systematically disadvantage already marginalized groups. The challenge is particularly acute in selection contexts, where the stakes are high and the pressure to achieve measurable diversity outcomes can lead to what \textcite{scheuerman2019computers} terms "categorical violence" - the harm done when complex human experiences are forced into rigid classification systems. As \textcite{winner1980artefacts} reminds us, technological systems themselves can embody political values and power relations, making it crucial to critically examine how diversity metrics are implemented and used.

\subsection{AI and Decision Support Systems for Hiring and Talent Selection}
While there is good reason to express scepticism towards predictive technology in recruitment, \textcite{Vereschak_Alizadeh_Bailly_Caramiaux_2024} demonstrate that, as long as AI systems prove trustworthy and well-designed, decision-makers will engage with and rely on these systems. Apprehension with these systems ranges from concerns about bias to disquiet about the distance between applicant and organisation \cite{Lashkari_Cheng_2023}. While concerns about biased AI systems can also be directed at human reviewers, the distance between applicant and organisation created by the introduction of automated decision support tools must be more carefully managed \cite{Leung_Zhang_Jibuti_Zhao_Klein_Pierce_Robert_Zhu_2020,Lashkari_Cheng_2023}. Research on decision subjects shows openness to AI involvement in hiring pipelines \cite{horodyski_applicants_2023}. Ultimately, flaws in present pipelines point to the need for better systems, and, despite a well-earned weariness of technologies developed for this sensitive field, AI and decision-support technologies may have a role to play \cite{kleinberg2018algorithmic,Vereschak_Alizadeh_Bailly_Caramiaux_2024,barocas2023fairness,huppenkothen2020entrofy,schumann2017diverse}.

New research explores a framing of individual applicant aptitude and overall group diversity \cite{noray2023systemic}, and many applications demonstrate that technology might improve both. For example, \textcite{bergman2021seven} show that replacing traditional testing mechanisms with prediction algorithms allows placement of students into remedial classes that improves student performance and increases minority representation in non-remedial courses. Similarly, \textcite{autor2008does} show that screening job applicants with personality tests increases worker productivity without reducing minority representation. Qualitative studies of HR practitioners' use of AI and decision-support systems have found that diversity is among the perceived benefits of such systems; \textcite{li2021algorithmic} find that recruiters and HR practitioners who used AI-enabled hiring software for sourcing and assessment reported higher diversity in their candidate pools.  \textcite{huppenkothen2020entrofy,kleinberg2018algorithmic,schumann2017diverse} demonstrate algorithms directly comparing applicant aptitude to group diversity, and all find improvements on both axes over traditional selection procedures. However, despite these widespread improvements, these technologies are rarely used in practice \cite{page_diversity_2017}, as the processes assumed by these technologies rarely align with those of selection organisations. More work is needed that begins with selectors' processes and diversity needs, and then constructs data-based support systems from there.

\section{Experimental Design}\label{sec:methods}
\subsection{Our Studies}
Our research is broken into two components: 15 one-to-one interviews with scholarship and talent investment selectors and two participatory design workshops with a subset of the 15 selectors. The relationship between these components is shown in Figure \ref{fig:flowchart} and details on Studies 1 and 2 can be found in Sections \ref{ssec:methods1} and \ref{ssec:methods2}, respectively.

\begin{figure}[htbp]
    \centering
    \includegraphics[width=.9\textwidth]{diversity/flowchart.png}
    \caption{This research begins with 15 interviews seeking to understand what selectors mean when they talk about diversity and how to support that, followed by two scenario-speed-dating activities where these selectors test several prototypes built based on the interviews.}
    \label{fig:flowchart}
\end{figure}

In the first session, we seek to ascertain how these selectors understand diversity, how they operationalise it in processes for selecting talented applicants, and how they envision using technology to assist them in that process. In the second session, we show these selectors six prototypes built in response to the interviews, then we aim to collaboratively design tools that can help them better consider diversity in selection.

\subsection{Positionality}
Following \textcite{venn-wycherley_realities_2024}, we state researcher positionality here. All authors endorse diversity as a societal and organisational value as discussed in Section \ref{ssec:value} (while sympathetic to the critiques of \textcite{Ahmed_2012,Warikoo_2019}); we thus contend that improving organisational capability to consider diversity is generally a positive development. The research team is comprised of three men and one woman; ethnically, two researchers are South Asian, while two are White; researchers represent three different primary nationalities; all researchers are affiliated with the University of Oxford.

\subsection{Participants}
We engaged selectors (N=15) from two scholarship and talent investment programmes. These participants are numbered P1-P15.

Before either study, we obtained informed consent from all participants to be included in both studies. All participants were given the option to recuse themselves from Study 1 at any point until publication, but, as Study 2 is a group workshop (and as we do not do participant-level attribution), participants were asked to recuse themselves before this second study. Two participants recused themselves before Study 2 but gave leave to be included in Study 1. Participants also gave consent to be recorded, and to have these recordings stored on a secure server. All recording, transcribing, and data analysis was conducted on secure servers. Ethics review was performed by the University of Oxford's Central University Research Ethics Committee.

\section[Study 1]{Study 1: Interviews and Thematic Analysis}\label{sec:study1}
\subsection{Methodology}\label{ssec:methods1}
Our interviews aim to answer three questions in each organisational context:

\begin{enumerate}
    \item What is diversity?
    \item Which elements of diversity matter in a selection context? Why?
    \item How could technology assist in operationalising diversity?
\end{enumerate}

In answering our first set of research questions, we follow \textcite{braun_using_2006}'s methodology for reflexive thematic analysis. We first conduct 45-minute semi-structured interviews with 15 selectors. In each interview, we first ask general questions about their selection methodology; we next ask specifically about diversity and its role in selection; we move on to \textcite{Knapp_Zeratzky_Kowitz_2016}'s `crazy 8s' exercise, where participants give eight feature requests in eight minutes; we conclude with a `magic app' exercise inspired by \textcite{blythe2014research}'s design fiction, where participants more thoroughly detail their ideal app. A question-by-question protocol for these interviews is supplied in Appendix \ref{app:divprotocol1}.

The lead author interviewed participants and then transcribed and anonymised the interviews. The lead author and another author (who wasn't present during the interviews) then independently `open-coded' each anonymised transcript to mitigate bias, looking for anything relevant to our research questions. The researchers met six times to discuss their open codes, then shared these codes with the remainder of their research team across four meetings; the researchers grouped codes into 6 themes and 18 subthemes after consensus was reached. These themes are detailed in Table \ref{tab:themes} and described in Section \ref{ssec:themes}.

\begin{table}[htbp]
    \centering
    \caption{Our three central subthemes all speak to the question ``Why diversity?''. Other themes and subthemes reflect types of diversity, concepts intertwined with diversity, and other considerations scholarship programmes must weigh against diversity desires.}
    \label{tab:themes}
    \begin{tabular}{|p{0.45\textwidth}|p{0.45\textwidth}|}
        \hline
        \multicolumn{2}{|c|}{\textbf{Themes and Subthemes}} \\
        \hline
        \textbf{Why Diversity?} & \textbf{Types of Diversity} \\
        \underline{Different perspectives} & \underline{Socioeconomic} \\
        \emph{...in the same room} & \emph{parental income} \\
        \underline{Representativeness} & \emph{parental education} \\
        \emph{...of a general population} & \emph{generational wealth} \\
        \emph{...of the eligible population} & \underline{Sex, gender, and sexuality} \\
        \emph{...of the applicant population} & \emph{sex} \\
        \emph{...of a target population} & \emph{gender identity} \\
        \underline{Contextualising applications} & \emph{sexual orientation} \\
         & \underline{Geography} \\
         & \emph{nationality} \\
         & \emph{the `Global South'} \\
         & \emph{region} \\
         & \underline{Race} \\
         & \emph{international categorisations of race} \\
         & \underline{Types of thinking} \\
         & \emph{subject area interest} \\
         & \emph{personality type} \\
         & \emph{core beliefs} \\
         & \emph{problem solving approaches} \\
         & \emph{political views} \\
        \hline
        \textbf{Operational Risks and Considerations} & \textbf{Fairness and Bias} \\
        \underline{Outreach} & \underline{Fairness} \\
        \underline{Support} & \emph{...to the applicants} \\
        \emph{...during the application process} & \emph{...to the world} \\
        \emph{...after selection} & \underline{Bias} \\
        \underline{Selectors} & \emph{measurement bias} \\
        \underline{Applicant fraud} & \emph{decision-makers' bias (prejudicial)} \\
         & \emph{decision-makers' unique perspective (probative)} \\
        \hline
        \textbf{Scholarship Goals} & \textbf{Merit} \\
        \underline{Impact} & \underline{Performance relative to disadvantage} \\
        \emph{...on all applicants} & \underline{Measurement} \\
        \emph{...on the selected scholars' performance} & \underline{Performance} \\
        \emph{...on the selected scholars' opportunities} & \\
        \emph{...by the selected scholars on the world} & \\
        \hline
    \end{tabular}
\end{table}

\subsection{Themes}\label{ssec:themes}
\subsubsection{Why Diversity?}
A central theme of our investigation revolved around the question of why diversity matters. To this end, we asked questions such as: ``What is diversity?'' and ``Why does diversity matter?''. Answers to: ``What is diversity?'' are visualised in Figure \ref{fig:diversity_is_teaser}; as hypothesised, these answers are vague and uninformative. Interestingly, though, answers to: ``Why does diversity matter?'' informed more specific definitions. We were able to cluster these more specific definitions into three central subthemes: `different perspectives', `representativeness', and `contextualising applications'. These are listed as subthemes of `Why Diversity' in Table \ref{tab:themes}.

\paragraph{Different Perspectives}
8 out of 15 participants mentioned that diversity was important because it brought different perspectives into the same room. This was seen as important for a few related reasons, e.g., the ability to see problems from different angles and the ability to make better decisions. Several participants referred to the: ``Benefits of diverse perspectives'' (P1). One said, when discussing their personal experience working with winners in a talent investment programme, that there is: ``Magic happening with lots of...diverse perspectives in the room'' (P14). People's experiences were particularly relevant here. As one participant writes: ``You want to have diverse perspectives from people who look different with different experiences'' (P2).

\paragraph{Representativeness}
9 out of 15 participants spoke of `representativeness'. This, we observed, was often spoken of in relationship to a larger population. Most frequently, participants spoke of the importance of having a cohort that was representative of the `eligible population'. I.e., one participant said: ``[You want] a community which is representative of where you are selecting young people from'' (P6). Others spoke of this in broader, more general terms: ``[You want] as broad a range of people as possible'' (P15). Participants identified the importance of building a cohort that variously: ``Reflects the population of the countries'' (P15) and is ``More representative of the national population than the STEM field already is'' (P10). Others talk about the representation of a particular target population, i.e.: ``Representation...because that gives you insight for the people that you're trying to serve....you have to be....reflective of your market'' (P9). Finally, selectors discussed the importance of representing an applicant population: ``[We want] a cohort that is representative of the pool'' (P10).

\paragraph{Contextualising Applications}
`Contextualising applications' was often spoken of by 6 out of 15 participants, most often in individual terms, speaking of identifying particular applicants in need of support, and then offering them a `boost' in the form of said support. I.e.: ``Identify those talents and specifically boost up people who are in need of support'' (P5). One participant identified `boosting' as a key metric for the programme: ``We need to know that...we have some level of...impact here, and...if all we're doing is supporting someone who is already on an amazing trajectory and then maybe that means we're not altering their trajectory at all. That's a question of efficiency of our dollars'' (P8).

In some cases, the need for support or boosting was identified with underrepresented or disadvantaged demographic groups. One participant said: ``The focus on gender has been to give the sex that has had the least opportunity the opportunity in this programme'' (P9).

\subsubsection{Types of Diversity}
Another central focus of our investigation was on different types of diversity. We asked participants to break down their understanding of diversity into different elements and to discuss why these elements were important. We found that participants identified a wide range of different types of diversity, which we clustered into subthemes. These subthemes are listed as subthemes of `Types of Diversity' in Table \ref{tab:themes}.

Notably, in addition to the standard demographic categories commonly considered `demographic diversity', participants identified a wide range of other types of diversity commonly termed `cognitive diversity' \cite{page_diversity_2010}. These included `subject areas of interest', `personality type', `core beliefs', `problem-solving approaches', and `political views'.

\paragraph{Socioeconomic}
All 15 participants identified socioeconomic diversity as particularly important in the context of a talent investment programme. One participant said: ``Socioeconomic [diversity] is probably the most important'' (P1). Another said: ``Socioeconomic background is number one from my perspective'' (P5). This was identified as particularly important for several reasons. Participants stated: ``Because right now the SAT, for example, is more highly correlated to socioeconomic status than it is to anything else'' (P5), and: ``It's a scholarship scheme, so I think it should be for kids who cannot afford normally the fees at the university'' (P7).

Outside of the standard categorisations by income and wealth, participants also identified: ``Familial education level'' (P5) or ``Socioeconomic backgrounds'' (P10) as a particularly important metric for understanding socioeconomic diversity in a scholarship context. This suggests that historical socioeconomic status is considered alongside present socioeconomic status. One participant noted that socioeconomic status varied in both meaning and measurement from country to country: ``For example, in Columbia, there's a whole society to organise on a 1 to 7 scale for socioeconomic status'' (P5).\footnote{Columbia's policy of socioeconomic stratification divides households into 6 strata; unlike traditional measurements such as income or quality of life, these strata only consider household location and accommodation and thus capture a different facet of socioeconomic status \cite{CHICAOLMO2020102560}.}

\paragraph{Sex, gender, and sexuality}
While all 15 participants noted some manner of sex, gender, and sexuality diversity as important, participants disagreed on the relative emphasis that should be placed on each. One participant noted: ``[Sex] is important. I think it will get diluted if we focus on identity gender because....the purpose of diversity on the gender aspect was to make sure that [men and women] were getting equal opportunities'' (P9). Another noted that, while sexual identification diversity was important in other contexts, they ``Wouldn't select for that'' (P1) in this context. Others listed `sexual orientation' and `gender' as important metrics for understanding diversity in a scholarship context. However, save for the participant who noted the distinction between sex and identity gender, participants expressed reluctance to discuss the relationships between these difficult concepts.

\paragraph{Geography}
14 of the 15 participants mentioned the importance of geography, citing a need for: ``[A] wide array of different geographical...representations'' (P8). Others spoke of a ``Regional distribution'' (P1), which we have included here.

In particular, emphasis was placed on geographic markers of socioeconomic status such as the `Global South', `indigenous communities', and `low-income countries'. One participant noted: ``Immigration status is tied so closely to socioeconomic status'' (P2), while another noted that ``[Geography] is connected to socioeconomics because we know there are some poorer countries and rich countries'' (P7). Others still asked questions like: ``Do they have a passport?.... Are they in a refugee camp?'' (P5).

Furthermore, participants saw it as important that their programmes had: ``Global reach'' (P7). They expressed a desire for: ``[A] diversity of people coming from a variety of places'' (P7).

\paragraph{Race}
While 11 out of 15 participants identified race as an important dimension of diversity, none suggested they would explicitly select racial diversity. Several participants instead noted the difficulty of measuring race in a global context: ``Racial categories obviously vary by country'' (P10). One participant noted: ``[In places] like Brazil or England, there are different categories of race than there are in the US...[In Brazil] there's a board of people who decide what people's race are'' (P5). In a global context, however, many participants pointed to relationships between geography and race and hence suggested diversifying across geography in place of race: ``If it's an international programme then you can use geography as a proxy'' (P5).

\paragraph{Types of Thinking}
4 out of 15 participants discussed a diversity in the types of thinking exhibited by applicants. One participant noted that: ``You want as much representation from different types of thinking as you can, because I want perspectives to be listened to equally'' (P2). This manifested in many ways.

Participants tended to express the belief that personality type diversity could improve group cohesion: ``With that understanding [of] personality types...be able to tell which...people would get on well with each other'' (P14). One participant suggested a ``Personality test'' (P12), and another specifically mentioned a desire to diversify across ``Openness'' (P2).

However, while personality type was seen as important, core beliefs were seen as even more so. One participant noted an interest in the diversity of ``Interests politically'' (P12), and expressed a desire for diversity of ``People's core beliefs...separate from religion'' (P12). Another also noted: ``I would try to have a good representation of...religious groups'' (P3).

\subsubsection{Operational Risks and Considerations}
While our study did not focus on the operational aspects of selection, several selectors' understanding of diversity was closely tied to the operational realities of selecting for and running a scholarship. In answering our questions, several participants identified operational risks or considerations that impacted their understanding of diversity. These are listed as subthemes of `Operational Risks and Considerations' in Table \ref{tab:themes}.

\paragraph{Outreach}
While our study was focused primarily on selecting a diverse cohort from a fixed applicant pool, 6 out of 15 participants answered questions from the perspective of outreach to grow a more diverse pool of applicants to select from. In particular, participants suggested that ``Using technology for....targeted outreach'' (P4) could help improve overall cohort diversity before selection even begins. One said: ``Giving you very clear signposting on where you may want to focus, you know further recruitment or outreach or whatever it might be to make sure that your programme is diverse at the end of the day'' (P6). Another added: ``You can target your outreach dollars to communities where you know that underrepresented talent exists'' (P10).

\paragraph{Support}
Similarly, 8 out of 15 participants suggested that technology could enable the support of applicants from underrepresented groups, which would also improve diversity. One participant suggested that technology could be used to provide: ``[Support] to keep people that you're attracting from underrepresented backgrounds and help them get across the finish line'' (P10). Participants focused on the: ``Support needed to actually get [applicants] through your programme'' (P10), i.e., supporting applicants after acceptance. Another suggested that technology could be used to provide support to applicants ``After selection'' (P15).

\paragraph{Selectors}
4 out of 15 participants noted that diversity did not apply merely to applicants. Instead, for programmes where a group of selectors assists in the selection process, ``Tracking the diversity of the selectors'' (P15) and ``[Monitoring] how they're scoring and reviewing applicants [for] prejudice or biases'' (P15).

\paragraph{Applicant fraud}
Finally, only 2 out of 15 participants expressed concern with selecting based on particular diversity characteristics, especially self-reported metrics of diversity characteristics, was the potential for applicant fraud, e.g. falsely reporting demographic or other attributes to increase their chances of acceptance. One participant requested: ``A fraud detector'' (P5), while another expressed a desire to ensure that the process ``Isn't super gameable'' (P10).

\subsubsection{Fairness and Bias}
Though not a type of diversity as we have understood it here, many participants referenced similarities between diversity metrics and metrics of fairness or bias (as in \textcite{zhao2023fairness}). Furthermore, several suggested that improving fairness while reducing bias would likely yield a more diverse cohort. These themes are reflected under `Fairness and Bias' in Table \ref{tab:themes}.

\paragraph{Fairness}
6 out of 15 participants discussed that it was important that applicants: ``Get fair chance on their on their academic merit'' (P7). This translated to an emphasis on ``Fairness in the assessment'' (P7).

However, participants also noted that ``The way the world works is unfair'' (P14), and found it important that the programme is: ``Making sure that the world is fairer by bringing more diversity to this world'' (P7). In this way, participants found: ``[The] representative thing....goes back to fairness'' (P15). One participant noted that: ``Affirmative action....can come across as unfair to some people, but....it's trying to balance things out when things have been so unequal for so long'' (P3). This supports \textcite{zhao2023fairness}'s positioning of fairness as related to, rather than in opposition to, diversity. 

\paragraph{Bias}
11 out of 15 participants discussed bias, often as both a human- and machine-decision-making problem. Many participants appealed to technology's ability to be comparatively impartial as an important mitigator of bias, i.e., one participant repeatedly requested a: ``Non-biased programme'' (P3); another stated a preference for: ``Data analysis to make decisions on who we should be supporting as opposed to having humans try to make those decisions with all their biases'' (P5). However, those same participants noted that common machine decision-making paradigms amplify bias and that it was important to be aware of this: ``AI has a lot of bias in it'' (P3).

Others noted the possibility for technology to elucidate biases in both humans and machines. One participant requested: ``Some kind of tool that can detect bias in a selection'' (P7).

\subsubsection{Programme Goals}
Selectors from both programmes identified goals for their programmes. These goals were discussed by both groups as an intended form of `impact' and both groups closely related achieving their goals to their expressions of why diversity mattered. While impact goals varied based on the type of impact and the affected party, we discuss these as `impact' in Table \ref{tab:themes}.

\paragraph{Impact}
5 out of 15 participants found key goals of their programme to include: ``Orient[ing] [scholars] towards social impact or using their talent for good'' (P8). They tended to encourage: ``[Scholars'] working towards something impactful throughout their career'' (P8).

\subsubsection{Merit}
Several participants reflected on the relationship between merit and diversity. While some participants saw these as competing goals, others saw them as complementary. In particular, complementary views often viewed merit as a form of performance relative to specific advantages or noted that many of our measurement tools are biased across our chosen diversity dimensions. These themes are reflected as subthemes of `Merit' in Table \ref{tab:themes}.

\paragraph{Performance relative to disadvantage}
2 out of 15 participants identified merit as something difficult to disentangle from performance. One participant noted that applicants may appear less qualified because they: ``Didn't have the chance; didn't have the opportunities'' (P2), while others with the opportunities will appear more qualified. Another participant began by asking: ``How good are their three A stars based on where they've come from?'' (P12), then proceeded to reflect that ``Your performance relative to your opportunity or maybe expected performance'' (P12) is a key indicator of merit.

\paragraph{Measurement}
Closely related, 3 out of 15 participants questioned our ability to measure merit independent of opportunity: ``[Whether they perform well] because they have the opportunity or because they are brilliant – I think that these two are really difficult to untangle'' (P7). Another noted that: ``Contextual factors mess up our otherwise seemingly objective measures of merit.... national context and family income is messing up your ability to measure the thing you actually care about'' (P10). They continued to note that they: ``Need to pay attention to [diversity] because it's messing up your measures of what you actually care about'' (P10).

\paragraph{Performance}
Finally, 2 out of 15 participants noted occasions where performance and diversity were ostensibly competing goals. However, even here, participants recognised that observed performance and actual merit may differ. One participant noted that: ``[The] overriding aim is for [the programme] to be as diverse as is possible but still meet a standard....relative score of like how good their application is based on all these kind of contextual factors'' (P12). Another requested a technology that helps discover how: ``Close you are to your idealised diversity targets and how close you are to maximising whatever it is you think you're maximising in your performance scores'' (P10).

\section[Study 2]{Study 2: Participatory Design}\label{sec:study2}
\subsection{Prototypes}
As a next step, we applied the results of our thematic analysis to design six prototypes following methodology from \textcite{Buchenau_Suri_2000}. These technologies aimed to help selectors better understand and operationalise diversity in their selection processes. We then present these prototypes to participants in participatory design workshops. These prototypes are shown in Figure \ref{fig:prototypes}.

Three of these prototypes (Figures \ref{fig:representativeness}, \ref{fig:entropy}, and \ref{fig:diversity}) present information about the range of possible cohorts participants must choose between. In addition to the themes uncovered in Section \ref{sec:study1}, these prototypes draw from the economic theory presented in Chapter \ref{ch:spf}. Meanwhile, the other three (Figures \ref{fig:demographic}, \ref{fig:impact}, and \ref{fig:advantage}) present information about an individual applicant relative to a given cohort and a given pool. Furthermore, five of the six prototypes were designed to satisfy definitions of diversity uncovered in Section \ref{sec:study1}.

\begin{figure}[htbp]
    \centering
    \begin{subfigure}[b]{0.3\textwidth}
        \includegraphics[width=\textwidth]{diversity/representativeness.png}
        \caption{Prototype \ref{fig:representativeness}: Cohort Representativeness}
        \label{fig:representativeness}
    \end{subfigure}
    \hfill
    \begin{subfigure}[b]{0.3\textwidth}
        \includegraphics[width=\textwidth]{diversity/entropy.png}
        \caption{Prototype \ref{fig:entropy}: Cohort Entropy}
        \label{fig:entropy}
    \end{subfigure}
    \hfill
    \begin{subfigure}[b]{0.3\textwidth}
        \includegraphics[width=\textwidth]{diversity/diversity.png}
        \caption{Prototype \ref{fig:diversity}: Cohort Diversity}
        \label{fig:diversity}
    \end{subfigure}

    \medskip

    \begin{subfigure}[b]{0.3\textwidth}
        \includegraphics[width=\textwidth]{diversity/demographic.png}
        \caption{Prototype \ref{fig:demographic}: Applicant Demographic Information}
        \label{fig:demographic}
    \end{subfigure}
    \hfill
    \begin{subfigure}[b]{0.3\textwidth}
        \includegraphics[width=\textwidth]{diversity/impact.png}
        \caption{Prototype \ref{fig:impact}: Applicant Demographic Impact on Cohort}
        \label{fig:impact}
    \end{subfigure}
    \hfill
    \begin{subfigure}[b]{0.3\textwidth}
        \includegraphics[width=\textwidth]{diversity/advantage.png}
        \caption{Prototype \ref{fig:advantage}: Applicant Advantage Scores}
        \label{fig:advantage}
    \end{subfigure}
    \caption{These figures depict the prototypes designed based on themes from Section \ref{sec:study1} and used in our participatory design workshops. They are reproduced at a larger scale in Appendix \ref{app:divfigures}}
    \label{fig:prototypes}
\end{figure}

Figures \ref{fig:demographic} and \ref{fig:impact} are based on the `representativeness' theme, while Figure \ref{fig:advantage} is based on the `contextualising applications' theme. Similarly, Figures \ref{fig:representativeness} and \ref{fig:entropy} draw a distinction in their measurements of diversity, based on the `representativeness' and the `different perspectives' themes, respectively. Figure \ref{fig:entropy}, in particular, defines and employs `entropy' as a metric. This reflects that when aiming to get different perspectives in the same room, the goal is not to represent any target population; rather, we desire that everyone in our group be as different from the remainder of the group as possible.

\subsection{Workshop Methodology}\label{ssec:methods2}
As the participants come from two separate talent investment programmes, we run one workshop for each group \cite{Buchenau_Suri_2000}. Before presenting to the broad audience of each group, we submit our figures to one primary contact (also a participant in the study) at each organisation, then run informal, 15-minute `pilot' one-on-one workshops with this primary contact. Here, we primarily sought approval to use these figures in a workshop with the broader team of selectors, but we also collected minor feedback and tweaked the prototypes based on this feedback. In one organisation's pilot workshop, the primary contact requested Figures \ref{fig:representativeness} and \ref{fig:entropy} be combined into one prototype with `Diversity' as the Y-axis, as the organisation already has an internal working definition of diversity (that incorporates what we mean by both representativeness and contextualising applications). This can be seen in Figure \ref{fig:diversity}. Participant grouping for the workshops is redacted to preserve participant anonymity; the results for this analysis are attributed by group, rather than individual, to reflect the cooperative nature of the task.

Our central research questions for this workshop are:

\begin{enumerate}
    \item What prototypes best promote diversity?
    \item What elements of these prototypes facilitate their success?
\end{enumerate}

\noindent Or, for each prototype: ``How and why does this prototype promote diversity?''. In each workshop, we ask participants to consider each prototype in turn and to discuss how they might use it in their selection process. We then ask participants to consider how these prototypes might fit into their current selection process, and how they might change their process to better incorporate these prototypes. Finally, we ask participants to consider how their current selection process might make the best use of these prototypes, and whether they think these prototypes would be beneficial. 

Following the methodology of \textcite{Gatian_1994,Griffiths_Johnson_Hartley_2007}, at the end of each workshop, we ask participants to highlight their favourite prototype.

A question-by-question protocol for these workshops can be found in Appendix \ref{app:divprotocol2}.

\subsection{Results}\label{ssec:results2}
\subsubsection{Participants Preferred Prototype \ref{fig:impact}}
As part of the workshop, participants were asked to mark their favourite prototypes \cite{Gatian_1994,Griffiths_Johnson_Hartley_2007}. These favourites have been collated in Table \ref{tab:favourites}, and it can be seen here that Prototype \ref{fig:impact} was by far the favourite in both groups.

\begin{table}[htbp]
    \centering
    \caption{This table tallies the number of participants who indicated that a given prototype was their favourite. The overwhelming favourite was prototype \ref{fig:impact}, which shows an individual applicant's impact on the cohort.}
    \label{tab:favourites}
    \begin{tabular}{lr}
        \toprule
        \textbf{Prototype} & \textbf{Favourites} \\
        \midrule
        Prototype \ref{fig:representativeness} & 1 \\
        Prototype \ref{fig:entropy} & 1 \\
        Prototype \ref{fig:diversity} & 0 \\
        Prototype \ref{fig:demographic} & 1 \\
        Prototype \ref{fig:impact} & 10 \\
        Prototype \ref{fig:advantage} & 0 \\
        \bottomrule
    \end{tabular}
\end{table}

\subsubsection{Both Groups Rely on Idiosyncratic Notions in Selection}
When participants were placed in a practical selection scenario and shown technology prototype information about (hypothetical) applicants, they compared applicants to the idiosyncratic profiles that they desired.

When evaluating Prototype \ref{fig:advantage}, G1 used advantage scores to seek out: ``Diamonds in the rough'' (G1), i.e., talented applicants from disadvantaged backgrounds who lacked the polish of their more privileged counterparts.

Prototype \ref{fig:entropy} was initially confusing to G2, as the `entropy' definition used was unfamiliar to the participants: ``Entropy is chaos in chemistry. How does this relate to our usage here?'' (G2). Thus, when interacting with Prototype \ref{fig:entropy}, G2 understood `entropy' to be a variety in cognitive skill and personality type. In particular, the group sought to ensure that the chosen cohort contained `glue' people, who improve overall cohort cohesion: ``For people working together, it's useful to have someone who is that `glue''' (G2). 

\subsubsection{Organisations and Participants Are Interested in Different Diversities}
When presented with Prototypes \ref{fig:representativeness} and \ref{fig:entropy}, G2 was given the demonstrative definitions of `representativeness' and `entropy' (visible on the prototypes). G2 quickly understood the differences between the figures to be that Prototype \ref{fig:representativeness} sought to support considerations of demographic representativeness, while Prototype \ref{fig:entropy} sought to support placing different perspectives (be they cognitive skill-sets or personality types) in the same room. Participants were then interested to know the relationship between these prototypes in practice: ``If we maximise based on [Prototype \ref{fig:representativeness}] scores, what would the Entropy scores be?'' (G2).

However, though individual participants took great interest in Prototype \ref{fig:representativeness}, G2 ultimately acknowledged that the programme was most interested in building a cohort from people with different perspectives to facilitate collaboration: ``For [our] cohorts, [we] want a balance [of personality types] and [we] want them to be collaborative'' (G2). At the same time: ``Let's track but not use [programme-specific measure of representativeness]'' (G2).

\subsubsection{Different Tools are Useful at Different Application Stages}
Participants in both groups variously expressed anxiety about measuring the relevant dimensions. ``What are our metrics and are they reliable?'' (G2) was echoed by several G2 participants. ``If it's all self-report, then we can't do anything with it'' (G1). 

However, both organisations noted that their selection processes involve a variety of stages. Different tools are useful in different stages. When speaking of Prototype \ref{fig:entropy}, one participant said: ``This is better post-interview than it is pre-interview'' (G2), as interviews will collect observational data on many of these characteristics. I.e., while certain tools may rely on measurements that cannot be collected until later stages, others will be more useful earlier in the selection process.

Similarly, different output modes were useful at different stages. Note that Prototypes \ref{fig:demographic} and \ref{fig:impact} contain similar information, but that Prototype \ref{fig:impact} displays this information in greater detail. When reviewing both prototypes, G1 found Prototype \ref{fig:demographic} preferable in the earlier stages of decision-making, while Prototype \ref{fig:impact} had the greatest utility later. ``[We] can't send [Prototype \ref{fig:impact}] as a pre-read, but [Prototype \ref{fig:demographic}] makes more sense in isolation, so better for a pre-read'' (G1). On Prototype \ref{fig:demographic} in particular, one participant noted: ``Helpful for the process, not so much for the [final cohort selection]'' (G1). Another said of Prototype \ref{fig:impact}: ``This has the most potential at the later stages of decision-making'' (G1).

At the other extreme, while the cohort-level tools did not spark discussions about particular individuals, they did spark earlier-stage, higher-level discussions. Most obviously, both programmes discussed specific tradeoffs between measured individual applicant aptitude and cohort diversity: ``Real decision-making always sees tradeoffs like these.... This chart helps you figure out the level of compromise you're willing to make on both axes'' (G2); ``[It] makes sense that top scoring candidates don't necessarily help you build the most diverse cohort.... 5-10\% [of the cohort] really get to a struggle between quality and diversity.... If this chart were real (rather than hypothetical), and you could see who you were losing, this would be useful'' (G1). (In the hypothetical,  participants from G1 settled closer to the centre of the frontier: ``[Our] target here is to look somewhere [from] red to yellow'' (G1). However, they also noted that ``Some candidates get a big diversity boost and score terribly'' (G1).)

In one case, participants also discussed broader, programme-level concerns. Participants spent time debating ``Is the programme needs-based or merit-based?'' (G1). They noted that ``This chart helps [facilitate that discussion]'' (G1) but did not ultimately conclude one way or the other.

\subsubsection{The Right Balance of Quantitative and Qualitative Information is Key}
Participants simultaneously expressed gratitude that the prototypes were as simplified as they were, and a desire for more detail. One participant from G1 noted that the prototypes are: ``Very constrained in terms of what is being shown.... This doesn't include all of the factors, but for the decision, that's good, because it prevents info overload'' (G1). However, participants G2 frequently requested additional quantitative information. In the case of the Prototype \ref{fig:impact} (participants' favourite prototype, as can be seen in Table \ref{tab:favourites}), participants requested many other metrics: ``Advantage score'' (G2), ``Impact on entropy'' (G2), ``[A summary of] these scores together'' (G2), and ``A composite impact on cohort diversity'' (G2).

Meanwhile, participants from G1 requested qualitative information: ``We need to know more about the applicants' backgrounds'' (G1). Other requested information included: ``Comments from selectors'' (G1) and ``A narrative summary...written by the selector team'' (G1). 

This suggests a discrepancy between the two groups' preference for the balance between qualitative and quantitative information. While G2 expressed a desire for unifying quantitative metrics: ``A single score for disadvantage [or] need, while recognising its flaws, could provide one read of an individual's circumstances'' (G2), G1 expressed a desire for individual, qualitative information: ``We're using quantitative to sift through qualitative.... [We] need to include comments from selectors'' (G1).


\section{Design Recommendations}
\begin{figure}[htbp]
    \centering
    \includegraphics[width=0.9\linewidth]{diversity/recommendations.png}
    \caption{This figure illustrates our four key design recommendations to others building tools to support the selection of diverse talent.}
    \label{fig:recommendations}
\end{figure}

\subsection{Design for a Specific Diversity}
In the initial interviews, participants were often vague when asked to define diversity. However, when asked to expand on why diversity is important, or on what dimensions of diversity they prioritised, it became clear that `diversity' included three separate (and sometimes competing) definitions. We have termed these: `representativeness', `different perspectives', and `contextualising applications'. When designing tools to assist a target organisation in considering diversity, we suggest designers first clarify through human-centric methods which definitions of diversity the target organisation seeks to consider, then designs to support those specific definitions. That is, designers should follow \textcite{VanKleek_Seymour_Binns_Shadbolt_2018}'s paradigm of `respectful' design and build technology to best serve the needs of the selectors who will use it.

\subsection{Design for Idiosyncrasy}
One key note revealed through this process is that different decision-making processes have philosophical underpinnings, desiderata, and anecdotal definitions that impact their selections. These idiosyncrasies should be discovered early in development and designed for in any technical solution. We suggest participatory design as a mechanism for achieving this. We observed a strong relationship between participant feedback in interviews and their satisfaction with the prototypes.

For example, both talent investment programmes we worked with have created specific personas they look for. For example, one group discussed `glue' people who helped groups function cohesively. The other group discussed `diamonds in the rough', talented youth systemically undervalued due to their backgrounds. Where these aspects were included, participants showed strong interest in the prototypes. Where they were excluded, participants often asked for these to be added.

\subsection{Design in Stages}
Much of what participants desired at one stage of decision-making was mutually exclusive with what they desired at other stages. For example, G1 desired Prototype \ref{fig:demographic} as a: ``Pre-read'' (G1) due to its simplicity but preferred the detail of Prototype \ref{fig:impact} ``In the room'' (G1). Thus, it is crucial for designers to consider what stage of decision-making their tools are designed to support and to design appropriate levels of detail, abstraction, and engagement accordingly.

\subsection{Design to Balance Qualitative and Quantitative}
Participants often noted that the prototypes were missing key qualitative information about applicants. This qualitative information is crucial in the holistic considerations of each applicant. However, when allowed to consider only qualitative information, participants obscure tradeoffs they are forced to make between different programme goals. In particular, while individual-level goals are often clear, cohort-level goals (such as diversity) are easier to delay or ignore. Thus, without quantitative tools to frame the discussion, participants noted that they were often forced to make cohort-level considerations ad-hoc and towards the end of their decision-making process.

Thus, while qualitative information is crucial, it is also important to present the quantitative information necessary to make these tradeoffs salient. Ultimately, final selection decisions are made by panels of trained selectors, but in the absence of both quantitative and qualitative information to guide these decision-makers, they would be forced to make decisions that are less well-informed than they could be.

\section{Discussion}\label{sec:divdisc}
\subsection{The Diversity Triangle}
In Section \ref{sec:study1}, participants variously identified the word `diversity' with three themes, which we have taken to signify definitions of diversity: `representativeness', `different perspectives', and `contextualising applications'. As these three definitions are central to our research questions, we privileged them over Section \ref{sec:study1}'s other themes in Section \ref{sec:study2}, where we engaged in scenario speed dating and experience prototyping designed to satisfy different definitions of diversity. In this section, we map the themes to the three definitions in the Diversity Triangle (Figure \ref{fig:div_triangle}).

\begin{figure}[htbp]
    \centering
    \includegraphics[width=\textwidth]{diversity/diversity_triangle.png}
    \caption{This figure depicts the Diversity Triangle, three differing definitions of diversity selectors expressed when discussing a diverse cohort. We also relate the Diversity Triangle to each other theme or subtheme participants mentioned.}
    \label{fig:div_triangle}
\end{figure}

\subsection{The Relationship Between the Diversity Triangle and Theories of Change}
\subsubsection{Representativeness}
Each of the definitions from the Diversity Triangle (Figure \ref{fig:div_triangle}) implies different programme values and a different theory of change. The representativeness
definition relies on social theories by which diversity is intrinsically valuable. Several participants seemed to believe representativeness here was intrinsically valuable, which aligns with the \textcite{morris1984origins} account under which contemporary diversity norms emerged from loci of oppression affecting underrepresented groups. However, some participants also noted instrumental reasons to value representativeness that align closely with \textcite{peters2021hidden,page2008difference}'s argument that people from different backgrounds often possess unique and germane knowledge. One participant noted that a team can: ``Better serve a community if they represent [that community]'' (P9). Another theory from \textcite{Friedler_Scheidegger_Venkatasubramanian_2016} discusses measurement bias. Notably, if we assume that talent is equally distributed across some partition, then the most talented cohort should also be representative. However, \textcite{Friedler_Scheidegger_Venkatasubramanian_2016} note that we often observe in practice that performance is not equally distributed across these partitions. This is likely due to both relevant differences between those groups and structural bias that causes differences in construct observability between groups (in fact, the relevance of said differences may also be due to structural biases) \cite{Friedler_Scheidegger_Venkatasubramanian_2016}. In this case, representativeness would be a proxy for the distributive notion of fairness \cite{Olsaretti_2018}. Thus, representativeness is pro-social: society is better served when resources are distributed to a representative group of people.

\subsubsection{Different Perspectives}
The argument for placing different perspectives in the same room is often instrumental. While `diversity' on the whole is often spoken of as an intrinsically valuable broader benefit, the aim of placing different people in the same room is often only to benefit the people in that room. \textcite{page_diversity_2010} argues that `cognitively' diverse groups outperform homogeneous groups on some tasks; some participants similarly contend that it improves cohort-level task performance. Other participants echo \textcite{wylie2006introduction}'s argument that it allows participants to better learn from each other. In either case, the benefit is primarily organisational rather than social.

\subsubsection{Contextualising Applications}
The argument for contextualising applications is twofold. Most often, participants make a systemic critique here. That is, the world is incredibly unjust, and we want to distribute resources differently, but in a talent selection process, we still have to operate in an unequal world. Thus, to correct that injustice, we must give more resources to those who have less. This could be seen as a form of distributive justice or `affirmative action' \cite{Olsaretti_2018}. Participants argue (perhaps relatedly) that appropriately contextualising applications results in more successful applications from disadvantaged groups; this, in turn, allows these applicants to have a positive impact on their groups. Relatedly, either due to measurement bias, or due to differences in performance brought on by disparate access to resources, support, and opportunities, we may find that applicants from disadvantaged backgrounds appear worse on paper. In either case, correcting this through contextualisation may also build a more fair selection process and a more just world, yielding great benefit to society. On the organisational side, if contextualising applicants allows for the admission of more applicants from marginalised groups, they may be better suited to critiques of existing power structures, as they may have more informative experiences of oppression \cite{mills2015blackness}.

\subsection{A Cautionary Note On Designing for Diversity}
Decisions such as scholarship selection have long-reaching impacts on the applicants. A successful applicant to a scholarship programme might thus attend a university they could not have otherwise attended; there, they will acquire skills and a network that will continue to impact them later in life; these impacts may even affect those close to them, as their better circumstances likely better the circumstances of their communities \cite{Dassin_Marsh_Mawer_2018}. Thus, it is simultaneously important to ensure that these opportunities are dispersed fairly and to ensure that all demographics are included among those selected.

This increases the importance of selecting a diverse group of people and mandates that we do all we can to do so. Quantitative decision support tools may have a role to play in this, but two problems remain: data collection and data processing. Primarily, diversity data is often self-reported. Thus, we cannot use this naively to generate diverse cohorts, as doing so would yield a competitive advantage to candidates who lie on their declarations. Secondly, processing data automatically has its drawbacks. AI systems are prone to bias \cite{Friedler_Scheidegger_Venkatasubramanian_2016}. And as these systems improve, it is unclear if they will help or harm our ability to select diverse cohorts.

However, it is important to note that both the problems of data collection and processing are not unique to this workflow. Algorithmically-supported diversity considerations may correct for data collection and processing issues elsewhere. Bias is a major consideration in data collection \cite{Friedler_Scheidegger_Venkatasubramanian_2016}; heterogeneous biases in measurements of talent may be corrected by ensuring diversity across these metrics: ``Assessments are usually biased measures of what we care about, and that opportunity often correlates with positive error terms in assessments' measure of underlying skills.... [Diversity considerations] correct for the inadvertent affirmative action against underprivileged individuals implicit in using biased assessments'' (P1). Similarly, much like AI systems, human selectors are prone to their own heterogeneous biases; in some sense, `design for idiosyncrasies' reflects the notion that decision support systems must at times help recognise and mitigate these biases.

\section{Limitations and Future Work}
In Studies 1 and 2 we build tools with which participants report satisfaction, but increasing participant satisfaction does not necessarily improve decision-making. In particular, technology that makes difficult decisions less painful may be well-received, while technology that makes these decisions more salient may be less popular but more impactful \cite{Lipton,miller_explainable_2023}. Our themes speak to participant perceptions of diversity and our design recommendations speak to participant desiderata from support tools. We assume that, in solving for these considerations, we can help organisations select better cohorts. However, we may find that these tools fail to improve decision-making concerning diversity in practice. Future work should investigate this possibility through the implementation of our prototypes in field settings.

\textcite{venn-wycherley_realities_2024} contend that HCI literature focused on educational contexts should consider both educator and student. Two distinctions distance our work from theirs: first, selection is distinct from pedagogy in that our decision subjects are not necessarily beneficiaries (while students are beneficiaries of their institution, applicants do not become beneficiaries unless they are selected); second, scholarships are distinct from educational institutions in that scholarship programme benefits primarily focus on assisting beneficiaries in accessing educational institutions, while educational institutions primarily educate beneficiaries. Both distinctions create distance between selectors and applicants beyond that between teachers and students and pose challenges in engaging decision subjects. Nonetheless, future work should engage scholarship applicants to understand their definitions, stances, and considerations concerning diversity. 

While not a limitation, we have intentionally set aside themes such as `outreach' and `selectors' that relate more to other aspects of selection processes than to the act of selection itself. We hope future work will consider these facets of selection programmes, especially when designing tools to support thinking around diversity.

\section{Conclusion}
This research answers the crucial question: ``What is diversity?'', from the perspective of scholarship and talent investment selectors. In doing so, we illuminate the multifaceted nature of diversity in scholarship selection and emphasise the critical need for tools that support considerations of diversity in decision-making. Our findings reveal that achieving true diversity involves navigating the complex interplay between three occasionally conflicting definitions: representativeness, diverse perspectives, and the contextualisation of applicants' backgrounds. By engaging in participatory design, we build six prototypes with our selectors; this process reveals four design recommendations: design for a specific diversity, design for idiosyncrasy, design in stages, and design to balance quantitative and qualitative. This work demonstrates that, when thoughtfully designed, technology can empower selection processes to be more equitable, inclusive, and transparent. The broader implication is that such advancements have the potential to reshape how diversity is operationalised, ensuring that it is both a measurable outcome and a core value in shaping the future of talent identification. As we move forward, the integration of these tools into real-world practices will be pivotal in fostering truly diverse and representative groups in global scholarship programmes and beyond. However, we caution that programmes seeking to implement these tools should ensure that they improve not only subjective selector perceptions of decisions but also the objective quality of those decisions themselves.
\chapter[A Possibility Frontier Approach to Diverse Talent Selection]{\label{ch:spf}A Possibility Frontier Approach to Diverse Talent Selection\footnote{This chapter is based on two papers showcasing research done in concert with Kadeem Noray. Both contributed equally to the research. One paper is currently under review as: Kadeem Noray and Neil Natarajan. 2024. "Selecting for Diverse Talent: Theory and Evidence." Under review at Economics of Talent Meeting, Fall 2024. The other is being prepared for submission as: Neil Natarajan and Kadeem Noray. 2024. "SPF: A Technology Probe Examining Diversity in Selection Processes.". This version draws from both publications; in doing so, it borrows from the rich tradition of talent-related research in economics, as well as the HCI work referenced in Chapters \ref{ch:xai}, \ref{ch:genai}, and \ref{ch:diversity}.}}

% [REJECTED] Cohort optimality – this chapter seems to focus on the optimal "cohort/team" form of selection rather than the aspirational/post-hoc; only in this context does the NP-completeness and complexity seem relevant.

\minitoc

\section{Motivation}
Chapter \ref{ch:diversity} designs six prototypes in cooperation with Rise and Ellison Scholars selectors. We ultimately find selector interest in all of these design prototypes. However, we warn as early as Chapter \ref{ch:context} of the dangers of placating subjective selector feedback in place of evaluating the impact on decision-making. (This is the very same issue with explainable AI we find in Chapter \ref{ch:xai}.) Thus, this chapter implements Prototype \ref{fig:diversity} in a field deployment with the Rise programme to gain a more objective evaluation of its impact on decision-making.\footnote{It should be noted here that Prototype \ref{fig:diversity} itself draws from theory presented in Section \ref{ssec:measurement}. Thus, the design of Prototype \ref{fig:diversity} implemented in this chapter draws as much from this chapter as from Chapter \ref{ch:diversity}.}

\section{Introduction}\label{sec:spfintro}
For comprehensive background on economic perspectives on selection and algorithmic decision-making in talent selection, see Chapter \ref{ch:context}, Section \ref{sec:ai_selection}. Here we focus on the specific theoretical contribution and field deployment this chapter presents.

The rise of diversity, equity, and inclusion initiatives suggests that various organisations (e.g., schools, firms, social impact programmes, etc.) are genuinely interested in selecting diverse talent. This is driven, at least in part, by recent declines in discrimination \cite{hsieh2019allocation}, increases in the perceived return to diversity \cite{deming2017growing, page_diversity_2017, noray2023systemic}, and increased social pressure for demographic representation \cite{minkin2023diversity}. In this chapter, we deploy technology based on Prototype \ref{fig:diversity} in a field study with Rise. We seek to understand whether this technology can help organisations select more diverse and talented cohorts.\footnote{Throughout the chapter, we use ``talent'', ``aptitude'', and ``performance'' interchangeably. In doing this, we recognise that organisations generally seek to optimise for vague and often conflicting notions of talent or aptitude, but do so via measurements of applicant performance on assessments or assignments. In practice, this process carries risks; e.g., heterogenous biases in metrics or assessment methods will lead some subgroups to appear less talented or apt than others, even when no true difference in talent or aptitude exists. On the whole, this chapter is more interested in measurements of diversity than of talent or aptitude; thus, while these risks are of crucial importance to fair selection processes, they are tangential to the focus of this chapter.}

To assess this technology's usefulness, we begin by developing a simple model for a central problem within the field of diversity: cohort selection. Organisations receive $N$ applications and must select $n<N$ individuals to construct a cohort from the total set of potential cohorts, each of which is indexed by $c$. The organisation seeks to maximise a preference function that increases both a cohort's talent (e.g., the mean performance on an ability measure) and a cohort's diversity (e.g., an inverse distance to a set of target proportions of demographic groups). This simultaneous optimisation problem yields a frontier, which we term the Selection Possibilities Frontier (SPF). The SPF represents the set of non-dominated cohorts, i.e., the set of cohorts that are not outperformed by any other cohort on both talent and diversity. We then use the SPF to implement Prototype \ref{fig:diversity} in the field, noting that a calculation or approximation of the SPF serves to produce the exact figure shown in Prototype \ref{fig:diversity}; we opt for a simple greedy estimation algorithm that relies on the submodularity, monotonicity, and non-negativity of diversity functions \cite{krause2014submodular, huppenkothen2020entrofy}.

Next, we turn to the empirical component of the chapter where we work with the Rise team to deploy this technology in the Rise Cycle 2023 selection. We find that, in Cycle 2021, the set of finalists could have been $18.5\%$ more diverse without any reduction in performance, or $17.8\%$ higher performing without any reduction in diversity. Similarly, in Cycle 2022, the finalists could have been $14.6\%$ more diverse without any reduction in performance, or $24.1\%$ higher performing without any reduction in diversity. This indicates that, in the absence of our SPF-based Decision-Support Tool (DST), Rise selects cohorts well within the frontier; i.e., both cohorts chosen were non-first-best as measured by the programme's definitions of diversity and performance. We also show that, when the programme was given access to the SPF in Cycle 2023, they significantly improved the efficiency of their selection process, choosing a cohort that was more diverse, higher-performing, and very near to the estimated SPF. This suggests that the DST aided their decision-making and allowed them to better optimise in the face of a complex choice.

Though it is encouraging that the technology deployed in this chapter improves organisational selection, we still require an understanding of why. In seeking to understand why programmes do not simply select on the frontier without support, we prove that this problem, though simple to state, is computationally complex. This is for two intuitive reasons. First, an individual's contribution to the diversity of the cohort depends on who else will be part of the cohort. Thus, to determine which cohort is the smallest distance from the predefined ``ideal'' diverse cohort, every possible cohort needs to be assessed. Second, when diversity preferences are for the representation of at least two non-mutually exclusive identities (e.g., ethnic minorities and women), optimally selecting members of one group constrains your ability to do the same for the other group. Formally, we show that, when we restrict our attention to functions formalising diversity as organisations measure it, the `Vertex Cover' problem, known to be $\mathbf{NP}$-hard, reduces to diversity maximisation. This implies that calculating the maximal diversity conditional on cohort performance (what we refer to as the Selection Possibilities Frontier or SPF) is also $\mathbf{NP}$-hard. Selecting on the SPF without a DST, then, is likely prohibitively costly for many organisations.

To account for this complexity, we augment our model of cohort selection by forcing organisations to incur a computational marginal cost for each unit of increased cohort diversity. This represents the fact that, to find a more diverse cohort, one must engage in the laborious process of composing potential cohorts and comparing them. This is in sharp contrast to finding more talented cohorts, which is comparatively simple -- and, therefore, modelled as costless -- because each individual's contribution to cohort talent is unrelated to the remainder of the cohort. This updated model appears to better describe organisational behaviour. This has two implications: (1) organisations will tend to select sub-optimal (i.e. non-first-best) cohorts and (2) organisations will improve on both performance and diversity if they gain access to a technology that reduces this computational marginal cost. 

As an aside, we apply our SPF estimation procedure as an ex-post DST to evaluate the efficacy of alternative screening and selection methods. To do this, we leverage two unique aspects of the programme. First, the programme collects both traditional merit-based measures -- including cognitive tests, written essays, and referring organisations -- as well as non-traditional measures -- including peer-reviewed video essays, gamified skill tests, and application platform behaviours. Second, the programme engaged in effectively no screening before receiving concrete projects from applicants, making it possible to estimate valid counterfactual diversity and performance of cohorts had they been screened in different ways. Leveraging these features, we find three key results. First, selecting only based on cognitive ability or traditional metrics would have improved cohort performance relative to random selection, but would significantly restrict the programme from reaching its diversity goals. By contrast, selecting only on peer reviews performs similarly on performance but improves diversity substantially. Second, all alternative selection methods we explore result in selecting cohorts well within the SPF and, therefore, leave substantial diversity and performance gains on the table. Third, the trade-off implied by the SPF between talent and diversity is steeper if traditional measures are used to measure talent than if applicant projects are used. 

\section{Theory and Methods}\label{sec:spfmethod}
\subsection{Evaluating Organisational Decision-Making}\label{ssec:measurement}
Central to this chapter is the desire to evaluate Prototype \ref{fig:diversity} on real results, rather than subjective satisfaction. However, there is no ground truth of decision-making in the scholarship programme. Thus, we define a model of talent selection and use this model in assessing a field deployment of the prototype. This model relies on a key object, the Selection Possibilities Frontier (SPF), to bound the range of possible cohort selections on two axes: average applicant performance on individualised metrics of talent, and overall group diversity. With this model, we can evaluate the programme's decisions in terms of their proximity to the SPF; the closer a programme is to the SPF, the more efficient its selection process. In this case, we deploy Prototype \ref{fig:diversity} with Rise, thus, we are interested in Rise's SPF. Rise does not share the precise aggregation method for their talent and diversity metrics, but a summary of the measurements they collect can be found in Appendix \ref{app:programmes}. For this chapter, it suffices to know that Rise has working definitions of both performance and diversity.

We start by considering a simple version of the organisation's optimisation problem. Organisations receive $N$ applications and must select $n<N$ individuals to form a cohort $c$ from the set of all potential cohorts $C$. The organisation prefers both that the selected cohort is higher-performing on some measure of talent and more diverse. For now, a cohort's diversity can be thought of as the inverse of a multidimensional measure of distance between the set of proportions of the cohort who belong to key demographic groups and a set of target proportions the organisation has for each group (we discuss definitions of diversity in more detail in Section \ref{subsec:dts_nphard}). If we let the performance and diversity of a given cohort $c$ be given by the functions $P(c)$ and $D(c)$, respectively, then the above description is equivalent to letting the organisation's preference function $F\Big(D(c),P(c)\Big)$ exhibit $F_D>0$, $F_P>0$, and $F_{DP}\geq0$, where subscripts indicate partial derivatives. 

\begin{figure}[htbp]
    \centering
    \caption{This figure depicts an example solution to an iteration of the selection problem, which is described in Equation \ref{eq:selection_simple}. The solid blue curve represents the Selection Possibilities Frontier (SPF), the dotted blue curve represents the organisation's indifference curve corresponding to the highest achievable utility, and the blue dot represents the diversity and performance of the optimal choice (i.e. the first-best solution). }
    \label{fig:model_spf}
    \includegraphics[width=\textwidth]{spf/model_spf.png} 
\end{figure}

Conceptually, this would represent a scenario where an organisation can observe $D$ and $P$ for every possible cohort and simply select the one that maximises $F(D,P)$. If we assume the organisation behaves rationally, we know the organisation will not choose dominated cohorts. Formally, $c^*$ can be the optimal cohort if and only if there exists no $c'$ such that $D(c')>D(c^*)$ and $P(c')\geq P(c^*)$ and there exists no $c'$ such that $D(c')\geq D(c^*)$ and $P(c')> P(c^*)$. We know that the optimal cohort must be in the set of non-dominated cohorts which we define as the Selection Possibilities Frontier (SPF). If we assume, for expositional purposes, that the SPF is continuous, we can represent it as the following function:

\begin{equation}
G(p) := \max\Big[D(c)|P(c) \geq p\Big]
\end{equation}

\noindent In words, $G(p)$ merely gives the highest possible diversity for every cohort performance level. The diverse talent selection problem, then, can be represented as choosing a cohort to maximise $F$ subject to a constraint that the choice be on the SPF. Formally: 

\begin{equation}
\max_{d,p} F\Big(d,p\Big) \text{ \bf{ s.t. } } d = G(p), \nonumber 
\end{equation}

\noindent which is equivalent to:

\begin{equation}
\max_{p} F\Big(G(p) ,p\Big). \label{eq:selection_simple}
\end{equation}

The solution to this simple version of the model is depicted in Figure \ref{fig:model_spf}. Notably, this model suggests that organisations should always select cohorts on the frontier, as all cohorts within the frontier are dominated by cohorts that are either at least as diverse and more talented or at least as talented and more diverse.

\subsection{Defining the Classes of Diversity and Performance Functions}\label{subsubsec:div_talent_def}
Without knowledge of the types of functions $D(c)$ and $P(c)$, Equation \ref{eq:selection_simple} proves difficult to instrument in practice. In this section, we define the classes of functions $D(c)$ and $P(c)$ that are relevant to the diverse talent selection problem. In particular, we will formalise \emph{proportional diversity} and \emph{count diversity} as kinds of diversity, and assume performance to be a real-valued individual-level metric, aggregated by summation.

In both cases, here, we work with Rise to identify their preferences w.r.t. different kinds of diversity and performance. We then implement their preferences as functions $D(c)$ and $P(c)$.

\paragraph{Proportional diversity} When organisations make statements like: ``we desire at least $x$ proportion of group $g$'', they are speaking of proportional diversity. But, since organisations aim to select cohorts of a specific size, we can reframe this goal as ``we desire at least $x*n$ individuals from group $g$'', where $n$ is the total number of applicants in the cohort.\footnote{This reframing will turn out to be helpful in Section \ref{sec:spf_alg} when we develop our SPF estimation strategy} If we let $\chi_g(c)$ be the proportion of $c$ in group $g$ and $\sigma_g(c)$ be the total number of applicants in $c$ who are in group $g$, this goal can be formalised into the proportional diversity function:

\begin{equation}
    \begin{split}
        \delta_{g}^{prop}(c,x) &:= n*\min(\chi_g(c), x) \\
        & := \frac{n* \min(\sigma_g(c), x*n)}{n} \\ 
        & := \min(\sigma_g(c), x*n). \label{eq:prop_div_function}
    \end{split}
\end{equation}

Note that the minimum function is used here to formalise ``at least'', so that the function only increases until the proportional threshold is met. If, for example, an organisation selecting 100 applicants would like their organisation to be at least $40\%$ female, the proportional diversity function associated with this goal is $\delta_{female}^p(c, 40) := \min(\vec{\mathbf{f}}*\vec{\mathbf{c}}, 40)$ where $\vec{\mathbf{f}}$ is a Boolean vector indicating which applicants are female and $\vec{\mathbf{c}}$ is a Boolean vector that indicates who is in cohort $c$. This can be thought of as an inverse distance along the dimension of group representation between cohort $c$ and an ideal cohort $c^*$ where $\sigma_g(c^*) = x*n$.

\paragraph{Count Diversity} The second common type of diversity is what we call \emph{Count diversity}. This formalises organisational statements like: ``We desire at least one person from $m$ groups''. To formalise this notion, let $\mathbb{I}(\cdot)$ be an indicator function that is equal to 1 if the condition within is true. We can now represent a count diversity preference as:

\begin{equation} 
    \begin{split}
        \delta_G^{count}(c,m) &:= \min\big(\sum_{g \in G}\mathbb{I}(\sigma_g(c)\geq 1), m\big), \label{eq:count_div_function}
    \end{split}
\end{equation}

\noindent where $G$ is the set of relevant groups the organisation wants to be represented by at least a single individual. This type of function is ideal for representing geographic representation goals where educational institutions, like colleges or scholarships, often have goals like ``we want a student from every state'' or ``we want as many countries as possible represented''. 

\paragraph{Overall Diversity} Ultimately, organisations care about all of their diversity goals, not just one. Thus, the diversity functions that are relevant for an organisation must be aggregated if we want to formalise an organisation's overall preference for diversity. We define this aggregation as an organisation's diversity score $D(c)$, which generally has the following form: 

\begin{equation}
A\big(\delta_{g_1}^{prop}(c,x_1),...,\delta_{g_K}^{prop}(c,x_K),\delta_{G_1}^{count}(c, m_1),...,\delta_{G_J}^{count}(c, m_J)\big), \nonumber
\end{equation}

\noindent where $A(\cdot)$ is an aggregator function. It is essential that $D(c)$ increases as a cohort gets ``closer'' to one of the underlying diversity goals because this is sufficient to identify cases when one cohort dominates another, even if the formalisation misses something subtle or difficult to articulate about the organisation's diversity preferences. A flexible but simple aggregator function is a weighted sum, where organisations can place different emphases on each of the goals. So, for the remainder of this chapter, we use diversity scores of the following form: 

\begin{equation}\label{eq:d_equation}
D(c,\vec{\mathbf{w}},\vec{\mathbf{x}},\vec{\mathbf{m}}, \vec{\mathbf{g}}, \vec{\mathbf{G}}) := \sum_{k\in K}w_k\delta_{g_k}^{prop}(c,x_k) + \sum_{j \in J}w_j\delta_{G_j}^{count}(c, m_j),
\end{equation}

\noindent where $\vec{\mathbf{w}},\vec{\mathbf{x}}, \vec{\mathbf{m}}, \vec{\mathbf{g}}, \vec{\mathbf{G}}$ are vectors of the organisation's weights, proportional targets, count targets, groups of interest to proportional diversity functions, and sets of groups of interest to count diversity functions, respectively.\footnote{Another attractive option is a CES aggregator because it allows for specifying the degree of substitutability between diversity goals, but this comes at a cost to interpretability, as many organisations don't regularly use CES aggregators. Nonetheless, the authors are currently working on establishing whether the estimation procedure presented in Section \ref{sec:spf_alg} is viable for a CES aggregator.} In general, we suppress the vector notation opting to refer to the diversity score as $D(c)$ where this doesn't lead to confusion.

\paragraph{Talent, Aptitude, or Performance} Relative to diversity, our definition of performance is simple. In general, organisations measure aptitude for their programme using an individualised metric, usually performance on some assessment or assignment. Common examples include test scores, essays, or grades for educational organisations or technical interviews for hiring in technology. More sophisticated (though uncommon) measures might be the predicted success of an individual based on a set of performance metrics. In this chapter, we assume that organisations already possess a real-valued talent metric $\rho_i$ evaluated at an individual level.\footnote{For more details on Rise's talent metrics, see Appendix \ref{app:programmes}.} A cohort's \emph{talent}, then is defined as the sum of the talent level of the individual members, which is given by:

\begin{equation}
P(c) := \sum_{i \in I_c}\rho_i,
\end{equation}

\noindent where $I_c$ is the set of all individuals $i$ in cohort $c$. Unlike diversity, $P(c)$ is straightforward because each individual's contribution is $\rho_i$ regardless of whoever else is in the cohort. Note that, as long as an organisation fixes their desired cohort size beforehand, optimising for the sum of $\rho_i$ is identical to optimising for mean $\rho_i$.

We represent an organisation's preference function $F$ as a weighted sum of performance and diversity functions. That is:

\begin{equation}\label{eq:f_spec}
F(D, P, c, \iota) := \iota*D(c)+(1-\iota)*P(c)
\end{equation}

\subsection{Implementing Prototype \ref{fig:diversity} in the Field}\label{sec:spf_alg}
As it happens, the SPF modelled in Figure \ref{fig:model_spf} can be used to implement Prototype \ref{fig:diversity} in the field. I.e., a calculation of the SPF using an organisation's preference function yields all of the data required to plot Prototype \ref{fig:diversity} (which is, as it happens, just a depiction of the SPF presented alongside contextual information designed to help selectors best understand the visualisation). However, as we will demonstrate in Theorem \ref{thm:specific-nphard}, calculating the SPF outright is unfeasible.\footnote{A keen reader may note that, under stricter conditions, others have already introduced algorithms for calculating the SPF outright. For example, \textcite{kleinberg2018algorithmic}'s algorithm can be easily extended to calculate the SPF when an organisation only possesses one proportional diversity preference.} Instead, we rely on a greedy algorithm to approximate the SPF.

Greedy optimisation is the practice of approximating an optimal solution to an iterative process by, at each iteration, making a choice that optimises the process at that iteration (i.e. ignoring iterations before and after) \cite{nemhauser1978analysis}. It is well known that greedy optimisation can be used to build near-optimal subsets of a given set when the objective function is non-negative, monotone, and submodular \cite{Feldman_Harshaw_Karbasi_2017,nemhauser1978analysis}. Though these conditions are not strictly necessary, results are not so clear when one of these conditions is dropped \cite{Feldman_Harshaw_Karbasi_2017}.

While non-negativity is self-explanatory (the objective function cannot be less than zero), monotonicity and submodularity deserve further clarification. In our context, monotonicity will require that cohorts are always more diverse than their smaller sub-cohorts while submodularity will require that an applicant's marginal effect on diversity for a cohort will be (weakly) less than their marginal effect on diversity for a smaller sub-cohort. More formally, a function $D$ defined on subsets of some universe $U$ is monotone if and only if 

\begin{equation}
    \label{eq:mononicity}
    \forall Y \subseteq U, X \subseteq Y: D(X)\leq D(Y),
\end{equation}

\noindent and is submodular if and only if

\begin{equation}
    \label{eq:submodularity}
    \forall Y \subseteq U, X \subseteq Y, x \in U \setminus Y: D(X \cup \{x\}) - D(X) \leq D(Y \cup \{x\}) - D(Y).
\end{equation}

This may appear constraining, but, luckily, diversity functions $\delta_g^{prop}(c)$ and $\delta_G^{count}(c)$, $D(c)$, the talent function $P(c)$, and $F(D,P)$ all satisfy these conditions as defined in Section \ref{subsubsec:div_talent_def}. We show this in Theorems \ref{thm:submodularity_additive}, \ref{thm:monotonicity_additive} and \ref{thm:f_sub_mon} in Appendix \ref{app:spfmath}.

Now that we have established the necessary restrictions on functions $F(D,P)$, we present a greedy algorithm that finds $c$ to optimise $F(D, P, c, \iota)$; by repeating this for various values of $\iota$, we obtain the frontier between $D(c)$ and $P(c)$ (i.e., the SPF). This algorithm relies on two observations. First, any point on the SPF can be represented as the maximum of a weighted sum $f(\iota,c) = \iota*D(c) + (1-\iota)*P(c)$ where $\iota \in [0,1]$. Second, any $f(\iota,c)$ is monotonic and submodular. In this context, the algorithm repeatedly maximises a weighted sum of diversity and talent, varying the weight put on each element in each maximisation. Formally, the algorithm maximises $f(\iota,c)$ $m$ times, where each iteration optimises $\iota = \frac{m_i}{m}$. Then, for each $f$, this algorithm builds each cohort $c$ from $c$ of size $0$ until size $n$ by adding an applicant \textit{not} in the current cohort $c$ ($u \in U \setminus c$) that yields the highest $f$ value (i.e., that maximises $f(c \cup \{u\})$). This algorithm is presented more formally in Algorithm \ref{alg:frontier}. 

\begin{algorithm}
    \caption{Greedy Frontier Optimisation}\label{alg:frontier}
    \begin{algorithmic}
    \State \textbf{For} each desired point on the frontier defined by $\iota \in [0, 1]$
    \State \hspace{\algorithmicindent} \textbf{Let} $f_{\iota} := \iota*P+(1-\iota)*D$ be weighted average of $P$ and $D$
    \State \hspace{\algorithmicindent} \textbf{Begin} with empty cohort $c = \vec{\mathbf{0}}$
    \State \hspace{\algorithmicindent} \textbf{While} cohort $c$ is less than the desired size ($|c| < k$)
    \State \hspace{\algorithmicindent} \hspace{\algorithmicindent} \textbf{Find} applicant $i$ such that adding $i$ to $c$ maximises $f_{\iota}(c + i)$
    \State \hspace{\algorithmicindent} \hspace{\algorithmicindent} $c := c + i$
    \end{algorithmic}
\end{algorithm}

It is well-known that the greedy algorithm yields a $\bigl( 1-\frac{1}{e} \bigr)$-approximation of any submodular, monotonic set function \cite{bordeaux_submodular_2014}. That is, the algorithm selects cohorts whose $f_\iota$ values are at least $\frac{1}{1-\frac{1}{e}}$ of the maximum $f_\iota$ any cohort of that size selected from the same applicant pool. For the avoidance of doubt, a proof of these approximation bounds is presented in Theorem \ref{thm:greedy-approximation} in Appendix \ref{app:greedy-proof}. Thus, the Greedy Frontier Optimisation algorithm returns points on a curve that $\bigl( 1-\frac{1}{e} \bigr)$-approximates the true SPF\footnote{In practice, the outputs of the greedy algorithm do not always themselves form a convex curve. We remove produced points that do not sit on the convex curve.}. We note that this is a worst-case approximation ratio and that the actual approximation ratio may be much better.

\section{A Field Study with Rise}\label{sec:spfresults}

We apply our methodology to evaluate our technology in a field deployment with Rise's Cycle 2023. Through this deployment, we document evidence that Rise selected finalists within the SPF -- consistent with the first prediction of our model -- and that Rise selected much closer to the SPF after they were given an SPF estimate to aid in the selection of their third cohort.

\paragraph{Evaluating Past Selection Decisions} Before implementing our technology, we use the methodology described in Section \ref{ssec:measurement} to determine the efficiency of past selection decisions. In particular, we analyse the finalist selection portion of the 2021 and 2022 application cycles, where the programme must construct a cohort of at most $500$ applicants from a pool of roughly $2000$. 

This analysis requires two steps: (1) applying Algorithm \ref{alg:frontier} to both cohorts to estimate the SPF and (2) comparing the actual talent and diversity levels of the finalist cohort to the estimated SPF. The model we developed in Section \ref{ssec:measurement} would suggest that this comparison should find that the chosen cohorts are on or near this frontier. 

\begin{figure}[htbp]
    \centering
    \begin{subfigure}[b]{0.4\textwidth}
        \includegraphics[width=\textwidth]{spf/yr1_spf_finalist.png}
        \caption{The SPF for the 2021 finalist selection process. In Cycle 2021, cohort diversity could have been improved by $15.2\%$ without any reduction in cohort performance, and cohort performance could have been improved by $15.6\%$ without any cost to diversity.}
        \label{fig:spf_2021}
    \end{subfigure}
    \hfill
    \begin{subfigure}[b]{0.4\textwidth}
        \includegraphics[width=\textwidth]{spf/yr2_spf_finalist.png}
        \caption{The SPF for the 2022 finalist selection process. In Cycle 2022, cohort diversity could have been improved by $13\%$ without any reduction in cohort performance, and cohort performance could have been improved by $19.6\%$ without any cost to diversity.}
        \label{fig:spf_2022}
    \end{subfigure}
    \caption{These figures depict the SPFs we estimate for the 2021 and 2022 finalist selection processes. The y-axis represents the diversity score while the x-axis represents average cohort performance (i.e. project scores). The green curve is our estimate of the cycle SPF, which represents the upper bound of diversity that is achievable at every level of cohort performance. The red dot depicts the actual level of diversity and performance of the finalists that were selected. The vertical and horizontal dashed red lines represent the maximum Pareto gain that was possible along the diversity and performance dimensions respectively. These figures are reproduced at a larger scale in Appendix \ref{app:spffigures}.}
    \label{fig:spf_2021_2022}
\end{figure}

The results from these two steps are depicted in Figure \ref{fig:spf_2021_2022}. Surprisingly, neither the 2021 nor 2022 finalist cohorts are chosen on the frontier. (We confirm that these apparent gaps between frontiers and chosen cohorts are statistically significant using a permutation test in Figure \ref{fig:permutation_tests}.)

These results confound the simple model from Section \ref{ssec:measurement}, which suggests that organisations should always select cohorts on the frontier, as all cohorts within the frontier are dominated by cohorts that are either at least as diverse and more talented or at least as talented and more diverse.

\paragraph{Evaluating Selection Decisions with Decision Support}
Now we turn to analysing what happened to selection in the talent investment programme when they were given access to our DST in Cycle 2023. Again, we first estimate the SPF. However, rather than immediately comparing chosen finalists to this estimate, we instead construct a functional implementation of Prototype \ref{fig:diversity} using this estimate.

\begin{figure}[htbp]
    \centering
    \caption{This figure displays the SPF-based DST provided to Rise selectors in Cycle 2023. In addition to the SPF itself, selectors were given access to a myriad of supporting information. While this information cannot all be presented here, much of it describes the candidate optima (i.e., the cohorts represented by the coloured dots). In particular, selectors were interested in the spread of performance scores in each cohort, as well as the extent to which each cohort satisfied programme diversity targets.}
    \label{fig:spf_dst}
    \includegraphics[width=.9\textwidth]{spf/spf_dst.png} 
\end{figure}

Selectors were provided with this DST to inform their decision-making process. The tool, as depicted in Figure \ref{fig:spf_dst}, presented the estimated SPF curve along with several pre-calculated candidate cohorts (the ``coloured dots''). For each of these candidate cohorts, selectors could review supporting information, including the distribution of performance scores and the extent to which various programme diversity targets were met. This allowed them to evaluate these specific, pre-defined options and understand the associated trade-offs. After selecting a desired cohort along the SPF, this cohort is used to inform a shadow price $\iota$. Participants were then repeatedly shown all available information on the next best applicant to add to the cohort according to chosen shadow price $\iota$ and asked to rule that candidate in or out.\footnote{Some details on information available to participants are provided in Appendix \ref{ssec:rise}, but most was excluded at the request of the programme.} This process was repeated until the desired size was reached. After the selection process was completed with the aid of this tool, we then compare the actual finalist cohort's diversity and talent to the SPF estimate.

% [FIXED] The experiment design needs significantly more information about how the DSTs were used: what were the participants given? Just the graph plus the info? Are these the only cohorts? What happened if they wanted to substitute people etc.? Could they get feedback on these? In short, this is not replicable as is – just requires more detail to make it so.

The results from this analysis are depicted in Figure \ref{fig:spf_2023}. Here we see notable differences in the selection patterns relative to Cycles 2021 and 2022. In particular, the Cycle 3 finalist cohort is nearly on the SPF, making the possible Pareto improvements in both directions no more than $2\%$. This suggests two things. First, it provides further evidence that selection decisions in Cycle 2021 and Cycle 2022 were, in fact, inefficient; had Rise known about the possibility of making Pareto improvements relative to their stated preferences, they likely would have changed their behaviour. Second, it provides evidence that the DST presented here actually influences the decisions of selectors. 

\begin{figure}[!htb]
    \centering
    \caption{This figure displays the SPF for the Cycle 2023 finalist cohort. Again, the y-axis represents the diversity score while the x-axis represents average cohort performance, the green curve is our estimate of the SPF, and the red dots depict the actual level of diversity and performance of the finalists that were selected. In this case, we overlay the finalist cohorts from 2021 and 2022 to provide a point of comparison. The diagonal dashed red line represents the distance in diversity-performance space between the Cycle 2022 cohort and the Cycle 2023 cohort. In Cycle 2023, there are no significant Pareto improvements in either diversity or performance.} 
    \label{fig:spf_2023}
    \includegraphics[width=.9\textwidth]{spf/yr3_spf_finalist.png}
\end{figure}

To determine whether the improvements are statistically significant, we leverage a permutation test depicted in Figure \ref{fig:permutation_tests}. The key comparison is between Cycle 2023 max Pareto improvements in talent and diversity (the solid blue vertical lines in both panels) and the corresponding 95 percentile of the random difference distributions (the dashed black vertical line). For both dimensions, the possible improvements are statistically insignificant. In contrast, Cycle 2021 and Cycle 2022 both display statistically significant max Pareto improvements. Ultimately, though this confounds the predictions about selector behaviour implied by the model in Section \ref{ssec:measurement}, it does suggest that the DST is effective in improving selection decisions.

\begin{figure}[htbp]
    \centering
    \caption{This figure displays permutation tests comparing the potential Pareto improvements along the diversity and talent dimensions to the distribution of differences on both dimensions from 1000 randomly drawn pairs of cohorts. The dashed black vertical line represents the 95 percentile of these differences. The solid vertical lines represent the maximum Pareto gain on performance and diversity in each application year. We interpret inefficiencies at or larger than the 95 percentile of the distribution as significant, thereby sticking to the conventional $\alpha$ value. While Cycles 2021 and 2022 both appear to have significant inefficiencies, Cycle 2023 does not.}
    \label{fig:permutation_tests}
    \includegraphics[width=.9\textwidth]{spf/permutation_tests.png} 
\end{figure}

\section{A Plausible Explanation for Selection Inefficiencies}\label{sec:spfexplanation}
\subsection{Why Are Organisations Selecting Pareto Inferior Cohorts?}\label{subsec:dts_nphard}
The results of our field study, specifically w.r.t. the 2021 and 2022 application cycles, suggest that organisations are not selecting cohorts on the SPF. This is surprising, as the SPF model suggests that organisations should always select cohorts on the frontier, as all cohorts within the frontier are dominated by cohorts that are either at least as diverse and more talented or at least as talented and more diverse. In conversation with Rise, we have come to two plausible explanations for why this might be.

First, it might be that the axes of performance and diversity fail to capture the full dimensionality of organisational preferences. One mechanism for this, supported by the revelation from Chapter \ref{ch:diversity} that programmes desire representations of specific idiosyncrasies, is that our axes fail to capture an idiosyncratic preference possessed by Rise selectors. Another mechanism, also supported by Chapter \ref{ch:diversity}, may come into play if the organisation's quantifications of talent do not capture the full scope of their concerns; i.e., if part of the measured ``talent'' an organisation selects for is only captured qualitatively, it cannot be factored into our model.

Second, however, it may be that there is an inherent cost associated with approaching the frontier. This would make organisational selections in the 2021 and 2022 application cycles second-best, in that they are optimal according to a model placing certain costs on their selections. 

The selection decisions in Cycle 2023 favour our second explanation; if the organisation had uncaptured preferences, we would expect these to play a similar role in Cycle 2023, and to result in an apparently suboptimal cohort selection. However, the cohort selected is, according to our model, Pareto optimal.\footnote{More specifically, our chosen cohort is not significantly within the frontier.} Thus, we must seek out the cause of this inherent cost.

\subsubsection{Diversity Causes Complexity}\label{subsubsec:nphard}
The most plausible source of cost in approaching the frontier is the impracticality of selection teams discovering the frontier by hand. This is particularly sensible, as we prove here that calculating the frontier outright is $\mathbf{NP}$-hard; that is, there are no known efficient algorithms that can calculate the SPF outright \cite{COPPERSMITH198527}.

To see why intuitively, consider a college that aims to accept some target fraction of Black applicants from a pool of Black and White applicants. And, assume the school wants to select as talented a class as possible, where talent is proxied for by test scores, grades, or some combination of the two. In this special case, as shown in \textcite{kleinberg2018algorithmic}, there exists a computationally easy algorithm to calculate the SPF shown below.\footnote{Technically, the algorithm presented by \textcite{kleinberg2018algorithmic} only optimises for the \emph{most diverse} point on the SPF. Thus, we have added the \textbf{Repeat} step (cycling through the algorithm with different representation thresholds) to enable their algorithm to trace out the SPF.}

\begin{algorithm}
    \caption{A Procedure For Calculating the SPF Based on \textcite{kleinberg2018algorithmic}}\label{alg:kleinberg}
    \begin{algorithmic}
        \State \textbf{Define} a minority group (Black) and mutually exclusive majority group (White), 
        \State \textbf{Rank} applicants within their group by test score,
        \State \textbf{Define} a target proportion of Black admits,
        \State \textbf{Select} Black applicants from the highest ranking down until the target is reached,
        \State \textbf{Select} the White applicants from the highest ranking down for the remaining slots,
        \State \textbf{Repeat} steps 1-5 for different thresholds of representation to trace out the SPF.
    \end{algorithmic}
\end{algorithm}

What allows this algorithm to work is the mutual exclusivity of the minority and majority groups. This allows one to transform the diverse talent selection problem into two separate talent maximisation problems where the organisation simply selects the most talented members of each group. This can be extended to any case where target proportions are defined at the level of mutually exclusive group level, even when multiple different demographic dimensions are considered. To be concrete, if the organisation cares about race and gender and, thus, has target proportions for black male, black female, white male, and white female applicants, then the problem can be broken into four separate talent maximisation problems where the most talented members of each group are selected until the target proportions are met for each group. 

But what happens if an organisation has preferences for the representation of non-mutually exclusive groups? (I.e., what if an organisation places nonzero weight on two proportional diversity functions?) To continue the running example, this would be analogous to a college that has target proportions for black applicants and female applicants, but not for each race by gender combination. This seemingly small change prevents an organisation from transforming the problem into simpler group-specific talent maximisation sub-problems. To see this, consider applying the \textcite{kleinberg2018algorithmic} algorithm to each group sequentially; this would mean selecting the best black applicants until reaching the target proportion, then doing the same for female applicants. If the most talented black applicants were male or if there were few talented white females in the pool, having allocated the black slots in this way forces the college to select less talented females than optimal (or, it may inhibit reaching the target proportion for females at all).\footnote{An alternative strategy might iterate over different intersectional targets that satisfy the original two targets; this strategy still suffers from non-polynomial growth.} In short, when diversity preferences are over non-mutually exclusive groups, we cannot cleanly and efficiently break the problem into simple talent maximisation subproblems for disjoint minority groups, so it is not clear how we might extend the \textcite{kleinberg2018algorithmic} algorithm to a general version of the diverse talent selection problem. 

The general diverse talent selection problem allows organisations to have preferences for the representation of an arbitrary number of overlapping (or disjoint) demographic groups. This aligns more closely with the diversity preferences of real-world organisations like colleges, firms, and social impact programmes, many of which aim to select personnel from various ethnicities, genders, classes, geographies, ideologies, and specialities. Organisations generally state their preferences using statements of the following form: ``the organisation desires at least $x\%$ of group $g$'' or ``the organisation desires at least one person from $m$ groups''. These types of diversity preferences are what is formalised in the function $D(c)$, which forms an integral part of the class of functions $F$; we demonstrate here that calculating $F$ is $\mathbf{NP}$-hard.\footnote{We do this via `reduction' to the Vertex Cover. A reduction is simple: $A \leq B$ (i.e., $A$ reduces to $B$) if and only if there exists a polynomial time algorithm that makes some polynomially bounded number of calls to $B$ and thus returns an answer to $A$. In other words, we say that $A$ is $\mathbf{NP}$-hard if and only if $\forall B \in \mathbf{NP} A \leq B$. It is clear to see, then, that if $B$ is $\mathbf{NP}$-hard and $A \leq B$, then $A$ is also $\mathbf{NP}$-hard. For more details on reductions, see \textcite{10.5555/1074100.1074233}.}

We now prove that, for the class of functions $F$ of the form from Equation \ref{eq:f_spec}, the problem of finding the optimal subset of size $k$ for any $f_i \in F$ is still computationally complex. This time, we rely on the assumption that $\mathbf{NP}$-hard problems are computationally complex. That is, Theorem \ref{thm:specific-nphard} holds. 

\begin{theorem}\label{thm:specific-nphard}
    Let $U$ be a `universe' set of size at least $N \geq 2*n$ and $F = \{f: \mathcal{P} (U) \rightarrow \mathbb{R}\}$ be the set of functions described in Equation \ref{eq:f_spec}. Then $Opt_{spec}(f_i, n) := argmax_{c \in U \land |c| = n}(f_i(c))$ is $\mathbf{NP}$-hard in $n$.
\end{theorem}

To do this, and to justify the significance of this result, we bring in the computational complexity of the Vertex Cover problem, which has been proven to be $\mathbf{NP}$-hard \cite{COPPERSMITH198527}. Vertex Cover can be seen in Theorem \ref{thm:vertexcover}.

\begin{theorem}\label{thm:vertexcover}
    Let $G = (V, E)$ be a graph. Let $VC(G, \kappa) := Cov | Cov \subseteq V \land |Cov| = \kappa \land \forall e \in E . \exists v \in Cov . v \in e$ be a function of $G$ that returns a set $Cov$ such that every edge in $G$ is incident on at least one vertex in $Cov$. Then $VC$ in $\mathbf{NP}$-hard in the number of vertices.
\end{theorem}

We now prove Theorem \ref{thm:specific-nphard} by reduction to Theorem \ref{thm:vertexcover}, assuming that there exists no polynomial time solution to Vertex Cover \cite{COPPERSMITH198527}.

\begin{proof}
Suppose for a contradiction that Theorem \ref{thm:specific-nphard} admits some polynomial-time solution $Alg_{spec}$. I.e., $Alg_{spec}(s_i, k )= argmax_{c \in U \land |c| = n}(s_i(c))$.

\begin{algorithm}
    \caption{An Algorithm for $VC(G = (V,E), \kappa)$}\label{alg:vc_spec}
    \begin{algorithmic}
        \State \textbf{Consider} $U := E$
        \State \textbf{Define} $\vec{\mathbf{g}} := \{g_i = v_i \in e | e \in E \land i \in |V|\}$ such that each $g_i$ has length $E$ and corresponds to whether an edge is incident on vertex $v_i$.
        \State \textbf{Return} $Opt_{spec}(1*D(c, \vec{\mathbf{1}}, \vec{\mathbf{0}}, \vec{\mathbf{0}}, \vec{\mathbf{g}}, \vec{\mathbf{0}})+ 0*P(c)) \geq k$
    \end{algorithmic}
\end{algorithm}

Then consider the algorithm $Alg_{VC}$ that is defined in Algorithm \ref{alg:vc_spec}. But this algorithm solves Vertex Cover in polynomial time relative to $Opt_{spec}$ and thus is a polynomial time solution to Vertex Cover. Assuming $\mathbf{P} \neq \mathbf{NP}$, contradiction!
\end{proof}

\subsection{Embedding Complexity into the Model}\label{subsec:dts_w_complexity}
Knowing the $\mathbf{NP}$-hardness of calculating the SPF outright, we can more comfortably assume that there exists a search cost in approaching the frontier; knowing that this $\mathbf{NP}$-hardness is driven by diversity targets, we can further suppose that this search cost is driven by diversity preferences. This leads us to a new model that incorporates complexity costs into the selection problem.

Consider a variation of the simple cohort selection problem (see Section \ref{ssec:measurement}) where the organisation can search for increasingly diverse cohorts at a cost. Let the amount of search effort be $e\in[0,1]$ and define the cost of search effort to be $\alpha p(e)$ where the cost is convex (i.e. $p_e>0$ and $p_{ee}>0$) and $\alpha$ is a constant that is inversely related to the quality of search technology available. Furthermore, we let the amount of search deterministically increase the maximum achievable diversity at each talent level, which is now given by $D^{SPF}*e$. The optimisation problem can then be rewritten as:

\begin{align}
&\max_{d,p,e} F\Big(d,p\Big) - \alpha p(e) \text{ \bf{ s.t. } } d = G(p)*e, \nonumber \\ 
& \implies \max_{p,e} F\Big(\underbrace{G(p)*e}_{\text{Info Cost}} ,p\Big) - \underbrace{\alpha p(e)}_{\text{Direct Cost}}. \label{eq:objective}
\end{align}

It is clear from the form of the organisation's new objective function in Equation \ref{eq:objective} that the complexity of maximising diversity imposes two kinds of costs: an information cost that represents the fact that the organisation will generally not know which cohort is on the SPF and a direct search cost. Also, when search costs are set to zero (i.e. $\alpha=0$), the problem collapses into the original problem because searching is costless and, therefore, maximised at $e=1$. But, when $\alpha>0$, the optimal cohort will now be inside of the SPF. This is because, for all $c'$ such that  $D(c')=G(p(c')=p')*e$ there exists $c^f$ on the SPF such that $D(c^f)=G(p(c^f)=p')$. Thus, as long as the optimal effort is below 1, any solution to this problem will result in selecting a cohort that is within the SPF and, therefore, non-first-best. The solution to the selection problem with complexity is depicted in Figure \ref{fig:model_complex}.

\begin{figure}[!htb]
    \centering
    \caption{This figure depicts an example solution to an iteration of the selection problem with complexity-induced search costs, which is described in Equation \ref{eq:objective}. As in Figure \ref{fig:model_spf}, the solid blue curve represents the SPF, the dotted blue curve represents the organisation's indifference curve corresponding to the highest achievable utility without search costs, and the blue dot represents the diversity and performance of the first-best solution. Additionally, the solid red curve represents the accessible frontier with optimal search, the dotted red curve represents the highest achievable utility with search costs, and the red dot represents the diversity and performance of the optimal cohort with search costs (i.e. the second-best solution).}
    \label{fig:model_complex}
    \includegraphics[width=1\textwidth,height=\textheight,keepaspectratio]{spf/model_complex.png} 
\end{figure}

Additionally, the extent of the inefficiency will tend to reduce as complexity costs reduce. We can see this by examining the comparative statics of the model. To simplify our derivation of the relevant comparative static, we refer to the organisation's objective function as $O(p,e)\equiv F\Big(G(p)*e ,p\Big) - \alpha p(e)$. Furthermore, we use subscripts on functions to refer to partial derivatives and we drop the arguments of functions where this does not confuse. The (necessary) first-order conditions from this model are, therefore, the following:

\begin{align}
O_p & \equiv F_{d}(G(p)e,p)G_p(p)e + F_p(G(p)e,p) = 0 \nonumber \\
O_e & \equiv F_{d}(G(p)e,p)G(p) - \alpha p_e(e) = 0. \nonumber
\end{align}

To ensure this is a maximum, we also need to assume that the (sufficient) second-order conditions hold. They are the following:

\begin{align}
O_{pp} \equiv &\;  e^2G_p^2F_{dd} + 2eG_pF_{dp} + eG_{pp}F_d + F_{pp} < 0, \nonumber \\
O_{ee}  \equiv &\;  G^2F_{dd} - \alpha p_{ee} < 0, \nonumber \\
&  \; O_{ee}O_{pp} - O_{ep}^2 > 0, \nonumber
\end{align}

\noindent where $O_{ep} \equiv O_{pe} \equiv eGG_pF_{dd} + F_dG_p + GF_{pd}$. Under these conditions, solutions to the first-order conditions both exist and guarantee a maximum. These solutions can be defined as $p^*(\alpha)$ and $e^*(\alpha)$. If we plug this into the first-order conditions and take a derivative with respect to $\alpha$, which governs the complexity costs, we get the following system of equations:

\begin{align}
& O_{pp}\frac{\partial p^*}{\partial \alpha} + O_{pe}\frac{\partial e^*}{\partial \alpha} + O_{p\alpha} \equiv 0, \nonumber \\
& O_{ep}\frac{\partial p^*}{\partial \alpha} + O_{ee}\frac{\partial e^*}{\partial \alpha} + O_{e\alpha} \equiv 0  \nonumber
\end{align}

\noindent where $O_{e\alpha} = -p_e$ and, essential for signing the comparative static, $O_{p\alpha} = 0$. We can then solve for $\frac{\partial e^*}{\partial \alpha}$ algebraically (or using Cramer's rule), which gives the following:

\begin{equation}
\frac{\partial e^*}{\partial \alpha} = \frac{-O_{e\alpha}O_{pp}}{O_{ee}O_{pp} - O_{ep}^2} + \cancelto{0}{\frac{O_{ep}O_{p\alpha}}{O_{ee}O_{pp} - O_{ep}^2}} = \frac{p_eO_{pp}}{O_{ee}O_{pp} - O_{ep}^2} < 0, \label{eq:comp_stat}
\end{equation}

\noindent where the the final inequality holds because of the signs assumed in the first and third second-order conditions. Thus, as complexity costs rise, optimal search effort decreases.

This model, thus, implies two predictions about organisational behaviour: (1) when complexity-induced search costs are sufficiently high, organisations will select cohorts within the SPF and (2) as computational costs are reduced, organisations will select cohorts that are closer to the SPF. We have already seen in Section \ref{sec:spfresults} that both predictions hold in practice.

\section{Alternative Applications of the SPF}\label{sec:spfapplications}
The main body of this chapter implements and evaluates Prototype \ref{fig:diversity} as an in-process DST by conducting Action Research with the Rise programme. However, in this section, we discuss potential ex-post applications of the SPF.

\paragraph{Comparing the Diversity Cost of Alternative Talent Measures} In some cases, organisations may have multiple alternative talent measures that seem equally valid as measures of individual ability. In this case, the tradeoff between each measure and diversity may help an organisation decide which talent measure they prefer.  Two cases where this might be relevant are in hiring and college admissions. In hiring, firms may have multiple measures that predict applicant productivity, but have many ways to weigh the various measures that are roughly equivalent for productivity prediction \cite{hartigan_fairness_1989}. This can happen if productivity is multidimensional (e.g., work per hour, tenure, spillovers on others, etc.), and different measures are correlated with some dimensions and not others. A similar problem can be found in college admissions, where, again, the college has multiple measures of applicant ability and may be close to indifferent about some set of ways of combining them when judging an applicant's talent \cite{tam2002new}. 

In these cases, the SPF estimation procedure allows an organisation to consider another dimension: which talent measures demand the sharpest tradeoffs against cohort diversity? I.e.: which measures yield the smallest SPFs? This is particularly relevant in cases where preferences may be lexicographic, meaning that an organisation wants to maximise talent first, then, conditional on doing so, choose the cohort among top talent cohorts that is the most diverse possible. It also is relevant in contexts where an organisation is not allowed, either legally or internally, to explicitly prioritise diversity in its selection criterion, but still wishes to promote diversity \cite{Bleemer_2023}. 

We demonstrate how to use SPF estimation to compare the diversity tradeoffs of two alternative talent measures. To do this, we estimate the SPF twice, once using each of the talent measures, and then compare the level of diversity at each percentile of both measures. In the case of indifference between the two talent measures on the talent dimension, the measure with higher maximum achievable diversity in the relevant percentile range should be chosen if the organisation cares at all about diversity. We use two measures of talent collected by Rise: a project-based measure and a traditional score. The results of this comparison are depicted in Figure \ref{fig:compare_div_tradeoffs}. Given that the programme does care about diversity, this would justify using project quality instead of the traditional score for selection. 

\begin{figure}[htbp]
    \centering
    \includegraphics[width=\textwidth]{spf/alt_merit_spfs.png} 
    \caption{This figure displays the SPF we estimated for the Cycle 2021 finalist cohort and an SPF based on a more traditional method of measuring performance (i.e. the average of cognitive ability and an essay assessment). The y-axis represents the diversity score while the x-axis represents the average cohort performance on projects or the traditional score. The vertical distance between the SPFs represents the difference in maximal diversity conditional on a cohort performing at a particular percentile of both scores. We see here that, above the 90th percentile of talent for both measures, the project quality measure strictly dominates the traditional score in diversity.}
    \label{fig:compare_div_tradeoffs}
\end{figure}
        
This method can also be applied to compare the diversity-talent tradeoff across application years. To do this, simply estimate SPFs for each application cycle and compare the level of diversity at each percentile of talent. As long as the diversity goals remain the same each year and cohort diversity is renormalized such that the most diverse cohort across all years becomes 1, organisations can compare across years to see whether differences in applicants across years better afford to get closer to their goals. Figure \ref{fig:diversity_across_cohorts} shows just this comparison. In general, the Cycle 2023 SPF allows for selecting more diverse cohorts at every level of talent than the other two cohorts. But, whether Cycles 2021 or 2022 allow for more diversity depends on where in the talent distribution the programme is interested in. Near the top of the talent distribution, Cycle 2022 has more diverse cohorts, but this flips as talent falls below the 94th percentile.

\begin{figure}[!htb]
    \centering
    \includegraphics[width=\textwidth,height=\textheight,keepaspectratio]{spf/spf_cohort_comparisons.png} 
    \caption{ This figure displays the SPFs we estimate for three finalist cohorts. The y-axis represents the diversity score while the x-axis represents average cohort performance (i.e. percentiles of mean project scores). The diversity target is held constant across cohorts, so differences in SPFs conditional on performance represent differences in the capacity to reach the same diversity target at a given level of performance.}
    \label{fig:diversity_across_cohorts}
\end{figure}

\paragraph{Evaluating Alternative Selection and Screening Approaches} Relatedly, organisations may consider using cheaper, but lower quality measures of talent to screen or select applicants. For example, firms may consider using metrics (e.g., cognitive or personality assessments) or recruiters to screen their applicants rather than allow each applicant to be assessed via an interview. In some cases, organisations may be considering replacing costlier selection measures and selecting applicants entirely based on cheaper information. Unlike before, we now assume that the initial metric captures talent much better than the new metric. Thus, rather than comparing different SPFs, we place cohorts selected using new metrics on the SPF estimate drawn using the original metric. If the new metric is not too much worse than the original metric, then the new metric may be a better choice.

Running selection counterfactuals can be done using two types of designs: the first we will refer to as a ``causal'' design and the second we call a ``suggestive'' design. A causal design requires an organisation to run a screening experiment where applicant talent is either evaluated randomly (or all applicants are evaluated). This allows organisations to avoid the selective labels problem whereby results become biased due to the selection of who gets evaluated and who doesn't. Avoiding this problem allows organisations to analyse representative samples, meaning that comparisons between alternatively selected samples and the estimated SPF should extend to the full population of applicants. Thus, barring any significant contextual changes, the results will be the same (in expectation) when used on another applicant pool (in this sense, the results are ``causal''). Alternatively, a counterfactual exercise can be conducted on selected data where only a selected set of individual talents are assessed. In this case, the applicability of the results to another applicant pool is merely ``suggestive'', hence the name ``suggestive design''. Causal designs, though more useful for decision-making, are also more costly, as they require running a selection experiment, which may force organisations to miss out on talent (additionally, using known-inferior selection methods may pose a fairness concern). 

To demonstrate, we return to the talent investment programme example where, in Cycle 2021, we can assess alternative selection procedures using a causal design. This is because, in Cycle 2021, the programme ran a selection experiment to determine whose projects were reviewed. In particular, the programme used a weighted sum of applicants' cognitive ability and peer assessments of their video essays to select the top $1500$ applicants who would receive project reviews. Of the remaining applicants, $500$ were chosen at random to be evaluated as well. This means that, from the total application pool of $2800$, a representative sample can be reconstructed by re-weighting the $500$ randomly assessed applicants such that they represent all $1300$ applicants who were below the project review threshold.

To demonstrate the use of the SPF estimate for counterfactual selection analysis, we compare the efficacy of three alternative selection strategies: the cognitive score, the traditional score, and the peer score. Because the cognitive score and the traditional score both use measures that are closely related to typical talent measures, this comparison also serves as a substantive comparison of traditional selection methodologies and more experimental ones, such as using applicant peer review. The results of this comparison are depicted in Figure \ref{fig:alt_screen}. Here we see that, on the dimension of talent (as measured by project quality) the cognitive ability score performs the worst of the three by far (over $10\%$ worse than the other two scores). The traditional score performs slightly better than the peer score on talent, but the peer score (and the cognitive ability score) performs slightly better than the traditional score on diversity. What is perhaps most striking, however, is that all three alternative selection approaches result in cohorts well within the frontier, indicating that Rise's actual metric far outstrips each hypothetical alternative.

    \begin{figure}[!htb]
    \centering
    \includegraphics[width=\textwidth]{spf/alt_screening_performance.png} 
    \caption{This figure displays various SPF estimates for the finalist cohort using the programme's notion of diversity, a `disadvantage' notion of diversity (drawn from the `contextualising applications' theme in Chapter \ref{ch:diversity}), and a `representativeness' notion of diversity (i.e., Prototype \ref{fig:representativeness}). The y-axes represent diversity scores while the x-axes represent average cohort performance. The green curves are our estimates of three alternative Cycle 2021 SPFs, which are estimates of the upper bound of diversity that is achievable at every level of cohort performance. Each dot represents the performance and diversity of cohorts had they been selected using only cognitive ability (blue), a combination of written essay judgements and cognitive ability (aka a ``traditional'' score, which is green), and just peer review (red).} \label{fig:alt_screen}
    \end{figure}

\paragraph{Evaluating According to Alternative Notions of Diversity} A similar process can help organisations understand the practical implications of different kinds of preferences over types of diversity. While we isolate three themes relating to definitions of diversity in Chapter \ref{ch:diversity}, we note that the Rise programme has a working understanding of what they mean by diversity, and did not wish to adopt any of these notions. In practice, Rise's diversity targets suggest both `representativeness' and `contextualising applications' (which they internally call `disadvantage' or `boostability', variously) as important to their consideration of diversity, while `different perspectives' do not appear in their decision-making process. Thus, Figure \ref{fig:alt_screen} also depicts results using two alternative notions of diversity based only on the disadvantage and representativeness portions of the Rise targets. The disadvantage diversity score puts maximal weight on representing those from various historically disadvantaged groups (e.g., being first-generation, poor, or female) while the representativeness diversity score only uses proportional targets that match the demographic distribution of the applicants. Evaluating each alternative selection method indicates that, while selecting on peer judgements or the traditional score both do substantially better than using cognitive ability alone on the talent dimension, using the peer score is by far the highest performing on both disadvantage and representativeness.

\section{Conclusion} \label{sec:conclusion}
While Chapter \ref{ch:diversity} focused on the theoretical and empirical aspects of diversity, this chapter has focused on the practical implications of diversity in selection. In doing so, we have introduced the notion of a selection possibilities frontier via a simple model of diverse talent selection, have implemented Prototype \ref{fig:diversity} as an in-process DST, and have demonstrated its use in practice. Analysing decision-making with and without our DST, we have shown that Rise selects talented and more diverse cohorts when given access to our SPF-based DST. To explain why, we showed that the diverse talent selection problem is $\mathbf{NP}$-hard and augmented our model with a notion of complexity costs; this new model predicts that organisations who are better and more cheaply able to approximate the frontier should find themselves closer to it. 

Finally, we have also shown that the SPF can be used ex-post to compare the diversity tradeoffs of alternative talent measures, evaluate alternative selection and screening approaches, and evaluate according to alternative notions of diversity. 

In an age of rapidly expanding interest in selecting from diverse talent pools (signalled by the growth of DEI), this chapter has wide-ranging policy implications. First, the chapter suggests that organisations will face extreme difficulty achieving their diversity goals unless they are willing to adopt more sophisticated selection technology. Second, this chapter contributes methods that are particularly useful for assessing the diversity impacts of alternative merit-based selection strategies. This extends beyond selection to related contexts like hiring, where not appropriately considering the diversity implications of selection strategies can result in lawsuits, and U.S. university admissions' non-merit-based selection has become a legal grey area despite university commitments to diversity. 
\chapter{\label{ch:discussion}Discussion}

\minitoc 

\section[The Role of AI Systems in Selection]{The Role of AI Systems in Selection: From Decision Support to Selection-Oriented AI}
\subsection{Current Challenges in AI for Selection}
\subsubsection{To Support or Supplant?}
AI tools have long been posed as both replacements and support systems for decision-makers both without and within selection processes \cite{barocas_big_2016,jacobs_how_2021,hildebrandt_law_nodate,yarger2020algorithmic,mattu_how_nodate}. Naïvely, implementations of AI systems supplant human decision-makers, often to disastrous effect \cite{mattu_how_nodate}.

\subsubsection{Who to Support?}
More human-centric AI systems, e.g., explainable AI (xAI), often instead serve as decision support tools (DSTs), and place the stakeholder (the selector) at the core of the decision-making process. These systems then focus on enhancing the experience of human decision-makers, offering tools that satisfy the subjective desiderata of selectors. \textcite{Lipton} critiques post-hoc notions of explainability because, in seeking to satisfy stakeholder desiderata, xAI tools may prove more misleading than insightful. While Chapter \ref{ch:xai} extends this critique to the practice of post-hoc justification more broadly, we find it applicable to all AI DSTs.

In addition to this, the problem of selection involves a complex interplay of interests. On the one hand, selection organisations are driven by the need to identify the best possible cohort of candidates, aiming to optimize organisational performance and success. On the other hand, candidates are motivated by the desire to be selected for opportunities that match their potential and aspirations. In practice, these two interests are at odds, and while DSTs might support one or the other, the need to support both decisions outstrips the capacity of any individual DST. But beyond these two core stakeholders lies a broader societal interest in the outcomes of selection processes. Society at large has a vested interest in ensuring that these processes uphold essential social values such as fairness, diversity, integrity, and justice. In the case of pro-social programmes such as global scholarships that desire that scholars do good with their careers, society similarly has an interest in ensuring that the most apt scholars are selected.

Centring these DSTs solely on the needs of either applicants or decision-makers within organisations would both suffer from the problem of subjective desires outlined by \textcite{Lipton} and fail to address the larger social implications of these selection decisions.\footnote{I.e., algorithms designed to streamline applicant evaluations may inadvertently reinforce existing biases, thereby undermining efforts to promote diversity and inclusion. Similarly, AI systems may prioritise ease of use and decision-making speed at the expense of fairness and transparency.} Thus, there is a need for DSTs that, rather than centring on a group of stakeholders, orient themselves around the task of selection itself.

\subsubsection{What to Support?}
A common DST paradigm sees practitioners making a series of similar kinds of decisions on different cases, e.g., deciding whether to grant a loan many times \cite{GiveMeSomeCredit,barocas_hidden_2020,ustun_actionable_2019,Rebitschek_Gigerenzer_Wagner_2021,10.1111/j.1467-954X.2007.00740.x}. However, we have explored here many challenges and decision points (enumerated in Table \ref{tab:full_decision_list}) not captured by this notion of decision-making. I.e., the conventional DST paradigm, wherein practitioners are supported for each decision, thinks only of a specific kind of in-process decision. It excludes both other in-process decisions and all ex-post decisions. Thus, any paradigm for DSTs oriented around the task of selection should seek to support all kinds of decisions.

\subsection{Proposing a New Paradigm: Selection-Oriented AI (SOAI)}
In response to these challenges, this thesis proposes a novel paradigm: SOAI. SOAI reimagines the role of AI systems in talent identification, and advocates for a shift away from the human-centric framework toward a hybrid selector-centred and selection-driven approach (wherein the design of AI systems is grounded in the social values that selection processes ought to uphold, but the values are supplied by the selectors making the decisions). While selectors remain important users of these systems, they are not the sole focus of design efforts. Instead, SOAI emphasises the importance of evaluating the broader social goals of AI-driven selection processes and seeks to help selectors achieve these goals \cite{batyavalue}.

The shift toward SOAI represents a necessary evolution in the use of AI systems in talent identification. By prioritising the social values that selection processes ought to reflect, SOAI challenges the current practitioner-centred approach and introduces a new standard for evaluating the success of AI in decision support. The implementation of SOAI can provide a path forward in creating AI systems that not only assist in the identification of talent but also ensure that the processes by which talent is selected are fair, just, and inclusive.

\section{Design Recommendations for SOAI Designers}
\subsection{Design for Specific Social Values}
While we wish to design to support all social values in the selection process, Chapter \ref{ch:diversity} demonstrates the difficulty of unpacking the social value of diversity; in Chapter \ref{ch:diversity}, we find success instead focusing on smaller component values that comprise diversity. Similarly, each decision point supported in Chapter \ref{ch:genai} implies a specific ontology about the role of generative AI in selection; these, too, stem from specific social values promoted by an organisation.

There is support for this from the literature. For example, literature on algorithmic fairness has long wrestled with contradictions between measurements of different kinds of fairnesses \cite{pmlr-v80-kearns18a}. While `individual fairness' draws on procedural notions of justice to ensure that applicants are treated equally regardless of differences in protected or irrelevant characteristics \cite{dwork_fairness_2012}, `group fairness' draws on distributive justice in seeking to achieve parity between different demographic groups \cite{Citron_2008,Olsaretti_2018}. There exists literature attempting to reconcile these notions: \textcite{pmlr-v28-zemel13} attempt to reconcile this in practice by simultaneously optimising for multiple fairness metrics; \textcite{lahoti2019ifairlearningindividuallyfair} seek only to optimise for individual fairness, and yet find increases in group fairness; and \textcite{binns_apparent_2019} contends that standard, blunt implementations of individual fairness should be replaced with a more nuanced formulation compatible with group fairness. Nonetheless, as they are often implemented, these two notions of fairness are often in conflict, and though designing to support both may be possible, it is liable, in a scholarship context, to create unclarity of the sort plaguing diversity, impeding programme desire to assess these concepts with specific targets. 

We suggest this generalises to SOAI practices in general: rather than designing around myriad values, only to find conflicting design implications of these disparate values, designers seeking to support social values in selection processes should focus on specific social values worthy of consideration.

\subsection{Identify Decision Points with the Decision Matrix}
In Chapter \ref{ch:context}, we conceive of selection as a series of decisions. Chapter \ref{ch:genai} expands on this, introducing the Decision Matrix framework to categorise the many decision points that selectors face according to their two most germane axes: the stakes of the decision and its stage in selection. This framework allows designers to reason about groups of decision points in much the same way that the explainable AI community reasons about groups of explainability techniques and to isolate desired or required properties of DSTs based on the taxonomy of the decision point they seek to support and to then determine which categories of decisions different genAI detectors are suitable to support\cite{ford_play_2020,kumar_problems_2020,doshi-velez_towards_2017,friedrich_taxonomy_2011,molnar_interpretable_2019}. 

In Chapter \ref{ch:xai}, we respond to criticisms isolated to \textcite{friedrich_taxonomy_2011}'s `post-hoc' explanations; here, the taxonomic distinctions are used in criticism to expand the scope of individual critiques \cite{barocas_hidden_2020,kumar_problems_2020}. We suggest the Decision Matrix can be used similarly, to discuss and critique decisions in a scholarship context.

However, we caution designers following this design recommendation to ensure that they also follow design recommendation \ref{ssec:real_change} and evaluate real change in addition to subjective desiderata. The Decision Matrix framework can be used to derive a set of a necessary, but perhaps not sufficient, properties of DSTs.

\subsection{Balance Qualitative and Quantitative Information in Presentation}
Human decision-makers often desire both a qualitative understanding of applicants and quantitative metrics to compare them. In Chapters \ref{ch:xai} and \ref{ch:diversity}, we find that selectors from Rise and Ellison Scholars seek to make decisions informed by both kinds of information; despite this, the desired balance between these modes of information varies based both on practitioner and type of decision. When quantitative information is neglected, practitioners are forced to make decisions on a case-by-case basis without important numerical context comparing applicants to a larger group; when qualitative information is neglected, practitioners are unable to consider applicants holistically. Developers following SOAI should consider the balance between quantitative and qualitative information in their systems, and design their systems to provide both when necessary.

We again find parallels in the fairness literature to this balance. Qualitative information enables the selectors' consideration of applicants as individuals, and this combines with the process of holistic review to create full pictures of applicants. \cite{dwork_fairness_2012}'s individual fairness holds a similar lens; rather than looking at applicants in terms of their place in the cohort, this notion of fairness demands equal treatment of applicants as people. However, quantitative information makes possible considerations of distributive notions of justice and group fairness principles \cite{Olsaretti_2018}, as decision-makers have access to the supporting information needed to contextualise applications relative to other members of protected groups. Notably, programmes with different ontologies governing what they consider fair will thus have different preferences considering the balance of quantitative and qualitative information in their systems. (This relationship is not absolute, though, as other differences in programmes may lead to differing priorities.)

\subsection{Evaluate Real Change in Addition to Subjective Satisfaction}\label{ssec:real_change}
\textcite{Lipton} critiques explainable AI (xAI) systems because they risk satisfying the subjective desires of the users while failing to improve objective outcomes. In Chapter \ref{ch:xai}, we confirm that this critique applies to some post-hoc justifications of model recommendations, as the justifications were found to yield an unwarranted increase in trust in human decision-makers. Thus, it is important to define and evaluate measures of the social values that DSTs intend to support; when evaluating these DSTs, they should not be evaluated human-centrically (i.e., according to their users' satisfaction), but should instead be evaluated on whether their employment improves social outcomes.

The fundamental challenge with evaluating ``real change'' in a selection context is the lack of ground truth; i.e., there are not, in general, ``correct'' or ``incorrect'' selection decisions, only those preferred by the organisation. In this thesis, we solve this problem by working with programmes to define measurable criteria that act as a surrogate for ``correct selection decisions''. In Chapter \ref{ch:xai}, these criteria are arrived at through an Action Research (AR) process and expressed in Figure \ref{fig:desiderata_matrix}, while in Chapter \ref{ch:spf}, these criteria are supplied directly by Rise, as the programme has internal metrics for both axes of the SPF. We recommend designers work with organisations to define surrogate criteria that can be used to evaluate the success of their systems.

\section{Implications}
\subsection{Algorithmic Fairness in a Selection Context}\label{ssec:fairness}
The work in this thesis has implications for the broader discussion of algorithmic fairness in selection processes. The design of SOAI DSTs has the potential to impact the lives of many of the world's most vulnerable people; it is thus imperative that these systems are designed fairly. However, the notion of fairness itself is complex and multifaceted. As \textcite{pmlr-v80-kearns18a} highlight, fairness can be understood in both procedural and distributive terms, and different methods of achieving fairness across different subgroups often conflict. Individual fairness is often discussed in the algorithmic fairness literature \cite{dwork_fairness_2012}; this is often contrasted with ``group'' fairness \cite{fleisher_whats_nodate,binns_apparent_2019,barocas2023fairness,Friedler_Scheidegger_Venkatasubramanian_2016}. Despite attempts to reconcile these differing notions of fairness, such as those by \textcite{binns_apparent_2019}, contradictions remain between metrics used to measure different forms of fairness; that is, decisions that may be ruled more fair by certain individual or procedural fairness measures might create group or distributive unfairness. What's worse, scholars disagree even on the best implementations of notions of fairness \cite{Friedler_Scheidegger_Venkatasubramanian_2016,binns_apparent_2019}, and differing interpretations conflict.

It is worth noting, then, that the work on generative AI detection in Chapter \ref{ch:genai} is built on a desire for procedural fairness, while the diversity goals of Chapters \ref{ch:diversity} and \ref{ch:spf}, in practice, accord closely with distributive notions. This raises the possibility that, via SOAI methods, researchers could determine socially beneficial fairness metrics to uphold in DSTs and build to support those.\footnote{This work, in particular, should not be done solely from the decision-maker's perspective. \textcite{10.1145/3351095.3372867} investigate applicant perceptions of appropriate fairness metrics; this work may be a good starting point for SOAI work in this field.}

%  - critical attention on algorithms Kitchin_2017
% I would be remiss to ignore the critical attention that algorithms have garnered as they proliferate through society. NISSENBAUM1998237 

\subsection{New Developments in AI for Selection}
The growing popularity of genAI has already dramatically increased the number of applications that job, university, and scholarship programmes must select from \cite{Kaashoek2024Impact}. While a blanket ban on genAI in application-writing may solve this \cite{h_holden_thorp_chatgpt_2023}, we find in Chapter \ref{ch:genai} that such a ban is unenforcible at present. We note in Chapter \ref{ch:genai} that our research is complicated by the rapidly changing nature of both genAI and detectors. Here, we extend this complication to SOAI as a whole. It may be that, as genAI development moves beyond retrieval-augmented generation to more complex architectures \cite{lewis2020retrieval}, such as integrated reasoning systems or agentic AI \cite{Shavit_O’Keefe_Eloundou_McMillan_Agarwal_Brundage_Adler_Campbell_Lee_Mishkin_et}, these systems will once again fundamentally change the process of selection. In light of this, SOAI is necessary to ensure that new, more powerful AI systems further the social aims of selection processes.

Of particular interest would be the development of AI systems capable of encoding domain knowledge in their structures, which could support decisions in a fundamentally different way. This could be particularly useful in automated essay scoring, where domain-specific knowledge is a significant problem \cite{elijahthesis}. If this is the case, then the work done in this thesis may serve as a precursor to the development of these systems and a guide for how to ensure that these systems are designed to support the social aims of selection processes.

\section{A Critical Reflection on the Position of this Research Within Structures of Power}\label{sec:reflexivity}
In a seminal provocation piece, \textcite{Barocas_Hood_Ziewitz_2013} question whether algorithms challenge or enforce existing power structures; in the case of this thesis, it seems unfortunately clear that these systems enforce existing power structures. For example, \textcite{Ahmed_2012} questions the role of diversity in enabling powerful institutions' dismissal of calls for real change; in helping these organisations better consider diversity, this research may also help them dismiss these calls. More generally designing AI systems for scholarship selection processes (often funded or administered by the most powerful people in the world), we risk contributing to an already vast power disparity between the world's richest and poorest. Scholarships such as those offered by Schmidt, MacArthur, Marshall, or Rhodes, while providing opportunities for individuals from disadvantaged backgrounds, serve to entrench their funders in institutional power structures. By working within these structures and aiming to optimize their selection processes, we may serve to maintain these power dynamics \cite{Ziegler_2008}. By writing this thesis at the University of Oxford, long decried among the most exclusionary institutions in the world \cite{Ziegler_2008}, we benefit from this power imbalance. By distancing this work from the disenfranchised decision subjects that the scholarship claims as its beneficiaries, we risk supporting this disparity in power. 

And yet, without this research, the University of Oxford would still stand. The scholarships of Schmidt, MacArthur, Marshall, and Rhodes would carry on, while selectors make difficult decisions with limited information. While the funding of these programs may carry with it the names of the elite, the programs themselves seek to support the intellectual development of their selected scholars, improve the lives of their target communities, and train talented young people to solve the world's most pressing problems. While the work in this thesis does not undertake to challenge the vast power gap separating decision-maker and decision subject, it does seek to improve the fairness and efficacy of decision-making processes within the existing framework. In doing so, this work enables the fair distribution of intellectual opportunity, seeks out scholars who improve the lives of their community members, and helps solve global problems.

\section{Limitations}
In scenarios outside the selection context, HCI research has generally sought to harmonise the conflicting needs of different user groups through participatory design. Yet, in the case of talent identification, the very nature of the selection process introduces a conflict of incentives between applicants and the selection organisation. Applicants are primarily motivated by the desire to be chosen, while selection organisations, guided by practitioners, seek to optimise the selected cohort according to (biased) measurements of (sometimes idiosyncratic) organisational preferences. This inherent misalignment of objectives raises significant concerns about how to balance the needs of both groups without compromising the overarching social values that SOAI seeks to uphold. Our solution in this thesis has been to centre the organisations and use data from past selections to evaluate their selection decisions based on stated and elicited preferences. However, this solution positions research from the position of the decision-makers, and a lack of engagement with decision subjects may limit the social benefit of the research.

Central to our solution is the notion that the broader social aims of the selection process (e.g., ensuring fairness, diversity, and social benefit) align more closely with practitioners than with applicants and ultimately supersede the immediate interests of both practitioners and applicants. However, we note that some overlap exists between the broader social goals of the selection process and the scholarship applicants' goals. We note that work already exists exploring this from a decision-subject standpoint; \textcite{10.1145/3351095.3372867} find that applicants prefer algorithmic decision-making to human decision-making according to both procedural and distributive notions of fairness. In light of this, SOAI's positioning as a paradigm for building DSTs, rather than a paradigm for building algorithmic decision-making systems, limits the scope of the research. Were it the case that algorithmic decision-making is preferable to algorithmically supported human decision-making, the SOAI paradigm would not be the most effective way to achieve the social aims of selection.

Setting aside the social benefit of the SOAI paradigm as a whole, specifics of SOAI, as it is implemented in this thesis, may limit the work done here. The Decision Matrix framework intentionally elides distinctions between decision points in favour of clarifying a focus on a decision's stage and stakes. This elision was arrived at in concert with participants from Rise, but it may not extend to other decision-making contexts. If it doesn't, though we still call for SOAI work in these contexts, our work may not be directly applicable.

\section{Future Work}
While the work in this thesis articulates SOAI as a paradigm for all AI design oriented around supporting selection problems, we develop and test this paradigm for three specific families of decision points. A straightforward extension of this work would apply SOAI principles to other decision points in selection processes. Natural candidates include: supporting essay judgements with automated essay scoring, where a large body of literature already seeks to score applicant essays via algorithm \cite{cozma_automated_2018,ramesh_automated_2022,wang_use_2022,elijahthesis}, but automated approaches continue to struggle with marking top or bottom essays \cite{elijahthesis}; supporting testing and test evaluation with automated scoring systems \cite{organisciak_beyond_2023,condon2014international}; and supporting pre-application portions of the outreach process, which was requested by several participants in Chapter \ref{ch:diversity}.

Though SOAI, as we investigate it here, aligns most closely with the interests of selectors, there is a need for human-centric work seeking to determine applicant perceptions of positive social outcomes. While work exists examining applicant perceptions of decisions made about them, \cite{pandey_applicants_2022,horodyski_applicants_2023}, this work often approaches research from a fairness or decision-subject-empowerment perspective. No work exists approaching applicant perspectives from the perspective of the ultimate social benefit of selection. Future work should seek to understand how applicants perceive the social outcomes of selection decisions, and how these perceptions can be used to design more effective AI systems for selection.

Though the Decision Matrix provides a useful framework for categorising decision points in selection processes, it is not exhaustive. Future work should seek to augment the Decision Matrix with additional axes that capture the complexity of selection decisions more fully. In particular, though selectors drew a distinction between individual- and group-level in-process decisions in Chapter \ref{ch:diversity} (and though the design prototypes reflect this distinction), the Decision Matrix does not capture this distinction. Future work should seek to augment the Decision Matrix to distinguish individual- from group-level distinctions and implement more individual-level decision support systems in practice.\footnote{Chapters \ref{ch:genai} and \ref{ch:spf} both avoid individual-level implementations in real decision-making pipelines due to risks associated with introduced unfairness or bias \cite{hartigan_fairness_1989,barocas2023fairness,pmlr-v80-kearns18a,Bastounis_Campodonico_vanderSchaar_Adcock_Hansen_2024,liang_gpt_2023}. Any work implementing tools at the individual-level should consider these risks first.}

\section{Conclusion}
In this thesis, we pioneer a new paradigm of AI design for selection processes, Selection-Oriented AI (SOAI). Chapters \ref{ch:xai} and \ref{ch:genai} find that existing AI systems often fail to meet the needs of selectors, particularly for in-process decision-making. In response, we propose a new paradigm, SOAI, which seeks to centre the design of AI DSTs not around the selectors but around the social aims of selection they seek to practice. Chapters \ref{ch:diversity} and \ref{ch:spf} apply SOAI principles to design DSTs to support considerations of diversity in selection; we implement a prototype designed to satisfy selector desires and find that it improves both diversity and performance outcomes in selection. We then provide a set of design recommendations for SOAI designers, including focusing on specific social values, identifying decision points with the Decision Matrix, balancing quantitative and qualitative information, and evaluating real change in addition to subjective satisfaction.

More broadly, the use of SOAI to support scholarship-specific selection decisions implies the potential to support and improve related decision-making processes, from other selection contexts (e.g., admissions or hiring) to non-selection decision-making contexts (e.g., programme outreach). With a technology-induced flattening of the world \cite{Friedman_2005}, more candidates from more parts of the world find themselves qualified for opportunities. Add to this the ease of application submission created by genAI assistants, and it is clear why applications to job, university, and scholarship opportunities have seen a dramatic increase in recent years \cite{Kaashoek2024Impact}. In light of this, we conclude with a call for SOAI across selection contexts; the need has never been more pressing.

%%%%% APPENDICES
% Starts lettered appendices, adds a heading in table of contents, and adds a
%    page that just says "Appendices" to signal the end of your main text.
\startappendices
% Add or remove any appendices you'd like here:
\chapter{\label{app:programmes}The Programmes we Study}

\minitoc
\section{Foreword to Appendix \ref{app:programmes}}
We work with two global scholarship and talent investment programmes: the Ellison Scholars Programme and the Rise Programme. Both programmes have asked that they not be identified in public-facing research, and thus we request that reviewers not share details on either programme and to protect the identities of the programmes where possible.

\begin{table}[htbp]
    \centering
    \caption{This table enumerates relevant measurements from the Rise and Ellison programmes.}
    \label{tab:measures}
    \begin{tabular}{l | p{0.3\textwidth} p{0.3\textwidth}}
        \toprule
        Measure Type & Rise & Ellison Scholars \\
        \midrule
        Cognitive Assessment & ICAR-Based Test \cite{condon2014international}; Roomworld \cite{Dumbalska_Bhatti_Ali_Summerfield_2023} & ICAR-Based Test \cite{condon2014international}; AUT \cite{guilford1967creativity,organisciak_beyond_2023} \\
        Essay Review & Peer; External Expert & AI \cite{xiao2024humanaicollaborativeessayscoring}; External Expert \\
        Grade and Achievement Review & None & External Expert \\
        Finalist Activity Review & Selector & Unknown \\
        \bottomrule
    \end{tabular}
  \end{table}
  
\section{The Rise Programme}\label{ssec:rise}
\subsection{Programme Overview}
Initiated from a $\$1$-billion investment, Schmidt Futures and the Rhodes Trust's Rise programme\footnote{https://www.risefortheworld.org/} finds and selects talented and disadvantaged 15-to-17-year-olds from around the world and helps them achieve their full career and service potential. Rise supports selected `winners' and `finalists' with a variety of benefits accessible at different points in their lives. We work with Rise from the programme's inception in 2021 until 2024. In each of these years, Rise has committed to selecting up to 100 winners and up to 500 finalists from their pool of applicants. In the four application cycles between 2021 and 2024, during which time Rise has selected 400 winners and nearly 2000 finalists from hundreds of thousands of started applications.

Rise uses a flexible benefits model, where winners (and, in some cases, finalists) gain access to a variety of potential resources, but utilise only resources they demonstrate a need for. For example, applicants who receive full scholarships to their university may not receive an academic scholarship from Rise. Programme benefits include academic scholarships, educational resources and programmes, networking opportunities, and even funding for winner-led startups.\footnote{As academic scholarships comprise a large portion of programme benefits, we speak about Rise as a ``scholarship and talent investment'' programme throughout. This is our own language and is intended to further anonymise the programme. Similarly, we replace the programme-specific term ``winners'' with ``scholars''.}

\subsection{The Selection Process}
The programme uses a two-stage selection process designed to be accessible to candidates from various global and socioeconomic backgrounds. In stage one, applicants submit various application materials asynchronously; Rise selects up to 500 finalists based on the quality of those materials and the programme's cohort composition goals. In stage two, finalists engage in one of several ``finalist days'' consisting of various collaborative live activities and an interview; after all finalist days are completed, Rise uses information from both stages to select 100 winners.

\begin{figure}[htbp]
    \centering
    \begin{subfigure}{.45\textwidth}
        \centering
        \includegraphics[width=\linewidth]{context/proj1.png}
        \caption{Sample Project 1}
        \label{sfig:can}
    \end{subfigure}
    \hfill
    \vspace{1em}
    \begin{subfigure}{.45\textwidth}
        \centering
        \includegraphics[width=\linewidth]{context/proj2.png}
        \caption{Sample Project 2}
        \label{sfig:cell}
    \end{subfigure}
    \hfill
    \vspace{1em}
    \begin{subfigure}{.45\textwidth}
        \centering
        \includegraphics[width=\linewidth]{context/proj3.png} 
        \caption{Sample Project 3}
        \label{sfig:app}
    \end{subfigure}
    \caption{The three panels in this figure depict slides from a programme presentation that highlighted three projects that were submitted as part of Rise's 2021 application cycle. Figure \ref{sfig:can} depicts an AI-dependent recycling bin paired with a mobile app to track and sort plastics. Figure \ref{sfig:cell} depicts schematics for a clean fuel cell using microorganisms and garbage. Figure \ref{sfig:app} depicts an app to help educate students about the importance of various technical and character skills. Names have been removed and pictures blurred to de-identify programme applicants.}
    \label{fig:example_projects}
\end{figure}

Stage one of selection occurs asynchronously (via smartphone, laptop, or, in rare cases, pen and paper) and in two parts. The first part requires applicants to submit an application form with their demographic information and either video or written essays. The first of these two essays explains a real-world problem the applicant wishes to solve, while the second discusses either how they consider themselves privileged or the challenges they have overcome.\footnote{Refinements to the application process between years all application materials change slightly over time. This change is most dramatic in the case of this second essay, where the focus of the essay changed from an applicant's declaration of their privilege to a description of a challenge the applicant has faced and how they overcame it. Changes appear throughout the application across years; we only detail them where it is relevant to our research.} In the second part, applicants complete a set of digital cognitive assessments and a project showcasing their talent.\footnote{In rare cases, technology or accessibility limitations prevented applicants from completing the cognitive assessment; these applicants were considered on the merits of their submitted materials.} The project showcase is a distinctive part of Rise's selection process whereby participants: (1) identify a problem they wish to solve, (2) research solutions to that problem, (3) implement those solutions, and (4) reflect on what they learned from the project. Participants submit one essay (video or text) on each of these four stages. Three example projects can be found in Figure \ref{fig:example_projects}. All applicants also submit a short written essay explaining their project and its significance. For an overview of the stage one selection design, see Figure \ref{fig:design}.

\begin{figure}[!htb]
    \centering
    \caption{This figure schematizes the key elements of the talent investment programme's data collection and selection process. }
    \label{fig:design}
    \includegraphics[width=\textwidth]{spf/selection_design_schematic.png} 
\end{figure}

Stage two of selection occurs synchronously (though still remotely) in one of several ``finalist days''. The finalist days consist of up to five activities of three types: presentations (where finalists present information about their project), group activities (where finalists collaborate to discuss and solve problems), or interviews (where finalists are interviewed). All activities were judged by a pool of adult `selectors' who assessed finalists according to a rubric. Winner selection decisions were made based both on data collected in stage two and information retained from stage one.

\subsection{Data Collection}
Across the application cycle, Rise collects a variety of data from applicants. This data includes traditional merit-based measures – including cognitive tests, written essays, and referrals – as well as non-traditional measures – including peer-reviewed video essays, gamified skill tests, and application platform behaviours. Many of these measures are used only for research purposes, and some are tangential to our research on supporting the selection process. We discuss relevant measures here. More detail on these measurements, especially for the 2021-2023 application cycles, can be found in Chapter \ref{ch:spf}, where findings depend on the specifics of Rise programme measurement. A comparison of the measurements used by Rise and the Ellison Scholars programme can be found in Table \ref{tab:measures}.

\paragraph{Cognitive Assessments}
Rise collects two data from two cognitive assessments taken by applicants. The first is based on the International Cognitive Assessment Resource (ICAR) \cite{condon2014international, subotic2020psychometric}, and has, in various selection cycles, incorporated four different item types: Cube Rotation, Number Sequence, Matrix Reasoning, and Verbal Reasoning. Applicants are given nonverbal and verbal sub-scores, which use a Bayesian generalized linear item response model \cite{burkner2021bayesian}. In some cycles, only the nonverbal score was used, while other cycles combined the two to create one singular score.

The second cognitive assessment consists of a gameified skills test called Roomworld \cite{Dumbalska_Bhatti_Ali_Summerfield_2023}.\footnote{Roomworld was created and scored by \textcite{Dumbalska_Bhatti_Ali_Summerfield_2023}; the specific scoring algorithm was not shared.}

\paragraph{Peer Review}
Stage one applicant essays were judged by two types of human evaluators: other applicants (peers) and adults with some expertise on the project topics (experts). Though \textcite{Anvari2021EffectivenessOP} provide evidence for the efficiency and effectiveness of peer review as a measurement of aptitude, peer review was (and remains) experimental \cite{Rahmatillah2022AnalyzingFA}. That said, \textcite{VanderSchee2022UsingCP} find that decision subjects of a blind peer review process experience just outcomes both according to the similar treatment and similar outcomes principles. Thus, though Rise treated peer reviews as experimental, peer scoring played an integral role in the Rise process. To collect peer reviews, each applicant was assigned to review 20 of each essay submitted. Each review consisted of Likert scale judgements designed to measure: intelligence, perseverance, empathy, integrity, sense of calling, and impact on the applicant.

\paragraph{Expert Review} 
Experts, on the other hand, were only asked to assess applicant project essays. Each reviewer was assigned several projects proportional to their capacity to review. Like peers, experts were asked to review different elements of the project, using Likert scales to gauge how effective the project was at accomplishing what the applicant intended and how impressive the project was relative to other projects in this field. 

\paragraph{Finalist Day Activities}
The finalist day activities were assessed by selectors through a mix of qualitative and quantitative measures. Each activity was scored on a rubric, and the scores were aggregated to create a final score for each finalist on each activity type. Additionally, selectors were given an option to provide specific qualitative feedback on applicants.

\section{The Ellison Scholars Programme}\label{ssec:ellison}
\subsection{Programme Overview}
Funded and administered by the Ellison Institute of Technology, The Ellison Scholars Programme\footnote{https://eit.org/ellisonscholars/} is a global scholarship programme that seeks to develop global technology innovators and leaders by finding and supporting talented people passionate about solving humanity’s most serious problems as they study at the University of Oxford and solve global problems through innovation. The programme seeks to select at least twenty scholars each year beginning in 2025.

As the Ellison Scholarship's inaugural cohort has yet to be selected, programme benefits have yet to be dispersed. However, the programme has committed to providing scholars with an academic scholarship to the University of Oxford and paid internships. These internships, as well as the programme as a whole, are organised around four humane endeavours: (1) Health and Medical Science, (2) Food Security and Sustainable Agriculture, (3) Climate Change and Clean Energy, and (4) Government Innovation and Era of Artificial Intelligence.

\subsection{The Selection Process}
As humane endeavours form a large part of the Ellison Scholarship, the selection process places special emphasis on suitability for these topics. The programme has applicants declare their humane endeavours of interest and assesses applicants relative to these humane endeavours. Additionally, as the programme seeks to pursue all four endeavours, diversity-like considerations require the programme to ensure that scholars are chosen for each humane endeavour. However, though many selectors on the Ellison Scholarship selection team are sympathetic to desires for demographic diversity, the programme's theory of change does not lend itself to explicit diversity considerations (unlike Rise)

The Ellison Scholarship employs a three-stage selection process. In stage one, applicants submit various application materials asynchronously; the programme selects semi-finalists based on the quality of those materials and the programme's cohort composition goals. In stage two, semi-finalists apply to the University of Oxford, and the University handles its internal selection process; programme applicants who receive Oxford scholarships are dubbed Finalists. In stage three, finalists engage in a series of synchronous activities before final decisions are made by the programme's board. 

In stage one of selection, applicants submit their demographic information; selections for humane endeavour, Oxford course, and preferred project; their education record; a list of achievements; and four written essays speaking to their suitability for the programme. These essays speak to the applicant's alignment to their chosen humane endeavour, alignment to their chosen course at Oxford, and their particular skills and archetype. After submitting this application, all applicants are invited to take a cognitive assessment assessing convergent and divergent reasoning.

In stage two, semi-finalists apply to the University of Oxford; in stage three, finalists engage in a series of synchronous activities before final decisions are made by the programme's board. As the Ellison Scholars programme is still selecting its inaugural cohort, neither stage two nor stage three have been enacted. Thus, we omit details on them here. 

\subsection{Data Collection}
The Ellison Institute collects and constructs several different aptitude measurements of applicants. This is primarily traditional merit-based measures, e.g., cognitive tests, written essays, or academic transcripts. Additionally, the programme constructs experimental measures from gathered data. We discuss relevant measures here. A comparison of the measurements used by Rise and the Ellison Scholars programme can be found in Table \ref{tab:measures}.

\paragraph{Cognitive Assessments} 
Much like the Rise programme, the Ellison Scholarship uses a cognitive assessment based on the International Cognitive Assessment Resource (ICAR) \cite{condon2014international,subotic2020psychometric}. Though the details of implementation differ, both programmes use the same four item types and the same scoring algorithm \cite{burkner2021bayesian}.

Additionally, the Ellison Scholarship relies on a divergent thinking assessment based on Guildford's Alternative Uses Task (AUT) \cite{guilford1967creativity}. This task is chosen for its ability to measure ``divergent thinking'', i.e., creativity or innovativeness \cite{dumas_measuring_2020,organisciak_beyond_2023}. Each question is timed at roughly 90 seconds per question, and each applicant is given ten questions. Applicants are scored according to \textcite{organisciak_beyond_2023}'s Open Creativity Scoring with Artificial Intelligence (OCSAI) scoring algorithm.\footnote{https://openscoring.du.edu/}

The Ellison Scholarship combined both ICAR and AUT scores to get an overall cognitive assessment score.

\paragraph{AI-driven Assessment of Essays}
The Ellison Scholarship employs an AI-based scoring method as a preliminary screen on applicants' four written essays. The programme's approach broadly follows the approach of \textcite{xiao2024humanaicollaborativeessayscoring}; the Ellison Scholars programme requested that specific details of the programme's implementation not be shared. After these four essays are scored, an overall AI-driven score of applicants is calculated.

\paragraph{Expert Assessment of Applications}
Stage one applicants whose test scores or AI-driven essay scores merited further consideration were judged by expert human evaluators in two types of reviews: anonymous reviews (where reviewers only had access to applicant essays, grades and achievements) and contextual reviews (where reviewers had access to supporting information such as references or applicant demographics). As compared to Rise's experts, the Ellison Institute engaged adult reviewers in a rigorous training process before qualifying them as expert reviewers.

In each review, experts judged applicants on axes related to specific programme goals (e.g., whether the applicant demonstrated an interest in the humane endeavour they chose). Anonymous and contextual reviews were ultimately pooled, and an overall review score was calculated.

\paragraph{Semifinalist and Finalist Assessment}
As the programme is still selecting its inaugural cohort, they have yet to undergo semi-finalist or finalist assessment. We thus omit the details of these assessments from this thesis.
\chapter{\label{app:protocolmockup}Study Protocols}

\minitoc

\section{Design Workshops from Chapter \ref{ch:xai}}\label{app:xaiprotocol}
We split our N=8 participants into two groups of 4 (G1 and G2) to run two participatory design workshops. As these are group discussions, actual programming deviates from the protocol slightly.

Our research question for both workshops is: ``Are SHAP explanations useful?''; however, to frame each workshop, we told both groups that we were interested in the answer to two questions: ``What does this technology tell us about the algorithm’s scores?'' and ``How do we envision this technology being used in future selection processes?''.

Following this, we gave both groups a 15-minute demonstration of the technology, where we described a sample case, gave some example insights, and answered any questions participants had. 

The main task for our workshops consists of hands-on cases examining a SHAP-based ``waterfall plot'' explanation of a programme applicant's score. An example case can be seen in Figure \ref{fig:sample_case}. Each case is presented to participants as a single slide on a presented slideshow, with additional questions asked by the researchers to prompt discussion. We show each group 5 different cases and spend an average of 10 minutes on each case.

For each case, we ask some of the following questions to prompt discussion:

\begin{enumerate}
    \item Why are we viewing this applicant?
    \item What comments are we responding to?
    \item What does the technology appear to say about this candidate?
    \item Does the technology address the comment we are responding to?
    \item What does this case say about the algorithm as a whole?
    \item Does this case necessitate changes to the algorithm?
\end{enumerate}

Finally, after all cases had been examined, we moved to a short reflection on the technology as a whole. We asked participants to answer:

\begin{enumerate}
    \item What did you think was useful about the technology presented?
    \item What was lacking?
    \item How might this be improved?
\end{enumerate}


\section{Interviews from Chapter \ref{ch:diversity}}\label{app:divprotocol1}
For the individual interviews, we follow a semi-structured protocol. Following the methodology of \textcite{braun_using_2006}, we do not limit our analysis to these questions. Instead, we deviate from this script as guided by the conversations and our overarching research questions, then we allow themes to emerge naturally from the data. Our interview research questions (also found in Section \ref{ssec:methods1}) are:

\begin{enumerate}
    \item What is diversity?
    \item Which elements of diversity matter in a selection context? Why?
    \item How could technology assist in operationalising diversity?
\end{enumerate}

We interviewed 15 individuals from two different talent identification organisations. We conduct each interview separately. We first ask a few questions about the factors that go into decision-making:

\begin{enumerate}
    \item We're going to take a step back and discuss a hypothetical selection scenario for a fellowship for a group of young people. In this scenario, you have full control over who is selected.
    \item Could you please list some things you think are important in deciding who to accept?
    \item (Can skip) Which of (these things) are about the individual applicant's performance?
    \item What are (these remaining things) about? 
    \item (Or:) Why are (these things) important?
\end{enumerate}

We then ask participants to define diversity, to break it down into elements, and to discuss why diversity is important:

\begin{enumerate}
    \item Now I want to talk about diversity. Keeping your list in mind, can you please define diversity?
    \item (If the definition is too short) Could you please elaborate on (pick apart)
    \item Why do we care about this definition of diversity?
    \item Now, if you were going to break your definition into some elements or facets, what would those be?
    \item (If they talk about holistic diversity) What considerations are important when looking at diversity holistically?
    \item (If elements are vague) How does (pick a metric) factor into your understanding of (element)?
    \item Which elements or considerations are most important?
    \item How do we measure these facets of diversity?
    \item (If this measurement isn't concrete) Imagine you had a ``magic metric'' that perfectly measured diversity. What does this metric do?
\end{enumerate}

Next, we run two short exercises from the participatory design literature. The first is called ``crazy 8s'', wherein participants are given 8 minutes to come up with 8 ideas. For these ideas, we ask participants to think about technologies that might help them better understand diversity in selection:

\begin{enumerate}
    \item Now I want to talk about technology we can build to support thinking about tradeoffs around diversity. Remember that we're stepping away from existing processes and solutions.
    \item We're going to start with an exercise called ``crazy 8s''. For the next 8 minutes, we're going to spend one minute each developing a technology that might help us better understand diversity in selection. These technologies don't have to make sense or be possible; I just want you to think of things that might help you think through diversity. This activity is difficult; don't worry if you find yourself struggling or sounding silly.
    \item Take a second to think about a technology. When you're ready, please describe it.
    \item (If the technology is unclear) Could you please elaborate on (unclear part)?
\end{enumerate}

The second is called ``the magic app'', wherein participants are asked to elaborate on a single idea for an application, waving away technical details as `magic':

\begin{enumerate}
    \item Now let's dig deeper into one hypothetical ``magic app'' designed to help us better understand tradeoffs around diversity in selection. The magic app can do anything you might want it to in any way you might want. What does your magic app do?
    \item (If the app has visuals) What do your visuals look like?
    \item (If the app is pure text) What sorts of visualisations might help you?
    \item (If the app has buttons or sliders) What do your buttons do?
    \item (If the app doesn't have any interactivity) How might you interact with this app?
    \item Now we are going to split the app out into different ``pages''
    \item (If they haven't already done this) The individual-level page: each applicant will have their individual-level page, which will say things about that applicant
    \item (If they haven't already done this) The cohort-level page: each possible cohort will have its cohort-level page, which will update any time we make changes to the cohort.
    \item (For each page) What happens on this page?
    \item What is the experience of using this page like?
    \item (If the page has visuals) What do your visuals look like?
    \item (If the page is pure text) What sorts of visualisations might help you?
    \item (If the page has buttons or sliders) What do your buttons do?
    \item (If the page doesn't have any interactivity) How might you interact with this app?
    \item (For each different feature of the page) What makes (feature) useful to you?
    \item Thank you! Is there anything else you would like to add?
\end{enumerate}

\section{Design Workshops from Chapter \ref{ch:diversity}}\label{app:divprotocol2}
For the participatory design workshops, we split our participants by organisation. As some individuals could not attend the second session, we have one group of 6 and another of 7. As these are much larger group discussions, we deviate further from our protocol.

The task for these workshops consists of hands-on sessions with different technologies. The technologies are designed and mocked up based on the thematic analysis of the interviews. These technologies are presented to participants via Miro, where they are free to interact with and annotate them. Our research questions for this workshop are:

\begin{enumerate}
    \item What prototypes best promote diversity?
    \item What elements of these prototypes facilitate their success?
\end{enumerate}

Or, for each prototype: ``How and why does this prototype promote diversity in talent identification?''. Again, though we write a list of questions targeted at these questions, we do not limit our analysis to these questions \cite{braun_using_2006}. Instead, we deviate from this script as guided by the conversations and our overarching research questions, then we allow themes to emerge naturally from the data \cite{braun_using_2006}. For each technology prototype shown, we have questions:

\begin{enumerate}
    \item This prototype describes... Are any of you familiar with this?
    \item In what follows, we're going to discuss this prototype. Let's start with: is it easy to read for you? What does it say? 
    \item What questions do you have upon seeing this prototype? Feel free to write these down.
    \item How would you use this prototype in a hypothetical selection procedure?
    \item How (else) would this prototype fit into your current selection procedure?
    \item How would your current selection procedure make the best use of this prototype? Would the process need to be changed? Do you think this would be beneficial?
\end{enumerate}

Finally, at the end of our workshop, after we have covered all of the prototypes, we ask participants to place a star next to their favourite prototype on the Miro board \cite{Gatian_1994,Griffiths_Johnson_Hartley_2007}.
\chapter{\label{app:math}Mathematics and Computation}

\minitoc

\section{ChatGPT Code Generation for Chapter \ref{ch:genai}}\label{app:prompt}

\section{Proofs of Submodularity and Monotonicity for Chapter \ref{ch:spf}}\label{app:spfmath}

\begin{theorem} 
    Submodularity is closed under weighted addition. \label{thm:submodularity_additive}
    \end{theorem}
    
    \begin{proof}
    \begin{equation}
        \label{eq:submodularity_additive}
        \begin{split}
            \forall Y \subseteq U, X \subseteq Y, x \in U \setminus Y, a \geq 0, b \geq 0: & F_1(X \cup \{x\}) - F_1(X) \leq F_1(Y \cup \{x\}) - F_1(Y) \\
            &\land F_2(X \cup \{x\}) - F_2(X) \leq F_2(Y \cup \{x\}) - F_2(Y) \\
            \implies & a*F_1(X \cup \{x\}) - a*F_1(X) \leq a*F_1(Y \cup \{x\}) - a*F_1(Y) \\
            &\land b*F_2(X \cup \{x\}) - b*F_2(X) \leq b*F_2(Y \cup \{x\}) - b*F_2(Y) \\
            \implies & a*F_1(X \cup \{x\}) - a*F_1(X) + b*F_2(X \cup \{x\}) - b*F_2(X) \\
            &\leq a*F_1(Y \cup \{x\}) - a*F_1(Y) + b*F_2(Y \cup \{x\}) - b*F_2(Y) \\
            \implies & a*F_1+b*F_2(X \cup \{x\}) - a*F_1+b*F_2(X) \\
            &\leq a*F_1+b*F_2(Y \cup \{x\}) - a*F_1+b*F_2(Y) \\ \nonumber
        \end{split}
    \end{equation}
    \end{proof}
    
    \begin{theorem} 
    Monotonicity is closed under weighted addition. \label{thm:monotonicity_additive}
    \end{theorem}
    \begin{proof}
    \begin{equation}
        \label{eq:mononicity_additive}
        \begin{split}
            \forall Y \subseteq U, X \subseteq Y, a \geq 0, b \geq 0: & F_1(X)\leq F_1(Y) \land F_2(X)\leq F_2(Y) \\
            \implies & a*F_1(X)\leq a*F_1(Y) \land b*F_2(X)\leq b*F_2(Y) \\
            \implies & a*F_1(X) + b*F_2(X) \leq a*F_1(Y) + b*F_2(Y) \\
            \implies & a*F_1+b*F_2(X) \leq a*F_1+b*F_2(Y) \\ \nonumber
        \end{split}
    \end{equation}
    \end{proof}
    
    \begin{theorem} 
    The class of functions $F$ as defined in Equation \ref{eq:f_spec} is submodular and monotone.\label{thm:f_sub_mon}
    \end{theorem}
    \begin{proof}
    By construction, $F$ is the weighted sum of $P$ and $D$. But $D$ is the weighted sum of functions $\delta_g^{prop}$ and $\delta_G^{count}$.
    It is trivial to see that $P$ is submodular and monotone. We have already demonstrated that $\delta_g^{prop}$ and $\delta_G^{count}$ are submodular and monotone. Thus, by Theorems \ref{thm:submodularity_additive} and \ref{thm:monotonicity_additive}, $F$ is submodular and additive.
    \end{proof}

\section{Proof that Algorithm \ref{alg:frontier} Approximates the SPF}\label{app:greedy-proof}
Here, we prove that the greedy approximation method introduced in Algorithm \ref{alg:frontier} for SPF is a $(1-\frac{1}{e})$-approximation for our standard class of diversity functions subject to a cardinality constraint \cite{nemhauser1978analysis}.

Recall that an organisation's preference function $f: 2^X -> \mathbb{R}^+$  maps a cohort (a set of applicants) to the weighted sum of a ``diversity score'' and a ``performance score'' (both non-negative, real numbers). Here, our applicant pool $X$ is represented as a set with applicants as its members, and possible cohorts $C \subseteq X$ are subsets of $X$. We prove Appendix \ref{app:spfmath} that $f$ is monotonic ($A \subseteq B -> f(A) \leq f(B)$) and submodular ($A \subseteq B \land x \notin B -> f_A(x) \geq f_B(x)$ (here, $f_S(e) = f(S \cup \{e\}) - f(S)$ denote the marginal gain of adding element $e$ to set $S$). We demonstrate elsewhere that many common understandings of diversity are represented by standard diversity functions.

\begin{theorem}\label{thm:greedy-approximation}
    Let $(S_0...S_k)$ be a sequence of sets where $S_0$ is the empty set and $S_{i>0}$ is defined by following Algorithm \ref{alg:frontier} with any $\iota$, $d$, and $p$. Further, let $O := argmax_S(f(S) : |S| = k)$ be the set of size $k$ that maximizes $f: = \iota*d+(1-\iota)*p$. Then $f(S_k) \geq (1 - \frac{1}{e})f(O)$.
\end{theorem}

\begin{proof}
By induction. Let ${o_1...o_k} = O$ be any ordering of the elements of $O$. Let ${s_i} := S_i - S_{i-1}$ be the element added to $S_{i-1}$ to form $S_i$.

By monotonicity, we have $\forall i . f(O) \leq f(O \cup S_i)$.  We can then write $f(O \cup S_i) = f(O \cup S_i) - f(S_i) + f(S_i) = \sum_{j=1}^{k} (f(S_i \cup {o_1...o_j}) - f(S_i \cup {o_1...o_{j-1}}))$. I.e., $f(O \cup S_i) = f(O \cup S_i) - f(S_i) + f(S_i) = \sum_{j=1}^{k} f_{S_i \cup {o_1...o_{j-1}}}(o_j)$.

By submodularity, $\forall j \in [1...k] . f_{S_i \cup {o_1...o_{j-1}}}(o_j) \leq f_{s_i}(o_j)$. Thus, $f(O) \leq f(S_i) + k*f_{S_i}(s_{i+1})$. 

Since Algorithm \ref{alg:frontier} guarantees that $\forall e \in X - S_i . f_{S_i}(e) \geq f_{S_i}(e)$, it follows that, at every stage, $f(S_{i+1}) - f(S_i) \geq \frac{1}{k} (f(O) - f(S_i))$.

Then induction yields $f(O) - f(S_k) \leq (1 - \frac{1}{k})^k f(O) \leq \frac{1}{e} f(O)$.
\end{proof}
\chapter{\label{app:figures}Reference Figures and Tables}

\minitoc

\section{Sample Explanations from Chapter \ref{ch:xai}}\label{app:xaifigures}

\begin{figure}[!htbp]
    \centering
    \includegraphics[width=0.9\linewidth]{xai/survey-shap.png}
    \caption{This figure shows sample SHAP explanations in the \emph{Salary} task. Features can be seen along the y-axis, while feature importance is shown based on the direction and magnitude of the associated bar.}
    \label{fig:shapsalaryfull}
\end{figure}

\begin{figure}[!hbtp]
    \centering
    \includegraphics[width=0.9\linewidth]{xai/survey-shap-2.png}
    \caption{This figure shows sample SHAP explanations in the \emph{Credit} task. Features can be seen along the y-axis, while feature importance is shown based on the direction and magnitude of the associated bar.}
    \label{fig:shapcreditfull}
\end{figure}

\begin{figure}[!hbtp]
    \centering
    \includegraphics[width=0.8\linewidth]{xai/survey-anchor.png}
    \caption{This figure shows sample Anchor explanations in the \emph{Salary} task. The explanation shows a set of rules that, when jointly followed, increases the likelihood that the model will yield the displayed prediction.}
    \label{fig:anchorsalaryfull}
\end{figure}

\begin{figure}[!hbtp]
    \centering
    \includegraphics[width=0.9\linewidth]{xai/survey-anchor-2.png}
    \caption{This figure shows sample Anchor explanations in the \emph{Credit} task. The explanation shows a set of rules that, when jointly followed, increases the likelihood that the model will yield the displayed prediction.}
    \label{fig:anchorcreditfull}
\end{figure}

\begin{figure}[!hbtp]
    \centering
    \includegraphics[width=0.6\linewidth]{xai/survey-confidence.png}
    \caption{This figure shows sample Confidence explanations in the \emph{Salary} task. The explanation is simply one sentence containing the model's confidence parameter.}
    \label{fig:confidencesalaryfull}
\end{figure}

\begin{figure}[!hbtp]
    \centering
    \includegraphics[width=0.6\linewidth]{xai/survey-confidence-2.png}
    \caption{This figure shows sample Confidence explanations in the \emph{Credit} task. The explanation is simply one sentence containing the model's confidence parameter.}
    \label{fig:confidencecreditfull}
\end{figure}

\newpage
\section{Images and Descriptions of Prototypes from Chapter \ref{ch:diversity}}\label{app:divfigures}

\begin{figure}[!hbtp]
    \centering
    \includegraphics[width=0.9\linewidth]{diversity/representativeness.png}
    \caption{This figure reproduces Prototype \ref{fig:representativeness} at a larger scale.}
    \label{fig:representativeness_full}
\end{figure}

\begin{figure}[!hbtp]
    \centering
    \includegraphics[width=0.9\linewidth]{diversity/entropy.png}
    \caption{This figure reproduces Prototype \ref{fig:entropy} at a larger scale.}
    \label{fig:entropy_full}
\end{figure}

\begin{figure}[!hbtp]
    \centering
    \includegraphics[width=0.9\linewidth]{diversity/diversity.png}
    \caption{This figure reproduces Prototype \ref{fig:diversity} at a larger scale.}
    \label{fig:diversity_full}
\end{figure}

\begin{figure}[!hbtp]
    \centering
    \includegraphics[width=0.9\linewidth]{diversity/demographic.png}
    \caption{This figure reproduces Prototype \ref{fig:demographic} at a larger scale.}
    \label{fig:demographic_full}
\end{figure}

\begin{figure}[!hbtp]
    \centering
    \includegraphics[width=0.9\linewidth]{diversity/impact.png}
    \caption{This figure reproduces Prototype \ref{fig:impact} at a larger scale.}
    \label{fig:impact_full}
\end{figure}

\begin{figure}[!hbtp]
    \centering
    \includegraphics[width=0.9\linewidth]{diversity/advantage.png}
    \caption{This figure reproduces Prototype \ref{fig:advantage} at a larger scale.}
    \label{fig:advantage_full}
\end{figure}

\newpage
\section{Figures and Tables for Chapter \ref{ch:spf}}\label{app:spffigures}
\begin{figure}[!hbtp]
    \centering
    \label{fig:spf_2021_full}
    \includegraphics[width=0.9\linewidth]{spf/yr1_spf_finalist.png} 
    \caption{This figure displays the SPF we estimate for the Cycle 2021 finalist selection process. The y-axis represents the diversity score while the x-axis represents the average cohort performance. The green curve is our estimate of the SPF, which represents the upper bound of diversity that is achievable at every level of cohort performance. The red dot depicts the actual level of diversity and performance of the finalists who were selected in Cycle 2021. The vertical and horizontal dashed red lines represent the maximum Pareto gain that was possible along the diversity and performance dimensions respectively. In particular, cohort diversity could have been improved by $15.2\%$ without any reduction in cohort performance. And, cohort performance could have been improved by $15.6\%$ without any cost to diversity.}
\end{figure}
    
\begin{figure}[!hbtp]
    \centering
    \label{fig:spf_2022_full}
    \includegraphics[width=0.9\linewidth]{spf/yr2_spf_finalist.png} 
    \caption{This figure displays the SPF we estimate for the Cycle 2022 finalist selection process. The y-axis represents the diversity score while the x-axis represents the average cohort performance. The green curve is our estimate of the SPF, which represents the upper bound of diversity that is achievable at every level of cohort performance. The red dot depicts the actual level of diversity and performance of the finalists who were selected in Cycle 2022. The vertical and horizontal dashed red lines represent the maximum Pareto gain that was possible along the diversity and performance dimensions respectively. In particular, cohort diversity could have been improved by $13\%$ without any reduction in cohort performance. And, cohort performance could have been improved by $19.6\%$ without any cost to diversity.}
\end{figure}


%%%%% REFERENCES

% JEM: Quote for the top of references (just like a chapter quote if you're using them).  Comment to skip.
% \begin{savequote}[8cm]
% The first kind of intellectual and artistic personality belongs to the hedgehogs, the second to the foxes \dots
%   \qauthor{--- Sir Isaiah Berlin \cite{berlin_hedgehog_2013}}
% \end{savequote}

\setlength{\baselineskip}{0pt} % JEM: Single-space References

{\renewcommand*\MakeUppercase[1]{#1}%
\printbibliography[heading=bibintoc,title={\bibtitle}]}


\end{document}
