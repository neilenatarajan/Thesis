% \begin{savequote}[8cm]
% Alles Gescheite ist schon gedacht worden.\\
% Man muss nur versuchen, es noch einmal zu denken.

% All intelligent thoughts have already been thought;\\
% what is necessary is only to try to think them again.
%   \qauthor{--- Johann Wolfgang von Goethe \cite{von_goethe_wilhelm_1829}}
% \end{savequote}

\chapter{\label{ch:iaicasestudy}A Human-Centric Approach to Identifying Talent With Intrinsically Interpretable Models [WIP]} % Maybe this is two chapters? The co-design could be one chapter, and the SPF itself could be another?

\minitoc

\section{Introduction}
(To-do)

\section{What Do We Mean When We Talk About `Diversity'?}
Talent Identification professionals frequently speak of diversity as a selection desideratum. This is variously used in the contexts of selected cohorts and selected individuals. Several definitions of diversity exist in the literature, but none encapsulate what TI professionals mean when they talk about diversity. We conduct a series of interviews with TI professionals investigating what they mean by "diversity", how they measure diversity, and why diversity is important. We conduct a thematic analysis of these interviews and report several themes related to the meaning of diversity. Finally, we discuss the significance of these insights and their applications to the development of purpose-built decision assistants designed to help TI professionals make diversity-conscious selection decisions.

(To-do)

\section{Co-designing Interpretable, Artificially Intelligent Decision Support Tools for Understanding Diversity}
The SPF (to be elaborated on below) represents one approach to resolving some of the concerns TI professionals have when discussing diversity. We interview TI professionals to understand what they would desire from tools designed to help them better address diversity concerns in selection. Furthermore, in addition to adapting the SPF methodology into a prototype designed to address specific TI professional concerns, we develop two other prototypes based on these interviews. Then, we run a scenario speed-dating exercise in order to understand the strengths and weaknesses of each approach. We evaluate each approach by how well it solves the problem it is intended to solve. Finally, we adapt these insights into a series of guidelines for how to design tools 

(To-do)