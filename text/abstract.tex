Selecting people for opportunities like jobs, loans, or scholarships is an unfortunate necessity of society. And while processes exist for people to make these decisions at scale, these processes are unequipped to handle the elevated demands of modernity. The work in this thesis explores the use of data-driven Decision Support Tools (DSTs) to improve selection processes, focusing on two global scholarship programs.

We frame our investigation in terms of the \emph{Decision Matrix} framework, categorising decisions by \emph{stage} (\emph{in-process} or \emph{ex-post}) and \emph{stakes} (\emph{high} or \emph{low}). We then explore the use of existing AI tools as DSTs, focusing on post-hoc explainable AI and generative AI detectors. We find them ineffective for \emph{in-process} decisions but useful \emph{ex-post}. We engage in participatory design to create six design prototypes to assist with \emph{in-process} decision-making, with a focus on diversity. Participants demonstrated enthusiasm to use these tools across the Decision Matrix. To validate this enthusiasm, we implemented one design as a technology probe and evaluated its impact. The selected cohort's diversity and performance improved, demonstrating the tool's ability to support high-stakes \emph{in-process} decisions. 

Our findings highlight the need for data-driven and AI-based DSTs across the \emph{Decision Matrix}. We propose \emph{Selection-Oriented AI}, a design paradigm focused on the social goals of selection, and provide design recommendations. We conclude with a call for AI-driven DSTs that balance practitioners' needs while optimising selection outcomes for social benefit.