Selecting people for opportunities like jobs, universities, loans, or scholarships pervades and shapes society. And while processes exist for people to make these decisions at scale, these processes are unequipped to handle the elevated demands of modernity. The work in this thesis explores the use of data-driven Decision Support Tools (DSTs) to improve selection processes, focusing on two global scholarship programmes.

We frame our investigation in terms of the \emph{Decision Matrix} framework, categorising decisions by stage (in-process or ex-post) and stakes (high or low). We then explore using existing AI tools as DSTs, focusing on post-hoc explainable AI and generative AI detectors. We find them ineffective for in-process decisions but useful ex-post. We engage in participatory design to create six design prototypes to assist with in-process decision-making, with a focus on diversity. Participants demonstrated enthusiasm for using these tools across the Decision Matrix. To validate this enthusiasm, we implemented one design as a technology probe and evaluated its impact. The selected cohort's diversity and performance improved, demonstrating the tool's ability to support high-stakes in-process decisions. 

Our findings highlight the need for data-driven and AI-based DSTs across the Decision Matrix. We propose \emph{Selection-Oriented AI}, a design paradigm focused on the social goals of selection, and provide design recommendations. We conclude with a call for AI-driven DSTs that balance practitioners' needs while optimising selection outcomes for social benefit.