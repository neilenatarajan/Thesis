In a globalised world, scholarship programs face pressures to select the ``best'' applicants fairly across regions, but current low-tech selection processes struggle to meet this burden. To address this, we studied two global scholarship programs, exploring data-driven Decision Support Tools (DSTs) for selection. One program was interested in utilising existing AI tools wherever possible. We explored using existing post-hoc explainable AI (xAI) tools as DSTs in a variety of contexts and using existing detection software as DSTs to understand generative AI (genAI)'s role in essays. We developed the \emph{Decision Matrix} framework, categorising decisions by \emph{stage} (\emph{in-process} or \emph{ex-post}) and \emph{stakes} (\emph{high} or \emph{low}), and applied it to assess both technologies as DSTs. Both proved ineffective for \emph{in-process} decisions and useful \emph{ex-post}.

To design a DST for \emph{in-process} decision-making, we engaged in participatory design with both programs, focusing on diversity considerations in selection. Six prototypes were created, with practitioners showing eagerness to use them across the \emph{Decision Matrix}. To validate this enthusiasm, we implemented one design as a technology probe and evaluated its impact. The selected cohort's diversity and performance improved, demonstrating the tool's ability to support high-stakes \emph{in-process} decisions.

Our findings highlight the need for data-driven and AI-based DSTs across the \emph{Decision Matrix}. We propose \emph{Selection-Oriented AI}, focused on the pro-social goals of selection. We conclude with a call for AI-driven DSTs that balance practitioners' needs while optimising selection outcomes for social benefit.