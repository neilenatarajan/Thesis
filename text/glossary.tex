% First parameter can be changed eg to "Glossary" or something.
% Second parameter is the max length of bold terms.

% [FIXED] Selection – May be useful to scope that this thesis explicitly deals with positive discrimination in talent selection (in the cohort or individual contexts as discussed above) as opposed to other aspects of selection.

% [FIXED] Section 1.2: I don't entirely agree that HCAI would centre the task around Selectors. It would centre around anyone involved in the process: selectors, their team/bosses, the applicants, etc…. if done properly! Perhaps it is correct to say that many people would choose to centre it around selection?

\begin{mclistof}{Glossary}{3.2cm}
    \item[HCI] Human-Computer Interaction is a subfield of Computer Science that deals primarily with how people interact with computers and to what extent computers are or are not developed for successful interaction with human beings. This thesis is a work of HCI.

    \item[PD] Participatory Design is a paradigm within HCI that engages participants as co-designers in an iterative design process, recognising the user as ideally positioned to understand user needs and preferences. Research outputs are usually designs and design recommendations driven by careful analysis of user feedback. Much of the work in this thesis is inspired by the PD paradigm.

    \item[AR] Action Research is a family of methods within HCI that engages a group of practitioners as co-researchers and co-participants in the research process; in this case, preparation is only one part of the research process, while action and reflection are equally valuable. Research outputs are ordinarily learnings that arise from the action. Much of the work in this thesis is inspired by the AR paradigm.

    \item[VSD] Value-Sensitive Design is a family of methods within HCI engaging participants, where particular values of participants are elicited and used as a guide for design. Research outputs are usually designs and design recommendations driven by careful analysis of user values. This thesis engages with VSD in supporting diversity.

    \item[HCC] Human-Centred Computing is a subfield of Computer Science that designs and develops computer systems around the needs and desires of a group of humans, thus `centring' that group of humans. This thesis's central contribution (Selection-Oriented AI) is offered in contrast to HCC.

    \item[AI] Artificial Intelligence is variously defined as the study of intelligent behaviour in computers \cite{wang2008you}, as computational requirements for tasks like perception or reasoning \cite{Leake2001ArtiicialI}, or as large models such as ChatGPT or DALL-E \cite{du2020ai}. AI is often construed as definitionally aspirational, i.e., it is taken as a given that current computer systems are not AI \cite{wang2008you}. In this thesis, any computer system that can be said to exhibit behaviour similar to human intelligence is included, and all work herein seeks to build or evaluate AI tools.

    \item[XAI] Explainable Artificial Intelligence is a subfield of AI that develops and assesses explanations that make AI systems more legible to a group of humans. Chapter \ref{ch:xai}, in particular, engages in a debate over the usefulness of XAI.

    \item[GenAI] Generative Artificial Intelligence is a subfield of AI that develops and assesses AI systems, usually large machine learning models, that generate new data, such as text, images, or audio. Chapter \ref{ch:genai} concerns itself with GenAI and the detection of GenAI.

    \item[Diversity] In its broadest sense, diversity refers to variety, difference, or heterogeneity within a given collection of entities. The seminal definition by Page describes it as: ``The heterogeneity of elements in a set about a class that takes different values, such as species in an eco-environment, or ethnicity in a population'' \cite{page_diversity_2010}. While this definition is broad enough for contexts such as ecology, a more nuanced understanding is required in the context of applicant selection (see Chapter \ref{ssec:context_diversity}).

    \item[HCAI] Human-Centred Artificial Intelligence is a subfield of HCC that concerns itself with AI systems, rather than all computer systems. This thesis's central contribution (Selection-Oriented AI) is offered in contrast to HCAI.

    \item[Selection] Occurs in a variety of forms throughout society, from recruitment to matchmaking. In this thesis, selection refers exclusively to the processes of scholarships and other academic or talent investment opportunities, with a primary interest in selection processes for social benefit rather than organisational benefit.

    \item[Selector] Practitioners responsible for making selection decisions, both direct and supporting, within selection teams and organisations. These practitioners are referred to as selectors, and tools are built to support their decision-making processes.

    \item[SOAI] Selection-Oriented AI, defined in this thesis, is a family of methodologies designed to achieve the social values of properly selecting scholars. In contrast to HCAI, which would centre the point of view of various stakeholders, SOAI orients itself around the social benefits of selection, deviating from the point of view of the selectors, applicants, and other stakeholders when their values differ.
\end{mclistof}