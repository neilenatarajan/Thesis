% \begin{savequote}[8cm]
% \textlatin{Neque porro quisquam est qui dolorem ipsum quia dolor sit amet, consectetur, adipisci velit...}

% There is no one who loves pain itself, who seeks after it and wants to have it, simply because it is pain...
%   \qauthor{--- Cicero's \textit{de Finibus Bonorum et Malorum}}
% \end{savequote}

% Proposed new structure:
% Global Scholarship Programs
    % A (Recent) History of Global Scholarship Programs
    % The Specific Programs we Work With
    % New and Pressing Selection Problems for these Programs
% Existing Quantitative Decision Support Approaches
% Decision Support Tools
% Generative AI Detectors
% Explainable AI
% Quantitative Modelling for Diversity


\chapter{\label{ch:methods}Methodology}

\minitoc

\section{Paradigms}
\subsection{Participatory Design}\label{ssec:participatory_design}
Participatory Design (PD) is a design paradigm in HCI emphasising active involvement of all stakeholders, particularly end-users, in the design process \cite{Hussain2014OverviewOV}. This approach recognises these users as best-positioned to speak to their subjective needs and preferences \cite{Hussain2014OverviewOV}, ensuring that designs meet user needs.

PD is a fundamentally collaborative process in which the designers and users work together to create solutions \cite{Tokranova2022ApplyingPD}. By involving users in each iteration of the development of a design, PD empowers participants to build tools that serve them \cite{Hussain2014OverviewOV}.

Historically, PD has placed special emphasis on inclusion of a broad variety of user groups \cite{Brankaert2019IntersectionsIH}. For example, \textcite{10.1145/3544549.3573821,10.1145/3544548.3580933,Chowdhury2023ReflectionsOO} employ child-centric PD. \textcite{Brankaert2019IntersectionsIH} call for (and employ) PD aiming to serve users with dementia.

This focus on inclusion, especially a focus on inclusion of oft-overlooked groups, is in itself a value-laden assumption of the PD process. That is, to employ PD is to assert the voices of the selected participants as valuable, both in the design of the tool and in the broader context of research.

\subsection{Action Research}\label{ssec:action_research}\footnote{This section reproduces text from Chapter \ref{ch:genai}.}
Action Research (AR) is a research philosophy that emphasises ``research with, rather than on, people'' \cite{bradbury_action_2003}. Rather than one specific method, AR is best seen as a collection of related methods all embodying this ethos, usually with the goal of producing research contributions useful to the target group of people \cite{lu_organizing_2023}. Among these are semiotic inspection \cite{DeSouza_Leitão_2009,Alvarado_Waern_2018} and participatory design  (PD) \cite{braun_using_2006,Griffiths_Johnson_Hartley_2007,blythe2014research,Knapp_Zeratzky_Kowitz_2016}. AR is most often used in the context of social work, but can be applied across a variety of fields \cite{dombrowski_social_2016,lu_organizing_2023}. 

In education, AR is often used in a classroom setting \cite{Mertler_2019}. \textcite{venn-wycherley_realities_2024} argue that it is crucial in this setting to perform AR on both educators (teachers) and educatees (students), as failing to do so is liable to yield contributions useful to one group but not the other. While this holds for classroom settings, engagement across the stakeholder map is less feasible or desirable in scholarship selection. Unlike teacher and student, who share the common goal that the student learn, selector and applicant are at cross purposes: practitioners seek to choose the `best' cohort of applicants (although they often disagree on what constitutes `best'), while applicants seek to be included in the chosen cohort \cite{bergman2021seven}. Thus, when elucidating the interests and desires of one group, the other will merely act as noise. (E.g., applicants who use genAI to assist in writing their application will, of course, oppose using systems that monitor genAI usage to disqualify applicants.)

AR is comparatively new to HCI \cite{Hayes_2011,lu_organizing_2023}, but its methods and philosophies closely mirror longstanding pillars of HCI \cite{Hayes_2011}. Much like PD and other HCI methods, AR seeks to democratise the research and design processes; unlike PD, AR extends beyond building solutions democratically, and sees learning through action as the ultimate research contribution \cite{Hayes_2011}. For example, AR sees all parties Become: ``Co-investigators of, co-participants in, and co-subjects of...the project'' \cite{Hayes_2011}.  Thus, research questions are formulated by and with participants, actions and interventions are designed by and with participants, and results are found by and with participants \cite{Hayes_2011}.

\subsection{The Relationship between PD and AR}
While PD and AR share the notion that research participants should be closely involved in the research process, they differ in their ontology.

The two differ in the nature of their research outputs. PD is primarily concerned with the design process of products, systems, or interfaces; thus, research contributions are primarily those finalised designs \cite{citation needed}. AR, in contrast, is primarily concerned with the learning that occurs through action; contributions in AR are more often learnings that occurred in the act of doing \cite{Hult1980TOWARDSAD}. 

They also differ in the kinds of research. While PD employs an iterative design process (and often involves evaluation), each stage in this iteration is a design stage. AR, in contrast, involves cycles of three distinct activities: planning (i.e., design), acting, and reflecting \cite{Hult1980TOWARDSAD}.

Finally, and most importantly, they differ in the relative role of the user. In PD, the user \emph{participates} in the act of design; a distinction remains between researchers, who facilitate discussion and implement user requirements, and participants, who discuss and require \cite{Hussain2014OverviewOV}. In AR, this distinction is elided, rendering the participant a co-researcher; it is not uncommon for participants to appear as authors in AR processes \cite{Hayes_2011}. In PD, the user is involved in the design process, while in AR, the user is involved in the research process.

\section{Methods}
\subsection{} % To-do: add in these methods.