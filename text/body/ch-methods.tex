\chapter{\label{ch:methods}Methodology}

\minitoc

\section{Methodological Framework of the Thesis}\label{sec:context_methodology}

This thesis employs a mixed-methods approach that combines computational evaluation with qualitative investigation of social and organizational contexts. This section outlines the key methodological frameworks that inform the research, grounding it in established practices while tailoring them to the unique demands of studying AI in global scholarship selection.

\subsection{Core Research Traditions}\label{ssec:context_core_traditions}
The methodological framework is grounded in several key traditions:
\begin{itemize}
    \item \textbf{Human-Computer Interaction (HCI) and Participatory Design (PD):} Following \textcite{blythe2014research} and \textcite{Knapp_Zeratzky_Kowitz_2016}, this thesis adopts PD methodologies that center the experiences and needs of user (in this case, scholarship selectors). This approach recognizes that effective DSTs cannot be designed in isolation from their contexts of use, but must emerge through collaborative engagement with stakeholders \cite{braun_using_2006}. The thesis also employs participatory design and action research methodologies (see below) to ensure that research questions and findings are grounded in the real-world experiences of scholarship selectors \cite{Hayes_2011}.
    \item \textbf{Action Research (AR):} Building on \textcite{Hayes_2011} and \textcite{bradbury_action_2003}, this thesis incorporates AR methodologies that emphasize "research with, rather than on, people." This approach is particularly important in the context of scholarship selection, where power dynamics and organizational constraints significantly shape the possibilities for technological intervention \cite{lu_organizing_2023}.
    \item \textbf{Value Sensitive Design (VSD):} The thesis draws on VSD frameworks \cite{VanKleek_Seymour_Binns_Shadbolt_2018} to ensure that algorithmic systems are designed with explicit attention to human values and their implications. This is crucial in selection contexts, where decisions have profound impacts on individuals' lives and opportunities.
    \item \textbf{Critical Algorithm Studies:} Following scholars such as \textcite{noble2018algorithms} and \textcite{oneill2016weapons}, this thesis adopts a critical perspective on algorithmic systems that examines not only their technical performance but their social and political implications. This perspective is essential for understanding how algorithmic DSTs may reproduce or challenge existing inequalities.
\end{itemize}

\subsection{Evaluation Methods}\label{ssec:context_evaluation_methods}
To assess both technical performance and socio-technical implications, the thesis employs:
\begin{itemize}
    \item \textbf{Computational Evaluation:} Standard machine learning evaluation metrics including accuracy, precision, recall, and F1-scores are used to assess algorithmic system performance. Recognizing the limitations of these metrics in capturing fairness concerns, the research also employs fairness-aware evaluation methods including stratification along demographic lines and calibration analysis \cite{mehrabi2021survey}.
    \item \textbf{Human-Centered Evaluation of XAI:} Following \textcite{doshi-velez_towards_2017}, the thesis employs both functionally-grounded evaluation (measuring how explanations affect task performance) and human-grounded evaluation (measuring how explanations align with human cognitive processes). This multi-faceted approach recognizes that technical performance alone is insufficient for understanding the social implications of algorithmic systems.
\end{itemize}

\subsection{Critical and Reflexive Stance}\label{ssec:context_reflexivity}
Drawing on critical algorithmic studies, the thesis adopts a reflexive approach that examines not only the technical properties of algorithmic systems but their social, political, and ethical implications \cite{seaver2017algorithms}. This includes attention to questions of power, representation, and accountability that are often overlooked in purely technical approaches to algorithm design. These approaches emphasize collaboration and shared ownership of the research process, recognizing that effective technological solutions must emerge from genuine engagement with user communities.

\section{Specific Methods Used}
\subsection{Online Surveys}
The practice of running online surveys to gather quantitative data is well-established and often used both within and without HCI \cite{zhao2023fairness,pillai_adoption_2020,krishna_disagreement_2022,mai_user_nodate,bansal_does_2021,binns_its_2018,dzindolet_role_2003,papenmeier_its_2022}. Chapter \ref{ch:xai} makes use of one such survey. We use Prolific Academic to gather participants and Formr to administer our survey \cite{binns_its_2018,Arslan_formr_2019}. We follow \textcite{caldwell_power_nodate} in designing our survey based on a power analysis of the statistical tests we intend to run on the output data.

\subsection{Design Workshops}
Chapters \ref{ch:xai} and \ref{ch:diversity} both make use of group design workshops to refine and evaluate design prototypes. Both follow an experience-prototype methodology \cite{Buchenau_Suri_2000}, and incorporate a few specific methodologies.

Both chapters follow \textcite{Zimmerman_Forlizzi_2017}'s scenario speed dating approach, which sees participants rapidly applying different design prototypes to (real or hypothetical) scenarios.

\textcite{Gatian_1994} has researchers asking participants to choose a favourite among a series of options as a means of comparison, while \textcite{Griffiths_Johnson_Hartley_2007} brings this method to HCI. Chapter \ref{ch:diversity} makes use of this method.

\subsection{Individual Interviews}
Chapter \ref{ch:diversity} makes use of one-on-one interviews with participants to first elucidate participant understanding of diversity. In these interviews, we incorporate several methods.

\textcite{Knapp_Zeratzky_Kowitz_2016}'s `crazy 8s' exercise sees participants give eight feature requests in eight minutes. Ordinarily, this exercise is done with a writing surface, but we have participants do this verbally.

\textcite{blythe2014research} introduces the concept of design fiction, where participants more detail their ideal app. We adapt this to create a ``magic app'', capable of doing anything the participant desires and asking the participant to describe this app.

\subsection{Quantitative Analysis}
Chapters \ref{ch:xai}, \ref{ch:genai}, and \ref{ch:spf} rely on several standard statistical tests. Primarily, we use Student's t-test \cite{Mishra_Singh_Pandey_Mishra_Pandey_2019}, the Analysis of Variance (ANOVA) \cite{Mishra_Singh_Pandey_Mishra_Pandey_2019}, Pearson's test of correlation \cite{Schober_Boer_Schwarte_2018}, Tukey's Honestly Significant Difference test \cite{Kim_2015}, and the Receiver Operating Characteristic curve \cite{hanley1989receiver}. Additionally, we develop a permutation test in Chapter \ref{ch:spf} based on \textcite{good2013permutation}.

\subsection{Qualitative Analysis}
Chapters \ref{ch:xai} and \ref{ch:diversity} engage in inductive thematic analyses of their qualitative results. In doing so, we follow the methodology introduced by \textcite{braun_using_2006} and developed in \textcite{braun_conceptual_2022,braun_toward_2023,noauthor_thematic_nodate}.

\section{Research Design}
Chapters \ref{ch:xai}, \ref{ch:genai}, \ref{ch:diversity}, and \ref{ch:spf} all detail studies conducted according to different research paradigms and employing different methodologies. Each chapter contains a self-encapsulated section on research design. However, Table \ref{tab:method_subsections} provides a high-level overview of the methods and paradigms employed in each chapter.

\begin{table}[htbp]
    \centering
    \begin{tabular}{|l|c|c|c|c|}
    \hline
    & \textbf{Chapter \ref{ch:xai}} & \textbf{Chapter \ref{ch:genai}} & \textbf{Chapter \ref{ch:diversity}} & \textbf{Chapter \ref{ch:spf}} \\
    \hline
    \textit{Participatory Design} & Yes & & Yes & \\ 
    \textit{Action Research} & & Yes & & Yes \\ 
    \textit{Value-Sensitive Design} & & & Yes & \\ 
    \hline
    \textit{Online Surveys} & Yes & & & \\ 
    \textit{Design Workshops} & Yes & & Yes & \\ 
    \textit{Individual Interviews} & & & Yes & \\ 
    \textit{Quantitative Analysis} & Yes & Yes & & Yes \\ 
    \textit{Qualitative Analysis} & Yes & Yes & Yes & \\
    \hline
    \end{tabular}
    \caption{This table answers, for each method or paradigm and each chapter: does this methodology appear in this chapter?}
    \label{tab:method_subsections}
\end{table}
