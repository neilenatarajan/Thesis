\chapter{\label{ch:intro}Introduction} 

% As noted below, your introduction does a nice job in setting the scene for the overall thesis. However, it could do better with properly clarifying what the research questions of the thesis are, and how all the chapters included fit into the overall narrative. 

\minitoc

\section{Motivation}

% [FIXED] Cohort/Team Selection vs Aspirational Quota Selection – The football example used to open the introduction is a fun example, but it would seem to differ significantly from the selection scenarios you focus on in the thesis. How does choosing candidates for scholarship applications, hiring, or university admissions differ from team/cohort selection, as in the football case?  For instance, this often requires assembling groups with specific, predefined compositions. In contrast, the other scenarios generally aim to meet aspirational diversity goals across the entire intake, with metrics analysed post-hoc rather than determining each group's precise makeup. This raises an important question: how do the methods presented in your thesis apply to strict compositional requirements compared to broader, trend-focused selection goals?  (You may, in fact, want to discuss this in 1.2. Scope and Terms as well–as all selection problems aren't equal, it might be useful to introduce fundamental terms relating to different kinds of candidate selection to better define the scope. )

Consider a prestigious global fellowship program receiving thousands of applications from aspiring scholars worldwide. The selection committee faces a complex decision: from this diverse pool of talented individuals, they must choose a cohort that will best advance the program's mission to develop future leaders who will tackle global challenges. Unlike a simple ranking of academic merit, this selection process involves balancing multiple, sometimes competing values. Should they prioritise applicants with the highest test scores, those who demonstrate the greatest potential for social impact, candidates from underrepresented regions, or individuals who have overcome significant adversity? The committee's approach will depend on how they interpret the program's goals and what they believe constitutes the ``best'' cohort, and this decision carries profound implications for both the selected scholars and the communities they will eventually serve.

This ``selection problem'' of choosing who to include echoes all through society. Employers select who they hire. Creditors select who they lend to. Universities select who they admit. And scholarship programmes such as our hypothetical one select who they award. Different organisations' goals and values will determine who they select. The employer may look for the best employee for a specific task, and the creditor seeks debtors who are sufficiently likely to repay loans. However, the universities and scholarship programmes differ. Their selection is not for organisational benefit, but rather for social one \cite{Warikoo_2019}. Thus, instead of selecting only the applicants who yield the best returns to the organisation, they seek to select those who are most deserving, those who will learn the most, those who have the greatest need, those whose presence in the programme will benefit others, or even those who will use their newfound education to most improve society.\footnote{Hirers and lenders are often bound either by law to select the most deserving, and even occasionally select those who will benefit others; however, in general, these institutions are motivated to select to maximise profits within the bounds of the law \cite{schmidt1998validity}.}

Although there is a wealth of research exploring when and how algorithms can make hiring or lending decisions (and whether they should) \cite{schmidt1998validity,schumann2017diverse,raghavan2020mitigating,horodyski_applicants_2023,Leung_Zhang_Jibuti_Zhao_Klein_Pierce_Robert_Zhu_2020}, there is a dearth of research exploring how algorithms can make or support selection decisions in the scholarship context, and what research exists in the university context often likens this problem more to hiring than to scholarship \cite{schumann2017diverse,Steel_Multiple_2018,ijcai2023p819}. Furthermore, just as research on hiring and lending finds flaws in many applications of algorithms \cite{raghavan2020mitigating,horodyski_applicants_2023,Peng_Nushi_Kıcıman_Inkpen_Suri_Kamar_2019}, the scant research that exists on human-led scholarship and university selection finds similar problems \cite{schumann2017diverse}.

In a world flattened by the global proliferation of technology \cite{Friedman_2005}, new global scholarship initiatives aim to select scholars from around the world. These programmes offer applicants all around the world an opportunity to access educational resources that may have previously been inaccessible but exacerbate the problems found in related works. If these programmes can select the ``best'' cohorts of applicants, they can deliver on their stated missions to improve the world through broadening access to elite higher education institutions and training scholars to solve the world's biggest problems. This thesis seeks to enable that mission by supporting these selection processes with algorithmic decision-support tools.

% [FIXED] Scope – Despite the title, the section doesn't establish a scope – e.g. what is in and what is out of scope of this thesis.  What sort of selection problems are within scope and what is not? More importantly, the questions you present in this section read like research questions; indeed, it is appropriate for introductions of theses to articulate a set of central research questions for the entire thesis.  However, it is unclear that these questions relate to some of the chapters; ensure they are able to admit and support the work on explanations (Ch 4), AI based plagiarism (Ch5), or even very much to Chapter 6.  Also, in RQ1 you mention applicant-based decisions and cohort-based decisions interchangeably – are they both in scope?
\section{Scope of the Thesis}

Global scholarship programs, such as Rise and Ellison Scholars with whom this research was conducted, face daunting challenges in their mission to select and cultivate future leaders. These challenges include navigating differing interpretations of the ``best'' cohort due to varying prioritizations of values, ensuring fair comparisons of applicants from diverse global contexts, and mitigating risks such as applicants ``gaming'' the system (e.g., by using generative AI to produce application materials misrepresentative of their aptitude). Existing low-tech decision-making systems are often unequipped to handle these newfound complexities \cite{Latzer_Hollnbuchner_Just_Saurwein_2014}. While selection practitioners (selectors) design innovative solutions, they frequently lack easy access to the information needed to robustly support their decision-making.

Not all selection problems are equivalent, and it is crucial to distinguish between different types of candidate selection to properly define this thesis's scope. We identify two primary categories of selection that differ fundamentally in their approach and objectives:

\paragraph{Cohort/Team Selection} involves assembling groups with specific, predefined compositions to fulfil particular functional requirements. Examples include selecting a sports team where specific positions must be filled, assembling a project team requiring particular skill sets, or creating working groups with predetermined role allocations. In these contexts, selectors typically know in advance how many individuals with particular characteristics they need (e.g., ``we need two defenders, two midfielders, and one goalkeeper''). The selection process is constrained by these structural requirements, and success is measured by how well the final composition matches the predetermined template.

\paragraph{Aspirational Quota Selection} aims to meet broader diversity and excellence goals across an entire intake, with composition metrics analysed post-hoc rather than determining each individual selection decision. This approach characterises most scholarship, university admissions, and similar social programmes. Selectors work toward aspirational targets (e.g., ``we hope to achieve geographic diversity'' or ``we aim to support underrepresented groups'') but do not have strict quotas that must be filled in predetermined proportions. Success is evaluated retrospectively by examining whether the overall cohort reflects the programme's values and mission, rather than whether specific compositional requirements were met.

This thesis engages with both modes of selection. While global scholarship programmes like Rise and Ellison Scholars primarily focus on \emph{Aspirational Quota Selection} in the hopes of achieving diversity and excellence throughout their programmes, both programmes feature programmatic elements of \emph{Cohort/Team Selection} in their selection processes.\footnote{In some cases, this cohort selection is a means of operationalising the aspirational goals of the programme in structures that allow stakeholders to evaluate the programme. In other cases, donors or other stakeholders have specific compositional desires for a given cohort.} The methods and tools developed herein are designed to support selection processes where values-based decision-making predominates over structural composition requirements, and where the ``best'' cohort is determined through balancing multiple competing objectives rather than fulfilling predetermined quotas.

% [FIXED] You discuss "competing values" abstractly. Can you give some examples of values?  Please include "values" as a key term, and examples of these.

\paragraph{Values} in the context of scholarship selection refer to the fundamental principles, beliefs, and priorities that guide decision-making about which applicants to select. These values reflect what selectors and organisations believe constitutes merit, worth, and potential for positive impact. Common values in scholarship selection include: \emph{academic excellence} (prioritising applicants with the highest test scores, grades, or demonstrated intellectual ability); \emph{potential for social impact} (favouring candidates who articulate compelling visions for addressing global challenges); \emph{diversity and inclusion} (seeking applicants from underrepresented backgrounds, regions, or perspectives); \emph{need-based considerations} (supporting those with limited access to educational opportunities); \emph{character and integrity} (valuing ethical leadership and moral reasoning); and \emph{resilience and perseverance} (recognising applicants who have overcome significant adversity). These values often compete with one another; an applicant with the highest academic credentials may not demonstrate the greatest potential for social impact, and those with the greatest need may not come from the most underrepresented regions. In real selection pipelines, these conflicts lead to vastly different understandings of the ``best'' cohort between and within organisations \cite{zimmerman_research_2014}. The challenge for selectors lies in balancing these competing values to select cohorts that best embody their programme's mission.

The primary decision point of overcoming this conflict is termed \emph{Selection}: choosing the most apt cohort of applicants. This central decision is supported by many subordinate decisions, such as: ``What criteria make one applicant (or cohort) more apt than another?'' and ``How can we apply these criteria to select the most apt cohort of applicants?''. Each subordinate decision is itself supported by further decisions regarding programme purpose, metrics, and their application.

The scope of this thesis is the development and evaluation of Decision Support Tools (DSTs) to enhance \emph{Selection} processes within global scholarship and talent investment programs. Specifically, it focuses on building and assessing DSTs that address three critical groups of subordinate decisions:

\begin{enumerate}
    \item Decisions influenced by the interpretability and trustworthiness of AI algorithms (Chapter \ref{ch:xai}).
    \item Decisions concerning the use of generative AI by applicants and its detection (Chapter \ref{ch:genai}).
    \item Decisions related to understanding, operationalising, and fostering diversity within selected cohorts (Chapters \ref{ch:diversity} and \ref{ch:spf}).
\end{enumerate}

This research is conducted in collaboration with two such scholarship organisations, Rise and Ellison Scholars. While this thesis engages with the ethical dimensions inherent in AI-supported selection, it does not aim to provide definitive solutions to all underlying societal inequities. Instead, it seeks to improve the tools and processes available to selectors working within the complex, values-driven paradigms of these programs. Work that falls outside this direct scope, such as an in-depth engagement with critical theory perspectives on selection, is acknowledged in Chapter \ref{ch:discussion}.

\section{Research Questions}

This thesis is guided by the following central research questions:

\begin{enumerate}
    \item[\textbf{RQ1:}] How can Decision Support Tools (DSTs) be effectively designed, implemented, and evaluated to aid selectors in global talent investment programs in making values-driven selection decisions that balance multifaceted objectives like excellence, diversity, and fairness, across both aspirational quota and cohort-based selection paradigms?
    \item[\textbf{RQ2:}] In what ways do emerging AI technologies, particularly explainable AI (XAI) and generative AI (GenAI), impact the decision-making processes, integrity, and perceived legitimacy of scholarship selection, and what frameworks or interventions can mitigate potential harms while leveraging benefits for selectors?
    \item[\textbf{RQ3:}] How can complex and evolving conceptualizations of diversity be translated into practical support mechanisms within DSTs to assist scholarship programs in achieving their diversity-related goals, and what is the real-world efficacy of such mechanisms when deployed in live selection processes?
\end{enumerate}

\section{Contributions} 
The contributions of this thesis are twofold. There are meta-level conceptual distinctions introduced, and also some substantive contributions associated with body chapters.

The meta-level contributions of this thesis are:

\begin{itemize}
    \item A list of decision points facing scholarship programmes uncovered through longitudinal HCI research with Rise and Ellison Scholars.
    \item The SOAI paradigm for designing AI systems that support one of the social benefits of good selection processes.
    \item A set of design recommendations for designers seeking to apply SOAI to build a DST to support selectors.
\end{itemize}

However, this thesis is composed of several papers that make more specific, core contributions to support tools seeking to solve issues of Explainability, Plagiarism, and Diversity. These are detailed in the relevant chapters but are also described here.

\paragraph{Explainability}
\begin{itemize}
    \item Quantitative findings indicating that the problem of explanation-induced unwarranted trust extends to generic post-hoc justifications, but that such criticism only applies in process (Chapter \ref{ch:xai}).
    \item Qualitative findings that post-hoc explanations, properly presented, can make useful ex-post DSTs (Chapter \ref{ch:xai}).
\end{itemize}

\paragraph{Plagiarism}
\begin{itemize}
    \item An evaluation of GenAI detectors GPTZero and Originality.ai on Rise's 2022 and 2023 application data (Chapter \ref{ch:genai}).
    \item The Decision Matrix framework for evaluating the suitability of AI systems as support tools for differing decision points.
    \item A case study using GPTZero to support two decision points facing Rise (Chapter \ref{ch:genai}).
\end{itemize}

\paragraph{Diversity}
\begin{itemize}
    \item The Diversity Triangle, categorising diversity-related themes according to our three definitions of diversity uncovered through inductive thematic analysis (Chapter \ref{ch:diversity}).
    \item Six design prototypes developed through PD for supporting the diversity needs of a given organisation (Chapter \ref{ch:diversity}).
    \item Design recommendations grounded in PD for system implementers supporting the diversity needs of a given organisation (Chapter \ref{ch:diversity}).
    \item A field deployment of Prototype \ref{fig:diversity} to the Rise selection process selecting $500$ finalists from a pool of $2000$ demonstrating the efficacy of this prototype in practice (Chapter \ref{ch:spf}).
    \item A demonstration of a hypothetical application of Prototype \ref{fig:diversity} as an ex-post DST (Chapter \ref{ch:spf}).
\end{itemize}

\section{Thesis Structure}
In a break from tradition, this thesis does not contain a labelled ``background'' or ``related works'' chapter; while some related work exists, most work is related only to one subordinate decision point and thus appears in the relevant chapter. Instead, Chapter \ref{ch:context} serves as an extended introduction and background chapter, including situating this thesis in related work. Following this, Chapter \ref{ch:methods} explores the paradigms that guide research design throughout this thesis, lists methods used throughout the thesis, and ties these methods to specific chapters. (The research designs of specific chapters or studies appear in the relevant chapter.)

Chapter \ref{ch:xai} responds to common criticisms of post-hoc XAI and explores this approach as a scholarship selection DST via PD workshops with selectors from Rise. Chapter \ref{ch:genai} engages selectors from Rise in an AR process and explores the role of generative AI in selection decisions. Chapter \ref{ch:diversity} engages selectors from both Rise and Ellison Scholars in participatory design to explore selector notions of diversity and potential ways to support these considerations; this chapter ultimately develops 6 design prototypes. Chapter \ref{ch:spf} implements one of these prototypes in a field deployment with Rise, evaluates that deployment, and explores other applications of the technology.

Chapter \ref{ch:discussion} discusses the scope of the thesis, including references to critical theory work falling outside the scope; the SOAI paradigm; design recommendations that developers can use to follow SOAI; methodological and technical limitations of the work herein; and this thesis's broader significance in a quickly changing landscape. 

\section{Papers}
\subsection{Archival and Under Review}
\begin{itemize}
    \item Kadeem Noray and Neil Natarajan. 2024. "Selecting for Diverse Talent: Theory and Evidence." Under review at Economics of Talent Meeting, Fall 2024.
    \item Neil Natarajan, Sruthi Viswanathan, Reuben Binns, Nigel Shadbolt. 2024. "'Diversity is Having the Diversity': Unpacking and Designing for Diversity in Applicant Selection." Under review at CHI 2025.
    \item Neil Natarajan, Reuben Binns, Ulrik Lyngs, Nigel Shadbolt. 2024. "XAI: Misleading In Process, but Useful Post Hoc." Under review at CHI 2025.
    \item Neil Natarajan, Elías Hanno, Logan Gittelson, Reuben Binns, Nigel Shadbolt. 2024. "What Are Generative AI Detectors Good For? Evaluating and Implementing with the Decision Matrix." Under review at CHI 2025.
\end{itemize}

\subsection{Peer Reviewed}
\begin{itemize}
    \item \fullcite{natarajan_detecting_2024}
    \item \fullcite{ijcai2023p819} 
    \item \fullcite{natarajan_trust_2023}
\end{itemize}



