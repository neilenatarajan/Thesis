\chapter{\label{ch:context}Background and Context}

% Should be unified and expanded to properly set the scene for the problems addressed in the thesis.
% ⭐️ Layout – It is not clear what the narrative structure of Ch 2 is, sections themselves seem disconnected and unordered – please come up with a more sensible ordering. For instance, you could start with defining the problem scope more clearly, then bring in a historical background on positive discrimination (from 6.3.1) up here, including discussing the civil rights movement etc. Background on the interdisciplinary treatment and discussion of diversity / diversities from 6.3.2 could be very welcome here as well. Perhaps actually avoiding defining diversity in Chapter 1 as a Key Term until then might be beneficial to avoid misleading readers that you are sticking only to the simplistic mathematical definition throughout.

% This section might also be an appropriate place where you provide the xAI background from Chapter 4 – connecting this to SOAI by casting the Decision Matrix and SPF from Ch 7 as kinds of xAI?  (Would this work?). Consolidating the background bits would allow each chapter to simply focus on novel findings. 

% ⭐️ Broader Background Expected – The thesis is quite thin on background for a DPhil thesis–DPhil theses are expected to provide a robust background establishing not only directly prior work but also the greater research context, research methods applied, and so on.  Overall as it stands, the experiment of putting “relevant background alongside each chapter” does not help, because a consolidated approach really allows for sufficient breadth of to be achieved in a simple singular place relating key elements, whereas separately these end up seeming individually thin and quite fragmented.  A consolidated and slightly more broader background chapter would be better.

\minitoc

\section{The Social Value of Selection}\label{sec:social_value}
Scholarship programmes offer long-term benefits to their chosen scholars, often under the theory that providing these benefits improves not only the welfare of the scholars themselves but of society as a whole \cite{DilraboJonbekova_Ruby_2023,Dassin_Marsh_Mawer_2018}. Theories of the mechanisms of this social benefit vary. Some theories rely on the future actions of the chosen scholars. For example, \textcite{Dassin_Marsh_Mawer_2018} argue that scholars are often empowered and disposed to devote themselves to solving global problems. \textcite{Dassin_Marsh_Mawer_2018} also note that these scholars may bring additional returns to their communities, thereby improving the welfare of a broad group of people (though they also express concern over `brain drain', where these scholars do not return to their communities). In contrast, others contend that the mere provision of scholarships to the correct recipients is itself pro-social. \textcite{minkin2023diversity} note that society benefits from making space for a breadth of perspectives; providing scholarships to those who would otherwise be unable to afford higher education may create that breadth of perspectives. Some evidence suggests that this broader range of perspectives brings additional benefits in the form of increased productivity \cite{autor2008does,noray2023systemic}. Besides the gain to organisations, though, some argue that the social mobility brought about by the existence of scholarship programmes yields inherent benefits to society \cite{Dassin_Marsh_Mawer_2018}. Under any of these theories, selecting the ``best'' applicants as scholars is clearly in society's best interest. However, as we explore throughout this thesis, different theories of change yield different definitions of ``best''.

Traditional selection is a human-led process, and many suggest it should remain that way \cite{Latzer_Hollnbuchner_Just_Saurwein_2014}. However, as the number of applicants to scholarship programmes grows, the need for scalable selection processes has grown; with traditional selection processes unequipped to handle these new challenges, organisations are forced to innovate, often turning to algorithmic solutions that offer to solve these difficult problems \cite{Latzer_Hollnbuchner_Just_Saurwein_2014}. This has led to the development of Decision Support Tools (DSTs) to support selection processes. These DSTs range from simple tools like automated essay scoring to more complex tools like AI-driven selection algorithms. The use of these tools is not without controversy. Critics argue that these tools may be biased \cite{dwork_fairness_2012}, or may dehumanise the selection process \cite{binns_its_2018}. However, while certain features of these tools may succumb to some critiques, evolving discussions about fairness in selection render older, human-led selection processes equally vulnerable to critique \cite{Ahnaf2023AHPAP,pmlr-v80-kearns18a}.\footnote{More discussion on the value-laden aspects of selection can be found in Chapter \ref{ch:discussion}. In particular, we revisit fairness in Chapter \ref{ssec:fairness}; we also discuss the position of this research in structures of power in Chapter \ref{sec:reflexivity}.}

\section{Related Work}
This thesis engages primarily with literature seeking to use AI tools to support decision-making in global scholarship selection processes. Unfortunately (or perhaps fortunately), this particular niche of literature is fairly sparse. While many tools do exist, especially those seeking to automate the job of the scholarship selector entirely, we are forced to look beyond this niche for a larger body of related literature. Primarily, we do this by considering other selection contexts (e.g., universities or recruiters). Note that while this body of literature is large, and contains work ranging from automated essay scoring to intellect testing \cite{cozma_automated_2018,condon2014international}, our work is only tangentially related to these fields. Some work explores applicant perceptions of scholarship selection processes \cite{10.1145/3351095.3372867}, but this work is primarily interested in the decision subjects and does not engage with the design of selection processes.

More closely related is the body of algorithmic fairness literature engaging with recruitment, much of which seeks to ensure that AI tools do not discriminate against protected classes \cite{dwork_fairness_2012}. However, as we explore in this chapter, disanalogies between recruitment and global scholarship selection limit the applicability of this work.

Similarly, much work explores the impact of algorithms on educational outcomes. For example, \textcite{NISSENBAUM1998237} explore the risk of algorithmic involvement in education dehumanising the experience. However, though these works are conducted in the same environment, they do not touch on selection itself.

This chapter also explores a model of selection as subordinate decisions and investigations supporting the ultimate decision about which subset of applicants to select as scholars. Many of these subordinate decisions relate to their bodies of literature; these are explored in the relevant chapters.

\section{On Working with Scholarship Programmes}
Unsurprisingly, little research explores this unique and novel context. In part, this is due to the necessity of longitudinal research with or on organisations. The novelty of these programmes, combined with an understandable desire to avoid the scrutiny arising from public identification in research, has made research with or on these programmes challenging.\footnote{Critics may point to the Rhodes Scholarship, founded in 1902, as an example of a long-standing global scholarship organisation \cite{Ziegler_2008}. However, the programme's history of intentional exclusionism challenges its claim to global reach; the Rhodes Scholarship did not accept women until legally mandated in 1977 \cite{Ziegler_2008}, and continued to discriminate against Black South Africans until the legalisation of the African National Congress and the end of Apartheid in 1991 \cite{Ziegler_2008}. Furthermore, though the Rhodes Scholarship does select a range of applicants from around the world, they do so via regional committees that compare applicants from similar backgrounds, thus sidestepping the central difficulty of global selection.}

Fortunately, this thesis engages with two such organisations, Rise and Ellison Scholars. Both are already seeking to embed AI and computational DSTs into their processes (e.g., Ellison Scholars already uses an AI-based automated essay scoring system). Both programmes have asked that they not be identified in public-facing research, and thus we request that reviewers not share details on either programme and to protect the identities of the programmes where possible. For more details on the programmes, see Appendix \ref{app:programmes}.

\begin{table}[htbp]
  \centering
  \caption{This table enumerates relevant challenges facing selectors from the Rise and Ellison Scholars selection teams. Challenges are drawn from discussions with selectors, where descriptions are framed in terms of decisions these programs make.}
  \label{tab:full_decision_list}
  \adjustbox{max width=\textwidth}{
  \begin{tabular}{l r p{0.33\linewidth}p{0.33\linewidth}}
      \toprule
      Challenges & Chapter(s) & Description & Supporting Information \\
      \midrule
      \emph{Refinement} & \ref{ch:xai} and \ref{ch:spf} & A programme may refine its scoring algorithm each year to better score applicants. & Explanations of perplexing AI-generated scores; information about implications of scoring methods for cohort diversity \\ 
      \emph{Diligence} & \ref{ch:genai} & A programme may make holistic decisions about when and how to consider applicants. & Information about which essays (and which parts of essays) were written by genAI; information about whether the genAI-written passages are hallucinations. \\ 
      \emph{Partners} & \ref{ch:genai} & A programme may determine whether to continue channel partnerships, which encourage and support applicants. & Whether any channel partners' affiliated applicants use genAI disproportionately. \\
      \emph{Pipeline} & \ref{ch:genai} & A programme may decide whether to modify their application material or process. & Information about the usage of genAI throughout the application pipeline. \\
      \emph{Gameability} & \ref{ch:genai} & A programme may decide how to modify their application material or process. & Information about how AI-generated essays are scored under the current application process. \\
      \emph{Disqualification} & \ref{ch:genai} & A programme may decide to disqualify an applicant that violates their application guidelines. & Information about whether essays violate application guidelines around genAI usage. \\
      \emph{Diversity} & \ref{ch:diversity} and \ref{ch:spf} & A programme may make cohort-level decisions regarding the diversity of their cohort. & Information about the diversity of possible cohorts. \\
      \emph{Contribution} & \ref{ch:diversity} and \ref{ch:spf} & A programme may make decisions about which applicants to move forward based on their contribution to diversity. & Information about the impact of including different applicants on cohort diversity. \\
      \bottomrule
  \end{tabular}
  }
\end{table}

We engage these programmes, variously, in AR, VSD, and PD. In working with the Rise and Ellison Scholars programmes to support solutions to the central \emph{Selection} decision, the programmes expressed interest in supporting many subordinate decisions and challenges. While some of them, such as automating the essay scoring process, fall outside the scope of this thesis, we isolate three families of challenges that engage with AI or HCI literature and are thus of both programme and research interest. These families are challenges supported by explainable AI algorithms, challenges arising from applicant usage of genAI, and challenges relating to the diversity of selected cohorts. Table \ref{tab:full_decision_list} enumerates challenges of interest to us. 