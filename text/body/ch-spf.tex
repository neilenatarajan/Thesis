\chapter[A Possibility Frontier Approach to Diverse Talent Selection]{\label{ch:spf}A Possibility Frontier Approach to Diverse Talent Selection\footnote{This chapter is based on two papers showcasing research done in concert with Kadeem Noray. Both contributed equally to the research. One paper is currently under review as: Kadeem Noray and Neil Natarajan. 2024. "Selecting for Diverse Talent: Theory and Evidence." Under review at Economics of Talent Meeting, Fall 2024. The other is being prepared for submission as: Neil Natarajan and Kadeem Noray. 2024. "SPF: A Technology Probe Examining Diversity in Selection Processes.". This version draws from both publications; in doing so, it borrows from the rich tradition of talent-related research in economics, as well as the HCI work referenced in Chapters \ref{ch:xai}, \ref{ch:genai}, and \ref{ch:diversity}.}}

% [REJECTED] Cohort optimality – this chapter seems to focus on the optimal "cohort/team" form of selection rather than the aspirational/post-hoc; only in this context does the NP-completeness and complexity seem relevant.

\minitoc

\section{Motivation}
In Chapter \ref{ch:diversity}, we co-designed six prototypes with selectors from Rise and Ellison Scholars, all of which garnered interest. However, as we caution in Chapter \ref{ch:context} and demonstrate in Chapter \ref{ch:xai}, subjective feedback is not a substitute for objective evaluation of a tool's impact on decision-making. Therefore, this chapter moves from design to deployment. We implement a functional version of Prototype \ref{fig:diversity} from the previous chapter and evaluate its real-world impact in a field deployment with the Rise programme.\footnote{It should be noted here that Prototype \ref{fig:diversity} itself draws from theory presented in Section \ref{ssec:measurement}. Thus, the design of Prototype \ref{fig:diversity} implemented in this chapter draws as much from this chapter as from Chapter \ref{ch:diversity}.}

\section{Introduction}\label{sec:spfintro}
For a comprehensive background on economic perspectives on selection and algorithmic decision-making, see Chapter \ref{ssec:context_ai_talent_selection}. This chapter builds upon the qualitative findings of Chapter \ref{ch:diversity} by presenting a quantitative framework and field deployment.

The rise of diversity, equity, and inclusion initiatives suggests that various organisations (e.g., schools, firms, social impact programmes, etc.) are genuinely interested in selecting diverse talent. This is driven, at least in part, by recent declines in discrimination \cite{hsieh2019allocation}, increases in the perceived return to diversity \cite{deming2017growing, page_diversity_2017, noray2023systemic}, and increased social pressure for demographic representation \cite{minkin2023diversity}. In this chapter, we deploy technology based on Prototype \ref{fig:diversity} in a field study with Rise. We seek to understand whether this technology can help organisations select more diverse and talented cohorts.\footnote{Throughout the chapter, we use `talent', `aptitude', and `performance' interchangeably. In doing this, we recognise that organisations generally seek to optimise for vague and often conflicting notions of talent or aptitude, but do so via measurements of applicant performance on assessments or assignments. In practice, this process carries risks; e.g., heterogenous biases in metrics or assessment methods will lead some subgroups to appear less talented or apt than others, even when no true difference in talent or aptitude exists. On the whole, this chapter is more interested in measurements of diversity than of talent or aptitude; thus, while these risks are of crucial importance to fair selection processes, they are tangential to the focus of this chapter.}

To ground and assess our technology, we first develop a model for cohort selection. We assume an organisation receives $N$ applications and must select a cohort of $n<N$ individuals. The organisation seeks to simultaneously maximise the cohort's talent (e.g., mean performance on an ability measure) and its diversity (e.g., proximity to target demographic proportions). This optimisation problem yields a trade-off, which we term the Selection Possibilities Frontier (SPF). The SPF represents the set of non-dominated cohorts—those that are not outperformed by any other cohort on both talent and diversity. The SPF framework provides the theoretical basis for Prototype \ref{fig:diversity}, which we implement using a greedy estimation algorithm that leverages the submodularity of diversity functions \cite{krause2014submodular, huppenkothen2020entrofy}.

Next, we present the results of deploying this tool with Rise during their 2023 selection cycle. Our analysis of their 2021 and 2022 cycles reveals that their selected finalist cohorts were well within the SPF, meaning they could have been substantially more diverse or higher-performing. For instance, the 2022 cohort could have been 14.6\% more diverse with no loss in performance, or 24.1\% higher-performing with no loss in diversity. This indicates that without our DST, Rise was making inefficient selections relative to their own stated goals. In contrast, when Rise used the SPF-based tool in 2023, they selected a cohort that was more diverse, higher-performing, and located very near the estimated frontier. This suggests the DST significantly improved the efficiency of their selection process.

To understand why organisations might select inefficient cohorts without such support, we prove that the underlying optimisation problem is computationally complex. When diversity preferences involve non-mutually exclusive identities (e.g., both ethnic minorities and women), the problem of finding the optimal cohort becomes $\mathbf{NP}$-hard. When these preferences are considered among an organisations myriad other preferences, the problem remains $\mathbf{NP}$-hard. This complexity makes it prohibitively costly for selectors to find the optimal frontier by hand, leading them to choose suboptimal cohorts. Our DST reduces this computational cost, enabling them to make more efficient decisions.

To account for this complexity, we augment our model of cohort selection by forcing organisations to incur a computational marginal cost for each unit of increased cohort diversity. This represents the fact that, to find a more diverse cohort, one must engage in the laborious process of composing potential cohorts and comparing them. This is in sharp contrast to finding more talented cohorts, which is comparatively simple and thus modelled as costless, because each individual's contribution to cohort talent is unrelated to the remainder of the cohort. This updated model appears to better describe organisational behaviour. This has two implications: (1) organisations will tend to select sub-optimal (i.e. non-first-best) cohorts and (2) organisations will improve on both performance and diversity if they gain access to a technology that reduces this computational marginal cost. 

As an aside, we apply our SPF estimation procedure as an ex-post DST to evaluate the efficacy of alternative screening and selection methods. To do this, we leverage two unique aspects of the programme. First, the programme collects both traditional merit-based measures -- including cognitive tests, written essays, and referring organisations -- as well as non-traditional measures -- including peer-reviewed video essays, gamified skill tests, and application platform behaviours. Second, the programme engaged in effectively no screening before receiving concrete projects from applicants, making it possible to estimate valid counterfactual diversity and performance of cohorts had they been screened in different ways. Leveraging these features, we find three key results. First, selecting only based on cognitive ability or traditional metrics would have improved cohort performance relative to random selection, but would significantly restrict the programme from reaching its diversity goals. By contrast, selecting only on peer reviews performs similarly on performance but improves diversity substantially. Second, all alternative selection methods we explore result in selecting cohorts well within the SPF and, therefore, leave substantial diversity and performance gains on the table. Third, the trade-off implied by the SPF between talent and diversity is steeper if traditional measures are used to measure talent than if applicant projects are used. 

\section{Theory and Methods}\label{sec:spfmethod}
\subsection{Evaluating Organisational Decision-Making}\label{ssec:measurement}
Central to this chapter is the desire to evaluate Prototype \ref{fig:diversity} on real results, rather than subjective satisfaction. However, there is no ground truth of decision-making in the scholarship programme. Thus, we define a model of talent selection and use this model in assessing a field deployment of the prototype. This model relies on a key object, the Selection Possibilities Frontier (SPF), to bound the range of possible cohort selections on two axes: average applicant performance on individualised metrics of talent, and overall group diversity. With this model, we can evaluate the programme's decisions in terms of their proximity to the SPF; the closer a programme is to the SPF, the more efficient its selection process. In this case, we deploy Prototype \ref{fig:diversity} with Rise, thus, we are interested in Rise's SPF. Rise does not share the precise aggregation method for their talent and diversity metrics, but a summary of the measurements they collect can be found in Appendix \ref{app:programmes}. For this chapter, it suffices to know that Rise has working definitions of both performance and diversity.

We start by considering a simple version of the organisation's optimisation problem. Organisations receive $N$ applications and must select $n<N$ individuals to form a cohort $c$ from the set of all potential cohorts $C$. The organisation prefers both that the selected cohort is higher-performing on some measure of talent and more diverse. For now, a cohort's diversity can be thought of as the inverse of a multidimensional measure of distance between the set of proportions of the cohort who belong to key demographic groups and a set of target proportions the organisation has for each group (we discuss definitions of diversity in more detail in Section \ref{subsec:dts_nphard}). If we let the performance and diversity of a given cohort $c$ be given by the functions $P(c)$ and $D(c)$, respectively, then the above description is equivalent to letting the organisation's preference function $F\Big(D(c),P(c)\Big)$ exhibit $F_D>0$, $F_P>0$, and $F_{DP}\geq0$, where subscripts indicate partial derivatives. 

\begin{figure}[htbp]
    \centering
    \caption{This figure depicts an example solution to an iteration of the selection problem, which is described in Equation \ref{eq:selection_simple}. The solid blue curve represents the Selection Possibilities Frontier (SPF), the dotted blue curve represents the organisation's indifference curve corresponding to the highest achievable utility, and the blue dot represents the diversity and performance of the optimal choice (i.e. the first-best solution). }
    \label{fig:model_spf}
    \includegraphics[width=\textwidth]{spf/model_spf.png} 
\end{figure}

Conceptually, this would represent a scenario where an organisation can observe $D$ and $P$ for every possible cohort and simply select the one that maximises $F(D,P)$. If we assume the organisation behaves rationally, we know the organisation will not choose dominated cohorts. Formally, $c^*$ can be the optimal cohort if and only if there exists no $c'$ such that $D(c')>D(c^*)$ and $P(c')\geq P(c^*)$ and there exists no $c'$ such that $D(c')\geq D(c^*)$ and $P(c')> P(c^*)$. We know that the optimal cohort must be in the set of non-dominated cohorts which we define as the Selection Possibilities Frontier (SPF). If we assume, for expositional purposes, that the SPF is continuous, we can represent it as the following function:

\begin{equation}
G(p) := \max\Big[D(c)|P(c) \geq p\Big]
\end{equation}

\noindent In words, $G(p)$ merely gives the highest possible diversity for every cohort performance level. The diverse talent selection problem, then, can be represented as choosing a cohort to maximise $F$ subject to a constraint that the choice be on the SPF. Formally: 

\begin{equation}
\max_{d,p} F\Big(d,p\Big) \text{ \bf{ s.t. } } d = G(p), \nonumber 
\end{equation}

\noindent which is equivalent to:

\begin{equation}
\max_{p} F\Big(G(p) ,p\Big). \label{eq:selection_simple}
\end{equation}

The solution to this simple version of the model is depicted in Figure \ref{fig:model_spf}. Notably, this model suggests that organisations should always select cohorts on the frontier, as all cohorts within the frontier are dominated by cohorts that are either at least as diverse and more talented or at least as talented and more diverse.

\subsection{Defining the Classes of Diversity and Performance Functions}\label{subsubsec:div_talent_def}
Without knowledge of the types of functions $D(c)$ and $P(c)$, Equation \ref{eq:selection_simple} proves difficult to instrument in practice. In this section, we define the classes of functions $D(c)$ and $P(c)$ that are relevant to the diverse talent selection problem. In particular, we will formalise \emph{proportional diversity} and \emph{count diversity} as kinds of diversity, and assume performance to be a real-valued individual-level metric, aggregated by summation.

In both cases, here, we work with Rise to identify their preferences w.r.t. different kinds of diversity and performance. We then implement their preferences as functions $D(c)$ and $P(c)$.

\paragraph{Proportional diversity} When organisations make statements like: ``we desire at least $x$ proportion of group $g$'', they are speaking of proportional diversity. But, since organisations aim to select cohorts of a specific size, we can reframe this goal as ``we desire at least $x*n$ individuals from group $g$'', where $n$ is the total number of applicants in the cohort.\footnote{This reframing will turn out to be helpful in Section \ref{sec:spf_alg} when we develop our SPF estimation strategy} If we let $\chi_g(c)$ be the proportion of $c$ in group $g$ and $\sigma_g(c)$ be the total number of applicants in $c$ who are in group $g$, this goal can be formalised into the proportional diversity function:

\begin{equation}
    \begin{split}
        \delta_{g}^{prop}(c,x) &:= n*\min(\chi_g(c), x) \\
        & := \frac{n* \min(\sigma_g(c), x*n)}{n} \\ 
        & := \min(\sigma_g(c), x*n). \label{eq:prop_div_function}
    \end{split}
\end{equation}

Note that the minimum function is used here to formalise ``at least'', so that the function only increases until the proportional threshold is met. If, for example, an organisation selecting 100 applicants would like their organisation to be at least $40\%$ female, the proportional diversity function associated with this goal is $\delta_{female}^p(c, 40) := \min(\vec{\mathbf{f}}*\vec{\mathbf{c}}, 40)$ where $\vec{\mathbf{f}}$ is a Boolean vector indicating which applicants are female and $\vec{\mathbf{c}}$ is a Boolean vector that indicates who is in cohort $c$. This can be thought of as an inverse distance along the dimension of group representation between cohort $c$ and an ideal cohort $c^*$ where $\sigma_g(c^*) = x*n$.

\paragraph{Count Diversity} The second common type of diversity is what we call \emph{Count diversity}. This formalises organisational statements like: ``We desire at least one person from $m$ groups''. To formalise this notion, let $\mathbb{I}(\cdot)$ be an indicator function that is equal to 1 if the condition within is true. We can now represent a count diversity preference as:

\begin{equation} 
    \begin{split}
        \delta_G^{count}(c,m) &:= \min\big(\sum_{g \in G}\mathbb{I}(\sigma_g(c)\geq 1), m\big), \label{eq:count_div_function}
    \end{split}
\end{equation}

\noindent where $G$ is the set of relevant groups the organisation wants to be represented by at least a single individual. This type of function is ideal for representing geographic representation goals where educational institutions, like colleges or scholarships, often have goals like ``we want a student from every state'' or ``we want as many countries as possible represented''. 

\paragraph{Overall Diversity} Ultimately, organisations care about all of their diversity goals, not just one. Thus, the diversity functions that are relevant for an organisation must be aggregated if we want to formalise an organisation's overall preference for diversity. We define this aggregation as an organisation's diversity score $D(c)$, which generally has the following form: 

\begin{equation}
A\big(\delta_{g_1}^{prop}(c,x_1),...,\delta_{g_K}^{prop}(c,x_K),\delta_{G_1}^{count}(c, m_1),...,\delta_{G_J}^{count}(c, m_J)\big), \nonumber
\end{equation}

\noindent where $A(\cdot)$ is an aggregator function. It is essential that $D(c)$ increases as a cohort gets ``closer'' to one of the underlying diversity goals because this is sufficient to identify cases when one cohort dominates another, even if the formalisation misses something subtle or difficult to articulate about the organisation's diversity preferences. A flexible but simple aggregator function is a weighted sum, where organisations can place different emphases on each of the goals. So, for the remainder of this chapter, we use diversity scores of the following form: 

\begin{equation}\label{eq:d_equation}
D(c,\vec{\mathbf{w}},\vec{\mathbf{x}},\vec{\mathbf{m}}, \vec{\mathbf{g}}, \vec{\mathbf{G}}) := \sum_{k\in K}w_k\delta_{g_k}^{prop}(c,x_k) + \sum_{j \in J}w_j\delta_{G_j}^{count}(c, m_j),
\end{equation}

\noindent where $\vec{\mathbf{w}},\vec{\mathbf{x}}, \vec{\mathbf{m}}, \vec{\mathbf{g}}, \vec{\mathbf{G}}$ are vectors of the organisation's weights, proportional targets, count targets, groups of interest to proportional diversity functions, and sets of groups of interest to count diversity functions, respectively.\footnote{Another attractive option is a CES aggregator because it allows for specifying the degree of substitutability between diversity goals, but this comes at a cost to interpretability, as many organisations don't regularly use CES aggregators. Nonetheless, the authors are currently working on establishing whether the estimation procedure presented in Section \ref{sec:spf_alg} is viable for a CES aggregator.} In general, we suppress the vector notation opting to refer to the diversity score as $D(c)$ where this doesn't lead to confusion.

\paragraph{Talent, Aptitude, or Performance} Relative to diversity, our definition of performance is simple. In general, organisations measure aptitude for their programme using an individualised metric, usually performance on some assessment or assignment. Common examples include test scores, essays, or grades for educational organisations or technical interviews for hiring in technology. More sophisticated (though uncommon) measures might be the predicted success of an individual based on a set of performance metrics. In this chapter, we assume that organisations already possess a real-valued talent metric $\rho_i$ evaluated at an individual level.\footnote{For more details on Rise's talent metrics, see Appendix \ref{app:programmes}.} A cohort's \emph{talent}, then is defined as the sum of the talent level of the individual members, which is given by:

\begin{equation}
P(c) := \sum_{i \in I_c}\rho_i,
\end{equation}

\noindent where $I_c$ is the set of all individuals $i$ in cohort $c$. Unlike diversity, $P(c)$ is straightforward because each individual's contribution is $\rho_i$ regardless of whoever else is in the cohort. Note that, as long as an organisation fixes their desired cohort size beforehand, optimising for the sum of $\rho_i$ is identical to optimising for mean $\rho_i$.

We represent an organisation's preference function $F$ as a weighted sum of performance and diversity functions. That is:

\begin{equation}\label{eq:f_spec}
F(D, P, c, \iota) := \iota*D(c)+(1-\iota)*P(c)
\end{equation}

\subsection{Implementing Prototype \ref{fig:diversity} in the Field}\label{sec:spf_alg}
As it happens, the SPF modelled in Figure \ref{fig:model_spf} can be used to implement Prototype \ref{fig:diversity} in the field. I.e., a calculation of the SPF using an organisation's preference function yields all of the data required to plot Prototype \ref{fig:diversity} (which is, as it happens, just a depiction of the SPF presented alongside contextual information designed to help selectors best understand the visualisation). However, as we will demonstrate in Theorem \ref{thm:specific-nphard}, calculating the SPF outright is unfeasible.\footnote{A keen reader may note that, under stricter conditions, others have already introduced algorithms for calculating the SPF outright. For example, \textcite{kleinberg2018algorithmic}'s algorithm can be easily extended to calculate the SPF when an organisation only possesses one proportional diversity preference.} Instead, we rely on a greedy algorithm to approximate the SPF.

Greedy optimisation is the practice of approximating an optimal solution to an iterative process by, at each iteration, making a choice that optimises the process at that iteration (i.e. ignoring iterations before and after) \cite{nemhauser1978analysis}. It is well known that greedy optimisation can be used to build near-optimal subsets of a given set when the objective function is non-negative, monotone, and submodular \cite{Feldman_Harshaw_Karbasi_2017,nemhauser1978analysis}. Though these conditions are not strictly necessary, results are not so clear when one of these conditions is dropped \cite{Feldman_Harshaw_Karbasi_2017}.

While non-negativity is self-explanatory (the objective function cannot be less than zero), monotonicity and submodularity deserve further clarification. In our context, monotonicity will require that cohorts are always more diverse than their smaller sub-cohorts while submodularity will require that an applicant's marginal effect on diversity for a cohort will be (weakly) less than their marginal effect on diversity for a smaller sub-cohort. More formally, a function $D$ defined on subsets of some universe $U$ is monotone if and only if 

\begin{equation}
    \label{eq:mononicity}
    \forall Y \subseteq U, X \subseteq Y: D(X)\leq D(Y),
\end{equation}

\noindent and is submodular if and only if

\begin{equation}
    \label{eq:submodularity}
    \forall Y \subseteq U, X \subseteq Y, x \in U \setminus Y: D(X \cup \{x\}) - D(X) \leq D(Y \cup \{x\}) - D(Y).
\end{equation}

This may appear constraining, but, luckily, diversity functions $\delta_g^{prop}(c)$ and $\delta_G^{count}(c)$, $D(c)$, the talent function $P(c)$, and $F(D,P)$ all satisfy these conditions as defined in Section \ref{subsubsec:div_talent_def}. We show this in Theorems \ref{thm:submodularity_additive}, \ref{thm:monotonicity_additive} and \ref{thm:f_sub_mon} in Appendix \ref{app:spfmath}.

Now that we have established the necessary restrictions on functions $F(D,P)$, we present a greedy algorithm that finds $c$ to optimise $F(D, P, c, \iota)$; by repeating this for various values of $\iota$, we obtain the frontier between $D(c)$ and $P(c)$ (i.e., the SPF). This algorithm relies on two observations. First, any point on the SPF can be represented as the maximum of a weighted sum $f(\iota,c) = \iota*D(c) + (1-\iota)*P(c)$ where $\iota \in [0,1]$. Second, any $f(\iota,c)$ is monotonic and submodular. In this context, the algorithm repeatedly maximises a weighted sum of diversity and talent, varying the weight put on each element in each maximisation. Formally, the algorithm maximises $f(\iota,c)$ $m$ times, where each iteration optimises $\iota = \frac{m_i}{m}$. Then, for each $f$, this algorithm builds each cohort $c$ from $c$ of size $0$ until size $n$ by adding an applicant \textit{not} in the current cohort $c$ ($u \in U \setminus c$) that yields the highest $f$ value (i.e., that maximises $f(c \cup \{u\})$). This algorithm is presented more formally in Algorithm \ref{alg:frontier}. 

\begin{algorithm}
    \caption{Greedy Frontier Optimisation}\label{alg:frontier}
    \begin{algorithmic}
    \State \textbf{For} each desired point on the frontier defined by $\iota \in [0, 1]$
    \State \hspace{\algorithmicindent} \textbf{Let} $f_{\iota} := \iota*P+(1-\iota)*D$ be weighted average of $P$ and $D$
    \State \hspace{\algorithmicindent} \textbf{Begin} with empty cohort $c = \vec{\mathbf{0}}$
    \State \hspace{\algorithmicindent} \textbf{While} cohort $c$ is less than the desired size ($|c| < k$)
    \State \hspace{\algorithmicindent} \hspace{\algorithmicindent} \textbf{Find} applicant $i$ such that adding $i$ to $c$ maximises $f_{\iota}(c + i)$
    \State \hspace{\algorithmicindent} \hspace{\algorithmicindent} $c := c + i$
    \end{algorithmic}
\end{algorithm}

It is well-known that the greedy algorithm yields a $\bigl( 1-\frac{1}{e} \bigr)$-approximation of any submodular, monotonic set function \cite{bordeaux_submodular_2014}. That is, the algorithm selects cohorts whose $f_\iota$ values are at least $\frac{1}{1-\frac{1}{e}}$ of the maximum $f_\iota$ any cohort of that size selected from the same applicant pool. For the avoidance of doubt, a proof of these approximation bounds is presented in Theorem \ref{thm:greedy-approximation} in Appendix \ref{app:greedy-proof}. Thus, the Greedy Frontier Optimisation algorithm returns points on a curve that $\bigl( 1-\frac{1}{e} \bigr)$-approximates the true SPF\footnote{In practice, the outputs of the greedy algorithm do not always themselves form a convex curve. We remove produced points that do not sit on the convex curve.}. We note that this is a worst-case approximation ratio and that the actual approximation ratio may be much better.

\section{A Field Study with Rise}\label{sec:spfresults}

We apply our methodology to evaluate our technology in a field deployment with Rise's Cycle 2023. Through this deployment, we document evidence that Rise selected finalists within the SPF -- consistent with the first prediction of our model -- and that Rise selected much closer to the SPF after they were given an SPF estimate to aid in the selection of their third cohort.

\paragraph{Evaluating Past Selection Decisions} Before implementing our technology, we use the methodology described in Section \ref{ssec:measurement} to determine the efficiency of past selection decisions. In particular, we analyse the finalist selection portion of the 2021 and 2022 application cycles, where the programme must construct a cohort of at most $500$ applicants from a pool of roughly $2000$. 

This analysis requires two steps: (1) applying Algorithm \ref{alg:frontier} to both cohorts to estimate the SPF and (2) comparing the actual talent and diversity levels of the finalist cohort to the estimated SPF. The model we developed in Section \ref{ssec:measurement} would suggest that this comparison should find that the chosen cohorts are on or near this frontier. 

\begin{figure}[htbp]
    \centering
    \begin{subfigure}[b]{0.4\textwidth}
        \includegraphics[width=\textwidth]{spf/yr1_spf_finalist.png}
        \caption{The SPF for the 2021 finalist selection process. In Cycle 2021, cohort diversity could have been improved by $15.2\%$ without any reduction in cohort performance, and cohort performance could have been improved by $15.6\%$ without any cost to diversity.}
        \label{fig:spf_2021}
    \end{subfigure}
    \hfill
    \begin{subfigure}[b]{0.4\textwidth}
        \includegraphics[width=\textwidth]{spf/yr2_spf_finalist.png}
        \caption{The SPF for the 2022 finalist selection process. In Cycle 2022, cohort diversity could have been improved by $13\%$ without any reduction in cohort performance, and cohort performance could have been improved by $19.6\%$ without any cost to diversity.}
        \label{fig:spf_2022}
    \end{subfigure}
    \caption{These figures depict the SPFs we estimate for the 2021 and 2022 finalist selection processes. The y-axis represents the diversity score while the x-axis represents average cohort performance (i.e. project scores). The green curve is our estimate of the cycle SPF, which represents the upper bound of diversity that is achievable at every level of cohort performance. The red dot depicts the actual level of diversity and performance of the finalists that were selected. The vertical and horizontal dashed red lines represent the maximum Pareto gain that was possible along the diversity and performance dimensions respectively. These figures are reproduced at a larger scale in Appendix \ref{app:spffigures}.}
    \label{fig:spf_2021_2022}
\end{figure}

The results from these two steps are depicted in Figure \ref{fig:spf_2021_2022}. Surprisingly, neither the 2021 nor 2022 finalist cohorts are chosen on the frontier. (We confirm that these apparent gaps between frontiers and chosen cohorts are statistically significant using a permutation test in Figure \ref{fig:permutation_tests}.)

These results confound the simple model from Section \ref{ssec:measurement}, which suggests that organisations should always select cohorts on the frontier, as all cohorts within the frontier are dominated by cohorts that are either at least as diverse and more talented or at least as talented and more diverse.

\paragraph{Evaluating Selection Decisions with Decision Support}
Now we turn to analysing what happened to selection in the talent investment programme when they were given access to our DST in Cycle 2023. Again, we first estimate the SPF. However, rather than immediately comparing chosen finalists to this estimate, we instead construct a functional implementation of Prototype \ref{fig:diversity} using this estimate.

\begin{figure}[htbp]
    \centering
    \caption{This figure displays the SPF-based DST provided to Rise selectors in Cycle 2023. In addition to the SPF itself, selectors were given access to a myriad of supporting information. While this information cannot all be presented here, much of it describes the candidate optima (i.e., the cohorts represented by the coloured dots). In particular, selectors were interested in the spread of performance scores in each cohort, as well as the extent to which each cohort satisfied programme diversity targets.}
    \label{fig:spf_dst}
    \includegraphics[width=.9\textwidth]{spf/spf_dst.png} 
\end{figure}

Selectors were provided with this DST to inform their decision-making process. The tool, as depicted in Figure \ref{fig:spf_dst}, presented the estimated SPF curve along with several pre-calculated candidate cohorts (the ``coloured dots''). For each of these candidate cohorts, selectors could review supporting information, including the distribution of performance scores and the extent to which various programme diversity targets were met. This allowed them to evaluate these specific, pre-defined options and understand the associated trade-offs. After selecting a desired cohort along the SPF, this cohort is used to inform a shadow price $\iota$. Participants were then repeatedly shown all available information on the next best applicant to add to the cohort according to chosen shadow price $\iota$ and asked to rule that candidate in or out.\footnote{Some details on information available to participants are provided in Appendix \ref{ssec:rise}, but most was excluded at the request of the programme.} This process was repeated until the desired size was reached. After the selection process was completed with the aid of this tool, we then compare the actual finalist cohort's diversity and talent to the SPF estimate.

% [FIXED] The experiment design needs significantly more information about how the DSTs were used: what were the participants given? Just the graph plus the info? Are these the only cohorts? What happened if they wanted to substitute people etc.? Could they get feedback on these? In short, this is not replicable as is – just requires more detail to make it so.

The results from this analysis are depicted in Figure \ref{fig:spf_2023}. Here we see notable differences in the selection patterns relative to Cycles 2021 and 2022. In particular, the Cycle 3 finalist cohort is nearly on the SPF, making the possible Pareto improvements in both directions no more than $2\%$. This suggests two things. First, it provides further evidence that selection decisions in Cycle 2021 and Cycle 2022 were, in fact, inefficient; had Rise known about the possibility of making Pareto improvements relative to their stated preferences, they likely would have changed their behaviour. Second, it provides evidence that the DST presented here actually influences the decisions of selectors. 

\begin{figure}[!htb]
    \centering
    \caption{This figure displays the SPF for the Cycle 2023 finalist cohort. Again, the y-axis represents the diversity score while the x-axis represents average cohort performance, the green curve is our estimate of the SPF, and the red dots depict the actual level of diversity and performance of the finalists that were selected. In this case, we overlay the finalist cohorts from 2021 and 2022 to provide a point of comparison. The diagonal dashed red line represents the distance in diversity-performance space between the Cycle 2022 cohort and the Cycle 2023 cohort. In Cycle 2023, there are no significant Pareto improvements in either diversity or performance.} 
    \label{fig:spf_2023}
    \includegraphics[width=.9\textwidth]{spf/yr3_spf_finalist.png}
\end{figure}

To determine whether the improvements are statistically significant, we leverage a permutation test depicted in Figure \ref{fig:permutation_tests}. The key comparison is between Cycle 2023 max Pareto improvements in talent and diversity (the solid blue vertical lines in both panels) and the corresponding 95 percentile of the random difference distributions (the dashed black vertical line). For both dimensions, the possible improvements are statistically insignificant. In contrast, Cycle 2021 and Cycle 2022 both display statistically significant max Pareto improvements. Ultimately, though this confounds the predictions about selector behaviour implied by the model in Section \ref{ssec:measurement}, it does suggest that the DST is effective in improving selection decisions.

\begin{figure}[htbp]
    \centering
    \caption{This figure displays permutation tests comparing the potential Pareto improvements along the diversity and talent dimensions to the distribution of differences on both dimensions from 1000 randomly drawn pairs of cohorts. The dashed black vertical line represents the 95 percentile of these differences. The solid vertical lines represent the maximum Pareto gain on performance and diversity in each application year. We interpret inefficiencies at or larger than the 95 percentile of the distribution as significant, thereby sticking to the conventional $\alpha$ value. While Cycles 2021 and 2022 both appear to have significant inefficiencies, Cycle 2023 does not.}
    \label{fig:permutation_tests}
    \includegraphics[width=.9\textwidth]{spf/permutation_tests.png} 
\end{figure}

\section{A Plausible Explanation for Selection Inefficiencies}\label{sec:spfexplanation}
\subsection{Why Are Organisations Selecting Pareto Inferior Cohorts?}\label{subsec:dts_nphard}
The results of our field study, specifically w.r.t. the 2021 and 2022 application cycles, suggest that organisations are not selecting cohorts on the SPF. This is surprising, as the SPF model suggests that organisations should always select cohorts on the frontier, as all cohorts within the frontier are dominated by cohorts that are either at least as diverse and more talented or at least as talented and more diverse. In conversation with Rise, we have come to two plausible explanations for why this might be.

First, it might be that the axes of performance and diversity fail to capture the full dimensionality of organisational preferences. One mechanism for this, supported by the revelation from Chapter \ref{ch:diversity} that programmes desire representations of specific idiosyncrasies, is that our axes fail to capture an idiosyncratic preference possessed by Rise selectors. Another mechanism, also supported by Chapter \ref{ch:diversity}, may come into play if the organisation's quantifications of talent do not capture the full scope of their concerns; i.e., if part of the measured ``talent'' an organisation selects for is only captured qualitatively, it cannot be factored into our model.

Second, however, it may be that there is an inherent cost associated with approaching the frontier. This would make organisational selections in the 2021 and 2022 application cycles second-best, in that they are optimal according to a model placing certain costs on their selections. 

The selection decisions in Cycle 2023 favour our second explanation; if the organisation had uncaptured preferences, we would expect these to play a similar role in Cycle 2023, and to result in an apparently suboptimal cohort selection. However, the cohort selected is, according to our model, Pareto optimal.\footnote{More specifically, our chosen cohort is not significantly within the frontier.} Thus, we must seek out the cause of this inherent cost.

\subsubsection{Diversity Causes Complexity}\label{subsubsec:nphard}
The most plausible source of cost in approaching the frontier is the impracticality of selection teams discovering the frontier by hand. This is particularly sensible, as we prove here that calculating the frontier outright is $\mathbf{NP}$-hard; that is, there are no known efficient algorithms that can calculate the SPF outright \cite{COPPERSMITH198527}.

To see why intuitively, consider a college that aims to accept some target fraction of Black applicants from a pool of Black and White applicants. And, assume the school wants to select as talented a class as possible, where talent is proxied for by test scores, grades, or some combination of the two. In this special case, as shown in \textcite{kleinberg2018algorithmic}, there exists a computationally easy algorithm to calculate the SPF shown below.\footnote{Technically, the algorithm presented by \textcite{kleinberg2018algorithmic} only optimises for the \emph{most diverse} point on the SPF. Thus, we have added the \textbf{Repeat} step (cycling through the algorithm with different representation thresholds) to enable their algorithm to trace out the SPF.}

\begin{algorithm}
    \caption{A Procedure For Calculating the SPF Based on \textcite{kleinberg2018algorithmic}}\label{alg:kleinberg}
    \begin{algorithmic}
        \State \textbf{Define} a minority group (Black) and mutually exclusive majority group (White), 
        \State \textbf{Rank} applicants within their group by test score,
        \State \textbf{Define} a target proportion of Black admits,
        \State \textbf{Select} Black applicants from the highest ranking down until the target is reached,
        \State \textbf{Select} the White applicants from the highest ranking down for the remaining slots,
        \State \textbf{Repeat} steps 1-5 for different thresholds of representation to trace out the SPF.
    \end{algorithmic}
\end{algorithm}

What allows this algorithm to work is the mutual exclusivity of the minority and majority groups. This allows one to transform the diverse talent selection problem into two separate talent maximisation problems where the organisation simply selects the most talented members of each group. This can be extended to any case where target proportions are defined at the level of mutually exclusive group level, even when multiple different demographic dimensions are considered. To be concrete, if the organisation cares about race and gender and, thus, has target proportions for black male, black female, white male, and white female applicants, then the problem can be broken into four separate talent maximisation problems where the most talented members of each group are selected until the target proportions are met for each group. 

But what happens if an organisation has preferences for the representation of non-mutually exclusive groups? (I.e., what if an organisation places nonzero weight on two proportional diversity functions?) To continue the running example, this would be analogous to a college that has target proportions for black applicants and female applicants, but not for each race by gender combination. This seemingly small change prevents an organisation from transforming the problem into simpler group-specific talent maximisation sub-problems. To see this, consider applying the \textcite{kleinberg2018algorithmic} algorithm to each group sequentially; this would mean selecting the best black applicants until reaching the target proportion, then doing the same for female applicants. If the most talented black applicants were male or if there were few talented white females in the pool, having allocated the black slots in this way forces the college to select less talented females than optimal (or, it may inhibit reaching the target proportion for females at all).\footnote{An alternative strategy might iterate over different intersectional targets that satisfy the original two targets; this strategy still suffers from non-polynomial growth.} In short, when diversity preferences are over non-mutually exclusive groups, we cannot cleanly and efficiently break the problem into simple talent maximisation subproblems for disjoint minority groups, so it is not clear how we might extend the \textcite{kleinberg2018algorithmic} algorithm to a general version of the diverse talent selection problem. 

The general diverse talent selection problem allows organisations to have preferences for the representation of an arbitrary number of overlapping (or disjoint) demographic groups. This aligns more closely with the diversity preferences of real-world organisations like colleges, firms, and social impact programmes, many of which aim to select personnel from various ethnicities, genders, classes, geographies, ideologies, and specialities. Organisations generally state their preferences using statements of the following form: ``the organisation desires at least $x\%$ of group $g$'' or ``the organisation desires at least one person from $m$ groups''. These types of diversity preferences are what is formalised in the function $D(c)$, which forms an integral part of the class of functions $F$; we demonstrate here that calculating $F$ is $\mathbf{NP}$-hard.\footnote{We do this via `reduction' to the Vertex Cover. A reduction is simple: $A \leq B$ (i.e., $A$ reduces to $B$) if and only if there exists a polynomial time algorithm that makes some polynomially bounded number of calls to $B$ and thus returns an answer to $A$. In other words, we say that $A$ is $\mathbf{NP}$-hard if and only if $\forall B \in \mathbf{NP} A \leq B$. It is clear to see, then, that if $B$ is $\mathbf{NP}$-hard and $A \leq B$, then $A$ is also $\mathbf{NP}$-hard. For more details on reductions, see \textcite{10.5555/1074100.1074233}.}

We now prove that, for the class of functions $F$ of the form from Equation \ref{eq:f_spec}, the problem of finding the optimal subset of size $k$ for any $f_i \in F$ is still computationally complex. This time, we rely on the assumption that $\mathbf{NP}$-hard problems are computationally complex. That is, Theorem \ref{thm:specific-nphard} holds. 

\begin{theorem}\label{thm:specific-nphard}
    Let $U$ be a `universe' set of size at least $N \geq 2*n$ and $F = \{f: \mathcal{P} (U) \rightarrow \mathbb{R}\}$ be the set of functions described in Equation \ref{eq:f_spec}. Then $Opt_{spec}(f_i, n) := argmax_{c \in U \land |c| = n}(f_i(c))$ is $\mathbf{NP}$-hard in $n$.
\end{theorem}

To do this, and to justify the significance of this result, we bring in the computational complexity of the Vertex Cover problem, which has been proven to be $\mathbf{NP}$-hard \cite{COPPERSMITH198527}. Vertex Cover can be seen in Theorem \ref{thm:vertexcover}.

\begin{theorem}\label{thm:vertexcover}
    Let $G = (V, E)$ be a graph. Let $VC(G, \kappa) := Cov | Cov \subseteq V \land |Cov| = \kappa \land \forall e \in E . \exists v \in Cov . v \in e$ be a function of $G$ that returns a set $Cov$ such that every edge in $G$ is incident on at least one vertex in $Cov$. Then $VC$ in $\mathbf{NP}$-hard in the number of vertices.
\end{theorem}

We now prove Theorem \ref{thm:specific-nphard} by reduction to Theorem \ref{thm:vertexcover}, assuming that there exists no polynomial time solution to Vertex Cover \cite{COPPERSMITH198527}.

\begin{proof}
Suppose for a contradiction that Theorem \ref{thm:specific-nphard} admits some polynomial-time solution $Alg_{spec}$. I.e., $Alg_{spec}(s_i, k )= argmax_{c \in U \land |c| = n}(s_i(c))$.

\begin{algorithm}
    \caption{An Algorithm for $VC(G = (V,E), \kappa)$}\label{alg:vc_spec}
    \begin{algorithmic}
        \State \textbf{Consider} $U := E$
        \State \textbf{Define} $\vec{\mathbf{g}} := \{g_i = v_i \in e | e \in E \land i \in |V|\}$ such that each $g_i$ has length $E$ and corresponds to whether an edge is incident on vertex $v_i$.
        \State \textbf{Return} $Opt_{spec}(1*D(c, \vec{\mathbf{1}}, \vec{\mathbf{0}}, \vec{\mathbf{0}}, \vec{\mathbf{g}}, \vec{\mathbf{0}})+ 0*P(c)) \geq k$
    \end{algorithmic}
\end{algorithm}

Then consider the algorithm $Alg_{VC}$ that is defined in Algorithm \ref{alg:vc_spec}. But this algorithm solves Vertex Cover in polynomial time relative to $Opt_{spec}$ and thus is a polynomial time solution to Vertex Cover. Assuming $\mathbf{P} \neq \mathbf{NP}$, contradiction!
\end{proof}

\subsection{Embedding Complexity into the Model}\label{subsec:dts_w_complexity}
Knowing the $\mathbf{NP}$-hardness of calculating the SPF outright, we can more comfortably assume that there exists a search cost in approaching the frontier; knowing that this $\mathbf{NP}$-hardness is driven by diversity targets, we can further suppose that this search cost is driven by diversity preferences. This leads us to a new model that incorporates complexity costs into the selection problem.

Consider a variation of the simple cohort selection problem (see Section \ref{ssec:measurement}) where the organisation can search for increasingly diverse cohorts at a cost. Let the amount of search effort be $e\in[0,1]$ and define the cost of search effort to be $\alpha p(e)$ where the cost is convex (i.e. $p_e>0$ and $p_{ee}>0$) and $\alpha$ is a constant that is inversely related to the quality of search technology available. Furthermore, we let the amount of search deterministically increase the maximum achievable diversity at each talent level, which is now given by $D^{SPF}*e$. The optimisation problem can then be rewritten as:

\begin{align}
&\max_{d,p,e} F\Big(d,p\Big) - \alpha p(e) \text{ \bf{ s.t. } } d = G(p)*e, \nonumber \\ 
& \implies \max_{p,e} F\Big(\underbrace{G(p)*e}_{\text{Info Cost}} ,p\Big) - \underbrace{\alpha p(e)}_{\text{Direct Cost}}. \label{eq:objective}
\end{align}

It is clear from the form of the organisation's new objective function in Equation \ref{eq:objective} that the complexity of maximising diversity imposes two kinds of costs: an information cost that represents the fact that the organisation will generally not know which cohort is on the SPF and a direct search cost. Also, when search costs are set to zero (i.e. $\alpha=0$), the problem collapses into the original problem because searching is costless and, therefore, maximised at $e=1$. But, when $\alpha>0$, the optimal cohort will now be inside of the SPF. This is because, for all $c'$ such that  $D(c')=G(p(c')=p')*e$ there exists $c^f$ on the SPF such that $D(c^f)=G(p(c^f)=p')$. Thus, as long as the optimal effort is below 1, any solution to this problem will result in selecting a cohort that is within the SPF and, therefore, non-first-best. The solution to the selection problem with complexity is depicted in Figure \ref{fig:model_complex}.

\begin{figure}[!htb]
    \centering
    \caption{This figure depicts an example solution to an iteration of the selection problem with complexity-induced search costs, which is described in Equation \ref{eq:objective}. As in Figure \ref{fig:model_spf}, the solid blue curve represents the SPF, the dotted blue curve represents the organisation's indifference curve corresponding to the highest achievable utility without search costs, and the blue dot represents the diversity and performance of the first-best solution. Additionally, the solid red curve represents the accessible frontier with optimal search, the dotted red curve represents the highest achievable utility with search costs, and the red dot represents the diversity and performance of the optimal cohort with search costs (i.e. the second-best solution).}
    \label{fig:model_complex}
    \includegraphics[width=1\textwidth,height=\textheight,keepaspectratio]{spf/model_complex.png} 
\end{figure}

Additionally, the extent of the inefficiency will tend to reduce as complexity costs reduce. We can see this by examining the comparative statics of the model. To simplify our derivation of the relevant comparative static, we refer to the organisation's objective function as $O(p,e)\equiv F\Big(G(p)*e ,p\Big) - \alpha p(e)$. Furthermore, we use subscripts on functions to refer to partial derivatives and we drop the arguments of functions where this does not confuse. The (necessary) first-order conditions from this model are, therefore, the following:

\begin{align}
O_p & \equiv F_{d}(G(p)e,p)G_p(p)e + F_p(G(p)e,p) = 0 \nonumber \\
O_e & \equiv F_{d}(G(p)e,p)G(p) - \alpha p_e(e) = 0. \nonumber
\end{align}

To ensure this is a maximum, we also need to assume that the (sufficient) second-order conditions hold. They are the following:

\begin{align}
O_{pp} \equiv &\;  e^2G_p^2F_{dd} + 2eG_pF_{dp} + eG_{pp}F_d + F_{pp} < 0, \nonumber \\
O_{ee}  \equiv &\;  G^2F_{dd} - \alpha p_{ee} < 0, \nonumber \\
&  \; O_{ee}O_{pp} - O_{ep}^2 > 0, \nonumber
\end{align}

\noindent where $O_{ep} \equiv O_{pe} \equiv eGG_pF_{dd} + F_dG_p + GF_{pd}$. Under these conditions, solutions to the first-order conditions both exist and guarantee a maximum. These solutions can be defined as $p^*(\alpha)$ and $e^*(\alpha)$. If we plug this into the first-order conditions and take a derivative with respect to $\alpha$, which governs the complexity costs, we get the following system of equations:

\begin{align}
& O_{pp}\frac{\partial p^*}{\partial \alpha} + O_{pe}\frac{\partial e^*}{\partial \alpha} + O_{p\alpha} \equiv 0, \nonumber \\
& O_{ep}\frac{\partial p^*}{\partial \alpha} + O_{ee}\frac{\partial e^*}{\partial \alpha} + O_{e\alpha} \equiv 0  \nonumber
\end{align}

\noindent where $O_{e\alpha} = -p_e$ and, essential for signing the comparative static, $O_{p\alpha} = 0$. We can then solve for $\frac{\partial e^*}{\partial \alpha}$ algebraically (or using Cramer's rule), which gives the following:

\begin{equation}
\frac{\partial e^*}{\partial \alpha} = \frac{-O_{e\alpha}O_{pp}}{O_{ee}O_{pp} - O_{ep}^2} + \cancelto{0}{\frac{O_{ep}O_{p\alpha}}{O_{ee}O_{pp} - O_{ep}^2}} = \frac{p_eO_{pp}}{O_{ee}O_{pp} - O_{ep}^2} < 0, \label{eq:comp_stat}
\end{equation}

\noindent where the the final inequality holds because of the signs assumed in the first and third second-order conditions. Thus, as complexity costs rise, optimal search effort decreases.

This model, thus, implies two predictions about organisational behaviour: (1) when complexity-induced search costs are sufficiently high, organisations will select cohorts within the SPF and (2) as computational costs are reduced, organisations will select cohorts that are closer to the SPF. We have already seen in Section \ref{sec:spfresults} that both predictions hold in practice.

\section{Alternative Applications of the SPF}\label{sec:spfapplications}
The main body of this chapter implements and evaluates Prototype \ref{fig:diversity} as an in-process DST by conducting Action Research with the Rise programme. However, in this section, we discuss potential ex-post applications of the SPF.

\paragraph{Comparing the Diversity Cost of Alternative Talent Measures} In some cases, organisations may have multiple alternative talent measures that seem equally valid as measures of individual ability. In this case, the tradeoff between each measure and diversity may help an organisation decide which talent measure they prefer.  Two cases where this might be relevant are in hiring and college admissions. In hiring, firms may have multiple measures that predict applicant productivity, but have many ways to weigh the various measures that are roughly equivalent for productivity prediction \cite{hartigan_fairness_1989}. This can happen if productivity is multidimensional (e.g., work per hour, tenure, spillovers on others, etc.), and different measures are correlated with some dimensions and not others. A similar problem can be found in college admissions, where, again, the college has multiple measures of applicant ability and may be close to indifferent about some set of ways of combining them when judging an applicant's talent \cite{tam2002new}. 

In these cases, the SPF estimation procedure allows an organisation to consider another dimension: which talent measures demand the sharpest tradeoffs against cohort diversity? I.e.: which measures yield the smallest SPFs? This is particularly relevant in cases where preferences may be lexicographic, meaning that an organisation wants to maximise talent first, then, conditional on doing so, choose the cohort among top talent cohorts that is the most diverse possible. It also is relevant in contexts where an organisation is not allowed, either legally or internally, to explicitly prioritise diversity in its selection criterion, but still wishes to promote diversity \cite{Bleemer_2023}. 

We demonstrate how to use SPF estimation to compare the diversity tradeoffs of two alternative talent measures. To do this, we estimate the SPF twice, once using each of the talent measures, and then compare the level of diversity at each percentile of both measures. In the case of indifference between the two talent measures on the talent dimension, the measure with higher maximum achievable diversity in the relevant percentile range should be chosen if the organisation cares at all about diversity. We use two measures of talent collected by Rise: a project-based measure and a traditional score. The results of this comparison are depicted in Figure \ref{fig:compare_div_tradeoffs}. Given that the programme does care about diversity, this would justify using project quality instead of the traditional score for selection. 

\begin{figure}[htbp]
    \centering
    \includegraphics[width=\textwidth]{spf/alt_merit_spfs.png} 
    \caption{This figure displays the SPF we estimated for the Cycle 2021 finalist cohort and an SPF based on a more traditional method of measuring performance (i.e. the average of cognitive ability and an essay assessment). The y-axis represents the diversity score while the x-axis represents the average cohort performance on projects or the traditional score. The vertical distance between the SPFs represents the difference in maximal diversity conditional on a cohort performing at a particular percentile of both scores. We see here that, above the 90th percentile of talent for both measures, the project quality measure strictly dominates the traditional score in diversity.}
    \label{fig:compare_div_tradeoffs}
\end{figure}
        
This method can also be applied to compare the diversity-talent tradeoff across application years. To do this, simply estimate SPFs for each application cycle and compare the level of diversity at each percentile of talent. As long as the diversity goals remain the same each year and cohort diversity is renormalized such that the most diverse cohort across all years becomes 1, organisations can compare across years to see whether differences in applicants across years better afford to get closer to their goals. Figure \ref{fig:diversity_across_cohorts} shows just this comparison. In general, the Cycle 2023 SPF allows for selecting more diverse cohorts at every level of talent than the other two cohorts. But, whether Cycles 2021 or 2022 allow for more diversity depends on where in the talent distribution the programme is interested in. Near the top of the talent distribution, Cycle 2022 has more diverse cohorts, but this flips as talent falls below the 94th percentile.

\begin{figure}[!htb]
    \centering
    \includegraphics[width=\textwidth,height=\textheight,keepaspectratio]{spf/spf_cohort_comparisons.png} 
    \caption{ This figure displays the SPFs we estimate for three finalist cohorts. The y-axis represents the diversity score while the x-axis represents average cohort performance (i.e. percentiles of mean project scores). The diversity target is held constant across cohorts, so differences in SPFs conditional on performance represent differences in the capacity to reach the same diversity target at a given level of performance.}
    \label{fig:diversity_across_cohorts}
\end{figure}

\paragraph{Evaluating Alternative Selection and Screening Approaches} Relatedly, organisations may consider using cheaper, but lower quality measures of talent to screen or select applicants. For example, firms may consider using metrics (e.g., cognitive or personality assessments) or recruiters to screen their applicants rather than allow each applicant to be assessed via an interview. In some cases, organisations may be considering replacing costlier selection measures and selecting applicants entirely based on cheaper information. Unlike before, we now assume that the initial metric captures talent much better than the new metric. Thus, rather than comparing different SPFs, we place cohorts selected using new metrics on the SPF estimate drawn using the original metric. If the new metric is not too much worse than the original metric, then the new metric may be a better choice.

Running selection counterfactuals can be done using two types of designs: the first we will refer to as a ``causal'' design and the second we call a ``suggestive'' design. A causal design requires an organisation to run a screening experiment where applicant talent is either evaluated randomly (or all applicants are evaluated). This allows organisations to avoid the selective labels problem whereby results become biased due to the selection of who gets evaluated and who doesn't. Avoiding this problem allows organisations to analyse representative samples, meaning that comparisons between alternatively selected samples and the estimated SPF should extend to the full population of applicants. Thus, barring any significant contextual changes, the results will be the same (in expectation) when used on another applicant pool (in this sense, the results are ``causal''). Alternatively, a counterfactual exercise can be conducted on selected data where only a selected set of individual talents are assessed. In this case, the applicability of the results to another applicant pool is merely ``suggestive'', hence the name ``suggestive design''. Causal designs, though more useful for decision-making, are also more costly, as they require running a selection experiment, which may force organisations to miss out on talent (additionally, using known-inferior selection methods may pose a fairness concern). 

To demonstrate, we return to the talent investment programme example where, in Cycle 2021, we can assess alternative selection procedures using a causal design. This is because, in Cycle 2021, the programme ran a selection experiment to determine whose projects were reviewed. In particular, the programme used a weighted sum of applicants' cognitive ability and peer assessments of their video essays to select the top $1500$ applicants who would receive project reviews. Of the remaining applicants, $500$ were chosen at random to be evaluated as well. This means that, from the total application pool of $2800$, a representative sample can be reconstructed by re-weighting the $500$ randomly assessed applicants such that they represent all $1300$ applicants who were below the project review threshold.

To demonstrate the use of the SPF estimate for counterfactual selection analysis, we compare the efficacy of three alternative selection strategies: the cognitive score, the traditional score, and the peer score. Because the cognitive score and the traditional score both use measures that are closely related to typical talent measures, this comparison also serves as a substantive comparison of traditional selection methodologies and more experimental ones, such as using applicant peer review. The results of this comparison are depicted in Figure \ref{fig:alt_screen}. Here we see that, on the dimension of talent (as measured by project quality) the cognitive ability score performs the worst of the three by far (over $10\%$ worse than the other two scores). The traditional score performs slightly better than the peer score on talent, but the peer score (and the cognitive ability score) performs slightly better than the traditional score on diversity. What is perhaps most striking, however, is that all three alternative selection approaches result in cohorts well within the frontier, indicating that Rise's actual metric far outstrips each hypothetical alternative.

    \begin{figure}[!htb]
    \centering
    \includegraphics[width=\textwidth]{spf/alt_screening_performance.png} 
    \caption{This figure displays various SPF estimates for the finalist cohort using the programme's notion of diversity, a `disadvantage' notion of diversity (drawn from the `contextualising applications' theme in Chapter \ref{ch:diversity}), and a `representativeness' notion of diversity (i.e., Prototype \ref{fig:representativeness}). The y-axes represent diversity scores while the x-axes represent average cohort performance. The green curves are our estimates of three alternative Cycle 2021 SPFs, which are estimates of the upper bound of diversity that is achievable at every level of cohort performance. Each dot represents the performance and diversity of cohorts had they been selected using only cognitive ability (blue), a combination of written essay judgements and cognitive ability (aka a ``traditional'' score, which is green), and just peer review (red).} \label{fig:alt_screen}
    \end{figure}

\paragraph{Evaluating According to Alternative Notions of Diversity} A similar process can help organisations understand the practical implications of different kinds of preferences over types of diversity. While we isolate three themes relating to definitions of diversity in Chapter \ref{ch:diversity}, we note that the Rise programme has a working understanding of what they mean by diversity, and did not wish to adopt any of these notions. In practice, Rise's diversity targets suggest both `representativeness' and `contextualising applications' (which they internally call `disadvantage' or `boostability', variously) as important to their consideration of diversity, while `different perspectives' do not appear in their decision-making process. Thus, Figure \ref{fig:alt_screen} also depicts results using two alternative notions of diversity based only on the disadvantage and representativeness portions of the Rise targets. The disadvantage diversity score puts maximal weight on representing those from various historically disadvantaged groups (e.g., being first-generation, poor, or female) while the representativeness diversity score only uses proportional targets that match the demographic distribution of the applicants. Evaluating each alternative selection method indicates that, while selecting on peer judgements or the traditional score both do substantially better than using cognitive ability alone on the talent dimension, using the peer score is by far the highest performing on both disadvantage and representativeness.

\section{Conclusion} \label{sec:conclusion}
While Chapter \ref{ch:diversity} focused on the theoretical and empirical aspects of diversity, this chapter has focused on the practical implications of diversity in selection. In doing so, we have introduced the notion of a selection possibilities frontier via a simple model of diverse talent selection, have implemented Prototype \ref{fig:diversity} as an in-process DST, and have demonstrated its use in practice. Analysing decision-making with and without our DST, we have shown that Rise selects talented and more diverse cohorts when given access to our SPF-based DST. To explain why, we showed that the diverse talent selection problem is $\mathbf{NP}$-hard and augmented our model with a notion of complexity costs; this new model predicts that organisations who are better and more cheaply able to approximate the frontier should find themselves closer to it. 

Finally, we have also shown that the SPF can be used ex post to compare the diversity tradeoffs of alternative talent measures, evaluate alternative selection and screening approaches, and evaluate according to alternative notions of diversity. 

In an age of rapidly expanding interest in selecting from diverse talent pools (signalled by the growth of DEI), this chapter has wide-ranging policy implications. First, the chapter suggests that organisations will face extreme difficulty achieving their diversity goals unless they are willing to adopt more sophisticated selection technology. Second, this chapter contributes methods that are particularly useful for assessing the diversity impacts of alternative merit-based selection strategies. This extends beyond selection to related contexts like hiring, where not appropriately considering the diversity implications of selection strategies can result in lawsuits, and U.S. university admissions' non-merit-based selection has become a legal grey area despite university commitments to diversity. 