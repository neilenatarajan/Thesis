\chapter[``Diversity is Having the Diversity'']{\label{ch:diversity}``Diversity is Having the Diversity'': Unpacking and Designing for Diversity in Applicant Selection\footnote{This chapter is based on a paper written in concert with Sruthi Viswanathan, Reuben Binns, and Nigel Shadbolt. The paper is currently under review as: Neil Natarajan, Sruthi Viswanathan, Reuben Binns, and Nigel Shadbolt. 2024. ```Diversity is Having the Diversity': Unpacking and Designing for Diversity in Applicant Selection.'' Under review at CHI 2025.}}

\minitoc

\section{Motivation}
Chapters \ref{ch:xai} and \ref{ch:genai} both reveal flaws of existing algorithmic support tools in their applications to decision points facing scholarship selection organisations: these tools lack the properties required to support in-process decision-making. In this chapter, we select a family of in-process decision points of particular interest to the literature: ensuring diversity in selection. This chapter, thus, seeks to understand what diversity is and how to support its consideration.
  
\section{Introduction}\label{sec:divintro}
For comprehensive background on diversity as a societal value, historical injustices in technological systems, and approaches to measuring diversity, see Chapters \ref{sec:context_historical_societal} and \ref{ssec:context_diversity}. Here we focus on the specific research gap Chapters \ref{ch:diversity} and \ref{ch:spf} address.

Processes for selecting people for jobs, universities, prizes, and other opportunities have often failed to reflect the diversity of their actual and potential candidate pool. Recognising this, various sectors have in recent years shifted towards recognising and promoting diversity through the establishment of a variety of related norms: DEI (Diversity, Equity, and Inclusion), EDI (Equality, Diversity and Inclusion), JEDI (Justice, Equality, Diversity, and Inclusion), DEIB (Diversity, Equity, Inclusion, and Belonging), etc. \cite{pinkett2023data,hsieh2019allocation,minkin2023diversity}. These norms are frequently operationalised through changes to application, evaluation, and decision-making procedures designed to result in greater representation of different demographic groups in final selection decisions \cite{pinkett2023data}. Such efforts are in part a response to widespread societal concerns about racial, gender, and other injustices, but have also frequently been justified in economic terms by evidence suggesting that diverse teams perform better than homogeneous ones on a variety of tasks \cite{deming2017growing,page_diversity_2017,noray2023systemic}. Meanwhile, the concept of diversity itself has become swept up in `culture war' discourse, criticised by right-wing commentators as part of a sinister `woke' agenda, and by progressives as mere window dressing that fails to meaningfully address deeper societal injustices. 

For practitioners on the ground, however, the question of how to meaningfully measure and promote diversity is a practical challenge. The proliferation of software for hiring and selection presents both complications and opportunities \cite{Lashkari_Cheng_2023}. AI-driven tools may discriminate \cite{chen2018investigating,li2020hiring,lambrecht2019algorithmic}, but they could also help mitigate human biases and increase diversity \cite{yarger2020algorithmic,avery2024does,will2023people, suhr2021does}. This context challenges Human-Computer Interaction (HCI) research to improve diversity in selection without amplifying existing inequities.

\begin{figure}
    \includegraphics[width=\textwidth]{diversity/diversity_is.png}
    \caption{This figure shows participant codes defining what ``diversity is''. In this chapter, we seek to answer: what do they mean and how do we design for that?}
    \label{fig:diversity_is_teaser}
\end{figure}

But to improve diversity, we must first understand it. We undertake two studies to do so. In Study 1, we conducted 15 one-to-one interviews with selection practitioners (selectors) from global scholarship programmes to understand how they define and operationalise diversity. Our inductive thematic analysis revealed the ad-hoc nature of current practices and surfaced three distinct definitions of diversity: as `different perspectives' in the same room; as `representativeness' of a target population; and as `contextualising applications' with information about an applicant's relative privilege. We conclude that technological interventions should first identify which definition(s) of diversity they aim to promote, and we construct the \emph{Diversity Triangle} (Figure \ref{fig:div_triangle}) as a conceptual guide.

In Study 2, we developed six design prototypes based on the Diversity Triangle and presented them to participants in participatory design workshops. The prototypes included tools for visualising cohort representativeness, measuring entropy (the average number of in-group differences), and assessing individual applicant (dis)advantage. They were 
presented to participants via participatory design workshops \cite{Zimmerman_Forlizzi_2017}. The workshops revealed that while quantitative tools are essential, selectors rely on their own idiosyncratic lenses and qualitative assessments to navigate diversity considerations.

This research contributes to understanding how data-driven tools can support diversity in selection. Our contributions are:

\begin{enumerate}
    \item Three practitioner-led definitions of diversity uncovered through inductive thematic analysis.
    \item The Diversity Triangle, a framework for categorising diversity-related themes.
    \item Design recommendations grounded in participatory design for building diversity-supporting tools.
\end{enumerate}

\noindent More broadly, this work demonstrates that by providing structured, data-supported approaches to diversity, organisations can better navigate the complexities of DEI (EDI, JEDI, DEIB, etc.). While differing in some respects, we believe these implications will generalise from scholarship selection to various other talent identification contexts, including recruitment for jobs and admission to universities. By helping these organisations achieve their desired outcomes in selection processes, we aim to ultimately contribute to and help build a more diverse society.

\section{Experimental Design}\label{sec:methods}
\subsection{Our Studies}
Our research is broken into two components: 15 one-to-one interviews with scholarship and talent investment selectors and two participatory design workshops with a subset of the 15 selectors. The relationship between these components is shown in Figure \ref{fig:flowchart} and details on Studies 1 and 2 can be found in Sections \ref{ssec:methods1} and \ref{ssec:methods2}, respectively.

\begin{figure}[htbp]
    \centering
    \includegraphics[width=.9\textwidth]{diversity/flowchart.png}
    \caption{This research begins with 15 interviews seeking to understand what selectors mean when they talk about diversity and how to support that, followed by two scenario-speed-dating activities where these selectors test several prototypes built based on the interviews.}
    \label{fig:flowchart}
\end{figure}

In the first session, we seek to ascertain how these selectors understand diversity, how they operationalise it in processes for selecting talented applicants, and how they envision using technology to assist them in that process. In the second session, we show these selectors six prototypes built in response to the interviews, then we aim to collaboratively design tools that can help them better consider diversity in selection.

\subsection{Positionality}
Following \textcite{venn-wycherley_realities_2024}, we state researcher positionality here. All authors endorse diversity as a societal and organisational value (while sympathetic to the critiques of \textcite{Ahmed_2012,Warikoo_2019}); we thus contend that improving organisational capability to consider diversity is generally a positive development. The research team is comprised of three men and one woman; ethnically, two researchers are South Asian, while two are White; researchers represent three different primary nationalities; all researchers are affiliated with the University of Oxford.

\subsection{Participants}
We engaged selectors (N=15) from two scholarship and talent investment programmes. These participants are numbered P1-P15.

Before either study, we obtained informed consent from all participants to be included in both studies. All participants were given the option to recuse themselves from Study 1 at any point until publication, but, as Study 2 is a group workshop (and as we do not do participant-level attribution), participants were asked to recuse themselves before this second study. Two participants recused themselves before Study 2 but gave leave to be included in Study 1. Participants also gave consent to be recorded, and to have these recordings stored on a secure server. All recording, transcribing, and data analysis was conducted on secure servers. Ethics review was performed by the University of Oxford's Central University Research Ethics Committee.

\section[Study 1]{Study 1: Interviews and Thematic Analysis}\label{sec:study1}
\subsection{Methodology}\label{ssec:methods1}
Our interviews aim to answer three questions in each organisational context:

\begin{enumerate}
    \item What is diversity?
    \item Which elements of diversity matter in a selection context? Why?
    \item How could technology assist in operationalising diversity?
\end{enumerate}

In answering our first set of research questions, we follow \textcite{braun_using_2006}'s methodology for reflexive thematic analysis. We first conduct 45-minute semi-structured interviews with 15 selectors. In each interview, we first ask general questions about their selection methodology; we next ask specifically about diversity and its role in selection; we move on to \textcite{Knapp_Zeratzky_Kowitz_2016}'s `crazy 8s' exercise, where participants give eight feature requests in eight minutes; we conclude with a `magic app' exercise inspired by \textcite{blythe2014research}'s design fiction, where participants more thoroughly detail their ideal app. A question-by-question protocol for these interviews is supplied in Appendix \ref{app:divprotocol1}.

The lead author interviewed participants and then transcribed and anonymised the interviews. The lead author and another author (who wasn't present during the interviews) then independently `open-coded' each anonymised transcript to mitigate bias, looking for anything relevant to our research questions. The researchers met six times to discuss their open codes, then shared these codes with the remainder of their research team across four meetings; the researchers grouped codes into 6 themes and 18 subthemes after consensus was reached. These themes are detailed in Table \ref{tab:themes} and described in Section \ref{ssec:themes}.

\begin{table}[htbp]
    \centering
    \caption{Our three central subthemes all speak to the question ``Why diversity?''. Other themes and subthemes reflect types of diversity, concepts intertwined with diversity, and other considerations scholarship programmes must weigh against diversity desires.}
    \label{tab:themes}
    \begin{tabular}{|p{0.45\textwidth}|p{0.45\textwidth}|}
        \hline
        \multicolumn{2}{|c|}{\textbf{Themes and Subthemes}} \\
        \hline
        \textbf{Why Diversity?} & \textbf{Types of Diversity} \\
        \underline{Different perspectives} & \underline{Socioeconomic} \\
        \emph{...in the same room} & \emph{parental income} \\
        \underline{Representativeness} & \emph{parental education} \\
        \emph{...of a general population} & \emph{generational wealth} \\
        \emph{...of the eligible population} & \underline{Sex, gender, and sexuality} \\
        \emph{...of the applicant population} & \emph{sex} \\
        \emph{...of a target population} & \emph{gender identity} \\
        \underline{Contextualising applications} & \emph{sexual orientation} \\
         & \underline{Geography} \\
         & \emph{nationality} \\
         & \emph{the `Global South'} \\
         & \emph{region} \\
         & \underline{Race} \\
         & \emph{international categorisations of race} \\
         & \underline{Types of thinking} \\
         & \emph{subject area interest} \\
         & \emph{personality type} \\
         & \emph{core beliefs} \\
         & \emph{problem solving approaches} \\
         & \emph{political views} \\
        \hline
        \textbf{Operational Risks and Considerations} & \textbf{Fairness and Bias} \\
        \underline{Outreach} & \underline{Fairness} \\
        \underline{Support} & \emph{...to the applicants} \\
        \emph{...during the application process} & \emph{...to the world} \\
        \emph{...after selection} & \underline{Bias} \\
        \underline{Selectors} & \emph{measurement bias} \\
        \underline{Applicant fraud} & \emph{decision-makers' bias (prejudicial)} \\
         & \emph{decision-makers' unique perspective (probative)} \\
        \hline
        \textbf{Scholarship Goals} & \textbf{Merit} \\
        \underline{Impact} & \underline{Performance relative to disadvantage} \\
        \emph{...on all applicants} & \underline{Measurement} \\
        \emph{...on the selected scholars' performance} & \underline{Performance} \\
        \emph{...on the selected scholars' opportunities} & \\
        \emph{...by the selected scholars on the world} & \\
        \hline
    \end{tabular}
\end{table}

\subsection{Themes}\label{ssec:themes}
\subsubsection{Why Diversity?}
A central theme of our investigation revolved around the question of why diversity matters. To this end, we asked questions such as: ``What is diversity?'' and ``Why does diversity matter?''. Answers to: ``What is diversity?'' are visualised in Figure \ref{fig:diversity_is_teaser}; as hypothesised, these answers are vague and uninformative. Interestingly, though, answers to: ``Why does diversity matter?'' informed more specific definitions. We were able to cluster these more specific definitions into three central subthemes: `different perspectives', `representativeness', and `contextualising applications'. These are listed as subthemes of `Why Diversity' in Table \ref{tab:themes}.

\paragraph{Different Perspectives}
8 out of 15 participants mentioned that diversity was important because it brought different perspectives into the same room. This was seen as important for a few related reasons, e.g., the ability to see problems from different angles and the ability to make better decisions. Several participants referred to the: ``Benefits of diverse perspectives'' (P1). One said, when discussing their personal experience working with winners in a talent investment programme, that there is: ``Magic happening with lots of...diverse perspectives in the room'' (P14). People's experiences were particularly relevant here. As one participant writes: ``You want to have diverse perspectives from people who look different with different experiences'' (P2).

\paragraph{Representativeness}
9 out of 15 participants spoke of `representativeness'. This, we observed, was often spoken of in relationship to a larger population. Most frequently, participants spoke of the importance of having a cohort that was representative of the `eligible population'. I.e., one participant said: ``[You want] a community which is representative of where you are selecting young people from'' (P6). Others spoke of this in broader, more general terms: ``[You want] as broad a range of people as possible'' (P15). Participants identified the importance of building a cohort that variously: ``Reflects the population of the countries'' (P15) and is ``More representative of the national population than the STEM field already is'' (P10). Others talk about the representation of a particular target population, i.e.: ``Representation...because that gives you insight for the people that you're trying to serve....you have to be....reflective of your market'' (P9). Finally, selectors discussed the importance of representing an applicant population: ``[We want] a cohort that is representative of the pool'' (P10).

\paragraph{Contextualising Applications}
`Contextualising applications' was often spoken of by 6 out of 15 participants, most often in individual terms, speaking of identifying particular applicants in need of support, and then offering them a `boost' in the form of said support. I.e.: ``Identify those talents and specifically boost up people who are in need of support'' (P5). One participant identified `boosting' as a key metric for the programme: ``We need to know that...we have some level of...impact here, and...if all we're doing is supporting someone who is already on an amazing trajectory and then maybe that means we're not altering their trajectory at all. That's a question of efficiency of our dollars'' (P8).

In some cases, the need for support or boosting was identified with underrepresented or disadvantaged demographic groups. One participant said: ``The focus on gender has been to give the sex that has had the least opportunity the opportunity in this programme'' (P9).

\subsubsection{Types of Diversity}
Another central focus of our investigation was on different types of diversity. We asked participants to break down their understanding of diversity into different elements and to discuss why these elements were important. We found that participants identified a wide range of different types of diversity, which we clustered into subthemes. These subthemes are listed as subthemes of `Types of Diversity' in Table \ref{tab:themes}.

Notably, in addition to the standard demographic categories commonly considered `demographic diversity', participants identified a wide range of other types of diversity commonly termed `cognitive diversity' \cite{page_diversity_2010}. These included `subject areas of interest', `personality type', `core beliefs', `problem-solving approaches', and `political views'.

\paragraph{Socioeconomic}
All 15 participants identified socioeconomic diversity as particularly important in the context of a talent investment programme. One participant said: ``Socioeconomic [diversity] is probably the most important'' (P1). Another said: ``Socioeconomic background is number one from my perspective'' (P5). This was identified as particularly important for several reasons. Participants stated: ``Because right now the SAT, for example, is more highly correlated to socioeconomic status than it is to anything else'' (P5), and: ``It's a scholarship scheme, so I think it should be for kids who cannot afford normally the fees at the university'' (P7).

Outside of the standard categorisations by income and wealth, participants also identified: ``Familial education level'' (P5) or ``Socioeconomic backgrounds'' (P10) as a particularly important metric for understanding socioeconomic diversity in a scholarship context. This suggests that historical socioeconomic status is considered alongside present socioeconomic status. One participant noted that socioeconomic status varied in both meaning and measurement from country to country: ``For example, in Columbia, there's a whole society to organise on a 1 to 7 scale for socioeconomic status'' (P5).\footnote{Columbia's policy of socioeconomic stratification divides households into 6 strata; unlike traditional measurements such as income or quality of life, these strata only consider household location and accommodation and thus capture a different facet of socioeconomic status \cite{CHICAOLMO2020102560}.}

\paragraph{Sex, gender, and sexuality}
While all 15 participants noted some manner of sex, gender, and sexuality diversity as important, participants disagreed on the relative emphasis that should be placed on each. One participant noted: ``[Sex] is important. I think it will get diluted if we focus on identity gender because....the purpose of diversity on the gender aspect was to make sure that [men and women] were getting equal opportunities'' (P9). Another noted that, while sexual identification diversity was important in other contexts, they ``Wouldn't select for that'' (P1) in this context. Others listed `sexual orientation' and `gender' as important metrics for understanding diversity in a scholarship context. However, save for the participant who noted the distinction between sex and identity gender, participants expressed reluctance to discuss the relationships between these difficult concepts.

\paragraph{Geography}
14 of the 15 participants mentioned the importance of geography, citing a need for: ``[A] wide array of different geographical...representations'' (P8). Others spoke of a ``Regional distribution'' (P1), which we have included here.

In particular, emphasis was placed on geographic markers of socioeconomic status such as the `Global South', `indigenous communities', and `low-income countries'. One participant noted: ``Immigration status is tied so closely to socioeconomic status'' (P2), while another noted that ``[Geography] is connected to socioeconomics because we know there are some poorer countries and rich countries'' (P7). Others still asked questions like: ``Do they have a passport?.... Are they in a refugee camp?'' (P5).

Furthermore, participants saw it as important that their programmes had: ``Global reach'' (P7). They expressed a desire for: ``[A] diversity of people coming from a variety of places'' (P7).

\paragraph{Race}
While 11 out of 15 participants identified race as an important dimension of diversity, none suggested they would explicitly select racial diversity. Several participants instead noted the difficulty of measuring race in a global context: ``Racial categories obviously vary by country'' (P10). One participant noted: ``[In places] like Brazil or England, there are different categories of race than there are in the US...[In Brazil] there's a board of people who decide what people's race are'' (P5). In a global context, however, many participants pointed to relationships between geography and race and hence suggested diversifying across geography in place of race: ``If it's an international programme then you can use geography as a proxy'' (P5).

\paragraph{Types of Thinking}
4 out of 15 participants discussed a diversity in the types of thinking exhibited by applicants. One participant noted that: ``You want as much representation from different types of thinking as you can, because I want perspectives to be listened to equally'' (P2). This manifested in many ways.

Participants tended to express the belief that personality type diversity could improve group cohesion: ``With that understanding [of] personality types...be able to tell which...people would get on well with each other'' (P14). One participant suggested a ``Personality test'' (P12), and another specifically mentioned a desire to diversify across ``Openness'' (P2).

However, while personality type was seen as important, core beliefs were seen as even more so. One participant noted an interest in the diversity of ``Interests politically'' (P12), and expressed a desire for diversity of ``People's core beliefs...separate from religion'' (P12). Another also noted: ``I would try to have a good representation of...religious groups'' (P3).

\subsubsection{Operational Risks and Considerations}
While our study did not focus on the operational aspects of selection, several selectors' understanding of diversity was closely tied to the operational realities of selecting for and running a scholarship. In answering our questions, several participants identified operational risks or considerations that impacted their understanding of diversity. These are listed as subthemes of `Operational Risks and Considerations' in Table \ref{tab:themes}.

\paragraph{Outreach}
While our study was focused primarily on selecting a diverse cohort from a fixed applicant pool, 6 out of 15 participants answered questions from the perspective of outreach to grow a more diverse pool of applicants to select from. In particular, participants suggested that ``Using technology for....targeted outreach'' (P4) could help improve overall cohort diversity before selection even begins. One said: ``Giving you very clear signposting on where you may want to focus, you know further recruitment or outreach or whatever it might be to make sure that your programme is diverse at the end of the day'' (P6). Another added: ``You can target your outreach dollars to communities where you know that underrepresented talent exists'' (P10).

\paragraph{Support}
Similarly, 8 out of 15 participants suggested that technology could enable the support of applicants from underrepresented groups, which would also improve diversity. One participant suggested that technology could be used to provide: ``[Support] to keep people that you're attracting from underrepresented backgrounds and help them get across the finish line'' (P10). Participants focused on the: ``Support needed to actually get [applicants] through your programme'' (P10), i.e., supporting applicants after acceptance. Another suggested that technology could be used to provide support to applicants ``After selection'' (P15).

\paragraph{Selectors}
4 out of 15 participants noted that diversity did not apply merely to applicants. Instead, for programmes where a group of selectors assists in the selection process, ``Tracking the diversity of the selectors'' (P15) and ``[Monitoring] how they're scoring and reviewing applicants [for] prejudice or biases'' (P15).

\paragraph{Applicant fraud}
Finally, only 2 out of 15 participants expressed concern with selecting based on particular diversity characteristics, especially self-reported metrics of diversity characteristics, was the potential for applicant fraud, e.g. falsely reporting demographic or other attributes to increase their chances of acceptance. One participant requested: ``A fraud detector'' (P5), while another expressed a desire to ensure that the process ``Isn't super gameable'' (P10).

\subsubsection{Fairness and Bias}
Though not a type of diversity as we have understood it here, many participants referenced similarities between diversity metrics and metrics of fairness or bias (as in \textcite{zhao2023fairness}). Furthermore, several suggested that improving fairness while reducing bias would likely yield a more diverse cohort. These themes are reflected under `Fairness and Bias' in Table \ref{tab:themes}.

\paragraph{Fairness}
6 out of 15 participants discussed that it was important that applicants: ``Get fair chance on their on their academic merit'' (P7). This translated to an emphasis on ``Fairness in the assessment'' (P7).

However, participants also noted that ``The way the world works is unfair'' (P14), and found it important that the programme is: ``Making sure that the world is fairer by bringing more diversity to this world'' (P7). In this way, participants found: ``[The] representative thing....goes back to fairness'' (P15). One participant noted that: ``Affirmative action....can come across as unfair to some people, but....it's trying to balance things out when things have been so unequal for so long'' (P3). This supports \textcite{zhao2023fairness}'s positioning of fairness as related to, rather than in opposition to, diversity. 

\paragraph{Bias}
11 out of 15 participants discussed bias, often as both a human- and machine-decision-making problem. Many participants appealed to technology's ability to be comparatively impartial as an important mitigator of bias, i.e., one participant repeatedly requested a: ``Non-biased programme'' (P3); another stated a preference for: ``Data analysis to make decisions on who we should be supporting as opposed to having humans try to make those decisions with all their biases'' (P5). However, those same participants noted that common machine decision-making paradigms amplify bias and that it was important to be aware of this: ``AI has a lot of bias in it'' (P3).

Others noted the possibility for technology to elucidate biases in both humans and machines. One participant requested: ``Some kind of tool that can detect bias in a selection'' (P7).

\subsubsection{Programme Goals}
Selectors from both programmes identified goals for their programmes. These goals were discussed by both groups as an intended form of `impact' and both groups closely related achieving their goals to their expressions of why diversity mattered. While impact goals varied based on the type of impact and the affected party, we discuss these as `impact' in Table \ref{tab:themes}.

\paragraph{Impact}
5 out of 15 participants found key goals of their programme to include: ``Orient[ing] [scholars] towards social impact or using their talent for good'' (P8). They tended to encourage: ``[Scholars'] working towards something impactful throughout their career'' (P8).

\subsubsection{Merit}
Several participants reflected on the relationship between merit and diversity. While some participants saw these as competing goals, others saw them as complementary. In particular, complementary views often viewed merit as a form of performance relative to specific advantages or noted that many of our measurement tools are biased across our chosen diversity dimensions. These themes are reflected as subthemes of `Merit' in Table \ref{tab:themes}.

\paragraph{Performance relative to disadvantage}
2 out of 15 participants identified merit as something difficult to disentangle from performance. One participant noted that applicants may appear less qualified because they: ``Didn't have the chance; didn't have the opportunities'' (P2), while others with the opportunities will appear more qualified. Another participant began by asking: ``How good are their three A stars based on where they've come from?'' (P12), then proceeded to reflect that ``Your performance relative to your opportunity or maybe expected performance'' (P12) is a key indicator of merit.

\paragraph{Measurement}
Closely related, 3 out of 15 participants questioned our ability to measure merit independent of opportunity: ``[Whether they perform well] because they have the opportunity or because they are brilliant – I think that these two are really difficult to untangle'' (P7). Another noted that: ``Contextual factors mess up our otherwise seemingly objective measures of merit.... national context and family income is messing up your ability to measure the thing you actually care about'' (P10). They continued to note that they: ``Need to pay attention to [diversity] because it's messing up your measures of what you actually care about'' (P10).

\paragraph{Performance}
Finally, 2 out of 15 participants noted occasions where performance and diversity were ostensibly competing goals. However, even here, participants recognised that observed performance and actual merit may differ. One participant noted that: ``[The] overriding aim is for [the programme] to be as diverse as is possible but still meet a standard....relative score of like how good their application is based on all these kind of contextual factors'' (P12). Another requested a technology that helps discover how: ``Close you are to your idealised diversity targets and how close you are to maximising whatever it is you think you're maximising in your performance scores'' (P10).

\section[Study 2]{Study 2: Participatory Design}\label{sec:study2}
\subsection{Prototypes}
As a next step, we applied the results of our thematic analysis to design six prototypes following methodology from \textcite{Buchenau_Suri_2000}. These technologies aimed to help selectors better understand and operationalise diversity in their selection processes. We then present these prototypes to participants in participatory design workshops. These prototypes are shown in Figure \ref{fig:prototypes}.

Three of these prototypes (Figures \ref{fig:representativeness}, \ref{fig:entropy}, and \ref{fig:diversity}) present information about the range of possible cohorts participants must choose between. In addition to the themes uncovered in Section \ref{sec:study1}, these prototypes draw from the economic theory presented in Chapter \ref{ch:spf}. Meanwhile, the other three (Figures \ref{fig:demographic}, \ref{fig:impact}, and \ref{fig:advantage}) present information about an individual applicant relative to a given cohort and a given pool. Furthermore, five of the six prototypes were designed to satisfy definitions of diversity uncovered in Section \ref{sec:study1}.

\begin{figure}[htbp]
    \centering
    \begin{subfigure}[b]{0.3\textwidth}
        \includegraphics[width=\textwidth]{diversity/representativeness.png}
        \caption{Prototype \ref{fig:representativeness}: Cohort Representativeness}
        \label{fig:representativeness}
    \end{subfigure}
    \hfill
    \begin{subfigure}[b]{0.3\textwidth}
        \includegraphics[width=\textwidth]{diversity/entropy.png}
        \caption{Prototype \ref{fig:entropy}: Cohort Entropy}
        \label{fig:entropy}
    \end{subfigure}
    \hfill
    \begin{subfigure}[b]{0.3\textwidth}
        \includegraphics[width=\textwidth]{diversity/diversity.png}
        \caption{Prototype \ref{fig:diversity}: Cohort Diversity}
        \label{fig:diversity}
    \end{subfigure}

    \medskip

    \begin{subfigure}[b]{0.3\textwidth}
        \includegraphics[width=\textwidth]{diversity/demographic.png}
        \caption{Prototype \ref{fig:demographic}: Applicant Demographic Information}
        \label{fig:demographic}
    \end{subfigure}
    \hfill
    \begin{subfigure}[b]{0.3\textwidth}
        \includegraphics[width=\textwidth]{diversity/impact.png}
        \caption{Prototype \ref{fig:impact}: Applicant Demographic Impact on Cohort}
        \label{fig:impact}
    \end{subfigure}
    \hfill
    \begin{subfigure}[b]{0.3\textwidth}
        \includegraphics[width=\textwidth]{diversity/advantage.png}
        \caption{Prototype \ref{fig:advantage}: Applicant Advantage Scores}
        \label{fig:advantage}
    \end{subfigure}
    \caption{These figures depict the prototypes designed based on themes from Section \ref{sec:study1} and used in our participatory design workshops. They are reproduced at a larger scale in Appendix \ref{app:divfigures}}
    \label{fig:prototypes}
\end{figure}

Figures \ref{fig:demographic} and \ref{fig:impact} are based on the `representativeness' theme, while Figure \ref{fig:advantage} is based on the `contextualising applications' theme. Similarly, Figures \ref{fig:representativeness} and \ref{fig:entropy} draw a distinction in their measurements of diversity, based on the `representativeness' and the `different perspectives' themes, respectively. Figure \ref{fig:entropy}, in particular, defines and employs `entropy' as a metric. This reflects that when aiming to get different perspectives in the same room, the goal is not to represent any target population; rather, we desire that everyone in our group be as different from the remainder of the group as possible.

\subsection{Workshop Methodology}\label{ssec:methods2}
As the participants come from two separate talent investment programmes, we run one workshop for each group \cite{Buchenau_Suri_2000}. Before presenting to the broad audience of each group, we submit our figures to one primary contact (also a participant in the study) at each organisation, then run informal, 15-minute `pilot' one-on-one workshops with this primary contact. Here, we primarily sought approval to use these figures in a workshop with the broader team of selectors, but we also collected minor feedback and tweaked the prototypes based on this feedback. In one organisation's pilot workshop, the primary contact requested Figures \ref{fig:representativeness} and \ref{fig:entropy} be combined into one prototype with `Diversity' as the Y-axis, as the organisation already has an internal working definition of diversity (that incorporates what we mean by both representativeness and contextualising applications). This can be seen in Figure \ref{fig:diversity}. Participant grouping for the workshops is redacted to preserve participant anonymity; the results for this analysis are attributed by group, rather than individual, to reflect the cooperative nature of the task.

Our central research questions for this workshop are:

\begin{enumerate}
    \item What prototypes best promote diversity?
    \item What elements of these prototypes facilitate their success?
\end{enumerate}

\noindent Or, for each prototype: ``How and why does this prototype promote diversity?''. In each workshop, we ask participants to consider each prototype in turn and to discuss how they might use it in their selection process. We then ask participants to consider how these prototypes might fit into their current selection process, and how they might change their process to better incorporate these prototypes. Finally, we ask participants to consider how their current selection process might make the best use of these prototypes, and whether they think these prototypes would be beneficial. 

Following the methodology of \textcite{Gatian_1994,Griffiths_Johnson_Hartley_2007}, at the end of each workshop, we ask participants to highlight their favourite prototype.

A question-by-question protocol for these workshops can be found in Appendix \ref{app:divprotocol2}.

\subsection{Results}\label{ssec:results2}
\subsubsection{Participants Preferred Prototype \ref{fig:impact}}
As part of the workshop, participants were asked to mark their favourite prototypes \cite{Gatian_1994,Griffiths_Johnson_Hartley_2007}. These favourites have been collated in Table \ref{tab:favourites}, and it can be seen here that Prototype \ref{fig:impact} was by far the favourite in both groups.

\begin{table}[htbp]
    \centering
    \caption{This table tallies the number of participants who indicated that a given prototype was their favourite. The overwhelming favourite was prototype \ref{fig:impact}, which shows an individual applicant's impact on the cohort.}
    \label{tab:favourites}
    \begin{tabular}{lr}
        \toprule
        \textbf{Prototype} & \textbf{Favourites} \\
        \midrule
        Prototype \ref{fig:representativeness} & 1 \\
        Prototype \ref{fig:entropy} & 1 \\
        Prototype \ref{fig:diversity} & 0 \\
        Prototype \ref{fig:demographic} & 1 \\
        Prototype \ref{fig:impact} & 10 \\
        Prototype \ref{fig:advantage} & 0 \\
        \bottomrule
    \end{tabular}
\end{table}

\subsubsection{Both Groups Rely on Idiosyncratic Notions in Selection}
When participants were placed in a practical selection scenario and shown technology prototype information about (hypothetical) applicants, they compared applicants to the idiosyncratic profiles that they desired.

When evaluating Prototype \ref{fig:advantage}, G1 used advantage scores to seek out: ``Diamonds in the rough'' (G1), i.e., talented applicants from disadvantaged backgrounds who lacked the polish of their more privileged counterparts.

Prototype \ref{fig:entropy} was initially confusing to G2, as the `entropy' definition used was unfamiliar to the participants: ``Entropy is chaos in chemistry. How does this relate to our usage here?'' (G2). Thus, when interacting with Prototype \ref{fig:entropy}, G2 understood `entropy' to be a variety in cognitive skill and personality type. In particular, the group sought to ensure that the chosen cohort contained `glue' people, who improve overall cohort cohesion: ``For people working together, it's useful to have someone who is that `glue''' (G2). 

\subsubsection{Organisations and Participants Are Interested in Different Diversities}
When presented with Prototypes \ref{fig:representativeness} and \ref{fig:entropy}, G2 was given the demonstrative definitions of `representativeness' and `entropy' (visible on the prototypes). G2 quickly understood the differences between the figures to be that Prototype \ref{fig:representativeness} sought to support considerations of demographic representativeness, while Prototype \ref{fig:entropy} sought to support placing different perspectives (be they cognitive skill-sets or personality types) in the same room. Participants were then interested to know the relationship between these prototypes in practice: ``If we maximise based on [Prototype \ref{fig:representativeness}] scores, what would the Entropy scores be?'' (G2).

However, though individual participants took great interest in Prototype \ref{fig:representativeness}, G2 ultimately acknowledged that the programme was most interested in building a cohort from people with different perspectives to facilitate collaboration: ``For [our] cohorts, [we] want a balance [of personality types] and [we] want them to be collaborative'' (G2). At the same time: ``Let's track but not use [programme-specific measure of representativeness]'' (G2).

\subsubsection{Different Tools are Useful at Different Application Stages}
Participants in both groups variously expressed anxiety about measuring the relevant dimensions. ``What are our metrics and are they reliable?'' (G2) was echoed by several G2 participants. ``If it's all self-report, then we can't do anything with it'' (G1). 

However, both organisations noted that their selection processes involve a variety of stages. Different tools are useful in different stages. When speaking of Prototype \ref{fig:entropy}, one participant said: ``This is better post-interview than it is pre-interview'' (G2), as interviews will collect observational data on many of these characteristics. I.e., while certain tools may rely on measurements that cannot be collected until later stages, others will be more useful earlier in the selection process.

Similarly, different output modes were useful at different stages. Note that Prototypes \ref{fig:demographic} and \ref{fig:impact} contain similar information, but that Prototype \ref{fig:impact} displays this information in greater detail. When reviewing both prototypes, G1 found Prototype \ref{fig:demographic} preferable in the earlier stages of decision-making, while Prototype \ref{fig:impact} had the greatest utility later. ``[We] can't send [Prototype \ref{fig:impact}] as a pre-read, but [Prototype \ref{fig:demographic}] makes more sense in isolation, so better for a pre-read'' (G1). On Prototype \ref{fig:demographic} in particular, one participant noted: ``Helpful for the process, not so much for the [final cohort selection]'' (G1). Another said of Prototype \ref{fig:impact}: ``This has the most potential at the later stages of decision-making'' (G1).

At the other extreme, while the cohort-level tools did not spark discussions about particular individuals, they did spark earlier-stage, higher-level discussions. Most obviously, both programmes discussed specific tradeoffs between measured individual applicant aptitude and cohort diversity: ``Real decision-making always sees tradeoffs like these.... This chart helps you figure out the level of compromise you're willing to make on both axes'' (G2); ``[It] makes sense that top scoring candidates don't necessarily help you build the most diverse cohort.... 5-10\% [of the cohort] really get to a struggle between quality and diversity.... If this chart were real (rather than hypothetical), and you could see who you were losing, this would be useful'' (G1). (In the hypothetical,  participants from G1 settled closer to the centre of the frontier: ``[Our] target here is to look somewhere [from] red to yellow'' (G1). However, they also noted that ``Some candidates get a big diversity boost and score terribly'' (G1).)

In one case, participants also discussed broader, programme-level concerns. Participants spent time debating ``Is the programme needs-based or merit-based?'' (G1). They noted that ``This chart helps [facilitate that discussion]'' (G1) but did not ultimately conclude one way or the other.

\subsubsection{The Right Balance of Quantitative and Qualitative Information is Key}
Participants simultaneously expressed gratitude that the prototypes were as simplified as they were, and a desire for more detail. One participant from G1 noted that the prototypes are: ``Very constrained in terms of what is being shown.... This doesn't include all of the factors, but for the decision, that's good, because it prevents info overload'' (G1). However, participants G2 frequently requested additional quantitative information. In the case of the Prototype \ref{fig:impact} (participants' favourite prototype, as can be seen in Table \ref{tab:favourites}), participants requested many other metrics: ``Advantage score'' (G2), ``Impact on entropy'' (G2), ``[A summary of] these scores together'' (G2), and ``A composite impact on cohort diversity'' (G2).

Meanwhile, participants from G1 requested qualitative information: ``We need to know more about the applicants' backgrounds'' (G1). Other requested information included: ``Comments from selectors'' (G1) and ``A narrative summary...written by the selector team'' (G1). 

This suggests a discrepancy between the two groups' preference for the balance between qualitative and quantitative information. While G2 expressed a desire for unifying quantitative metrics: ``A single score for disadvantage [or] need, while recognising its flaws, could provide one read of an individual's circumstances'' (G2), G1 expressed a desire for individual, qualitative information: ``We're using quantitative to sift through qualitative.... [We] need to include comments from selectors'' (G1).


\section{Design Recommendations}
\begin{figure}[htbp]
    \centering
    \includegraphics[width=0.9\linewidth]{diversity/recommendations.png}
    \caption{This figure illustrates our four key design recommendations to others building tools to support the selection of diverse talent.}
    \label{fig:recommendations}
\end{figure}

\subsection{Design for a Specific Diversity}
In the initial interviews, participants were often vague when asked to define diversity. However, when asked to expand on why diversity is important, or on what dimensions of diversity they prioritised, it became clear that `diversity' included three separate (and sometimes competing) definitions. We have termed these: `representativeness', `different perspectives', and `contextualising applications'. When designing tools to assist a target organisation in considering diversity, we suggest designers first clarify through human-centric methods which definitions of diversity the target organisation seeks to consider, then designs to support those specific definitions. That is, designers should follow \textcite{VanKleek_Seymour_Binns_Shadbolt_2018}'s paradigm of `respectful' design and build technology to best serve the needs of the selectors who will use it.

\subsection{Design for Idiosyncrasy}
One key note revealed through this process is that different decision-making processes have philosophical underpinnings, desiderata, and anecdotal definitions that impact their selections. These idiosyncrasies should be discovered early in development and designed for in any technical solution. We suggest participatory design as a mechanism for achieving this. We observed a strong relationship between participant feedback in interviews and their satisfaction with the prototypes.

For example, both talent investment programmes we worked with have created specific personas they look for. For example, one group discussed `glue' people who helped groups function cohesively. The other group discussed `diamonds in the rough', talented youth systemically undervalued due to their backgrounds. Where these aspects were included, participants showed strong interest in the prototypes. Where they were excluded, participants often asked for these to be added.

\subsection{Design in Stages}
Much of what participants desired at one stage of decision-making was mutually exclusive with what they desired at other stages. For example, G1 desired Prototype \ref{fig:demographic} as a: ``Pre-read'' (G1) due to its simplicity but preferred the detail of Prototype \ref{fig:impact} ``In the room'' (G1). Thus, it is crucial for designers to consider what stage of decision-making their tools are designed to support and to design appropriate levels of detail, abstraction, and engagement accordingly.

\subsection{Design to Balance Qualitative and Quantitative}
Participants often noted that the prototypes were missing key qualitative information about applicants. This qualitative information is crucial in the holistic considerations of each applicant. However, when allowed to consider only qualitative information, participants obscure tradeoffs they are forced to make between different programme goals. In particular, while individual-level goals are often clear, cohort-level goals (such as diversity) are easier to delay or ignore. Thus, without quantitative tools to frame the discussion, participants noted that they were often forced to make cohort-level considerations ad-hoc and towards the end of their decision-making process.

Thus, while qualitative information is crucial, it is also important to present the quantitative information necessary to make these tradeoffs salient. Ultimately, final selection decisions are made by panels of trained selectors, but in the absence of both quantitative and qualitative information to guide these decision-makers, they would be forced to make decisions that are less well-informed than they could be.

\section{Discussion}\label{sec:divdisc}
\subsection{The Diversity Triangle}
In Section \ref{sec:study1}, participants variously identified the word `diversity' with three themes, which we have taken to signify definitions of diversity: `representativeness', `different perspectives', and `contextualising applications'. As these three definitions are central to our research questions, we privileged them over Section \ref{sec:study1}'s other themes in Section \ref{sec:study2}, where we engaged in scenario speed dating and experience prototyping designed to satisfy different definitions of diversity. In this section, we map the themes to the three definitions in the Diversity Triangle (Figure \ref{fig:div_triangle}).

\begin{figure}[htbp]
    \centering
    \includegraphics[width=\textwidth]{diversity/diversity_triangle.png}
    \caption{This figure depicts the Diversity Triangle, three differing definitions of diversity selectors expressed when discussing a diverse cohort. We also relate the Diversity Triangle to each other theme or subtheme participants mentioned.}
    \label{fig:div_triangle}
\end{figure}

\subsection{The Relationship Between the Diversity Triangle and Theories of Change}
\subsubsection{Representativeness}
Each of the definitions from the Diversity Triangle (Figure \ref{fig:div_triangle}) implies different programme values and a different theory of change. The representativeness
definition relies on social theories by which diversity is intrinsically valuable. Several participants seemed to believe representativeness here was intrinsically valuable, which aligns with the \textcite{morris1984origins} account under which contemporary diversity norms emerged from loci of oppression affecting underrepresented groups. However, some participants also noted instrumental reasons to value representativeness that align closely with \textcite{peters2021hidden,page2008difference}'s argument that people from different backgrounds often possess unique and germane knowledge. One participant noted that a team can: ``Better serve a community if they represent [that community]'' (P9). Another theory from \textcite{Friedler_Scheidegger_Venkatasubramanian_2016} discusses measurement bias. Notably, if we assume that talent is equally distributed across some partition, then the most talented cohort should also be representative. However, \textcite{Friedler_Scheidegger_Venkatasubramanian_2016} note that we often observe in practice that performance is not equally distributed across these partitions. This is likely due to both relevant differences between those groups and structural bias that causes differences in construct observability between groups (in fact, the relevance of said differences may also be due to structural biases) \cite{Friedler_Scheidegger_Venkatasubramanian_2016}. In this case, representativeness would be a proxy for the distributive notion of fairness \cite{Olsaretti_2018}. Thus, representativeness is pro-social: society is better served when resources are distributed to a representative group of people.

\subsubsection{Different Perspectives}
The argument for placing different perspectives in the same room is often instrumental. While `diversity' on the whole is often spoken of as an intrinsically valuable broader benefit, the aim of placing different people in the same room is often only to benefit the people in that room. \textcite{page_diversity_2010} argues that `cognitively' diverse groups outperform homogeneous groups on some tasks; some participants similarly contend that it improves cohort-level task performance. Other participants echo \textcite{wylie2006introduction}'s argument that it allows participants to better learn from each other. In either case, the benefit is primarily organisational rather than social.

\subsubsection{Contextualising Applications}
The argument for contextualising applications is twofold. Most often, participants make a systemic critique here. That is, the world is incredibly unjust, and we want to distribute resources differently, but in a talent selection process, we still have to operate in an unequal world. Thus, to correct that injustice, we must give more resources to those who have less. This could be seen as a form of distributive justice or `affirmative action' \cite{Olsaretti_2018}. Participants argue (perhaps relatedly) that appropriately contextualising applications results in more successful applications from disadvantaged groups; this, in turn, allows these applicants to have a positive impact on their groups. Relatedly, either due to measurement bias, or due to differences in performance brought on by disparate access to resources, support, and opportunities, we may find that applicants from disadvantaged backgrounds appear worse on paper. In either case, correcting this through contextualisation may also build a more fair selection process and a more just world, yielding great benefit to society. On the organisational side, if contextualising applicants allows for the admission of more applicants from marginalised groups, they may be better suited to critiques of existing power structures, as they may have more informative experiences of oppression \cite{mills2015blackness}.

\subsection{A Cautionary Note On Designing for Diversity}
Decisions such as scholarship selection have long-reaching impacts on the applicants. A successful applicant to a scholarship programme might thus attend a university they could not have otherwise attended; there, they will acquire skills and a network that will continue to impact them later in life; these impacts may even affect those close to them, as their better circumstances likely better the circumstances of their communities \cite{Dassin_Marsh_Mawer_2018}. Thus, it is simultaneously important to ensure that these opportunities are dispersed fairly and to ensure that all demographics are included among those selected.

This increases the importance of selecting a diverse group of people and mandates that we do all we can to do so. Quantitative decision support tools may have a role to play in this, but two problems remain: data collection and data processing. Primarily, diversity data is often self-reported. Thus, we cannot use this naively to generate diverse cohorts, as doing so would yield a competitive advantage to candidates who lie on their declarations. Secondly, processing data automatically has its drawbacks. AI systems are prone to bias \cite{Friedler_Scheidegger_Venkatasubramanian_2016}. And as these systems improve, it is unclear if they will help or harm our ability to select diverse cohorts.

However, it is important to note that both the problems of data collection and processing are not unique to this workflow. Algorithmically-supported diversity considerations may correct for data collection and processing issues elsewhere. Bias is a major consideration in data collection \cite{Friedler_Scheidegger_Venkatasubramanian_2016}; heterogeneous biases in measurements of talent may be corrected by ensuring diversity across these metrics: ``Assessments are usually biased measures of what we care about, and that opportunity often correlates with positive error terms in assessments' measure of underlying skills.... [Diversity considerations] correct for the inadvertent affirmative action against underprivileged individuals implicit in using biased assessments'' (P1). Similarly, much like AI systems, human selectors are prone to their own heterogeneous biases; in some sense, `design for idiosyncrasies' reflects the notion that decision support systems must at times help recognise and mitigate these biases.

\section{Limitations and Future Work}
In Studies 1 and 2 we build tools with which participants report satisfaction, but increasing participant satisfaction does not necessarily improve decision-making. In particular, technology that makes difficult decisions less painful may be well-received, while technology that makes these decisions more salient may be less popular but more impactful \cite{Lipton,miller_explainable_2023}. Our themes speak to participant perceptions of diversity and our design recommendations speak to participant desiderata from support tools. We assume that, in solving for these considerations, we can help organisations select better cohorts. However, we may find that these tools fail to improve decision-making concerning diversity in practice. Future work should investigate this possibility through the implementation of our prototypes in field settings.

\textcite{venn-wycherley_realities_2024} contend that HCI literature focused on educational contexts should consider both educator and student. Two distinctions distance our work from theirs: first, selection is distinct from pedagogy in that our decision subjects are not necessarily beneficiaries (while students are beneficiaries of their institution, applicants do not become beneficiaries unless they are selected); second, scholarships are distinct from educational institutions in that scholarship programme benefits primarily focus on assisting beneficiaries in accessing educational institutions, while educational institutions primarily educate beneficiaries. Both distinctions create distance between selectors and applicants beyond that between teachers and students and pose challenges in engaging decision subjects. Nonetheless, future work should engage scholarship applicants to understand their definitions, stances, and considerations concerning diversity. 

While not a limitation, we have intentionally set aside themes such as `outreach' and `selectors' that relate more to other aspects of selection processes than to the act of selection itself. We hope future work will consider these facets of selection programmes, especially when designing tools to support thinking around diversity.

\section{Conclusion}
This research answers the crucial question: ``What is diversity?'', from the perspective of scholarship and talent investment selectors. In doing so, we illuminate the multifaceted nature of diversity in scholarship selection and emphasise the critical need for tools that support considerations of diversity in decision-making. Our findings reveal that achieving true diversity involves navigating the complex interplay between three occasionally conflicting definitions: representativeness, diverse perspectives, and the contextualisation of applicants' backgrounds. By engaging in participatory design, we build six prototypes with our selectors; this process reveals four design recommendations: design for a specific diversity, design for idiosyncrasy, design in stages, and design to balance quantitative and qualitative. This work demonstrates that, when thoughtfully designed, technology can empower selection processes to be more equitable, inclusive, and transparent. The broader implication is that such advancements have the potential to reshape how diversity is operationalised, ensuring that it is both a measurable outcome and a core value in shaping the future of talent identification. As we move forward, the integration of these tools into real-world practices will be pivotal in fostering truly diverse and representative groups in global scholarship programmes and beyond. However, we caution that programmes seeking to implement these tools should ensure that they improve not only subjective selector perceptions of decisions but also the objective quality of those decisions themselves.