% \begin{savequote}[8cm]
% \textlatin{Cor animalium, fundamentum e\longs t vitæ, princeps omnium, Microco\longs mi Sol, a quo omnis vegetatio dependet, vigor omnis \& robur emanat.}

% The heart of animals is the foundation of their life, the sovereign of everything within them, the sun of their microcosm, that upon which all growth depends, from which all power proceeds.
%   \qauthor{--- William Harvey \cite{harvey_exercitatio_1628}}
% \end{savequote}

\chapter{\label{app:math}Mathematics and Computation}

\minitoc

\section{ChatGPT Code Generation for Chapter \ref{ch:genai}}\label{app:prompt}

\section{Proofs of Submodularity and Monotonicity for Chapter \ref{ch:spf}}\label{app:spfmath}

\begin{theorem} 
    Submodularity is closed under weighted addition. \label{thm:submodularity_additive}
    \end{theorem}
    
    \begin{proof}
    \begin{equation}
        \label{eq:submodularity_additive}
        \begin{split}
            \forall Y \subseteq U, X \subseteq Y, x \in U \setminus Y, a \geq 0, b \geq 0: & F_1(X \cup \{x\}) - F_1(X) \leq F_1(Y \cup \{x\}) - F_1(Y) \\
            &\land F_2(X \cup \{x\}) - F_2(X) \leq F_2(Y \cup \{x\}) - F_2(Y) \\
            \implies & a*F_1(X \cup \{x\}) - a*F_1(X) \leq a*F_1(Y \cup \{x\}) - a*F_1(Y) \\
            &\land b*F_2(X \cup \{x\}) - b*F_2(X) \leq b*F_2(Y \cup \{x\}) - b*F_2(Y) \\
            \implies & a*F_1(X \cup \{x\}) - a*F_1(X) + b*F_2(X \cup \{x\}) - b*F_2(X) \\
            &\leq a*F_1(Y \cup \{x\}) - a*F_1(Y) + b*F_2(Y \cup \{x\}) - b*F_2(Y) \\
            \implies & a*F_1+b*F_2(X \cup \{x\}) - a*F_1+b*F_2(X) \\
            &\leq a*F_1+b*F_2(Y \cup \{x\}) - a*F_1+b*F_2(Y) \\ \nonumber
        \end{split}
    \end{equation}
    \end{proof}
    
    \begin{theorem} 
    Monotonicity is closed under weighted addition. \label{thm:monotonicity_additive}
    \end{theorem}
    \begin{proof}
    \begin{equation}
        \label{eq:mononicity_additive}
        \begin{split}
            \forall Y \subseteq U, X \subseteq Y, a \geq 0, b \geq 0: & F_1(X)\leq F_1(Y) \land F_2(X)\leq F_2(Y) \\
            \implies & a*F_1(X)\leq a*F_1(Y) \land b*F_2(X)\leq b*F_2(Y) \\
            \implies & a*F_1(X) + b*F_2(X) \leq a*F_1(Y) + b*F_2(Y) \\
            \implies & a*F_1+b*F_2(X) \leq a*F_1+b*F_2(Y) \\ \nonumber
        \end{split}
    \end{equation}
    \end{proof}
    
    \begin{theorem} 
    The class of functions $F$ as defined in Equation \ref{eq:f_spec} is submodular and monotone.\label{thm:f_sub_mon}
    \end{theorem}
    \begin{proof}
    By construction, $F$ is the weighted sum of $P$ and $D$. But $D$ is the weighted sum of functions $\delta_g^{prop}$ and $\delta_G^{count}$.
    It is trivial to see that $P$ is submodular and monotone. We have already demonstrated that $\delta_g^{prop}$ and $\delta_G^{count}$ are submodular and monotone. Thus, by Theorems \ref{thm:submodularity_additive} and \ref{thm:monotonicity_additive}, $F$ is submodular and additive.
    \end{proof}

\section{Proof that Algorithm \ref{alg:frontier} Approximates the SPF}\label{app:greedy-proof}
Here, we prove that the greedy approximation method introduced in Algorithm \ref{alg:frontier} for SPF is a $(1-\frac{1}{e})$-approximation for our standard class of diversity functions subject to a cardinality constraint \cite{nemhauser1978analysis}.

Recall that an organisation's preference function $f: 2^X -> \mathbb{R}^+$  maps a cohort (a set of applicants) to the weighted sum of a ``diversity score'' and a ``performance score'' (both non-negative, real numbers). Here, our applicant pool $X$ is represented as a set with applicants as its members, and possible cohorts $C \subseteq X$ are subsets of $X$. We prove Appendix \ref{app:spfmath} that $f$ is monotonic ($A \subseteq B -> f(A) \leq f(B)$) and submodular ($A \subseteq B \land x \notin B -> f_A(x) \geq f_B(x)$ (here, $f_S(e) = f(S \cup \{e\}) - f(S)$ denote the marginal gain of adding element $e$ to set $S$). We demonstrate elsewhere that many common understandings of diversity are represented by standard diversity functions.

\begin{theorem}\label{thm:greedy-approximation}
    Let $(S_0...S_k)$ be a sequence of sets where $S_0$ is the empty set and $S_{i>0}$ is defined by following Algorithm \ref{alg:frontier} with any $\iota$, $d$, and $p$. Further, let $O := argmax_S(f(S) : |S| = k)$ be the set of size $k$ that maximizes $f: = \iota*d+(1-\iota)*p$. Then $f(S_k) \geq (1 - \frac{1}{e})f(O)$.
\end{theorem}

\begin{proof}
By induction. Let ${o_1...o_k} = O$ be any ordering of the elements of $O$. Let ${s_i} := S_i - S_{i-1}$ be the element added to $S_{i-1}$ to form $S_i$.

By monotonicity, we have $\forall i . f(O) \leq f(O \cup S_i)$.  We can then write $f(O \cup S_i) = f(O \cup S_i) - f(S_i) + f(S_i) = \sum_{j=1}^{k} (f(S_i \cup {o_1...o_j}) - f(S_i \cup {o_1...o_{j-1}}))$. I.e., $f(O \cup S_i) = f(O \cup S_i) - f(S_i) + f(S_i) = \sum_{j=1}^{k} f_{S_i \cup {o_1...o_{j-1}}}(o_j)$.

By submodularity, $\forall j \in [1...k] . f_{S_i \cup {o_1...o_{j-1}}}(o_j) \leq f_{s_i}(o_j)$. Thus, $f(O) \leq f(S_i) + k*f_{S_i}(s_{i+1})$. 

Since Algorithm \ref{alg:frontier} guarantees that $\forall e \in X - S_i . f_{S_i}(e) \geq f_{S_i}(e)$, it follows that, at every stage, $f(S_{i+1}) - f(S_i) \geq \frac{1}{k} (f(O) - f(S_i))$.

Then induction yields $f(O) - f(S_k) \leq (1 - \frac{1}{k})^k f(O) \leq \frac{1}{e} f(O)$.
\end{proof}