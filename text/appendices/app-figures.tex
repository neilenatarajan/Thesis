\chapter{\label{app:figures}Reference Figures and Tables}

\minitoc

\section{Sample Explanations from Chapter \ref{ch:xai}}\label{app:xaifigures}

\begin{figure}[!htbp]
    \centering
    \includegraphics[width=0.9\linewidth]{xai/survey-shap.png}
    \caption{This figure shows sample SHAP explanations in the \emph{Salary} task. Features can be seen along the y-axis, while feature importance is shown based on the direction and magnitude of the associated bar.}
    \label{fig:shapsalaryfull}
\end{figure}

\begin{figure}[!hbtp]
    \centering
    \includegraphics[width=0.9\linewidth]{xai/survey-shap-2.png}
    \caption{This figure shows sample SHAP explanations in the \emph{Credit} task. Features can be seen along the y-axis, while feature importance is shown based on the direction and magnitude of the associated bar.}
    \label{fig:shapcreditfull}
\end{figure}

\begin{figure}[!hbtp]
    \centering
    \includegraphics[width=0.8\linewidth]{xai/survey-anchor.png}
    \caption{This figure shows sample Anchor explanations in the \emph{Salary} task. The explanation shows a set of rules that, when jointly followed, increases the likelihood that the model will yield the displayed prediction.}
    \label{fig:anchorsalaryfull}
\end{figure}

\begin{figure}[!hbtp]
    \centering
    \includegraphics[width=0.9\linewidth]{xai/survey-anchor-2.png}
    \caption{This figure shows sample Anchor explanations in the \emph{Credit} task. The explanation shows a set of rules that, when jointly followed, increases the likelihood that the model will yield the displayed prediction.}
    \label{fig:anchorcreditfull}
\end{figure}

\begin{figure}[!hbtp]
    \centering
    \includegraphics[width=0.6\linewidth]{xai/survey-confidence.png}
    \caption{This figure shows sample Confidence explanations in the \emph{Salary} task. The explanation is simply one sentence containing the model's confidence parameter.}
    \label{fig:confidencesalaryfull}
\end{figure}

\begin{figure}[!hbtp]
    \centering
    \includegraphics[width=0.6\linewidth]{xai/survey-confidence-2.png}
    \caption{This figure shows sample Confidence explanations in the \emph{Credit} task. The explanation is simply one sentence containing the model's confidence parameter.}
    \label{fig:confidencecreditfull}
\end{figure}

\newpage
\section{Images and Descriptions of Prototypes from Chapter \ref{ch:diversity}}\label{app:divfigures}

\begin{figure}[!hbtp]
    \centering
    \includegraphics[width=0.9\linewidth]{diversity/representativeness.png}
    \caption{This figure reproduces Prototype \ref{fig:representativeness} at a larger scale.}
    \label{fig:representativeness_full}
\end{figure}

\begin{figure}[!hbtp]
    \centering
    \includegraphics[width=0.9\linewidth]{diversity/entropy.png}
    \caption{This figure reproduces Prototype \ref{fig:entropy} at a larger scale.}
    \label{fig:entropy_full}
\end{figure}

\begin{figure}[!hbtp]
    \centering
    \includegraphics[width=0.9\linewidth]{diversity/diversity.png}
    \caption{This figure reproduces Prototype \ref{fig:diversity} at a larger scale.}
    \label{fig:diversity_full}
\end{figure}

\begin{figure}[!hbtp]
    \centering
    \includegraphics[width=0.9\linewidth]{diversity/demographic.png}
    \caption{This figure reproduces Prototype \ref{fig:demographic} at a larger scale.}
    \label{fig:demographic_full}
\end{figure}

\begin{figure}[!hbtp]
    \centering
    \includegraphics[width=0.9\linewidth]{diversity/impact.png}
    \caption{This figure reproduces Prototype \ref{fig:impact} at a larger scale.}
    \label{fig:impact_full}
\end{figure}

\begin{figure}[!hbtp]
    \centering
    \includegraphics[width=0.9\linewidth]{diversity/advantage.png}
    \caption{This figure reproduces Prototype \ref{fig:advantage} at a larger scale.}
    \label{fig:advantage_full}
\end{figure}

\newpage
\section{Figures and Tables for Chapter \ref{ch:spf}}\label{app:spffigures}
\begin{figure}[!hbtp]
    \centering
    \label{fig:spf_2021_full}
    \includegraphics[width=0.9\linewidth]{spf/yr1_spf_finalist.png} 
    \caption{This figure displays the SPF we estimate for the Cycle 2021 finalist selection process. The y-axis represents the diversity score while the x-axis represents the average cohort performance. The green curve is our estimate of the SPF, which represents the upper bound of diversity that is achievable at every level of cohort performance. The red dot depicts the actual level of diversity and performance of the finalists who were selected in Cycle 2021. The vertical and horizontal dashed red lines represent the maximum Pareto gain that was possible along the diversity and performance dimensions respectively. In particular, cohort diversity could have been improved by $15.2\%$ without any reduction in cohort performance. And, cohort performance could have been improved by $15.6\%$ without any cost to diversity.}
\end{figure}
    
\begin{figure}[!hbtp]
    \centering
    \label{fig:spf_2022_full}
    \includegraphics[width=0.9\linewidth]{spf/yr2_spf_finalist.png} 
    \caption{This figure displays the SPF we estimate for the Cycle 2022 finalist selection process. The y-axis represents the diversity score while the x-axis represents the average cohort performance. The green curve is our estimate of the SPF, which represents the upper bound of diversity that is achievable at every level of cohort performance. The red dot depicts the actual level of diversity and performance of the finalists who were selected in Cycle 2022. The vertical and horizontal dashed red lines represent the maximum Pareto gain that was possible along the diversity and performance dimensions respectively. In particular, cohort diversity could have been improved by $13\%$ without any reduction in cohort performance. And, cohort performance could have been improved by $19.6\%$ without any cost to diversity.}
\end{figure}
