\chapter{\label{app:programmes}The Programmes we Study}

\minitoc
\section{Foreword to Appendix \ref{app:programmes}}
We work with two global scholarship and talent investment programmes: the Ellison Scholars Programme and the Rise Programme. Both programmes have asked that they not be identified in public-facing research, and thus we request that reviewers not share details on either programme and to protect the identities of the programmes where possible.

\begin{table}[htbp]
    \centering
    \caption{This table enumerates relevant measurements from the Rise and Ellison programmes.}
    \label{tab:measures}
    \begin{tabular}{l | p{0.3\textwidth} p{0.3\textwidth}}
        \toprule
        Measure Type & Rise & Ellison Scholars \\
        \midrule
        Cognitive Assessment & ICAR-Based Test \cite{condon2014international}; Roomworld \cite{Dumbalska_Bhatti_Ali_Summerfield_2023} & ICAR-Based Test \cite{condon2014international}; AUT \cite{guilford1967creativity,organisciak_beyond_2023} \\
        Essay Review & Peer; External Expert & AI \cite{xiao2024humanaicollaborativeessayscoring}; External Expert \\
        Grade and Achievement Review & None & External Expert \\
        Finalist Activity Review & Selector & Unknown \\
        \bottomrule
    \end{tabular}
  \end{table}
  
\section{The Rise Programme}\label{ssec:rise}
\subsection{Programme Overview}
Initiated from a $\$1$-billion investment, Schmidt Futures and the Rhodes Trust's Rise programme\footnote{https://www.risefortheworld.org/} finds and selects talented and disadvantaged 15-to-17-year-olds from around the world and helps them achieve their full career and service potential. Rise supports selected `winners' and `finalists' with a variety of benefits accessible at different points in their lives. We work with Rise from the programme's inception in 2021 until 2024. In each of these years, Rise has committed to selecting up to 100 winners and up to 500 finalists from their pool of applicants. In the four application cycles between 2021 and 2024, during which time Rise has selected 400 winners and nearly 2000 finalists from hundreds of thousands of started applications.

Rise uses a flexible benefits model, where winners (and, in some cases, finalists) gain access to a variety of potential resources, but utilise only resources they demonstrate a need for. For example, applicants who receive full scholarships to their university may not receive an academic scholarship from Rise. Programme benefits include academic scholarships, educational resources and programmes, networking opportunities, and even funding for winner-led startups.\footnote{As academic scholarships comprise a large portion of programme benefits, we speak about Rise as a ``scholarship and talent investment'' programme throughout. This is our own language and is intended to further anonymise the programme. Similarly, we replace the programme-specific term ``winners'' with ``scholars''.}

\subsection{The Selection Process}
The programme uses a two-stage selection process designed to be accessible to candidates from various global and socioeconomic backgrounds. In stage one, applicants submit various application materials asynchronously; Rise selects up to 500 finalists based on the quality of those materials and the programme's cohort composition goals. In stage two, finalists engage in one of several ``finalist days'' consisting of various collaborative live activities and an interview; after all finalist days are completed, Rise uses information from both stages to select 100 winners.

\begin{figure}[htbp]
    \centering
    \begin{subfigure}{.45\textwidth}
        \centering
        \includegraphics[width=\linewidth]{context/proj1.png}
        \caption{Sample Project 1}
        \label{sfig:can}
    \end{subfigure}
    \hfill
    \vspace{1em}
    \begin{subfigure}{.45\textwidth}
        \centering
        \includegraphics[width=\linewidth]{context/proj2.png}
        \caption{Sample Project 2}
        \label{sfig:cell}
    \end{subfigure}
    \hfill
    \vspace{1em}
    \begin{subfigure}{.45\textwidth}
        \centering
        \includegraphics[width=\linewidth]{context/proj3.png} 
        \caption{Sample Project 3}
        \label{sfig:app}
    \end{subfigure}
    \caption{The three panels in this figure depict slides from a programme presentation that highlighted three projects that were submitted as part of Rise's 2021 application cycle. Figure \ref{sfig:can} depicts an AI-dependent recycling bin paired with a mobile app to track and sort plastics. Figure \ref{sfig:cell} depicts schematics for a clean fuel cell using microorganisms and garbage. Figure \ref{sfig:app} depicts an app to help educate students about the importance of various technical and character skills. Names have been removed and pictures blurred to de-identify programme applicants.}
    \label{fig:example_projects}
\end{figure}

Stage one of selection occurs asynchronously (via smartphone, laptop, or, in rare cases, pen and paper) and in two parts. The first part requires applicants to submit an application form with their demographic information and either video or written essays. The first of these two essays explains a real-world problem the applicant wishes to solve, while the second discusses either how they consider themselves privileged or the challenges they have overcome.\footnote{Refinements to the application process between years all application materials change slightly over time. This change is most dramatic in the case of this second essay, where the focus of the essay changed from an applicant's declaration of their privilege to a description of a challenge the applicant has faced and how they overcame it. Changes appear throughout the application across years; we only detail them where it is relevant to our research.} In the second part, applicants complete a set of digital cognitive assessments and a project showcasing their talent.\footnote{In rare cases, technology or accessibility limitations prevented applicants from completing the cognitive assessment; these applicants were considered on the merits of their submitted materials.} The project showcase is a distinctive part of Rise's selection process whereby participants: (1) identify a problem they wish to solve, (2) research solutions to that problem, (3) implement those solutions, and (4) reflect on what they learned from the project. Participants submit one essay (video or text) on each of these four stages. Three example projects can be found in Figure \ref{fig:example_projects}. All applicants also submit a short written essay explaining their project and its significance. For an overview of the stage one selection design, see Figure \ref{fig:design}.

\begin{figure}[!htb]
    \centering
    \caption{This figure schematizes the key elements of the talent investment programme's data collection and selection process. }
    \label{fig:design}
    \includegraphics[width=\textwidth]{spf/selection_design_schematic.png} 
\end{figure}

Stage two of selection occurs synchronously (though still remotely) in one of several ``finalist days''. The finalist days consist of up to five activities of three types: presentations (where finalists present information about their project), group activities (where finalists collaborate to discuss and solve problems), or interviews (where finalists are interviewed). All activities were judged by a pool of adult `selectors' who assessed finalists according to a rubric. Winner selection decisions were made based both on data collected in stage two and information retained from stage one.

\subsection{Data Collection}
Across the application cycle, Rise collects a variety of data from applicants. This data includes traditional merit-based measures – including cognitive tests, written essays, and referrals – as well as non-traditional measures – including peer-reviewed video essays, gamified skill tests, and application platform behaviours. Many of these measures are used only for research purposes, and some are tangential to our research on supporting the selection process. We discuss relevant measures here. More detail on these measurements, especially for the 2021-2023 application cycles, can be found in Chapter \ref{ch:spf}, where findings depend on the specifics of Rise programme measurement. A comparison of the measurements used by Rise and the Ellison Scholars programme can be found in Table \ref{tab:measures}.

\paragraph{Cognitive Assessments}
Rise collects two data from two cognitive assessments taken by applicants. The first is based on the International Cognitive Assessment Resource (ICAR) \cite{condon2014international, subotic2020psychometric}, and has, in various selection cycles, incorporated four different item types: Cube Rotation, Number Sequence, Matrix Reasoning, and Verbal Reasoning. Applicants are given nonverbal and verbal sub-scores, which use a Bayesian generalized linear item response model \cite{burkner2021bayesian}. In some cycles, only the nonverbal score was used, while other cycles combined the two to create one singular score.

The second cognitive assessment consists of a gameified skills test called Roomworld \cite{Dumbalska_Bhatti_Ali_Summerfield_2023}.\footnote{Roomworld was created and scored by \textcite{Dumbalska_Bhatti_Ali_Summerfield_2023}; the specific scoring algorithm was not shared.}

\paragraph{Peer Review}
Stage one applicant essays were judged by two types of human evaluators: other applicants (peers) and adults with some expertise on the project topics (experts). Though \textcite{Anvari2021EffectivenessOP} provide evidence for the efficiency and effectiveness of peer review as a measurement of aptitude, peer review was (and remains) experimental \cite{Rahmatillah2022AnalyzingFA}. That said, \textcite{VanderSchee2022UsingCP} find that decision subjects of a blind peer review process experience just outcomes both according to the similar treatment and similar outcomes principles. Thus, though Rise treated peer reviews as experimental, peer scoring played an integral role in the Rise process. To collect peer reviews, each applicant was assigned to review 20 of each essay submitted. Each review consisted of Likert scale judgements designed to measure: intelligence, perseverance, empathy, integrity, sense of calling, and impact on the applicant.

\paragraph{Expert Review} 
Experts, on the other hand, were only asked to assess applicant project essays. Each reviewer was assigned several projects proportional to their capacity to review. Like peers, experts were asked to review different elements of the project, using Likert scales to gauge how effective the project was at accomplishing what the applicant intended and how impressive the project was relative to other projects in this field. 

\paragraph{Finalist Day Activities}
The finalist day activities were assessed by selectors through a mix of qualitative and quantitative measures. Each activity was scored on a rubric, and the scores were aggregated to create a final score for each finalist on each activity type. Additionally, selectors were given an option to provide specific qualitative feedback on applicants.

\section{The Ellison Scholars Programme}\label{ssec:ellison}
\subsection{Programme Overview}
Funded and administered by the Ellison Institute of Technology, The Ellison Scholars Programme\footnote{https://eit.org/ellisonscholars/} is a global scholarship programme that seeks to develop global technology innovators and leaders by finding and supporting talented people passionate about solving humanity’s most serious problems as they study at the University of Oxford and solve global problems through innovation. The programme seeks to select at least twenty scholars each year beginning in 2025.

As the Ellison Scholarship's inaugural cohort has yet to be selected, programme benefits have yet to be dispersed. However, the programme has committed to providing scholars with an academic scholarship to the University of Oxford and paid internships. These internships, as well as the programme as a whole, are organised around four humane endeavours: (1) Health and Medical Science, (2) Food Security and Sustainable Agriculture, (3) Climate Change and Clean Energy, and (4) Government Innovation and Era of Artificial Intelligence.

\subsection{The Selection Process}
As humane endeavours form a large part of the Ellison Scholarship, the selection process places special emphasis on suitability for these topics. The programme has applicants declare their humane endeavours of interest and assesses applicants relative to these humane endeavours. Additionally, as the programme seeks to pursue all four endeavours, diversity-like considerations require the programme to ensure that scholars are chosen for each humane endeavour. However, though many selectors on the Ellison Scholarship selection team are sympathetic to desires for demographic diversity, the programme's theory of change does not lend itself to explicit diversity considerations (unlike Rise)

The Ellison Scholarship employs a three-stage selection process. In stage one, applicants submit various application materials asynchronously; the programme selects semi-finalists based on the quality of those materials and the programme's cohort composition goals. In stage two, semi-finalists apply to the University of Oxford, and the University handles its internal selection process; programme applicants who receive Oxford scholarships are dubbed Finalists. In stage three, finalists engage in a series of synchronous activities before final decisions are made by the programme's board. 

In stage one of selection, applicants submit their demographic information; selections for humane endeavour, Oxford course, and preferred project; their education record; a list of achievements; and four written essays speaking to their suitability for the programme. These essays speak to the applicant's alignment to their chosen humane endeavour, alignment to their chosen course at Oxford, and their particular skills and archetype. After submitting this application, all applicants are invited to take a cognitive assessment assessing convergent and divergent reasoning.

In stage two, semi-finalists apply to the University of Oxford; in stage three, finalists engage in a series of synchronous activities before final decisions are made by the programme's board. As the Ellison Scholars programme is still selecting its inaugural cohort, neither stage two nor stage three have been enacted. Thus, we omit details on them here. 

\subsection{Data Collection}
The Ellison Institute collects and constructs several different aptitude measurements of applicants. This is primarily traditional merit-based measures, e.g., cognitive tests, written essays, or academic transcripts. Additionally, the programme constructs experimental measures from gathered data. We discuss relevant measures here. A comparison of the measurements used by Rise and the Ellison Scholars programme can be found in Table \ref{tab:measures}.

\paragraph{Cognitive Assessments} 
Much like the Rise programme, the Ellison Scholarship uses a cognitive assessment based on the International Cognitive Assessment Resource (ICAR) \cite{condon2014international,subotic2020psychometric}. Though the details of implementation differ, both programmes use the same four item types and the same scoring algorithm \cite{burkner2021bayesian}.

Additionally, the Ellison Scholarship relies on a divergent thinking assessment based on Guildford's Alternative Uses Task (AUT) \cite{guilford1967creativity}. This task is chosen for its ability to measure ``divergent thinking'', i.e., creativity or innovativeness \cite{dumas_measuring_2020,organisciak_beyond_2023}. Each question is timed at roughly 90 seconds per question, and each applicant is given ten questions. Applicants are scored according to \textcite{organisciak_beyond_2023}'s Open Creativity Scoring with Artificial Intelligence (OCSAI) scoring algorithm.\footnote{https://openscoring.du.edu/}

The Ellison Scholarship combined both ICAR and AUT scores to get an overall cognitive assessment score.

\paragraph{AI-driven Assessment of Essays}
The Ellison Scholarship employs an AI-based scoring method as a preliminary screen on applicants' four written essays. The programme's approach broadly follows the approach of \textcite{xiao2024humanaicollaborativeessayscoring}; the Ellison Scholars programme requested that specific details of the programme's implementation not be shared. After these four essays are scored, an overall AI-driven score of applicants is calculated.

\paragraph{Expert Assessment of Applications}
Stage one applicants whose test scores or AI-driven essay scores merited further consideration were judged by expert human evaluators in two types of reviews: anonymous reviews (where reviewers only had access to applicant essays, grades and achievements) and contextual reviews (where reviewers had access to supporting information such as references or applicant demographics). As compared to Rise's experts, the Ellison Institute engaged adult reviewers in a rigorous training process before qualifying them as expert reviewers.

In each review, experts judged applicants on axes related to specific programme goals (e.g., whether the applicant demonstrated an interest in the humane endeavour they chose). Anonymous and contextual reviews were ultimately pooled, and an overall review score was calculated.

\paragraph{Semifinalist and Finalist Assessment}
As the programme is still selecting its inaugural cohort, they have yet to undergo semi-finalist or finalist assessment. We thus omit the details of these assessments from this thesis.